\procedure{Relative à la gestion des différends}

%Se doter d'une procédure de gestion des différents.
% Avant la composition du comité, conciliation / médiation
% À qui on dépose la plainte? -> Président, VPAL, CE
% Composition d'un sous-comité de discipline.
%- Une personne déléguée des SVE désignée par la directrice des SVE
%- Un membre de la faculté désigné par le secrétaire de faculté
%- Un étudiant désigné par le CA ou le CE de l'AGEG
% Reccomandations au CA
% Faire attention à la légitimité d'agir de la faculté.

\preambule{En tant qu’association étudiante, l'AGEG veut promouvoir un milieu sain et sécuritaire dans lequel les personnes sont traitées avec respect et dignité. Par cette procédure, elle s'engage à prendre toutes les mesures, les décisions et les actions qui s’imposent pour redresser les situations de conflit.}

\partie{Champs d'application}
\alinea{Cette procédure s’applique, de manière générale, à tous les membres de l'AGEG incluant sans s'y limiter le personnel, les personnes administrantes, les personnes exécutantes, les personnes qui participent aux activités organisées par l'Association et les personnes de l’extérieur qui, pour des motifs liés au travail, aux études ou à l’utilisation des services,  fréquentent l'AGEG (fournisseurs, clients, partenaires, etc.).}

\partie{Rôles et responsabilités}
%Une personne reçoit la plainte, l'apporte au CE. Le CE détermine une personne pour la conciliation

\article{Conseil d'administration}
\alinea{}


\article{Comité exécutif}

\alinea{Le comité exécutif intervient au besoin pour prévenir ou régler des situations à risque dans leur unité.}
\alinea{Le comité exécutif s'engage à mettre en place les moyens pour faire cesser le harcèlement psychologique à l'AGEG.}
\alinea{Le comité exécutif s'engage à rediriger les personnes directement impliquées dans le conflit vers des ressources d'aide psychologique.}


\article{Vice-présidence aux employés et aux communications}
\alinea{\add{Dans le cas d'une plainte provenant du personnel,} la vice-présidence aux employés et aux communications reçoit les plaintes, \replace{fait enquête}{s'informe de la situation et} dirige les employés vers des ressources spécialisées au besoin. \remove{recommande des mesures disciplinaires ou des interventions pour régler des situations conflictuelles ou suite au traitement d'une plainte.} Dans le cas où la vice-présidence aux employés et aux communications est directement impliquée dans le conflit, c'est la présidence qui reçoit la plainte et qui gère la situation.}

\article{Vice-présidence XX}
\alinea{ vice-présidence XX reçoit les plaintes et se met au courant de la situation, dirige les membre vers des ressources spécialisées au besoin, \remove{recommande des mesures disciplinaires ou des interventions pour régler des situations conflictuelles ou suite au traitement d'une plainte}. Dans le cas où la vice-présidence XX est directement impliquée dans le conflit, c'est la présidence qui reçoit la plainte et qui gère la situation.}


\partie{La procédure applicable}

\article{Le dépôt de la plainte}
\alinea{Tout membre ou membre du personnel qui estime faire l'objet ou être témoin de harcèlement ou de violence peut, s'il le désire, en discuter avec la vice-présidence XX, la présidence ou tout autre membre du comité exécutif.}
\alinea{Le membre ou membre du personnel qui désire faire une plainte doit remplir le formulaire de dépôt de plainte de l'AGEG et le faire parvenir à la vice-présidence XX. Si celle-ci est en conflit dans la situation, le membre ou membre du personnel peut décider de remettre son formulaire directement à la présidence. Si celle-ci est en conflit, le membre ou membre du personnel peut décider de remettre son formulaire directement à une personne administratrice annuelle.}
\alinea{Si nécessaire, une intervention d'urgence est mise en place pour faire cesser immédiatement le harcèlement ou le comportement violent.}
\alinea{Les victimes de harcèlement ou de violence en milieu de travail sont invitées
à consulter un professionnel de la santé de leur choix pour un traitement ou une référence dans le cas d'une blessure ou d’un symptôme défavorable.}

\article{Traitement de la plainte}
\alinea{La vice-présidence aux employés et aux communications procède à une enquête diligente, impartiale et discrète.}
\alinea{L'enquête débute dès le dépôt de la plainte.}
\alinea{La vice-présidence aux employés et aux communications peut, si elle le juge nécessaire, demander l’assistance d’un consultant externe, ou déléguer à cette personne l’entière responsabilité du
traitement de la plainte, à condition d’en aviser la personne plaignante.}
\alinea{La personne plaignante, la personne visée par la plainte et les témoins identifiés sont rencontrés individuellement, accompagnés s’ils le désirent d’une personne de
leur choix. Des versions écrites et signées des personnes rencontrées sont recueillies.}
\alinea{À la suite de l’enquête, une analyse est effectuée et une décision est prise soit sur le fondement (en partie ou en totalité), soit sur le non-fondement des allégations de la plainte.}
\alinea{La personne plaignante et la personne visée par la plainte sont informées de la décision.}
\alinea{Des mesures disciplinaires ou administratives peuvent aussi être
recommandées.}
\alinea{Le rapport écrit final de l’enquête doit être fourni au conseil d'administration dans les six semaines du début de l’enquête.}

\article{Les mesures disciplinaires}
\alinea{Le conseil d’administration s’engage à prendre les mesures propres à sanctionner toute conduite adoptée en contravention avec la présente politique.}
\alinea{Les mesures envers la personne dont la conduite a été jugée harcelante ou inappropriée dépendent notamment de la nature, des circonstances et de la gravité des incidents reprochés.}

\article{Confidentialité}
\alinea{Toute plainte sera traitée confidentiellement.}
\alinea{Certains renseignements personnels pourront devoir être divulgués à des tiers, mais strictement dans la mesure où cela est nécessaire au traitement de la plainte.}
\alinea{Toute personne impliquée dans la gestion d’une plainte sera tenue de ne pas discuter des faits entourant la plainte avec d’autres personnes, sauf à des fins autorisées par la Loi, par cette politique ou à des fins de consultation auprès d’un conseiller externe, le cas échéant.}

\alinea{Les personnes qui, de bonne foi, se prévalent de la politique sont protégées contre les représailles qui
pourraient être prises à leur endroit parce qu’elles ont utilisé la politique. Il en est de même des témoins.}

\article{Les plaintes frivoles ou vexatoires}
\alinea{Une plainte doit être sérieuse et faite de bonne foi.}
\alinea{Un employé ou un membre de l'AGEG qui déposerait une plainte frivole ou une accusation de mauvaise foi ou avec l’intention de nuire se verra imposer des mesures disciplinaires.}