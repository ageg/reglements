\chapter*{Définitions}
\addcontentsline{toc}{chapter}{Définitions}
\begin{description}
\item[Groupe étudiant :]{Groupe accrédité par l'AGEG, que ce soit un groupe technique (affilié ou interne), un groupe de l'AGEG ou une promotion.}
\item[Groupe technique :]{Ensemble des groupes internes et des groupes affiliés.}
\item[Groupe technique interne :]{Groupe d'étudiants ou d'étudiantes qui réalise des activités d'ingénierie ou relié à l'ingénierie à partir des ressources de la corporation tel que défini dans l'entente entre le groupe et l'AGEG.}
\item[Groupe technique partenaire]{Organisation  partenaire dont  le  groupe  d’étudiant ou d'étudiantes  réalise  des  activités  d’ingénierie  ou  reliée  à l’ingénierie  à  partir  de  ressources  provenant  d’une  autre  corporation. Le partenariat  avec  l’AGEG  est  régi  par  l’entente  entre  le  groupe  et l’AGEG.}
\item[Groupe de l’AGEG :]{Groupe créé par règlement, dont la sélection des membres et le suivi des activités est effectué par le conseil d'administration.}

\end{description}
\vspace{5mm}
\begin{description}
\item[Promotion :]{Ensemble des étudiants et étudiantes qui compléteront dans la même année scolaire leur baccalauréat.}
\item[Année scolaire :]{Année débutant le premier jour de la session d'automne et se terminant la dernière journée de la session d'été.}
\item[Année finissante :]{Intervalle de temps entre 2 dates de passations consécutives, tel que défini dans les ententes inter-promo.}
\item[Promotion sortante :]{Promotion qui, suite à la date de passation, effectue présentement la fin de sa dernière session au baccalauréat et qui a encore des obligations légales et financières envers l’AGEG.}
\item[Promotion finissante :]{Promotion qui effectue présentement son année finissante.}
\item[Promotion  de première année (ou entrante):]{Promotion qui effectue présentement sa première année scolaire au baccalauréat.}
\item[Promotion de deuxième année :]{Promotion qui effectue présentement sa deuxième année scolaire au baccalauréat.}
\item[Promotion de troisième année :]{Promotion qui effectue présentement sa troisième année scolaire au baccalauréat.}
\item[Promotion de quatrième année :]{Promotion qui effectue présentement sa quatrième année scolaire au baccalauréat jusqu'à sa date de passation tel que définit dans l'entente inter-promo.}
\end{description}
