\section{Délégation de la Compétition Québécoise d’Ingénierie}

\article{Préambule}
\alinea{La délégation de la Compétition Québécoise d'Ingénierie (CQI) est un groupe de l'AGEG dont l'objectif est d'organiser la participation des membres de l'AGEG à la Compétition Québécoise d'Ingénierie puis à la Compétition Canadienne d'Ingénierie.}

\article{Nomination des membres de la délégation}

\alinea{Le chef ou les co-chefs de la délégation sont sélectionnés par le CA à la session d'hiver, sous recommandation d'un comité de nomination dirigé par la VPEX.}

\alinea{Le comité de nomination des co-chefs est formé par le CA lors du CA2 de l'hiver, sous recommandation de la VPEX, et est constitué de deux personnes administratrices et de la VPEX. Une priorité est donnée aux membres du CA ayant déjà participé à la CQI.}

\alinea{Le comité de nomination est responsable d'organiser une sélection ouverte à tous les membres de l'AGEG. Cette sélection doit être annoncée au moins 2 semaines à l'avance.}

\alinea{Le chef ou les co-chefs sont responsables d'organiser les sélections (pré-CQI) pour le reste de la délégation en collaboration avec l'université hôtesse de la CQI. Le chef ou les co-chefs ont le droit de participer à la sélection.}

\alinea{Si des places sont vacantes après les sélections, les chefs ou les co-chefs peuvent combler la délégation de la façon qui leur convient.}

\alinea{Les critères de sélection des chefs ou des co-chefs sont les suivants:}
\sousalinea{Être membre de l'AGEG}
\sousalinea{Motivations des candidats }
\sousalinea{Idées des candidats quant à la CQI}
\sousalinea{Implications antérieures et actuelles des candidats au niveau universitaire}
\sousalinea{Participation antérieure des candidats à la CQI}
\sousalinea{Tous autres points jugés pertinents.}


\article{Gestion de la délégation}
\alinea{Le chef ou les co-chefs de la délégation doivent produire un rapport sur l’année de son mandat après la tenue de la compétition y incluant la liste de tous les membres de la délégation au plus tard pour le CA2 de la session d’hiver.}
\alinea{Un des chefs est responsable de la gestion financière de la délégation.}
\alinea{Le chef ou les co-chefs sont responsable de faire le lien entre la délégation et l'AGEG.}
\alinea{La VPEX de l'AGEG est responsable de s'assurer de la bonne gestion de la délégation.}


\article{Rôles du Chef ou des co-chefs}
\alinea{S'occuper de la gestion financière du groupe.}
\alinea{Superviser le travail fait par les délégués.}
\alinea{S’occuper des communications à l’intérieur de la délégation.}
\alinea{Faire le lien entre l’Université hôtesse et la délégation.}
\alinea{Appuyer les membres sélectionnés pour représenter le Québec à la Compétition Canadienne d'Ingénierie dans l'organisation de leur délégation.}


\article{Démission ou destitution}
\alinea{Une démission d'un chef ou d’un co-chef de la délégation doit être transmise à la VPEX.}
\alinea{Un chef ou un co-chef peut se voir expulsé de la délégation par le CA de l’AGEG ou l’assemblée générale de l’AGEG pour des raisons jugées valables.}
\alinea{Un poste laissé vacant sera pourvu de façon intérimaire par la VPEX de l’AGEG jusqu’à la nomination d’une nouvelle personne par le CA de l’AGEG.}
