\chapter{Administration}

\section{Gestion des documents administratifs}

\article{Archivage et accès aux documents}
\alinea{La Vice-Présidence aux affaires légales (VPAL) gère, uniformise et conserve les documents confidentiels et les contrats de la corporation. La Vice-Présidence aux affaires internes (VPAI) gère, uniformise et conserve les autres documents de la corporation. Elles doivent s'assurer de mettre en place tout dispositif jugé nécessaire pour s'assurer de la préservation des documents.}
\alinea{La VPAL est responsable d'archiver toutes les versions des règlements généraux.}
\alinea{La Présidence (Prez), la VPAL, la  Vice-Présidence aux affaires financières (VPAF) et la coordination administrative possède un accès au coffre et à tous les documents administratifs de la corporation.}
\alinea{La  PREZ est responsable de fournir, au besoin, les accès aux documents confidentiels à d'autres personnes.}
\alinea{Chaque membre de l’AGEG responsable d’une activité corporative doit produire et rendre les rapports et les comptes rendus relatifs à ses activités.}

\article{Documents accessibles aux membres}
\alinea{L’ensemble des membres de l’AGEG a accès en lecture à la dernière version des règlements généraux, du cahier de procédure et du cahier de position. Cette version doit être disponible sur le site internet de la corporation.}
\alinea{L’ensemble des membres de l’AGEG a accès en lecture aux procès-verbaux de la corporation, hormis ceux sous huis-clos.}
\alinea{L’ensemble des membres de l’AGEG a accès en lecture aux rapports de fin de mandat des exécutants.}

\article{Documents financiers de l’AGEG}
\alinea{Ces documents doivent être conservés dans les classeurs et le logiciel comptable. Après cinq ans, les originaux sont envoyés au service des archives de l’Université de Sherbrooke.}

\section{Rencontre de transition et rapport de fin de mandat}
\alinea{Une rencontre de transition doit être effectuée pour chacun des postes entre l’exécutant actif et son homologue qui reprend son poste afin d’assurer une continuité dans les dossiers en cours. Si une rencontre physique ne peut être effectuée, un rapport de transition doit être produit par l’exécutant sortant.}
\alinea{Un rapport de fin de mandat doit être écrit et archivé par chacun des exécutants sortants et servir de mémoire au bénéfice des sessions à venir dans un délai maximal de deux semaines après la fin du mandat.}
\alinea{Le rapport de fin de mandat doit contenir les projets réalisés, leurs conclusions et leurs recommandations.}

\section{Gestion des systèmes informatiques}
\alinea{La VPAI est responsable de nominer et encadrer un directeur informatique pour l'appuyer dans ces tâches.}
\alinea{La direction informatique et la VPAI sont responsables notamment du système d'archivage de l'AGEG, des boîtes courriels, de la plateforme électorale, du site web et de l'affichage électronique.}


\section{Gestion des ressources humaines}
\alinea{La VPAF approuve les heures du coordination administrative et de la personne adjointe administrative.}
\alinea{La VPAI s'assure de la gestion des personnes préposées à l'accueil ainsi que de leurs heures de travail.}
\alinea{La VPAL doit s'assurer du respect des normes du travail.}


\article{Évaluation des personnes employées}
\alinea{L’évaluation des personnes employées a pour but de conserver un dossier des performances de nos ressources humaines. L’évaluation permet de juger de la qualité du travail de la personne employée.}
\alinea{La VPAI réalise les évaluations des personnes préposées à l'accueil durant des rencontres individuelles et produit les rapports associés  pour chaque personne évaluée.}
\alinea{La PREZ réalise les évaluations des autres personnes employées durant des rencontres individuelles et produit les rapports associés pour chaque personne évaluée.}
\alinea{Les évaluations doivent être faites en utilisant un gabarit pré-établit.}
\sousalinea{La première partie de l’évaluation doit être faite tout juste après l’embauche de la personne employée.}
\sousalinea{La deuxième partie doit être réalisée au plus court entre le mi-mandat et un semestre.}
\sousalinea{La troisième partie doit être réalisée tout juste avant la fin du mandat de la personne employée ou avant un renouvellement de contrat.}
\alinea{Les parties 1 et 2 de l’évaluation sont disponibles aux membres du CE et du CA.}
\alinea{La partie 3 de l’évaluation doit être conservée en format papier dans les classeurs de la corporation et être détruites 5 ans après le dernier mandat de la personne employée.}
