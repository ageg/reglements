\section{Comité organisateur de l’Oktoberfest}

\article{Préambule}
\alinea{Le comité organisateur de l'Oktoberfest est un groupe de l'AGEG dont l'objectif est d'organiser l'Oktoberfest.}

\article{Nomination du comité organisateur}

\alinea{Le comité organisateur est nominé par le CA à la session d'automne, sous recommandation d'un comité de nomination dirigé par le VPAS.}

\alinea{Le comité de nomination du comité exécutif est formé par le CA lors du CA3 de l'automne, sous recommandation du VPAS, et est constitué de deux personnes administratrices, du VPEX et d'un ancien membre du CO de l'Oktoberfest. Une priorité est donnée aux membres du CA ayant déjà participé à l'organisation de l'Oktoberfest.}

\alinea{Le comité de nomination est responsable d'organiser une sélection ouverte à tous les membres de l'AGEG. Cette sélection doit être annoncé au moins 2 semaines à l'avance.}

\alinea{Les personnes se présentant pour le poste de président assistent le comité de nomination pour la sélection des autres membres du comité organisateur.}

\alinea{Un poste du comité exécutif laissé vacant sera comblé par un membre de l’AGEG nommé par le conseil organisateur.}

\alinea{Les critères de sélection du comité organisateur de l’Oktoberfest sont}
\sousalinea{Le président du comité organisateur doit être un membre de l'AGEG. Une priorité est donné aux membres de l'AGEG pour les autres postes.}
\sousalinea{Compétences des candidats;}
\sousalinea{Idées des candidats quant à l'Oktoberfest;}
\sousalinea{Implications antérieures et actuelles des candidats au niveau universitaire;}
\sousalinea{Participation antérieure des candidats quant à l'organisation de l'Oktoberfest;}
\sousalinea{Représente la relève de l'implication étudiante;}
\sousalinea{Tout autre point jugé pertinent.}


\article{Mandat du comité organisateur}
\alinea{Le comité organisateur de l’Oktoberfest a comme mandat d’organiser, de tenir et de superviser l'Oktoberfest. Il est responsable de former les sous-comités par un appel général.}
\alinea{Il doit transmettre les informations sur l’avancement de la planification des activités au conseil d'administration.}
\alinea{Il doit recruter et former les bénévoles nécessaires au bon déroulement de l'Oktoberfest.}
\alinea{Il doit prioriser les membres de l'AGEG ainsi que les groupes techniques de l'AGEG lors du recrutement des bénévoles.}\color{black}
\alinea{Il doit s'assurer de donner les formations nécessaires : Éduc’alcool, premiers-soins (5 \%) et toute autre formation jugée nécessaire.}
\alinea{Il doit veiller à l’application du guide sur le développement durable et de l’écoresponsabilité de l’AGEG.}

\article{Supervision du comité organisateur par l'AGEG}
\alinea{L’AGEG, représentée par le VPAS, s’assure du suivi des objectifs et de la mission de l'Oktoberfest. En cas de non-respect des valeurs exprimées à travers les objectifs et la mission de l'Oktoberfest, l’AGEG a un droit de veto sur toutes activités découlant du comité organisateur.}

\article{Gestion de l'Oktoberfest}
\alinea{En conformité avec la mission, les valeurs et les objectifs du présent règlement, le comité organisateur de l’Oktoberfest a le pouvoir décisionnel sur tous les aspects organisationnels des sous-comités et des activités reliées à l’événement.}
\alinea{Le comité organisateur de l’Oktoberfest choisit la façon d’élire les directeurs des sous-comités.}
\alinea{Le comité organisateur de l’Oktoberfest se réserve le droit de destituer un membre ou un directeur d’un sous-comité advenant un comportement inadmissible avant ou durant la tenue des activités de l'événement.}
\alinea{Le comité organisateur doit rester sobre tout au long de l'évènement.}

\article{Président}
\alinea{Le Président doit être membre de l’AGEG;}
\alinea{Être cosignataire du compte bancaire de l'Oktoberfest avec le trésorier;}
\alinea{Superviser le travail du comité organisateur et des sous-comités;}
\alinea{S’occuper des communications à l’intérieur du comité organisateur;}
\alinea{Convoquer et diriger les réunions du comité organisateur de l’Oktoberfest et des sous-comités;}
\alinea{S’assurer que les dossiers de chacun des VP avancent normalement;}
\alinea{Faire le lien entre le comité et le CA de l’AGEG, en étant présent au minimum au CA4 de l’hiver, au CA 2 et 3 de l’été et au CA1 de l’automne pour faire part des avancements du comité au CA;}
\alinea{Produire un rapport final sur la planification et le déroulement des activités de l'activité, qui sera déposé au CA3 de la session d’automne;}
\alinea{Il doit s'assurer que les tâches des postes laissés vacants soient effectuées à chaque session;}


\article{Trésorier}
\alinea{Être cosignataire du compte bancaire de l'Oktoberfest avec le président;}
\alinea{S'assurer de la sécurité de l'argent au cours de la soirée;}
\alinea{Apprendre le fonctionnement des caisses électroniques, leur mise en place, la production de rapports, etc.;}
\alinea{Communiquer les besoins en matière d'électricité au VP logistique;}
\alinea{Communiquer les besoins en matière d’éclairage au VP Logistique et Audio-visuel}
\alinea{S'assurer de la formation du personnel pour la tenue des caisses}
\alinea{Commander les bocks pour l’événement, au plus tard pour la mi-août;}
\alinea{Prendre les commandes de bocks des associations du campus de Sherbrooke et des associations externes;}
\alinea{Commander tous les billets requis pour l'événement (entrée, remplissages, etc.);}
\alinea{S'assurer d'avoir la monnaie nécessaire pour la soirée;}
\alinea{S'occuper de la gestion des factures et de leur entrée adéquate dans le logiciel comptable de l’AGEG;}
\alinea{Obtenir des assurances pour l'Oktoberfest;}


\article{VP Logistique}
\alinea{Définir un emplacement et fixer une date pour la tenue de l'Oktoberfest de l’année suivante, après l’événement de l’année en cours;}
\alinea{Assurer une communication constante avec les responsables de l’endroit où se tient l’Oktoberfest;}
\alinea{Planifier et louer le matériel nécessaire à la tenue de l'Oktoberfest(toilettes, véhicule, walkie-talkie, hôtel, etc.);}
\alinea{Faire le plan  du site de l'Oktoberfest (électrique, entrée d'eau, terrasse, etc.);}
\alinea{S’assurer de décorer l’endroit de façon à recréer une ambiance bavaroise, et coordonner le tout avec les équipements audiovisuel;}
\alinea{Déterminer l'horaire de la soirée (ouverture, bouffe, ouvertures, fermetures) les horaires de montage, de démontage, des bénévoles et de la soirée.}
\alinea{S’assurer d’un transport adéquat des participants entre le lieu de l’événement et les différents campus de l’Université de Sherbrooke;}
\alinea{Coordonner les demandes en électricité, eau et espace de la brasserie;}


\article{VP Sécurité}
\alinea{S'assurer d'avoir une équipe professionnelle de sécurité en conformité avec le nombre d'agents nécessaires sur la convention avec l'établissement hôte;}
\alinea{S’assurer de la présence d’une équipe de premiers soins et d’urgence adéquate (ambulance, équipe premiers soins, etc.);}
\alinea{S’assurer de la communication entre l’organisation et l’équipe de sécurité en place;}
\alinea{Informer les membres de tous les comités des procédures de résolution de problèmes;}
\alinea{S'occuper des communications entre la police de Sherbrooke et l'événement;}
\alinea{Communiquer avec le service incendie de la ville de Sherbrooke pour leur faire part de nos plans
et confirmer les capacités;}
\alinea{S’assurer que l’analyse de risque est à jour;}
\alinea{S'assurer que le plan de contingence est à jour;}
\alinea{S’assurer de la présence d'un service de raccompagnement sur les lieux de l'événement}


\article{VP Audio-visuel}
\alinea{Communiquer les besoins en matière d'électricité au VP logistique;}
\alinea{Coordonner les demandes d’équipement visuel (éclairage, écrans, projecteurs) avec le VP Logistique, VP bière et le VP Communication}
\alinea{Réserver le système de son et le matériel visuel de l'Oktoberfest;}
\alinea{Trouver un directeur animation;}
\alinea{Veiller à l'installation et la désinstallation des systèmes audio-visuels;}\alinea{S'assurer de la présence d'un groupe de type bavarois;}
\alinea{Déterminer l'horaire audiovisuel de la soirée (animation, groupes musicaux, DJ);}


\article{VP Communications}
\alinea{Créer le design d’impression pour les bocks de l'Oktoberfest;}
\alinea{S'assurer d'avoir les logos des associations étudiantes du campus et de l'externe en format vectoriel pour l’impression du bock;}
\alinea{Rédiger les articles et communiqués officiels;}
\alinea{Promouvoir l'événement à l'interne et à l'externe du campus;}
\alinea{Commander les articles promotionnels;}


\article{VP Bière}
\alinea{S'assurer d'avoir l'alcool nécessaire au bon déroulement de l'activité;}
\alinea{Obtenir le permis d'alcool;}
\alinea{Communiquer les besoins en matière d'électricité au VP logistique;}
\alinea{S'assurer d'avoir le personnel et les installations nécessaires pour le service d'alcool;}


\article{VP Bouffe}
\alinea{S'assurer d'avoir la nourriture nécessaire au bon déroulement de l'activité;}
\alinea{Communiquer les besoins en matière d'électricité et d'eau au VP logistique;}
\alinea{S'assurer d'avoir le personnel pour le service de nourriture;}
\alinea{S'assurer  d'avoir les équipements de cuisson et de service nécessaire;}
\alinea{Commander et préparer la nourriture pour maximiser le nombre d’assiettes bavaroises servies;}
\alinea{Prévoir un plan de service de nourriture et de nettoyage;}

\article{Gestion de l'Oktoberfest}
\alinea{Le trésorier et le président est responsable de la gestion financière de la délégation.}
\alinea{Le président est responsable de faire le lien entre le comité et l'AGEG.}
\alinea{Le VPAS de l'AGEG est en charge de s'assurer de la bonne gestion du comité.}
\alinea{Étant donné que le comité organisateur de l'Oktoberfest utilise certains services de l'AGEG, celui-ci se doit de payer une partie des frais administratifs reliés aux salaires et honoraires professionnels de l'AGEG. Ces frais sont fixe et de l'ordre de 10 000\$.}
\alinea{L'Oktoberfest ne doit pas engendrer de déficit;}
\alinea{Advenant un surplus, celui-ci devra être remis à l’AGEG à la fin de l’activité;}
\alinea{Advenant un déficit, le comité organisateur devra en justifier clairement les causes au conseil d’administration dès que ce déficit est connu. Il devra aussi produire des recommandations dans le but que la situation soit régularisée au cours de la prochaine édition.}
\alinea{Le comité présente ces avancements lors du CA4 de l’hiver, du CA 2 et 3 de l’été et au CA1 de l’automne en incluant une présentation de leur budget.}
\alinea{Le comité doit produire un rapport final sur la planification et le déroulement des activités de l'activité, qui sera déposé au CA3 de la session d’automne. Ce rapport inclus un bilan financier final.}
\alinea{Le comité organisateur ne sera pas rémunéré. Cependant, peu importe l’état des finances de l’événement, le comité aura à sa disposition un budget de 1000~\$ pour financer un souper de début de mandat et un souper de fin de mandat pour les membres du comité et les directeurs.}

\sousarticle{Démission ou expulsion}
\alinea{Toute démission de tout membre du comité devra parvenir par écrit au président du comité et devra être transmise par celui au VPAS.}
\alinea{Un membre du comité peut se voir expulsé du comité par le CA de l’AGEG ou l’assemblée générale de l’AGEG pour des raisons jugées valables.}
