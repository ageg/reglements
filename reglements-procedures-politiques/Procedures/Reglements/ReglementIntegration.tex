\section{Comité organisateur des intégrations des nouveaux étudiants}


\article{Préambule}

\alinea{Le comité organisateur des intégrations est un groupe de l'AGEG dont l’objectif est d'organiser l'intégration des nouveaux étudiants en génie. Les objectifs des intégrations sont:}

\sousalinea{L'intégration de la nouvelle promotion de génie au milieu de vie de l’Université de Sherbrooke}

\sousalinea{La transmission des informations essentielles à l’intégration des nouveaux étudiants}

\sousalinea{La mise en œuvre d’activités variées répondant aux besoins de chaque étudiant}

\sousalinea{Le développement de liens fraternels entre les étudiants de la Faculté, nouveaux comme anciens}

\sousalinea{Le développement d’un sentiment d’appartenance}


\article{Nomination du comité}

\alinea{Le comité organisateur de l’intégration est formé par le CA à l'automne suite à une procédure d'élection réalisé par un président d'élection.}

\alinea{Un des membres du Comité organisateur de la dernière année est nommé président d’élection pour le nouveau Comité organisateur par le VPAS de la session d'automne. Le VPAS occupe le rôle de président d'élection si personne ne se présente.}

\alinea{La période de mise en candidature et de vote, le nombre de postes ouverts et la procédure de nomination sont décidés par le président d’élection. Le président d'élection doit s'assurer de privilégié les membres issus de la plus récente promotion.}

\alinea{Les mentors et l'agent de liaison sont nominés par le CA à l'automne sous recommandation d'un comité de nomination dirigé par le VPAS.}

\alinea{Le comité de nomination du comité organisateur est formé par le CA lors du CA1 de l'automne, sous recommandation du VPAS, et est constitué de deux personnes administratrices, du VPAS, du VPAX et d'un ancien membre du CO. Une priorité est donnée aux membres du CA ayant déjà participé à l'organisation des intégrations.}

\alinea{Le comité de nomination est responsable d'organiser une sélection ouverte à tous les membres de l'AGEG. Cette sélection doit être annoncé au moins 2 semaines à l'avance.}


\article{Mandat du comité organisateur}

\alinea{Le comité organisateur de l’intégration a comme mandat d’organiser, de tenir et de superviser les activités d’intégration. Il est responsable de  transmettre les informations sur l’avancement de la planification des activités au décanat et au CA.}

\alinea{Recruter et former tous les bénévoles nécessaires à l’activité d’intégration pour assurer le soutien des nouveaux étudiants dès leur arrivée à la faculté. Les formations recommandées sont Éduc-alcool, premiers-soins (5~\%) et MAPAQ.}

\alinea{Avertir les bénévoles et les chefs d’équipe que les chansons et activités à caractère sexuel, dégradantes ou ne participant pas à l’intégration ne seront pas acceptées sur les prémisses de l’université}

\alinea{Veiller à l’élimination complète de toute forme d’intimidation, d’abus de pouvoir et de risques pour la santé}


\article{Supervision du comité organisateur}

\alinea{L’AGEG, représenté par l’agent de liaison et les mentors, s’assure du suivi des objectifs et de la mission de l’intégration. En cas de non-respect des valeurs exprimées à travers les objectifs et la mission de l’intégration ou en cas de conflit avec les autorités de l’université, l’AGEG a un droit de veto sur toutes activités découlant du Comité organisateur.}


\article{Rôle de l'agent de liaison}

\alinea{S’assurer que la sécurité et l’intégrité de tous les membres sont respectées durant l’ensemble des activités sur le campus comme à l’extérieur de l’Université;}

\alinea{Faire le lien entre le comité et le CA, la sécurité de l’Université, la Faculté et les départements;}

\alinea{Être en charge, durant l’intégration, du respect des directives de sécurité adoptées par le CO;}

\alinea{Produire un rapport sur les incidents importants s’étant déroulés durant la semaine;}

\alinea{Il doit rester sobre, disponible et présentable tout au long de l’événement;}

\alinea{Doit remettre le présent règlement au comité organisateur;}

\alinea{Porte-parole des intégrations auprès des institutions externes;}


\article{Rôle des mentors}

\alinea{Assister le Comité organisateur de l'intégration dans ses prises de décisions selon son expérience, en participant à chacune des réunions;}

\alinea{Diriger le Comité organisateur de l’intégration vers les personnes-ressources de l’Université et de la Faculté pour assurer
le bon déroulement des activités d’intégration;}

\alinea{Assister le comité organisateurs lors des sélections des participants aux intégrations;}

\alinea{Avoir un comportement exemplaire lors de l’évenement.}

\alinea{Un des mentors agit en tant que chef des animateurs}


\article{Gestion des intégrations}

\alinea{Le Comité organisateur de l’intégration a le pouvoir décisionnel sur tous les aspects organisationnels et des activités reliées à l’intégration;}

\alinea{Le Comité organisateur de l’intégration choisit la façon d’élire les directeurs et l'ensemble des bénévoles;}

\alinea{Le Comité organisateur de l’intégration se réserve le droit de destituer ou d'exclure toutes personnes advenant un comportement inadmissible avant ou durant la tenue des activités d’intégration;}

\alinea{Le comité organisateur fait le suivi avec le VPAI au sujet des articles promotionnels et des possibilités d'achat commun;}

\alinea{Le comité organisateur veille à l'application de la politique de développement durable et de l'écoresponsabilité de l'Université de Sherbrooke;}

\alinea{Le comité organisateur doit s'assurer du rayonnement des étudiants et de l'association lors des activités d'intégration;}

\alinea{Le comité organisateur est responsable de la planification et du bon déroulement de la sécurité des événements.}

\alinea{L’organisation et les activités de l’intégration ne doivent pas engendrer de déficit;}

\alinea{Advenant un surplus, celui-ci devra être remis à l’AGEG à la fin de l’activité;}

\alinea{La corporation s’engage à fournir une commandite au comité exécutif de l’intégration du comité organisateur de l’intégration après approbation du rapport d’avancement détaillant l’organisation, le choix et le déroulement des activités ainsi qu'un budget prévisionnel.}


\article{Gestion du comité organisateur}

\alinea{Le comité organisateur fait une rencontre mensuelle de suivi avec l'agent de liaison;}

\alinea{Le comité organisateur s'assure de la prise de notes et des procès-verbaux lors des réunions;}

\alinea{Le président et le trésorier du comité organisateur sont responsables de la gestion financière.}

\alinea{Le VPAS de l'AGEG est en charge de s'assurer de la bonne gestion du comité.}

\alinea{Le comité organisateur rapport d’avancement déposé au CA4 de la session d'hiver et au CA3 de l'été;}
