\section{AGEG Band}


\article{Préambule}

\alinea{L'AGEG Band est un groupe de l'AGEG dont l'objectif est de former à chaque session, un groupe de musique pour les activités sociales de l'AGEG.}


\article{Nomination des membres de l'AGEG Band}

\alinea{Les membres de l'AGEG Band sont sélectionnés par le CA, deux sessions avant leur session d'activité, sous recommandation d'un comité de nomination dirigé par la VPAS.}

\alinea{Le comité de nomination des membres de l'AGEG Band est formé par le CA, sous recommandation de la VPAS, et est constitué de deux personnes administratrices, de la VPAS, du directeur de l'AGEG Band de la session en cours, d'un autre membre actif de l'AGEG Band et d'un ou plusieurs anciens membres de l'AGEG Band. Les membres du comité de nomination ne peuvent pas être candidat.}

\alinea{Le comité de nomination est responsable d'organiser une sélection ouverte à tous les membres de l'AGEG. Cette sélection doit être annoncée au moins 2 semaines à l'avance et une liste de chansons à apprendre doit être rendue disponible.}

\alinea{Le comité de nomination peut aussi faire des recommandations pour combler une formation incomplète. Pour ce faire, le comité doit aussi inclure au moins un membre de cette formation.}%cette phrase maque de sens

\alinea{En cas de vacances, au début de la session d'activité, les membres de l'AGEG Band peuvent ajouter des membres à leur formation.}

\alinea{Une formation typique comprend 1 chanteur, 2 guitaristes, 1 bassiste et 1 batteur. Le comité de sélection est libre de recommander une composition différente.}

\alinea{Les critères pour la sélection des membres de l'AGEG Band sont:}

\sousalinea{La personne doit être membre de l'AGEG}

\sousalinea{Le talent musical doit être de niveau suffisant pour être fonctionnel dans un groupe}

\sousalinea{Expériences musicales et implications, antérieures et actuelles}

\sousalinea{Versatilité du répertoire et du style de musique}

\sousalinea{Attitude, charisme et présence de scène}

\sousalinea{Motivation, intérêt et compréhension de ce que représente l'AGEG Band.}


\article{Gestion de l'AGEG Band}

\alinea{Une personne à la direction de AGEG Band est nommée parmi les membres de l'AGEG Band chaque session pour faire la liaison entre l'AGEG Band et l'AGEG.}

\alinea{La personne à la direction de l'AGEG Band est responsable de faire approuver les dépenses du groupe par le CA.}

\alinea{La VPAS de l'AGEG est responsable de s'assurer de la bonne gestion de l'AGEG Band.}

\alinea{L'AGEG Band doit conserver une liste des évènements auxquels il a participé ou prévoit participer chaque session. L'AGEG Band doit participer à au moins 3 évènements par session.}

\alinea{L'AGEG Band doit conserver une liste du matériel du groupe et s'assurer de conserver celui-ci en bon état.}


\article{Spectacles payants}

\alinea{L’AGEG Band doit organiser au minimum un (1) spectacle payant au cours de la session afin d’assurer son développement. Des recommandations et une liste de contacts sont fournies dans le guide de l’AGEG Band.}

\alinea{Les spectacles payants pourront être organisés avec ou sans la collaboration d’un ou des groupes étudiants. L’AGEG Band devra cependant respecter les semaines d’activités sociales comme prescrit par les règlements de l’AGEG.}

\sousalinea{Si l'AGEG Band joue dans un événement organisé en collaboration avec un ou des groupes étudiants, un minimum de 25\% de la répartition des profits de cet événement devra être remis à l'AGEG Band et déposé dans le fonds AGEG Band.}


\article{Style de musique}

\alinea{Les membres de l'AGEG Band sont responsables de choisir les chansons qu'ils performent. Cependant, leurs choix doivent respecter les critères suivants :}

\sousalinea{Le répertoire doit inclure la chanson thème de la promotion finissante (si applicable).}

\sousalinea{Chaque spectacle doit comporter un minimum de (2) chansons francophones.}


\article{Local de pratique}

\alinea{L'AGEG doit allouer un minimum de 1000\$ par session, de son budget courant, pour la location d'un local de pratique pour l'AGEG Band.}

\alinea{La sélection du local de pratique est entérinée par le CA de l'AGEG. Une copie du contrat de location doit être conservée par l'AGEG pour toute sa période de validité.}

\alinea{Le local de pratique ne doit être partagé avec aucun autre groupe. Seuls les membres actifs et passifs de l'AGEG Band et la VPAS doivent avoir les clés en leur possession.}

\alinea{Le local de pratique doit être gardé sous alarme en tout temps en dehors des heures de pratique.}


\article{Équipement et assurances}

\alinea{La liste des équipements appartenant à l’AGEG Band, excluant les équipements personnels des membres, doit être mise à jour et fournie à l’AGEG à la fin de chaque mandat.}

\alinea{Cette liste servira de preuve de matériel auprès de l’AGEG pour que les équipements soient enregistrés comme appartenant à l’AGEG pour pouvoir bénéficier des assurances.}

\alinea{Tout nouvel équipement acheté par l’AGEG Band doit être ajouté à la liste des équipements dans le guide de l’AGEG Band.}


\article{Démission ou expulsion}

\alinea{La démission d’un membre de l’AGEG Band devra parvenir par écrit à la VPAS .}

\alinea{Les membres de l'AGEG Band peuvent prendre la décision d'expulser un de leur membre, s'ils ont un motif valable. L'expulsion du membre doit être approuvée par la VPAS.}
