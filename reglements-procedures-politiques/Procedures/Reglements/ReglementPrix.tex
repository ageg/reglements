\chapter{Prix}

\section{Prix Jacques Bazinet}

\article{Mission de l’attribution du Prix Bazinet}
\alinea{La mission du Prix Bazinet est de rendre hommage annuellement aux enseignants les plus appréciés des étudiants durant leur baccalauréat. Ce prix permet de reconnaître l’effort des enseignants et leur disponibilité auprès des étudiants, se décrivant par une bonne relation enseignant/étudiant s’étalant même jusqu’à l’extérieur de leurs missions titulaires. Aux fins de cette procédure, le terme « enseignant » correspondra aux professeurs, chargés de cours ou techniciens qui participent à la formation des étudiants du premier cycle.}


\article{Gestion du prix Jacques Bazinet}
\alinea{Un prix est décerné pour chacun des programmes de baccalauréat de la faculté de génie.}
\alinea{La VPAP distribue les feuilles de vote à tous les représentants de concentration de la promotion finissante au début de la dernière session du baccalauréat (S8).}
\alinea{La VPAP s’assure que les représentants feront bien comprendre la procédure du vote et le but du prix aux étudiants.}
\alinea{La VPAP s’assure que les représentants font circuler les feuilles de vote et les lui remette avant la date déterminée.}
\alinea{La VPAP comptabilise les votes avant le gala, prépare les prix et les remet au gala du mérite étudiant organisé par la Faculté de génie.}

\partie{Vote}
\alinea{Chaque étudiant finissant membre de l’AGEG a le droit de vote.}
\alinea{Les étudiants devront voter en inscrivant le nom de l’enseignant méritant selon leur programme.}
\alinea{Les formulaires doivent être identifiés, selon le programme, avant leur remise aux représentants de l'AGEG pour avoir un contrôle à leur retour.}
\alinea{Les formulaires doivent être disponibles à l'AGEG ou via courriel, accompagnés d’une lettre décrivant la nature du prix Jacques Bazinet, les critères et la procédure à suivre pour voter.}

\sousarticle{Comptabilisation}
\alinea{Les votes doivent être comptabilisés au moins un mois avant le gala du mérite étudiant, afin d’avoir assez de temps pour préparer les prix.}
\alinea{Les trois meilleurs enseignants par catégorie seront nommés. Celui qui sera convoqué lors du gala du mérite étudiant sera l’enseignant ayant recueilli le plus de votes.}

\section{Prix Yvan Néron}

\article{Mission de l’attribution du Prix Yvan Néron}
\alinea{La mission du Prix Yvan Néron est de rendre hommage annuellement à un membre personnel non-enseignant apprécié des étudiants durant leur baccalauréat. Ce prix permet de reconnaître l’effort et la disponibilité auprès des étudiants du personnel non-enseignant, se décrivant par une bonne relation personnel/étudiant s’étalant même jusqu’à l’extérieur de leurs missions administratives.  Aux fins de ce règlement, le terme «~enseignant~» correspondra aux professeurs, chargés de cours ou techniciens qui participent à la formation des étudiants du premier cycle.}

\article{Gestion du prix Yvan Néron}
\alinea{Le CE de l'AGEG définit le récipiendaire et le fait accepter en CA au plus tard lors du CA2 de l’automne;}
\alinea{Le CE de l'AGEG prépare les prix et les remet au gala du mérite étudiant organisé par la Faculté de génie.}