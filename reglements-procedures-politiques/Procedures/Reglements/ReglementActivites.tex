\chapter{Procédures sur les activités sociales}

\section{Organisation des Jeudis détentes}

\article{Allocation des "Jeudis détentes"}
\alinea{L'AGEG est réputé détenir les "Jeudis détentes" et les alloue à la promotion finissante.}
\alinea{Sur résolution du conseil d’administration, pour une durée indéterminée et pour des raisons exceptionnelles, l’AGEG peut retirer l’organisation des "Jeudi détente" à la promotion finissante. L’AGEG devient donc responsable de l’organisation des "Jeudi détente" pour cette période.}

\article{Droit d'exclusivité}
\alinea{La finissante est le seul groupe étudiant autorisé, en dehors de l'AGEG elle-même, à faire de la vente lors des jeudis détentes excepté les groupes qui organisent une activité sociale après la fin de l'évènement dans le cadre d'une semaine de financement.}
\alinea{La finissante peut permettre à n'importe quel groupe de faire de la vente durant ces événements. }

\article{Coupons de consommations}
\alinea{L'AGEG peut distribuer des coupons de consommations pour des activités ou de la représentation. La finissante est compensée au prix coûtant pour ces coupons.}

\article{Réservation des locaux}
\alinea{L'AGEG devra, pourvu que l'administration de la Faculté lui en reconnaisse le pouvoir, réserver l'usage exclusif du salon des étudiants et de la radio pour les "Jeudi détente"  à la promotion finissante.}


\article{Taxe administrative de la finissante}
\alinea{Étant donné que la promotion finissante utilise certains services de l'AGEG, celle-ci se doit de payer une partie des frais administratif reliés au salaire des comptables et honoraires professionnels des comptables externes de l'AGEG.}
\alinea{La promotion finissante doit verser une somme équivalente à 5~\% de l’ensemble de ses profits lorsqu’elle était finissante au Fonds d'administration de l’AGEG lors de la passation des pouvoirs à la session d’automne.}
\alinea{De cette somme peuvent être retirées les Contributions de la finissante, soit de l’argent que le comité donne déjà à l’AGEG ou ses membres par l’entremise des autres ententes. L'approbation des Contributions se fait par la VPAF à l'automne lors de la passation des pouvoirs. Les Contributions de la finissante sont : }
\sousalinea{Tout don de la finissante vers un fonds de l’AGEG n'étant pas prévu par les Règlements de l'AGEG ou son cahier de procédure.}
\sousalinea{Tout don de la finissante vers un groupe de l'AGEG ou un groupe technique de l’AGEG. Ces dons incluent le financement du verre de la rentrée aux intégrations.}
\sousalinea{Tout don issu de la finissante qui est au service de l'AGEG. Ces dons incluent les réparations effectuées sur les locations des services de l'AGEG.}

\article{Contribution de la finissante}
\alinea{La promotion finissante doit verser une somme équivalente à 5~\% de l’ensemble de ses profits lorsqu’elle était finissante au Fonds de donation de l’AGEG lors de la passation des pouvoirs à la session d’automne.}
\alinea{La promotion finissante doit verser une somme équivalente à 5~\% de l’ensemble de ses profits lorsqu’elle était finissante au Fonds de subvention de l’AGEG lors de la passation des pouvoirs à la session d’automne.}


\section{Code de conduite}

\article{Préambule}
\alinea{L’objectif du code de conduite est de faire perdurer la présence des membres et des employés de l’AGEG lors des événements universitaires externes en leur imposant des conséquences en cas de non respect du dit code de conduite pour éviter des comportements répréhensibles.L’AGEG se porte garant uniquement de ses membres et de ses employés participant à un événement universitaire externe dont le transport est organisé par l’AGEG. Cependant, le Comité du code de conduite peut sévir contre tous membres ayant eu un comportement inadéquat lors d’un événement universitaire externe.}

\article{Fonctionnement du code de conduite}
\alinea{Le code de conduite doit être signé par les membres et employés participant à un événement universitaire externe, avant la tenue de celui-ci.}
\alinea{Un membre ou un employé n’ayant pas signé le code de conduite ne pourra pas participer à l’événement universitaire externe.}
\alinea{Le VPAS tient une liste des personnes exclues des événements universitaires externes et s’assure que la liste est bien à jour en collaboration avec le VPEX.}
\alinea{Le VPEX assure un suivi avec les autres associations lorsqu’un membre de l’AGEG a eu un comportement inadéquat lors d’un événement universitaire externe.}
\alinea{Le Comité du code de conduite détermine la gravité des sanctions imposées au membre ayant commis un délit.}

\article{Formation et composition du Comité du code de conduite}
\alinea{Le Comité du code de conduite sera formé avant le CA1 de chaque session.}
\alinea{Le VPAS siégera sur le Comité du code de conduite.}
\alinea{Le CA de l’AGEG désigne un membre du CA de l’AGEG ainsi qu'un membre de l'AGEG pour siéger sur le Comité du code de conduite }
\alinea{Les futurs membres du Comité ne doivent pas avoir reçu de pénalité de la part du Comité du code de conduite depuis au minimum un an.}

\article{Convocation}
\alinea{Le VPAS s’occupe de convoquer le Comité lorsqu’un incident survient lors d’un évènement universitaire ou d’un évènement universitaire externe.}
\sousarticle{Quorum}
\alinea{Pour toute assemblée officielle du comité, il est nécessaire d’avoir la présence de tous les membres du comité. Aucune décision ne sera prise à moins que cette assemblée n’ait le quorum requis.}
\alinea{Les décisions du comité se prendront par vote à majorité. Chaque membre du comité a droit à un vote.}

\article{Décision}
\sousarticle{Critères}
\alinea{La gravité des sanctions imposées au membre ayant commis un délit est déterminé selon les critères suivants :}
\sousalinea{Gravité du délit commis}
\sousalinea{Récurrence d'un délit par le membre}
\sousalinea{Attitude du participant face au délit commis et face aux organisateurs de l'évènement suite au délit}

\sousarticle{Délai}
\alinea{Le comité doit évaluer les dossiers à l’intérieur des 2 semaines suivant l’événement.}
\alinea{Le comité doit aviser le membre de l’AGEG de sa décision avant l'échéance de ce délai.}
\sousarticle{Appel}
\alinea{Un membre pourra porter une décision du Comité du code de conduite en appel devant le CA de l'AGEG s'il juge que cette décision est disproportionnée.}
\alinea{La décision du CA de l'AGEG, incluant le refus d'entendre l'appel, est finale et sans appel.}
