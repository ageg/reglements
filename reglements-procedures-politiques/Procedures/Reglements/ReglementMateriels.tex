\chapter{Matériels et locaux}

\section{Locaux de l'AGEG}


\article{Locaux appartenant à l'AGEG}

\alinea{Les locaux suivants appartiennent à l'AGEG : C1-1016, C1-2044, C1-2045, C1-2046, C1-2047, C1-2048, C1-2058, C1-2058-1, C1-2058-2, C1-2058-3, C1-2116, C1-2116-1, C1-2116-2, C1-2116-3 et C1-5010.}


\article{Entreposage de la nourriture}

\alinea{La nourriture ne peut être entreposée ailleurs que dans le C1-2045, le C1-2047, le C1-2048 et le C1-2044.}

\alinea{La nourriture entreposée dans les réfrigérateurs et les congélateurs doit servir exclusivement pour les activités des semaines de financement, les jeudis-détentes ou les activités organisées par l'AGEG.}

\alinea{Tout groupe étudiant peut conserver de la nourriture dans les réfrigérateurs et les congélateurs à condition que celle-ci soit identifiée au nom du groupe, sanitaire et non périmée.}

\alinea{Après sa semaine de financement, le groupe doit vider sa nourriture du réfrigérateur ou avoir un arrangement avec le groupe suivant.}

\alinea{Toute nourriture ou contenant de nourriture qui n'a pas besoin d'être réfrigéré doit se trouver sur les tablettes.}

\alinea{Toute nourriture ou contenant de nourriture déposé par terre, et ce, même momentanément, devra être disposé de façon convenable immédiatement.}

\alinea{La direction MAPAQ peut, à n'importe quel moment, et ce, sans préavis, inspecter la nourriture se trouvant dans les locaux et disposer de tout ce qui semble impropre à la consommation et/ou à la vente ainsi que tout ce qui n'est pas identifié.}

\sousalinea{Si la direction MAPAQ a eu à intervenir plus de deux fois dans une même session, ou à trois reprises à l'intérieur de trois sessions consécutives, envers un groupe, le groupe perdra son droit de profiter des semaines de financement pour les trois sessions suivantes, à moins d'avis contraire de la Vice-Présidence aux affaires internes (VPAI).}


\article{Entreposage du matériel dans le C1-2047}

\alinea{Tout groupe technique n'ayant pas déjà un local alloué peut y entreposer du matériel.}

\alinea{Tout groupe ayant une semaine de financement à venir peut y entreposer du matériel.}

\alinea{Tout groupe entreposant du matériel a la responsabilité d'envoyer au moins un membre pour aider à faire un ménage important une fois par session, à la demande de la VPAI ou de la Vice-Présidences aux affaires extracurriculaires (VPAX).}

\alinea{Le nom du groupe doit être indiqué clairement sur le matériel.}

\sousalinea{Tout le matériel qui n'est pas clairement identifié, rangé dans des contenants non identifiés ou non rangé sur des étagères identifiées est réputé appartenir à l'AGEG. Celle-ci pourra le donner à un autre groupe ou en disposer autrement, à moins d'avis contraire de la VPAI.}

\alinea{Le matériel entreposé ne doit pas être salissant.}

\sousalinea{Les sacs de sable doivent être rangés dans un bac de plastique.}

\sousalinea{Les sacs de sable doivent être sortis des bacs à l'extérieur du local.}

\sousalinea{Tout groupe n'ayant pas nettoyé et rangé de manière appropriée son matériel salissant dans des délais raisonnables après l'avertissement de la VPAI verra son matériel saisi et perdra sa prochaine semaine de financement.}

\alinea{Aucun matériel ne doit bloquer l'accès et la circulation.}

\alinea{Aucun matériel ne peut se trouver dans le corridor de circulation ou les zones identifiées.}

\alinea{Tous les chariots doivent être placés à l'endroit clairement identifié.}


\article{Direction MAPAQ}

\alinea{La direction MAPAQ doit passer au moins une fois par semaine pour inspecter le contenu du réfrigérateur et des congélateurs ainsi que la nourriture.}

\alinea{La direction MAPAQ peut, à n'importe quel moment, et ce, sans préavis, inspecter le contenu du réfrigérateur et des congélateurs, et jeter tout ce qui semble impropre à la consommation et/ou à la vente ainsi que tout ce qui n'est pas identifié.}

\alinea{La direction MAPAQ doit jeter toute nourriture qui se trouve sur le plancher, ou à moins de dix centimètres de celui-ci, même si elle est emballée et qu'elle n'a pas besoin d'être réfrigérée, sans préavis.}


\article{Journée de ménage}

\alinea{Chaque session, la VPAI ou la VPAX doit prévenir tous les groupes actifs de la tenue du grand ménage et rappeler l'obligation de chaque groupe locataire du local de déléguer au moins un membre lors de cette journée, à moins d'une exemption accordée explicitement par la VPAI.}

\alinea{Tout groupe n'ayant pas contribué lors de la journée de ménage, et qui avait du matériel dans les locaux, perdra l'accès à sa prochaine semaine de financement, à moins d'avis contraire de la VPAI.}

\section{Système de son}


\article{Gestion du système de son}

\alinea{La VPAI est la personne exécutante responsable de la location ainsi que de l'équipement de la radio.}

\alinea{La direction radio de la promotion finissante est responsable de veiller au bon état du système de son lors de chaque Jeudi détente.}


\article{Location du kit de son mobile ou du kit de son fixe de la radio.}

\alinea{La location de l’équipement de la radio doit être faite selon les règles suivantes :}

\sousalinea{La réservation doit être faite via le formulaire d’emprunt disponible à l’AGEG.}

\sousalinea{Le dépôt nécessaire doit être acquitté lors de la sortie du matériel.}

\sousalinea{Le dépôt est fixé à 500~\$.}

\alinea{Le matériel doit être remis dans le même état qu’il était lors de la sortie.}

\alinea{Seul la direction radio, la présidence, la Vice-Présidence aux affaires financières (VPAF) ainsi que la VPAI ont accès à ce local en tout temps.}

\alinea{L’équipement est disponible en tout temps pendant la semaine. La promotion finissante à priorité sur l’équipement lors des "Jeudi détente". Afin de louer l’équipement les fins de semaines, la personne emprunteuse doit venir le chercher le vendredi avant la fermeture des bureaux et le ramener le jour ouvrable suivant avant 10 h.}

\alinea{La manutention du kit de son mobile doit être faite avec le chariot et les housses prévues à cet effet.}


\article{Dommage, bris ou perte d’équipement}

\alinea{En cas de dommage, bris ou perte d’équipement de la part d’un groupe qui emprunte l’équipement de la radio, la totalité du dépôt sera retenue jusqu’à la réparation ou le remplacement de l’équipement. Les modalités de remboursement seront les suivantes :}

\sousalinea{Si le coût de la réparation ou du remplacement de l’équipement est inférieur au montant du dépôt, l’AGEG remboursera la différence au groupe.}

\sousalinea{Si le coût de la réparation ou du remplacement de l’équipement est supérieur au montant du dépôt, le groupe perdra complètement son dépôt et sera exigé de défrayer tout montant supérieur à 500~\$.}

\section{Bandana de Génie}


\article{Gestion du bandana de génie}

\alinea{La direction bandana veille au bon fonctionnement du comité bandana et supervise le bon déroulement de la remise du S d'or.}

\alinea{Le comité bandana veille à s’assurer que le bandana conserve son statut d’article d’exception et représente la tradition et l’appartenance au génie sherbrookois.}

\alinea{Le comité bandana s’assure de l’implantation des valeurs en lien avec le bandana chez la promotion entrante lors de la semaine d’intégration.}

\alinea{Le comité bandana s’assure de rendre disponible aux personnes étudiantes l’accès aux écussons.}

\alinea{Le comité bandana peut recommander l’imposition de conditions particulières sur l’utilisation du bandana sous l’approbation du CE de l’AGEG.}

\alinea{Le comité bandana peut proposer une personne récipiendaire d’un bandana honorifique sous l'approbation du CA de l'AGEG.}

\alinea{Le comité bandana décide des activités qu’il veut organiser et des achats qu’il veut effectuer selon la mission, la vision et les valeurs véhiculées par l’AGEG.}

\alinea{Les achats du comité devront être approuvés par le CA de l'AGEG.}


\article{Personnes porteuses et récipiendaires}

\alinea{Toute personne étudiante membre de l’AGEG a droit à l’achat d’un bandana de génie.}

\alinea{Tout professeur ou membre de l’administration de la Faculté de génie de Sherbrooke a droit à l’achat d’un bandana de génie.}

\alinea{Toute personne ayant fait au moins trois (3) sessions de travail au sein de l’AGEG a droit à un bandana.}

\alinea{Toute personne ayant fait valoir les valeurs du bandana de génie et s’étant démarquée dans la valorisation des personnes étudiantes de génie à travers ses actions peut se faire attribuer un bandana honorifique.}


\article{Composition du comité bandana}

\alinea{La direction du comité est nommée par le VPAI et doit être entérinée par le CE de l’AGEG.}

\alinea{La direction bandana recrute entre une et cinq personnes étudiantes inscrites en session d’étude lors de leur mandat actif pour combler le comité bandana.}

\alinea{Les principaux critères de sélection seront la motivation à contribuer à la valorisation du bandana et le respect des traditions des personnes fondatrices du bandana.}

\alinea{Le mandat du comité bandana est d’une session.}


\article{Décision}

\sousarticle{Bandana honorifique}

\alinea{Sur recommandation du comité bandana, au plus un bandana honorifique peut être donné par session.}

\alinea{Les écussons à apposer sont à la discrétion du comité bandana.}

\alinea{La proposition doit être amenée en CA et la décision relève des personnes administratives.}

\sousarticle{S d’or}

\alinea{Toute personne possédant un bandana peut recevoir un S d’Or. S’il s’agit d’une personne étudiante de première année, il doit attendre à la remise durant sa deuxième session.}

\alinea{Le comité bandana est responsable d’approuver les demandes de S d’or.}

\alinea{Les formulaires doivent être disponibles à tous auparavant et la publicisation de l’événement doit avoir été suffisante.}

\alinea{Un bandana refait, car perdu ou volé, peut orner le S doré; au jugement du comité bandana.}


\article{Achat, perte et rachat d’un bandana}

\alinea{Le prix d’un bandana incluant l’écusson de concentration, l’écusson de promotion, l'écusson bandanom et le S d’or est fixé à 22,50~\$ à l'intégration ou avant la réception du bandana lors de la cérémonie d'engagement.}

\alinea{Lors de l'achat d'un premier bandana, après la cérémonie d'engagement où la personne aurait dû recevoir son bandana, le prix du bandana incluant l'écusson de concentration et l'écusson de promotion est fixé au coût de 25,00~\$}

\alinea{Lors de la perte de son bandana, une personne étudiante aura la chance de racheter son bandana incluant l'écusson de concentration et l'écusson de promotion au coût de 25,00~\$, sur acceptation du comité bandana. Afin d’éviter que le bandana soit trop répandu, une personne ne peut recevoir un troisième bandana.}

\alinea{Sur recommandation du comité bandana, les exceptions devront être statuées par le CE de l’AGEG.}


\article{Gestion de l’inventaire}

\alinea{Le comité bandana doit s'assurer, avant le denier CA de sa session, d'avoir les budgets nécessaires pour maintenir les inventaires essentiels au bon fonctionnement du comité pour la période qui va du dernier CA au CA1 de la session suivante.}

\alinea{Le comité bandana devra s’assurer de faire des commandes judicieuses (s’assurer avant achat que la commande s’écoulera facilement et qu’il existe une demande) et s’assurer que l’inventaire s’écoule par des ventes régulières d’écussons.}

\alinea{Le comité devra faire un inventaire à chaque fin de session.}

\alinea{Un suivi des ventes devra être fait pour permettre la mise à jour de la valeur de l'inventaire.}

\section{Objets promotionnels}


\article{Gestion des objets promotionnels}

\alinea{Les objets promotionnels sont pris en charge par le Comité des Objets Promotionnels Sensationnels (COPS)}

\alinea{Le directeur COPS est nommé par le CE pour un mandat d'une session.}

\alinea{Le comité est responsable de l'inventaire des objets promotionnels. Il doit assurer une saine gestion de l'inventaire, promouvoir les articles et s'assurer d'offrir des périodes de ventes adéquates.}

\alinea{Le comité peut présenter une ou plusieurs idées d'objet promotionnel à chaque CA de la session active. Le CA décidera si l'article promotionnel est adéquat pour les besoins de l'AGEG.}

\alinea{Le décanat pourra acheter au prix coûtant les articles promotionnels de l’AGEG.}


\article{Lancement et distribution}

\alinea{La date de lancement de l’objet devra être annoncée à l’avance par les méthodes promotionnelles habituelles de l’AGEG.}

\alinea{Une personne ne pourra acheter qu’un nombre limité d’objets lors du lancement. À la fin de la période de lancement, toute personne peut s’en procurer le nombre voulu.}

\sousalinea{La période de lancement suggérée est de 2 jours ouvrables et sera décidée par le COPS.}

\alinea{Pour les objets promotionnels à tirage limité, la moitié de l’inventaire de ces objets sera réservée pour la prochaine session si cela est pertinent.}

\alinea{La distribution ne pourra en aucun cas favoriser un ou des membres de l’AGEG en particulier.}

\section{Affichage sur les babillards}


\article{Validité d’une affiche}

\alinea{Pour être valide, une affiche doit respecter toutes les caractéristiques suivantes:}

\alinea{L’affiche doit posséder le sceau de l’AGEG apposé par les personnes préposées à l'accueil;}

\alinea{L'affiche ne doit pas avoir dépassée sa date d’expiration;}

\sousalinea{La durée maximale pour l’affichage d’une activité étudiante est de 3 semaines, après quoi, l’affiche sera retirée même si l’activité n’a pas eu lieu;}

\sousalinea{Les affiches sont retirées des babillards lorsque la date de l’activité est dépassée;}

\sousalinea{Les petites annonces ont une durée d’affichage de 2 semaines;}

\sousalinea{Les activités sociales de génie ont une durée d’affichage limitée à 1 semaine avant ladite activité;}

\alinea{L'affiche ne doit pas être plus grandes que le format 11 pouces par 17 pouces.}


\article{Quantité des affiches}

\alinea{Le nombre d’affiches maximum pouvant être estampillé pour une activité est de six.}

\alinea{Le nombre maximum d’affiches par babillard est de un.}


\article{Affichage permis}

\alinea{Toute activité organisée par les groupes étudiants de génie, durant leur semaine d’activités sociales.}

\alinea{Toute activité organisée par l’AGEG.}

\alinea{Toutes informations d’intérêt général pour les personnes étudiantes.}

\alinea{Les "Jeudi détente"  organisés par la promotion finissante et/ou sortante.}

\alinea{Tout autre party dont la publicité a été explicitement autorisée par le groupe étudiant qui s’est vu attribuer la semaine en début de session.}


\article{Affichage proscrit}

\alinea{La publicité n’ayant aucun lien avec les personnes étudiantes.}

\alinea{La publicité d’un bar qui contrevient avec le party de la semaine.}

\alinea{L’affiche contrevenant aux règlements de la Faculté ou à une loi établie.}

\sousarticle{Groupe étudiant ou promotion}

\alinea{Le CE d’un groupe étudiant peut refuser l’affichage de toute autre activité se tenant durant la semaine qui lui a été attribuée en début de session et qui compromettrait la rentabilité de son activité sociale.}

\alinea{Les seules affiches pouvant porter l’appellation « Party de Génie » sont celles des partys organisés par les groupes étudiants durant leur semaine de financement.}


\article{Babillards électroniques}

\alinea{L'affichage sur les babillards électroniques suit les mêmes règles que pour l'affichage sur les babillards.}

\sousarticle{Jeudi détente}

\alinea{Les babillards électroniques sont mis à la disposition du groupe organisant les "Jeudi détente" pour la durée de ceux-ci.}

\alinea{Le comité exécutif de l'AGEG peut retirer tout média considéré inadmissible lors des "Jeudi détente".}
