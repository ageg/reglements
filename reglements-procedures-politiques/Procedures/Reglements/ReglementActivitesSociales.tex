\section{Semaine de financement des groupes étudiants}


\article{Distribution des semaines de financement}

\alinea{Le VPAS est responsable de tenir une réunion pour la distribution des semaines d’activités sociales dans les quatres dernières semaines de la session précédente.}

\alinea{Le VPAS et le VPAX sont responsable de publiciser auprès des groupes la tenue de cette réunion.}

\alinea{Le VPAS est responsable d'identifier les évènements susceptibles d'affecter le déroulement des semaines de financements et d'en informer les groupes.}

\alinea{L'AGEG à le droit exclusif de se réserver des semaines avant la distribution. L'AGEG n'a pas à se jumeler pour ses semaines.}

\alinea{La promotion finissante obtient le premier choix de semaine d'activités sociales. La promotion finissante n’a pas à se jumeler pour sa semaine.}

\alinea{Les promotions non-finissantes suivent par ordre d’ancienneté, avec un choix chacune. Elle doivent se jumeler si elles sont sollicitées.}

\sousalinea{À la session d’automne, la promotion entrante sera automatiquement jumelée à la promotion de deuxième année.}

\alinea{Les groupes étudiants obtiennent ensuite leur choix par tirage au sort.}

\sousalinea{Lorsque toutes les semaines sont attribués, le tirage au sort continue et le groupe tirés doit se jumeler avec un autre groupe ou promotion (exculant la promotion finissante et l’AGEG)}

\sousalinea{Les groupes peuvent se jumeler avec une équipe avant d’être tiré au sort, si les deux groupes sont d’accord.}

\alinea{S’il reste encore des semaines disponibles, une fois que tous les groupes ont obtenu leur semaine d’activités sociales, elles seront réservées à l’AGEG. L'AGEG pourra les distribuer à sa discrétion.}

\alinea{Si un groupe est absent lors de la distribution, il sera laissé à la discrétion de l’AGEG de lui octroyer une semaine. Si un groupe, à l’exception de l’AGEG et de la promotion finissante, est resté seul à la suite de la réunion de distribution des semaines, il sera automatiquement jumelé à ce dernier.}


\article{Gestion des semaines de financements}

\alinea{Aucune activité sociale ne peut être organisée par un groupe sauf pendant sa semaine d’activités sociales. Il est reconnu que d’autres groupes peuvent organiser des activités sociales; ces activités doivent être approuvées par le groupe qui détient la semaine durant laquelle l’activité doit être organisée.}

\alinea{Aucune activité sociale ne peut se tenir au même moment que le "Jeudi détente", sauf sous approbation de l’organisateur du "Jeudi détente".}

\alinea{Lorsqu’une activité sociale est coorganisée par plus d’un groupe, la distribution des profits entre ces groupes est laissée à leur discrétion.}

\alinea{La publicité relative à une activité sociale, sous toutes ses formes, ne peut débuter hors de la semaine d’activités sociales, sauf sous autorisation du groupe ayant les droits d’affichage pour ladite semaine.}

\alinea{En cas de conflit lié à la distribution des profits entre deux groupes associés pour l’organisation d’un party, l’AGEG servira d’arbitre auprès des parties en conflit.}

\alinea{En cas de non-respect du présent règlement par un groupe, l’AGEG pourra retirer le droit à une semaine d’activités sociales pour la prochaine session où le groupe sera actif. Dans le cas d’une récidive, ou lorsque la sanction ne s’applique pas, l’AGEG se réserve le droit d’appliquer des sanctions plus sévères ou adéquates.}

\alinea{Dans l’éventualité où l’AGEG est partie prenante dans un conflit, le CA sera appelé à trancher, sa décision étant finale; dans tous les autres cas, la décision du CE de l’AGEG sera finale}

\alinea{Les objets promotionnels vendus ou distribués par les groupes étudiants ne doivent pas encourager la consommation d'alcool, la violence ou être à caractère sexuelle. En cas de doute, le groupe doit contacter le VPAX.}
