\chapter{Représentation associatif}

\section{Délégation à la FCEG et à la CRÉIQ}


\article{Formation de la délégation pour le congrès de la FCEG}


\article{Le conseil d'administration détermine le nombre maximum de personnes dans la délégation chaque année, selon le budget, lors du CA1 de la session d'automne.}


\article{Le conseil d'administration sélectionne un membre du premier cycle pour agir en tant que direction du comité de sélection et deux autres membres du premier cycle pour former le comité de sélection de la délégation au congrès de la FCEG lors du CA1 de la session d'automne.}

\alinea{La direction du comité de sélection est responsable de publiciser les sélections de la délégation et de recevoir les candidatures.}

\alinea{Suite à la rencontre entre le comité et les personnes candidates, le comité forme une délégation et nomme une des personnes candidats en tant que chef de la délégation.}%est ce qu'on laisse chef de déleg ?

\alinea{Suite aux délibérations du comité, la direction est responsable de rédiger une recommandation pour le CA avec les justifications nécessaires.}

\alinea{Le CA doit entériner les recommandations du comité de sélection.}

\alinea{Les critères de sélection pour la ou le chef de délégation sont les suivants :}

\sousalinea{Faire partie du CE de l’AGEG au moment du congrès, de préférence au poste de Vice-Présidence aux affaires externes (VPEX);}

\sousalinea{Avoir une connaissance appropriée de l’anglais.}

\alinea{Les critères de sélection pour les membres de la délégation sont les suivants :}

\sousalinea{Motivation des personnes candidates;}

\sousalinea{Implications antérieures et actuelles dans le milieu universitaire;}

\sousalinea{Objectifs futurs dans l’implication universitaire;}

\sousalinea{Leurs attentes et objectifs par rapport au congrès.}


\article{Représentation aux autres instances de la FCEG}

\alinea{La VPEX en poste est la personne représentante de l'AGEG à toutes les instances de la FCEG.}

\alinea{La VPEX est responsable de trouver des membres du premier cycle pour l'accompagner au besoin. La VPEX de la prochaine session, la ou le chef de la délégation au congrès de la FCEG et les membres de la délégation ont priorité, dans cet ordre.}

\section{Délégation aux instances et aux congrès de la FEUS}


\article{Gestion de la délégation aux instances de la FEUS}

\alinea{La Vice-Présidence aux affaires universitaires (VPAU) est la représentante officielle de l’AGEG à la FEUS.}

\alinea{La VPAU est responsable de la délégation aux instances et aux congrès de la FEUS et s'occupe de choisir les membres de la délégation.}

\alinea{La délégation choisit ses positions selon les valeurs et les intérêts de l’AGEG et de ses membres.}

\alinea{La position finale de la délégation est donnée par la VPAU de l’AGEG.}

\alinea{Si la délégation ne reçoit pas d’indications de la VPAU, chaque membre vote selon ses valeurs profondes en considérant qu’elle ou il représente l’opinion d’une partie des membres de l’AGEG}

\alinea{Advenant l’impossibilité pour la VPAU d’assister à une instance ou à un congrès, elle peut nominer un membre de sa délégation pour agir représenter l’AGEG au sein de cette instance.}

\section{Représentation lors des référendums de la FEUS}


\article{Politique}

\alinea{Un maximum de quatre (4) personnes représentantes actives par comité est autorisé.}

\alinea{Seuls les membres de la FEUS et toute personne acceptée par le CE/CA de l’AGEG sont autorisés à faire de la mobilisation.}

\alinea{Les personnes représentantes des comités partisans doivent venir s’identifier au local de l’AGEG.}

\alinea{Les personnes représentantes doivent être identifiées conformément aux modalités de la FEUS et du contrat référendaire.}

\alinea{Toute mobilisation partisane doit être faite à l’intérieur de la cafétéria et du Salon de la faculté de génie.}

\alinea{Le maraudage dans les classes et sur les terrains de la faculté est strictement interdit.}

\alinea{Toute mobilisation partisane est strictement interdite lors des évènements sociaux de la faculté de génie.}

\alinea{Toute personne contrevenant à l’une ou l’autre de ces règles pourra se voir interdire l’accès à la faculté de génie.}

\section{Délégation au congrès du REMDUS}


\article{Gestion de la délégation au congrès du REMDUS}

\alinea{La Vice-Présidence aux cycles supérieurs (VPCS) est la représentante officielle de l’AGEG au REMDUS.}

\alinea{La VPCS est responsable de la délégation au congrès du REMDUS ainsi que de choisir les membres de la délégation.}

\alinea{La délégation choisit ses positions selon les valeurs et les intérêts de l’AGEG et de ses membres;}

\alinea{La position finale de la délégation est donnée par la VPCS de l’AGEG.}

\alinea{Si la délégation ne reçoit pas d’indications de la VPCS, chaque membre vote selon ses valeurs profondes en considérant qu’elle ou il représente l’opinion d’une partie des membres de l’AGEG.}

\alinea{Advenant l’impossibilité pour la VPCS d’assister au congrès, elle peut nominer un membre de sa délégation pour représenter l’AGEG au sein de cette instance.}
