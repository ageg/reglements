\chapter{Procédures financières}

\section{Fonds d'administration}
\alinea{Le fonds d'administration est créé selon les pratiques comptables habituelles et a pour but de gérer efficacement les surplus et les déficits réels de la corporation, ainsi que de donner à l’AGEG la capacité de gérer les mauvaises créances sans avoir recours à son budget général.}
\alinea{C’est dans ce fonds que sont placés les déficits ou les surplus réels de l’association à la fin de l’année financière.}
\alinea{Le financement de ce fonds est laissé à la discrétion du conseil d’administration, qui doit s’assurer que son solde reste positif en tout temps. Les montants des chèques non encaissés un an après leur date d’émission sont déposés dans ce fonds.}
\alinea{Le conseil d’administration a la liberté d’autoriser un transfert à l’extérieur de ce fonds. Cependant, le solde du fonds doit demeurer supérieur à 15 000~\$.}

\section{Fonds des Finissantes}
\alinea{Le fonds des Finissantes est créé selon les pratiques comptables habituelles et a pour but de contenir l'argent restant des finissantes afin de financer leurs retrouvailles.}
\alinea{Le surplus des promotions sortantes est transféré dans le fonds des finissantes une fois que les activités de fin de baccalauréat sont terminés.}
\alinea{La coordination administrative de l'AGEG doit conserver le détails des surplus versés par chaque promotion au sein du fonds des finissantes.}
\alinea{Les promotions peuvent utiliser, avec l'approbation du VPAF de l'AGEG, les sommes qu'elles y ont versées pour l'organisation de leur retrouvailles.}
\alinea{Tout argent non utilisé après 10 ans sera déposé dans le fonds d'administration.}

\section{Fonds de Meuble}
\alinea{Le fonds des Finissantes est créé selon les pratiques comptables habituelles et sert à payer les frais reliés aux :}
\sousalinea{objets mobiles servant à l'aménagement des locaux.}
\sousalinea{biens tangibles destinés à être utilisés d’une manière durable durant le cycle d’exploitation de l’entreprise. On distingue dans cette catégorie les constructions, le matériel industriel, les agencements et les installations techniques, le matériel de transport, les équipements de bureau.}
\alinea{6 \% du montant total des cotisations de l’AGEG de tous les membres doit être alloué à ce fonds.}

\section{Fonds de la Direction}
\alinea{Le fonds de la Direction est créé selon les pratiques comptables habituelles et permet de financer des initiatives étudiantes.}
\alinea{La source principale d'argent du Fonds de la Direction provient du décanat tel qu'établit par l'Entente Décanat-AGEG.}
\article{Attribution du fonds}
\alinea{Les montants du fonds de la direciton sont alloués par le CA suite à une recommandation par le comité conjoint d'attribution.}
\alinea{Les critères d’attribution suivant permettent au comité conjoint d'attribution d’évaluer et d’analyser les demandes afin de soumettre les meilleures recommandations au conseil d’administration:}
\sousalinea{La Faculté de Génie de Sherbrooke et l’AGEG doivent être représentées à l’extérieur ou à l’intérieur du campus.}
\sousalinea{L’activité ou le projet doit augmenter le sentiment d’appartenance à la Faculté et à l’AGEG.}
\sousalinea{Si le groupe demandeur est un groupe étudiant, l'activité ne doit pas être récurrente (Moins de deux~(2) occurrences financées par le fonds au cours des quatre~(4) dernières années.)}


\section{Fonds de Donation}
\alinea{Le fonds de la Direction est créé selon les pratiques comptables habituelles et permet d'appuyer des organismes de charité.}
\alinea{Toutes les sommes recueillies lors d’une activité de levée de fonds à but charitable devront être déposées dans ce fonds avant leur attribution, sauf si les activités ont été organisés pour appuyer des organismes précis.}
\article{Attribution du fonds}
\alinea{Les montants du fonds de donations sont alloués par le CA à des organismes de charité suite à une recommandation par le comité conjoint d'attribution.}
\alinea{Les critères d’attribution suivant permettent au comité conjoint d'attribution d’évaluer et d’analyser les demandes afin de soumettre les meilleures recommandations au conseil d’administration:}
\sousalinea{Les valeurs de l'organisme doit rejoindre celles de l’AGEG}
\sousalinea{L'organisme doit oeuvre dans la région de Sherbrooke et ne pas avoir de lien direct avec la Faculté de génie (Ne pas être un groupe technique de l'AGEG par exemple);}
\sousalinea{Besoins financiers de l'organisme;}
\sousalinea{Le comité doit favoriser les organismes dans lesquelles des membres de l'AGEG participent.}
\sousalinea{Le comité doit favoriser les organismes qui offrent une visibilité à l'association et à ses membres}

\section{Fonds de Subventions}
\alinea{Le Fonds de Subventions est créé selon les pratiques comptables habituelles et permet aux groupes d’obtenir des subventions pour leurs activités et leurs équipements pour ainsi promouvoir l’implication au sein de la communauté étudiante et le rayonnement du génie à l’Université de Sherbrooke au travers de ces activités.}
\alinea{Au CA1 de l'automne, un montant de 5000\$ est transféré du Fonds d'administration vers le Fonds de subvention afin de contribuer au financement des délégations de l'AGEG aux compétitions québécoises de la CRÉIQ. Le comité peut attribuer une subvention différente de ce montant.}
\alinea{Au CA1 de l'hiver, un montant de 3000\$ est transféré du Fonds d'administration vers le Fonds de subvention afin de contribuer au financement de la délégation de l'AGEG à la Compétition canadienne d'ingénierie. Le comité peut attribuer une subvention différente de ce montant.}

\article{Attribution du fonds}
\alinea{Les montants du fonds de donations sont alloués par le CA à des organismes de charité suite à une recommandation par le comité du fonds de subvention.}
\alinea{Les sommes attribuées par le comité du fonds de subvention qui ne sont pas dépensée par un groupe  à la fin de l'année financière sont reportés à la prochaine année financière.}
\alinea{Les critères d’attribution suivant permettent au comité du fonds de subventions d’évaluer et d’analyser les demandes afin de soumettre les meilleures recommandations au conseil d’administration:}
\sousalinea{Besoin financier du groupe;}
\sousalinea{Saine gestion financière du groupe;}
\sousalinea{Effort de recrutement de membres et diversité au sein du groupe (concentration, promotion);}
\sousalinea{Le groupe doit favoriser l’apprentissage de ces membres;}
\sousalinea{Le groupe doit favoriser le sentiment d’appartenance à la Faculté de génie;}
\sousalinea{Le groupe doit favoriser participer à diverses activités facultaires organisées par l’AGEG (portes ouvertes);}
\sousalinea{Le groupe doit favoriser contribuer au rayonnement de la faculté à l’extérieur par des actions concrètes (compétitions, actions à l’international, aide humanitaire, dimensions humaines en génie, média, etc.);}
\sousalinea{Le groupe doit se conformer aux différents règlements de l'AGEG et respecter ces obligations envers l'AGEG}


\section{Fonds d'équipement étudiant}
\alinea{Le fonds d'équipement étudiant est créé selon les pratiques comptables habituelles et a pour but de gérer les montants de la cotisation du fonds d'équipement étudiant. Ces montants permettent notamment de financer l'accès au studio de création.}

\section{Comité conjoint d'attribution des fonds de la direction et de donation}
\article{Composition et fonctionnement du comité}
\alinea{Le CA nomme, au CA1 de la session d'automne, un directeur et un représentant par promotion pour former, avec le VPAF, le comité conjoint d'attribution du fonds de la direction et de donation pour un mandat d'un an. En cas de vacances du poste de directeur, le VPAF agit comme directeur du comité conjoint d'attribution.}
\alinea{Le directeur du comité conjoint d'attribution est responsable de recevoir les demandes pour les fonds de la direction et de donation et d'organiser une rencontres avec le comité conjoint d'attribution.}
\alinea{Suite à la rencontre entre le comité et les demandeurs, le comité peut, à majorité absolu, attribuer des sommes de l'un ou l'autre des fonds en respectant les critères respectifs de chaque fonds.}
\alinea{Suite au délibération du comité, le directeur est responsable de rédiger une recommandation pour le CA avec les justifications nécessaires. }
\alinea{Le CA doit entériné les recommandations du comité d'attribution.}
\article{Répartition des fonds disponibles}
\alinea{La totalité des fonds doivent être attribuée à chaque année financière.}
\alinea{Le comité se rencontre lorsque le directeur le juge nécessaire.}
\alinea{À la fin de l’année financière, soit avant le CA4 de l’automne, le comité se rencontre pour attribuer l’argent restant dans les fonds et pour faire un bilan qui sera présenté au CA4 de l’automne.}

\section{Comité du fonds de subvention}
\article{Composition et fonctionnement du comité}
\alinea{Le CA nomme, au CA1 de la session d'automne, un directeur et 3 membres de l'AGEG pour former, avec le VPAF et le VPAX, le comité du fonds de subvention pour un mandat d'un an. En cas de vacances du poste de directeur, le VPAF agit comme directeur du comité. En cas de vacance d'un autre poste, le directeur est responsable de nommer un membre de l'AGEG pour combler le comité.}
\alinea{Tous les membres du comité doivent être présent à chaque rencontre.}
\alinea{Le comité se réunit une fois à l'automne et une fois à l'hiver pour rencontrer les groupes étudiants éligibles. Suite à ces rencontres, le comité peut, à majorité absolu attribuer des sommes du fonds en respectant les critères respectifs de chaque fonds.}
\alinea{À la formation d'un nouveau groupe étudiant éligible, le groupe peut soumettre une demande en dehors des périodes d'attribution prévue. Dans ce cas, le comité rencontre le groupe et soumet une recommandation au CA de l'AGEG pour une subvention.}
\alinea{De façon extraordinaire le comité peut recevoir toute demandes pertinentes en dehors des périodes prévue de subvention qu'elle provienne d'un groupe étudiant ou non. Dans ce cas, le comité rencontre le groupe et soumet une recommandation au CA de l'AGEG pour une subvention.}
\alinea{Le directeur du comité est responsable de d'informer les groupes éligibles de la procédure pour soumettre une demande de subvention au comité et d'organiser les rencontres du comité.}
\alinea{Suite aux délibérations du comité, le directeur est responsable de rédiger une recommandation pour le CA avec les justifications nécessaires. }
\alinea{Le CA doit entériné les recommandations du comité du fonds de subvention.}
\alinea{Le comité peut recommander d'imposer des conditions particulières aux groupes sur les montants financés.}
\article{Répartition des fonds disponibles}
\alinea{La totalité du Fonds doit être attribuée à chaque année. Il est recommandé de prévoir environ la moitié du budget annuel à chaque période d'attribution.}
\article{Exigences et éligibilité}
\alinea{Pour appuyer leur demande, les groupes étudiants éligibles doivent fournir au comité un bilan financier, la liste des rapports d’incidents, le nom du coordonnateur de santé et sécurité et un bilan des activités}
\alinea{Les groupes doivent faire parvenir leur demande avant la date fixée et être présent à la rencontre avec les membres du comité pour obtenir une subvention. Si le comité le juge pertinent, elle peut néanmoins offrir une subvention à un groupe qui ne respecte pas un de ces critères de façon exceptionnel.}
\alinea{Tous les groupes techniques, la délégation des jeux de génie, la délégation de la compétition québécoise d'ingénierie et la délégation de la compétition canadienne d'ingénierie sont éligibles pour une subvention.}

\section{Financement du comité des cycles supérieurs}
\alinea{Pour permettre au comité des cycles supérieurs d'organiser des activités sociales pour ses membres malgré qu'elle ne puisse pas organiser de jeudis détentes, 22\% du montant des cotisations de l'AGEG provenant de membres aux cycles supérieurs est réservé pour ce comité.}
\alinea{Les sommes non utilisées à la fin de l'année ne sont pas reportées à l'année suivante.}


