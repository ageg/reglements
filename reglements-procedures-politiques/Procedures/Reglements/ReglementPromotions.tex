\section{Conseil exécutif de promotion}

\article{Élections du CE de promotion}
\alinea{Le conseil exécutif de promotion est élu selon les modalités suivantes, à moins de dispositions contraires dans une charte votée lors d'une assemblée des membres de la promotion:}
\sousalinea{Lors de la première session d'activité d'une promotion, celle-ci est parrainée par le conseil exécutif de la promotion précédente qui l'aide à organiser une assemblée générale pour nommer un conseil exécutif. La promotion précédente aide les élus de la nouvelle promotion tout au long de la session.}
\sousalinea{Pour tous les autres sessions, les membres du conseil exécutif de promotion sont élus lors d'une assemblée générale se tenant dans l'une des deux sessions précédentes le début de leur mandat.}
\sousalinea{En cas de vacance, de démission d’un membre du conseil exécutif ou de perte d'éligibilité, il sera de la responsabilité du conseil exécutif de promotion de combler le poste une par élection parmi les membres concernés.}

\article{Mandat du CE de promotion}
\alinea{Le conseil exécutif de promotion a les mandats suivants:}
\sousalinea{De servir d'intermédiaire entre le conseil de la corporation et ses membres.}
\sousalinea{D’assurer la présence d’au moins un membre par conseil exécutif de promotion, en tant que représentant ou représentante de promotion au conseil d’administration de la corporation tel que spécifié aux règlements généraux de la promotion.}
\sousalinea{D'annoncer les assemblées générales, référendums et réunions du conseil d'administration ou de toute autre activité pertinente.}
\sousalinea{De diffuser l’information relative à la corporation~: sujets traités en conseil d’administration, services offerts.}
\sousalinea{De servir de porte-parole auprès de l'AGEG concernant les plaintes, demandes, souhaits et ainsi de suite, des membres de leur promotion.}
\sousalinea{De favoriser les activités de promotion.}
