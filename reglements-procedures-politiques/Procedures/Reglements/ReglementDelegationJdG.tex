\section{Délégation des Jeux De Génie de l’Université de Sherbrooke}


\article{Préambule}

\alinea{La délégation de Jeux de génie est un groupe de l'AGEG dont l'objectif est d'organiser la participation des membres de l'AGEG aux Jeux De Génie du Québec.}


\article{Nomination des membres de la délégation}

\alinea{Le comité exécutif de la délégation est sélectionné par le CA à la session d'hiver, sous recommandation d'un comité de nomination dirigé par la VPEX.}

\alinea{Le comité de nomination du comité exécutif est formé par le CA lors du CA2 de l'hiver, sous recommandation de la VPEX, et est constitué de deux personnes administratrices : la VPEX et l'ancien chef de la délégation. Une priorité est donnée aux membres du CA ayant déjà participé aux JDG.}

\alinea{Le comité de nomination est responsable d'organiser une sélection ouverte à tous les membres de l'AGEG. Cette sélection doit être annoncée au moins 2 semaines à l'avance.}

\alinea{Les personnes se présentant pour le poste de chef ou de co-chef assistent le comité de nomination pour la sélection des autres membres du comité exécutif.}

\alinea{Le comité exécutif est responsable d'organiser les sélections pour le reste de la délégation. Les membres du comité exécutif sont automatiquement membres de la délégation.}

\alinea{Un poste du comité exécutif laissé vacant sera pourvu par un membre de l’AGEG nommé par le conseil exécutif de la délégation des Jeux De Génie.}

\alinea{Les critères de sélection pour le comité exécutif de la délégation sont:}

\sousalinea{Être membre de l'AGEG}

\sousalinea{Avoir un comité de sélection le plus uni possible}

\sousalinea{Motivations des candidats}

\sousalinea{Idées des candidats quant aux Jeux De Génie}

\sousalinea{Implications antérieures et actuelles des candidats au niveau universitaire}

\sousalinea{Participation antérieure des candidats aux Jeux De Génie}

\sousalinea{Tout autre point jugé pertinent}

\alinea{Le comité exécutif est formé de:}

\sousalinea{Un ou une chef ou deux co-chefs}

\sousalinea{Trésorier}

\sousalinea{Vice-présidence aux Commanditaires}

\sousalinea{Vice-présidence Logistique (facultatif)}

\sousalinea{Vice-présidence Social}

\sousalinea{Vice-présidence Articles Promotionnels}

\sousalinea{Vice-présidence Machine}

\sousalinea{Vice-présidence Arts et Culture}

\sousalinea{Vice-présidence Entrepreneuriat}


\article{Rôles du Chef ou des co-chefs}

\alinea{Le chef ou les co-chefs sont cosignataire du compte avec le trésorier de la délégation des Jeux De Génie.}

\alinea{Superviser le travail fait par les vice-présidences et s’assurer que les différents mandats des vice-présidences sont accomplis.}

\alinea{S’occuper des communications avec l'AGEG, avec l'Université hôtesse et avec les membres de la délégation.}

\alinea{Représenter la délégation à la CRÉIQ.}

\alinea{Convoquer et diriger les réunions de l’exécutif de la délégation.}

\alinea{Responsable de former les équipes pour les différentes compétitions des Jeux De Génie.}

\alinea{Responsable de trouver le matériel nécessaire aux différentes compétitions.}

\alinea{S’assurer que tous les participants sont au courant de leurs tâches durant les Jeux.}

\alinea{S'assurer de faire participer tous les membres de la délégation.}

\alinea{Dans le cas où il y a un co-chef, cette personne est responsable des tâches de la VP logistique}

\sousarticle{Rôles du Trésorier}

\alinea{Cosignataire du compte avec le président de la délégation des Jeux De Génie.}

\alinea{Responsable du budget de la délégation.}

\alinea{Responsable de créer le cahier de commandites et d'assister la VP Commandite dans la recherche de commanditaires.}

\alinea{Assister la VP Social dans l'organisation des activités de financement.}

\sousarticle{Rôles de la VP Commandites}

\alinea{S'occuper des commanditaires, de faire la promotion de la délégation auprès des étudiants, auprès de la population et des médias et être responsable des archives qui concernent la délégation.}

\alinea{Assister la VP Social dans l'organisation des activités de financement.}

\sousarticle{Rôles de la VP Social}

\alinea{Responsable d'organiser le recrutement des membres, de préparer les activités de cohésion d'équipe et de prendre part à l'organisation des activités de financement.}

\sousarticle{Rôles de la VP Articles Promotionnels}

\alinea{Responsable des uniformes ainsi que de la mise en œuvre de la thématique de la délégation.}

\sousarticle{Rôles de la VP Machine}

\alinea{Responsable de l'équipe de conception et de la réalisation de la machine.}

\sousarticle{Rôles de la VP Logistique}

\alinea{Responsable de l'organisation des minis-jeux, de la compétition machine II et du transport de la délégation aux Jeux.}

\alinea{Assister la VP Social dans le recrutement des membres.}

\alinea{Dans le cas où une personne co-chef est élue, les tâches de la VP Logistique seront transférées au co-chef et ce poste restera vacant.}

\sousarticle{Rôles de la VP Arts et Culture}

\alinea{Responsable d'organiser les pratiques des compétitions sociales (génie en herbe, débat oratoire, improvisation) et des évènements culturels organisés par la délégation durant les Jeux (décors, dégrises, spectacle d'ouverture).}

\alinea{Responsable de la réalisation et du tournage de la vidéo présentant la machine et de la mise à jour du site internet de la délégation.}

\alinea{Assister la VP aux Articles Promotionnels pour la confection des uniformes.}

\sousarticle{Rôles de la VP Entrepreneuriat}

\alinea{Responsable de l'équipe entrepreneuriale pour ladite compétition.}


\article{Gestion de la délégation}

\alinea{Le chef ou les co-chefs de la délégation doivent produire un rapport sur l’année de son mandat après la tenue de la compétition y incluant la liste de tous les membres de la délégation au plus tard pour le CA2 de la session d’hiver.}

\alinea{Le trésorier et un des chefs sont responsables de la gestion financière de la délégation.}

\alinea{Le chef ou les co-chefs sont responsables de faire le lien entre la délégation et l'AGEG.}

\alinea{La VPEX de l'AGEG est en charge d'assurer de la bonne gestion de la délégation.}


\article{Gestion du comité exécutif}

\sousarticle{Quorum}

\alinea{Pour toute assemblée officielle du comité, il est nécessaire d’avoir la majorité des membres. Aucune décision ne sera prise à moins que cette assemblée n’ait le quorum requis.}

\alinea{Les décisions du comité se prendront par vote à majorité. Chaque membre a droit à un vote. En cas d’égalité, le ou la chef a un vote de plus.}

\sousarticle{Démission ou expulsion}

\alinea{Toute démission de tout membre du comité devra parvenir par écrit au Chef de la délégation des Jeux De Génie.}

\alinea{Un membre du comité peut se voir expulsé du comité par le CA de l’AGEG ou par l’assemblée générale de l’AGEG pour des raisons jugées valables.}
