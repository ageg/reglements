\section{Délégation des Jeux de génie de l’Université de Sherbrooke}


\article{Préambule}
\alinea{La délégation de Jeux de génie est un groupe de l'AGEG dont l'objectif est d'organiser la participation des membres de l'AGEG aux Jeux de génie du Québec.}

\article{Nomination des membres de la délégation}

\alinea{Le comité exécutif de la délégations est nominé par le CA à la session d'hiver, sous recommandation d'un comité de nomination dirigé par le VPEX.}

\alinea{Le comité de nomination du comité exécutif est formé par le CA lors du CA2 de l'hiver, sous recommandation du VPEX, et est constitué de deux personnes administratrices, du VPEX et de l'ancien chef de la délégation. Une priorité est donnée aux membres du CA ayant déjà participé aux JDG.}

\alinea{Le comité de nomination est responsable d'organiser une sélection ouverte à tous les membres de l'AGEG. Cette sélection doit être annoncé au moins 2 semaines à l'avance.}

\alinea{Les personnes se présentant pour le poste de chef ou de co-chef assistent le comité de nomination pour la sélection des autres membres du comité exécutif.}

\alinea{Le comité exécutif est responsable d'organiser les sélections pour le reste de la délégation. Les membres du comité exécutif sont automatiquement membres de la délégation.}

\alinea{Un poste du comité exécutif laissé vacant sera comblé par un membre de l’AGEG nommé par le conseil exécutif de la délégation des Jeux de génie.}

\alinea{Les critères de sélection pour le comité exécutif de la délégation sont:}
\sousalinea{Être membre de l'AGEG}
\sousalinea{Avoir un comité de sélections le plus uni possible }
\sousalinea{Motivations des candidats}
\sousalinea{Idées des candidats quant aux Jeux de génie}
\sousalinea{Implications antérieures et actuelles des candidats au niveau universitaire}
\sousalinea{Participation antérieure des candidats aux Jeux de génie}
\sousalinea{Tout autre point jugé pertinent}

\alinea{Le comité exécutif est formé de:}
\sousalinea{Un chef ou deux co-chefs}
\sousalinea{Trésorier}
\sousalinea{Vice-président aux commanditaires}
\sousalinea{Vice-président logistique (facultatif)}
\sousalinea{Vice-président social}
\sousalinea{Vice-président articles promotionnels}
\sousalinea{Vice-président machine}
\sousalinea{Vice-président arts et culture}
\sousalinea{Vice-président entrepreneurial}

\article{Rôles du Chef ou co-chef}
\alinea{Le chef ou co-chef est cosignataire du compte avec le trésorier de la délégation des Jeux de génie.}
\alinea{Il supervise le travail fait par les vice-présidents et s’assure que les différents mandats des vice-présidents sont accomplis.}
\alinea{Il s’occupe des communications avec l'AGEG, l'Université hôtesse et avec les membres de la délégation.}
\alinea{Il représente la délégation à la CRÉIQ.}
\alinea{Il convoque et dirige les réunions de l’exécutif de la délégation.}
\alinea{Il est responsable de former les équipes pour les différentes compétitions des Jeux de génie.}
\alinea{Il est responsable de trouver le matériel nécessaire aux différentes compétitions.}
\alinea{Il s’assure que tous les participants sont au courant de leurs tâches durant les Jeux.}
\alinea{Il doit voir à faire participer tous les membres de la délégation.}
\alinea{Dans le cas où il y a un co-chef, il est responsable des tâches du VP logistique}


\sousarticle{Rôles du Trésorier}
\alinea{Il est cosignataire du compte avec le président de la délégation des Jeux de génie.}
\alinea{Il est responsable du budget de la délégation.}
\alinea{Il est responsable de créer le cahier de commandites d'assister le VP commandite dans la recherche de commanditaires.}
\alinea{Il assiste le VP social dans l'organisation des activités de financement.}


\sousarticle{Rôles du VP commandites}
\alinea{Il s'occupe des commanditaires, de faire la promotion de la délégation auprès des étudiants, de la population et des médias et est responsable des archives qui concernent la délégation. }
\alinea{Il assiste le VP social dans l'organisation des activités de financement.}


\sousarticle{Rôles du VP social}
\alinea{Il est responsable d'organiser le recrutement des membres, de préparer les activités de cohésion d'équipe et de prendre part à l'organisation des activités de financement.}


\sousarticle{Rôles du VP articles promotionnels}
\alinea{Il est responsable des uniformes ainsi que de la mise en œuvre de la thématique de la délégation.}

\sousarticle{Rôles du VP machine}
\alinea{Il est responsable de l'équipe de conception et de réalisation de la machine.}


\sousarticle{Rôles du VP logistique}
\alinea{Il est responsable de l'organisation des minis-jeux, de la compétiton machine II et du transport de la délégation aux jeux.}
\alinea{Il assiste le VP social dans le recrutement des membres.}
\alinea{Dans le cas où un co-chef est élu, les tâches du VP logistique seront transférées au co-chef et ce poste restera vacant.}


\sousarticle{Rôles du VP arts et culture}
\alinea{Il est responsable d'organiser les pratiques des compétitions sociales (génie en herbe, débat oratoire, improvisation) et des évènements culturels organisés par la délégation durant les jeux (décors, dégrises, spectacle d'ouverture). }
\alinea{Il est responsable de la réalisation et du tournage de la vidéo présentant la machine et de la mise à jour du site internet de la délégation.}
\alinea{Il assiste le VP aux articles promotionnels de la confection des uniformes.}


\sousarticle{Rôles du VP Entrepreneuriat}
\alinea{Il est responsable de l'équipe entrepreneuriale pour ladite compétition.}


\article{Gestion de la délégation}
\alinea{Le chef ou les co-chefs de la délégation doivent produire un rapport sur l’année de son mandat après la tenue de la compétition y incluant la liste de tous les membres de la délégation au plus tard pour le CA2 de la session d’hiver.}
\alinea{Le trésorier et un des chefs est responsable de la gestion financière de la délégation.}
\alinea{Le chef ou les co-chefs sont responsable de faire le lien entre la délégation et l'AGEG.}
\alinea{Le VPEX de l'AGEG est en charge de s'assurer de la bonne gestion de la délégation.}

\article{Gestion du comité exécutif}

\sousarticle{Quorum}
\alinea{Pour toute assemblée officielle du comité, il est nécessaire d’avoir la majorité des membres. Aucune décision ne sera prise à moins que cette assemblée n’ait le quorum requis.}
\alinea{Les décisions du comité se prendront par vote à majorité. Chaque membre a droit à un vote. En cas d’égalité, le chef a un vote de plus.}

\sousarticle{Démission ou expulsion}
\alinea{Toute démission de tout membre du comité devra parvenir par écrit au Chef de la délégation des Jeux de génie.}
\alinea{Un membre du comité peut se voir expulsé du comité par le CA de l’AGEG ou l’assemblée générale de l’AGEG pour des raisons jugées valables.}
