\reglement{Relatif à la délégation de la Compétition québécoise d’ingénierie}

\preambule{L’objectif du présent règlement est de donner des lignes directrices au comité de sélection de l’exécutif de la délégation de la CQI, ainsi qu’au président ou aux co-présidents de la délégation afin d’assurer le bon déroulement des activités et de la représentation sherbrookoise lors de l’événement.}

\partie{Dispositions générales}
\article{Mission de la délégation de la Compétition québécoise d’ingénierie}
\alinea{Permettre à l’élite de nos étudiants de compétitionner dans un esprit de fraternité avec d’autres étudiants de la province dans différentes disciplines.}
\alinea{Permettre aux étudiants de génie de l’Université de Sherbrooke de rencontrer des étudiants des autres facultés de génie du Québec dans le cadre de la compétition.}
\alinea{Rapprocher les membres de l’AGEG.}

\article{Vision de la délégation de la Compétition québécoise d’ingénierie}
\alinea{La vision de la délégation de la Compétition québécoise d’ingénierie est d’assurer la continuité et le bon fonctionnement des activités de la délégation de l’Université de Sherbrooke lors de la compétition.}

\article{Valeurs de la délégation de la CQI}
\valeurs
{Ouverture}{Offrir à tous les étudiants inscrits au baccalauréat une compétition académique d’envergure}
{Engagement}{Permettre de représenter Sherbrooke avec une délégation complète}
{Intégrité}{Assurer une sélection impartiale des représentants de l’AGEG à la CQI dans le but d’avoir des représentants d’élite}
{Fraternité}{Favoriser un sentiment d’appartenance des membres envers la Faculté et l’Université}

\article{Rôles et pouvoirs}
\sousarticle{AGEG}
\alinea{Elle forme le comité de sélection de la présidence de la délégation lors du CA2 de la session d’hiver.}
\alinea{Elle vote les recommandations du comité remises lors du CA3 de la session d’hiver.}

\sousarticle{Comité de sélection de la présidence de la délégation}
\alinea{Il rencontre les différents candidats au poste de présidence de la délégation pour la CQI de l’Université de Sherbrooke.}
\alinea{Il remet ses recommandations lors du CA3 de la session d’hiver.}

\sousarticle{VPEX de l’AGEG}
\alinea{Il préside le comité de sélection de la présidence de la délégation pour la CQI.}
\alinea{Il est responsable d’annoncer les formulaires de demandes ainsi que les modalités de sélection de la présidence de la délégation.}
\alinea{Il recueille les demandes avant la date butoir à la session d’hiver.}
\alinea{Dans le cas où le président ou les co-présidents de la délégation ne peuvent pas organiser une partie ou l’ensemble des compétitions de sélection, il organise une pré-CQI, appelée CSI (Compétition sherbrookoise d’ingénierie), et il est responsable de l’organisation de cette ou ces compétitions.}

\sousarticle{Président ou co-président de la délégation de la CQI}
\alinea{Il est signataire du compte de la délégation de Sherbrooke pour la CQI.}
\alinea{Il supervise le travail fait par les délégués.}
\alinea{Il s’occupe des communications à l’intérieur de la délégation.}
\alinea{Il fait le lien entre l’AGEG et la délégation.}
\alinea{Il fait le lien entre l’Université hôtesse et la délégation.}
\alinea{Il doit produire un rapport sur l’année de son mandat après la tenue de la compétition y incluant la liste de tous les membres de la délégation au plus tard pour le CA2 de la session d’hiver.}
\alinea{Il est responsable de motiver les membres de la délégation avant et pendant la CQI.}
\alinea{Il participe à la compétition québécoise d’ingénierie en tant que participant s’il a été sélectionné lors des qualifications ou en tant qu’observateur s’il n’est pas participant dans aucunes des compétitions.}
\alinea{Il est président de la délégation sherbrookoise à la Compétition canadienne d’ingénierie si}
\sousalinea{Une ou des équipes se sont classés pour la CCI et}
\sousalinea{Il fait partie d’une de ces équipes.}
\alinea{Le cas échéant, il devra nommer un participant de la CCI et l’accompagner lors de ses démarches, tout en conservant le titre de signataire du compte.}
\alinea{Il nomme un signataire supplémentaire du compte de la délégation lorsque les membres de la délégation sont choisis. Il doit choisir en priorité un membre du CE de l’AGEG ou un membre du CA de l’AGEG qui est également membre de la délégation.}
\alinea{Il est responsable de la tenue sécuritaire des évènements en lien avec la compétition.
\alinea{Il est responsable de la tenue du budget de la délégation et le remplit selon les normes établies par le règlement 50 sur les finances des groupes techniques et des promotions}
\alinea{Il est responsable de l'organisation de la Compétition Sherbrookoise d'Ingénierie (CSI)}

\partie{Partie II :	Comité de sélection de la présidence de la délégation de la CQI}}
\article{Composition du comité}

\sousarticle{Membres du comité}
\alinea{Le comité est formé de 2 membres de l’AGEG nommés par le CA ainsi que du VPEX actif de l’AGEG.}

\sousarticle{Rémunération}
\alinea{Les membres du comité ne recevront aucune rémunération pour leur fonction.}
\article{Nomination et démission}

\sousarticle{Nomination des membres du comité}
\alinea{La durée du mandat des membres du comité est d’une session : la session d’hiver.}
\alinea{La priorité sera donnée aux membres du CA ayant déjà participé à la CQI.}

\sousarticle{Démission}
\alinea{Toute démission de tout membre du comité devra parvenir par écrit au VPEX de l’AGEG et sera effective après son approbation.}
\alinea{Un membre du comité peut se voir expulsé du comité par le CA de l’AGEG ou l’assemblée générale de l’AGEG pour des raisons jugées valables.}
\alinea{Un poste laissé vacant sera comblé par un membre de l’AGEG, nommé par le CE de l’AGEG.}

\article{Décision}
\sousarticle{Critères}
\alinea{Les critères de sélection du conseil exécutif de la délégation de la CQI permettent au comité d’évaluer et d’analyser les demandes afin de soumettre les meilleures recommandations au conseil d’administration.}
\alinea{Les critères sont les suivants :}
\sousalinea{Motivations des candidats }
\sousalinea{Idées des candidats quant à la CQI}
\sousalinea{Implications antérieures et actuelles des candidats au niveau universitaire}
\sousalinea{Participation antérieure des candidats à la CQI}
\sousalinea{Tous autres points jugés pertinents.}

\partie{Présidence de la délégation de la CQI}
\article{Rémunération}
\alinea{Le président ou les co-présidents ne recevront aucune rémunération pour leur mandat.}

\article{Nomination et démission}
\sousarticle{Nomination du président ou des co-présidents}
\alinea{La durée du mandat est d’un an, débutant au CA3 de la session d’hiver, se terminant lors du CA2 de la session d'hiver suivante, à la remise du rapport.}
\alinea{La présidence de la délégation peut être assurée par un président ou par deux co-présidents.}

\sousarticle{Démission}
\alinea{Une démission du président ou d’un co-président de la délégation doit être transmise au CE de l’AGEG et sera effective après son approbation.}
\alinea{Le président ou les co-présidents peuvent se voir expulsés de la délégation par le CA de l’AGEG ou l’assemblée générale de l’AGEG pour des raisons jugées valables.}
\alinea{Un poste laissé vacant sera comblé de façon intérimaire par le VPEX de l’AGEG jusqu’à la nomination d’une nouvelle personne par le CA de l’AGEG.}

\article{Décision}
\sousarticle{Critères}
\alinea{Les critères de sélection des participants de la délégation de la CQI permettent au VPEX et au président ou aux co-présidents de la délégation d’évaluer et d’analyser les demandes afin de choisir les meilleurs candidats.}
\alinea{Lorsque plusieurs équipes se présentent dans une même discipline, le guide de compétition pré-CQI doit être utilisé, guide qui est envoyé par le comité organisateur de la CQI avant la mi-session d’automne, pour faire compétitionner entre elles les équipes d’une même catégorie.}
\alinea{Les critères d’évaluation sont les mêmes que ceux reçus dans la grille d’évaluation pour les compétitions pré-CQI.}
\alinea{Dans la situation où le guide des compétitions pré-CQI ne contient pas toutes les informations nécessaires, le président ou les co-présidents de la délégation peuvent se baser sur les guides des compétitions des années précédentes. Il en revient donc au président ou aux co-présidents d’organiser une compétition de sélection qui sera la plus représentative de la compétition québécoise d’ingénierie à venir.}

\adoption{3 avril 2016}{7 avril 2016}