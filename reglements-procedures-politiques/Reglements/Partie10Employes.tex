\chapter{Employés}


\article{Coordination administrative}
\sousarticle{Définition du poste}
\alinea{La personne faisant la coordination administrative s'occupe de la tenue des livres comptables de l'AGEG, de l'archivage des documents importants, des relations avec les comptables de la corporation et de la paie des personnes employées. Elle effectue la production des différents documents financiers, bancaires et gouvernementaux. Elle est la personne de référence pour les finances des groupes de l'entreprise.}

\sousarticle{Rémunération}
\alinea{Le salaire de la personne faisant la coordination administrative est négocié à l'embauche et aux renouvellements du contrat de travail. Le contrat de travail est approuvé en huis clos par le conseil d'administration.}


\article{Adjointe ou adjoint à l'administration} 
\sousarticle{Définition du poste}
\alinea{La personne adjointe à l'administration offre du soutien à la coordination administrative dans les différentes tâches au sein de la corporation.}

\sousarticle{Rémunération}
\alinea{Le salaire de l'adjointe administrative ou l'adjoint administratif est négocié à l'embauche et aux renouvellements du contrat de travail. Le contrat de travail est approuvé en huis clos par le conseil d'administration.}


\article{Préposée ou préposé à l'accueil} 
\sousarticle{Définition du poste}
\alinea{La personne préposée à l'accueil de l'AGEG est responsable d'accueillir les membres au local de l'association. Elle réalise diverses tâches de services aux membres. Elle se doit d'avoir de l'entregent, d'être polyvalente, d'avoir de l'initiative et d'aimer travailler avec le public. Finalement, elle doit avoir une très bonne maîtrise du français écrit.}

\sousarticle{Rémunération}
\alinea{Le salaire de la préposée ou du préposé à l'accueil est présenté sur la convention de travail de la personne employée et n'est pas négociable. }


\article{Préposée ou préposé à l'accueil senior}
\sousarticle{Définition du poste}
\alinea{La personne préposée à l'accueil senior est une personne employée ayant réalisée au moins deux (2) sessions de travail au poste de préposée ou préposé à l'accueil et dont les évaluations semestrielles attestent d'un leadership exemplaire. En plus d'effectuer les tâches de la personne préposée à l'accueil, la personne préposée à l'accueil senior est aussi responsable de la formation et de la coordination des personnes préposées à l'accueil.}

\sousarticle{Rémunération}
\alinea{Le salaire de la préposée ou du préposé à l'accueil senior est présenté sur la convention de travail de l'employé et n'est pas négociable.}


\article{Création et modification de poste}
\alinea{Le conseil d'administration de l'AGEG est responsable de la création de nouveaux postes.}

\alinea{La modification et le renouvellement des contrats et conventions de travail doivent être approuvés par le conseil d'administration de l'AGEG.}


\article{Durée des mandats de travail}
\alinea{La durée des contrats de travail de la coordination administrative et de la personne adjointe administrative est d'un an. Le conseil d'administration se réserve le droit de renouveler les contrats à leurs expirations.}

\alinea{La durée des mandats de travail des personnes préposées à l'accueil et de la personne préposée à l'accueil senior est d'une session. Le comité exécutif se réserve le droit de renouveler les postes à leurs expirations.}


\article{Embauche}
\alinea{Le comité exécutif de l'AGEG recueille les candidatures et choisit les personnes candidates retenues, pour ensuite les passer en entrevue.}

\alinea{Le conseil d'administration de l'AGEG adopte les contrats de travail ou les modifications au contrat de travail des personnes employées.}

\alinea{Le comité exécutif de l'AGEG approuve l'embauche des personnes préposées à l'accueil.}

\alinea{Le comité exécutif de l'AGEG est en charge d'embaucher suffisamment de personnes préposées à l'accueil pour couvrir toutes les heures d'ouverture de l'AGEG.}

\alinea{Les personnes préposées à l’accueil doivent avoir été sélectionnées avant le début de la session pour assurer leur entrée en fonction à la première journée ouvrable de la session.}

\alinea{Les personnes préposées à l’accueil doivent être éligibles au programme études-travail de la fondation FORCE.}

\alinea{L'embauche de tout autre personne employée est approuvée par le conseil d'administration de l'AGEG.}


\article{Démission}
\alinea{Le conseil d'administration de l'AGEG devra être averti de toute démission de tout employé.}

\alinea{Un poste laissé vacant devra être pourvu par une nouvelle personne employée le plus tôt possible.}


\article{Renvoie}
\alinea{Une personne employée peut se voir renvoyer par le conseil d'administration de l'AGEG pour des raisons jugées valables.}

\alinea{Un poste laissé vacant devra être pourvu par une nouvelle personne employée le plus tôt possible.}


\article{Cessation d'emploi}
\alinea{Un avis de cessation d'emploi doit être produit et donné à la personne employé.}


\article{Accès au document privilégiés}
\alinea{Les personnes siégeant au conseil d'administration sont tenues de ne pas divulguer les informations privilégiées auxquelles elles ont accès.}


\article{Conflit d'intérêts}
\alinea{Tout personne à l'emploi de la corporation en conflit d'intérêts est responsable de faire savoir rapidement à la présidence qu'elle est en conflit d'intérêts par rapport à un dossier qu'elle traite dans le cadre de ses fonctions.}


\article{Harcèlement et violences à caractère sexuel}
\alinea{L'association s'engage à fournir un milieu de travail sans harcèlement ou violences à caractère sexuel à ses personnes employées.}

\alinea{L'AGEG est dotée d'une politique contre le harcèlement et les violences à caractère sexuel adoptée par le conseil d'administration. Cette politique est un document public accessibles à tous les membres de la corporation.}