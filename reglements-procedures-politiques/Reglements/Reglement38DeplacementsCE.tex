\reglement{Relatif au remboursement des frais de déplacement du comité exécutif}

\preambule{Attendu que l’Association générale des étudiants en génie de l’Université de Sherbrooke (« AGEG ») a besoin, pour acquérir le matériel de fonctionnement courant, de se déplacer dans la Ville de Sherbrooke pour effectuer lesdits achats, attendu que ces déplacements engendrent des dépenses en carburant qui ne devraient pas être assumées par l’exécutant réalisant ces achats, les administrateurs de l’AGEG adoptent le règlement suivant :}

\partie{Dispositions générales}
\article{Mission du présent règlement}
\alinea{Permettre le remboursement des frais de déplacement.}
\alinea{Empêcher les abus sur les remboursements des frais de déplacement.}

\article{Vision de l’AGEG par rapport aux frais de déplacement}
\alinea{Assurer que les exécutants ne soient pas tenus d’assumer les coûts relatifs au fonctionnement de l’association.}

\article{Valeurs dans le remboursement des frais de déplacement du CE}
\valeurs
{Ouverture}{S’assurer que tous peuvent être membres du CE, sans limitation sur leur capacité financière}
{Engagement}{Offrir des services de qualité}
{Intégrité}{Être égal envers tous les membres du CE}
{Fraternité}{Promouvoir la participation dans le CE}

\article{Rôles et pouvoirs}
\sousarticle{Le CE de l’AGEG}
\alinea{Il gère la distribution des fonds alloués pour le déplacement.}

\partie{Modalités d’application}
\article{Budget}
\alinea{Nonobstant tout autre règlement ou décision, le montant autorisé par le CA pour l’application de ce présent règlement ne peut jamais être dépassé, sous aucune circonstance. }
\alinea{Le montant alloué au CE pour ses frais de déplacement est fixé à 75~\$ par session, à partir des affaires courantes de la session.}

\adoption{20 février 2011}{31 mars 2011}