\reglement{Relatif à l’élection régulière du comité exécutif passif et du conseil d'administration}

\preambule{L’objectif du présent règlement est de définir le processus d’élection régulier du comité exécutif passif et du conseil d'administration de l'AGEG, afin de favoriser une gestion efficace et continue entre les différents comités exécutifs et les conseils d'administration.
\newline Pour les élections relatives à un poste vacant, les procédures à appliquer sont décrites dans la charte de l'AGEG.}

\partie{Dispositions générales}
\article{Mission de l’AGEG par rapport à l’élection du comité exécutif passif et du conseil d'administration}
\alinea{Le but des procédures d'élections de l'AGEG est de rendre l’AGEG fonctionnelle en favorisant une transition efficace entre les différents comités exécutifs et conseils d'administration.}

\article{Vision de l’AGEG par rapport à l’élection du comité exécutif passif et du conseil d'administration}
\alinea{La vision de l’AGEG est d’avoir un processus d’élection transparent et équitable envers tous les membres de la corporation.}

\article{Valeurs de l’AGEG à propos de l’élection du comité exécutif passif et du conseil d'administration}
\valeurs
{Intégrité}{Assurer une sélection impartiale des dirigeants de l'AGEG}
{Fraternité}{Favoriser un sentiment d’appartenance des membres face à leur association}
{Engagement}{Fournir aux membres la possibilité d’acquérir des qualités professionnelles grâce à l’implication}
{Ouverture}{Faire preuve de démocratie dans l’élection des dirigeants de l'AGEG}

\article{Rôles et pouvoirs}
\sousarticle{CA}
\alinea{Reçoit et entérine le rapport du président d'élection.}

\sousarticle{CE}
\alinea{Désigne le président d’élection.}

\sousarticle{Président d’élection}
\alinea{Détermine les dates de la présente période électorale.}
\sousalinea{Si, pour quelque raison que ce soit, les dates de la période électorale ne peuvent pas respecter les barèmes fixés dans le présent règlement, le président d'élection devra faire approuver son calendrier électoral par le comité exécutif actif.}
\alinea{Reçoit les candidatures, prépare les bulletins de vote et veille à la bonne marche de l'élection, le tout en conformité avec les dispositions du présent règlement.}
\alinea{Établit et publicise les modalités du vote.}
\alinea{S’adjoint de scrutateurs le jour de l’élection.}
\alinea{Favorise la participation au scrutin à l’aide de publicité le jour du scrutin.}
\alinea{Publicise les résultats du scrutin.}
\alinea{Dépose un rapport d’élection au dernier CA de la session durant laquelle se tient le scrutin.}
\alinea{Doit être étudiant ou ancien étudiant de l'université de Sherbrooke et ne peut être candidat.}


\sousarticle{Scrutateur}
\alinea{S’occupe du scrutin lors du jour de l’élection.}
\alinea{Doit être étudiant ou ancien étudiant de l'Université de Sehrbrooke et ne peut être candidat.}

\sousarticle{Candidat au comité exécutif passif}
\alinea{Doit être membre régulier de la corporation.}
\alinea{Doit être membre régulier actif de la corporation à la session pour laquelle il se présente.}
\alinea{Doit remplir son formulaire de mise en candidature.}

\sousarticle{Candidat au conseil d'administration}
\alinea{Doit être membre régulier de la corporation.}
\alinea{Doit être membre régulier de la corporation à la période pour laquelle il se présente.}
\alinea{Doit remplir son formulaire de mise en candidature.}

\partie{Fonctionnement de l’élection}
\article{Lancement de la période d’élection}
\alinea{Le comité exécutif désigne un président d’élection.}
\alinea{Le président d’élection détermine les dates de la présente campagne électorale.}

\article{Mise en candidature }
\alinea{Tous les candidats aux postes d’officiers passifs doivent remplir le formulaire de mise en candidature.}
\alinea{Le président d'élection recevra les candidatures des candidats.}
\alinea{La période de mise en candidature doit durer un minimum de cinq (5) jours ouvrables.}
\sousalinea{Si possible, la période de mise en candidature doit englober la soirée d'implication de l'AGEG.}

\article{Campagne électorale}
\alinea{Un minimum de trois (3) jours ouvrables suivants la période de mise en candidature sont destinés à la campagne électorale.}
\alinea{Les candidatures seront publiées par le président d'élection dès le début de la période de campagne électorale.}
\alinea{Pour les postes où aucune candidature n'aura été présentée à la fin de la période de mise en candidature régulière, la période de mise en candidature sera prolongée pendant toute la campagne électorale.}

\article{Jour du scrutin}
\alinea{Le jour suivant la campagne électorale est destiné au scrutin.}
\sousalinea{Si possible, une assemblée générale d'élection devra être tenue le jour du scrutin.}
\sousalinea{Si certains postes n'ont toujours pas reçus de candidatures valides, les candidatures spontanées pour ces postes seront recevables au cours de l'assemblée générale.}
\alinea{Le jour du scrutin, la seule publicité permise sera celle prévue par le président d'élection. Elle devra favoriser la participation au scrutin.}
\alinea{Le jour du scrutin doit être avant la fin des cours de la présente session.}

\sousarticle{Méthode de scrutin}
\alinea{La méthode de scrutin par défaut est le scrutin universel, selon les modalités établies par le président d'élection.}


\sousarticle{Élection du comité exécutif passif}
\alinea{Pour chacun des postes, le vote comprend un vote de confiance. L’électeur a donc le choix entre un vote pour un des candidats au poste et un vote contre tous les candidats au poste (la chaise).}

\sousarticle{Élection du conseil d'administration}
\alinea{Si le président d'élection juge que son système de vote lui en offre la possibilité, il devra soumettre chacun des candidats à un vote de confiance. Sinon, le vote de confiance sera extrapolé directement du vote entre tous les candidats.}
\sousalinea{S'il y a un vote de confiance explicite, chaque candidat devra récolter la majorité simple des voix exprimées, sinon il sera retiré de la course.}
\sousalinea{S'il y a un vote de confiance implicite, chaque candidat devra récolter au moins la moitié plus un de la moyenne des voix possibles par candidat, sinon, il sera retiré de la course. La formule qu'un candidat devra respecter pour gagner son vote de confiance est exprimée à l'équation \ref{eq:confiance}.
\begin{equation}\label{eq:confiance}
 V \geq
 \begin{cases}
  \frac{n}{2}+1 & p \geq c\\
  \frac{n*p}{2*c}+1 & p<c\\
 \end{cases}
\end{equation}
Où:
\begin{itemize}
 \item[] $V$ est le nombre de voix récoltées par un candidat;
 \item[] $p$ le nombre de postes ouverts au CA;
 \item[] $n$ le nombre de votes exprimés;
 \item[] $c$ le nombre de candidats.
\end{itemize}
}
\alinea{S'il y a moins de candidats au conseil d'administration que de postes ouverts, tous les candidats gagnant leur vote de confiance seront réputés élus par acclamation.}
\alinea{S'il y a plus de candidats que de postes ouverts, les électeurs devront pouvoir voter un nombre maximal de candidats égalant le nombre de postes ouverts.}

\article{Résultat de l’élection}
\alinea{Un candidat au comité exécutif passif sera élu s'il récolte la majorité simple des voix.}
\alinea{Les postes au conseil d'administration seront attribués par ordre décroissant aux candidats ayant gagné leur vote de confiance et ayant récolté le plus grand nombre de votes pour son investiture parmi tous les candidats.}
\alinea{Le président d’élection doit publiciser les résultats du scrutin.}
\alinea{Le président d’élection doit déposer un rapport au dernier CA de la session durant laquelle se tient le scrutin.}

\article{Litiges}
\alinea{En cas de litige, le président d'élection déterminera les procédures à suivre.}

\adoption{21 octobre 2018}{27 novembre 2018}