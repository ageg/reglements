\reglement{Relatif à l’élection régulière du comité exécutif passif et du conseil d'administration}

\preambule{L’objectif du présent règlement est de définir le processus d’élection régulier du comité exécutif passif et du conseil d'administration de l'AGEG, afin de favoriser une gestion efficace et continue entre les différents comités exécutifs et les conseils d'administration.
\newline Pour les élections relatives à un poste vacant, les procédures à appliquer sont décrites dans le Règlement 1 du présent cahier.}

\partie{Dispositions générales}
\article{Mission de l’AGEG par rapport à l’élection du comité exécutif passif et du conseil d'administration}
\alinea{Le but des procédures d'élections de l'AGEG est de rendre l’AGEG fonctionnelle en favorisant une transition efficace entre les différents comités exécutifs et conseils d'administration.}

\article{Vision de l’AGEG par rapport à l’élection du comité exécutif passif et du conseil d'administration}
\alinea{La vision de l’AGEG est d’avoir un processus d’élection transparent et équitable envers tous les membres de la corporation.}

\article{Valeurs de l’AGEG à propos de l’élection du comité exécutif passif et du conseil d'administration}
\valeurs
{Intégrité}{Assurer une sélection impartiale des dirigeants de l'AGEG}
{Fraternité}{Favoriser un sentiment d’appartenance des membres face à leur association}
{Engagement}{Fournir aux membres la possibilité d’acquérir des qualités professionnelles grâce à l’implication}
{Ouverture}{Faire preuve de démocratie dans l’élection des dirigeants de l'AGEG}

\article{Rôles et pouvoirs}
\sousarticle{CA}
\alinea{Reçoit et entérine le rapport du président d'élection. Il est à noter que les candidats à l'élection ne sont pas en conflit d'intérêt lors de l'entérinement du rapport puisque le CA ne vote que sur les procédures d'élection et non sur les candidats.}

\sousarticle{CE}
\alinea{Désigne le directeur d’élection.}

\sousarticle{Directeur d’élection}
\alinea{Détermine les dates de la présente période électorale.}
\sousalinea{Si, pour quelque raison que ce soit, les dates de la période électorale ne peuvent pas respecter les barèmes fixés dans le présent règlement, le directeur d'élection devra faire approuver son calendrier électoral par le comité exécutif actif.}
\alinea{Prépare la plateforme de vote, dans le cas d'un vote électronique.}
\alinea{Reçoit les candidatures, prépare les bulletins de vote et veille à la bonne marche de l'élection, le tout en conformité avec les dispositions du présent règlement.}
\alinea{Établit et publicise les modalités du vote.}
\alinea{S’adjoint de scrutateurs le jour de l’élection, seulement si le vote doit être mené sous format papier.}
\alinea{Favorise la participation au scrutin à l’aide de publicité durant la période de scrutin.}
\alinea{Dépose un rapport d’élection au dernier CA de la session durant laquelle se tient le scrutin, lequel n'inclut aucun résultat de candidats.}
\alinea{Dépose un document indiquant le résultat du vote au dernier CA de la session durant laquelle se tient le scrutin, uniquement à la suite de l'entérinement du rapport par le CA.}
\alinea{Publicise les résultats du scrutin après l'entérinement par le CA.}
\alinea{Doit être étudiant ou ancien étudiant de l'Université de Sherbrooke et ne peut être candidat.}

\sousarticle{Scrutateur}
\alinea{S’occupe du scrutin lors du jour de l’élection dans le cas de votes en format papier.}
\alinea{Doit être étudiant ou ancien étudiant de l'Université de Sherbrooke et ne peut être candidat.}

\sousarticle{Candidat au comité exécutif passif}
\alinea{Doit être membre régulier de la corporation.}
\alinea{Doit être membre régulier actif de la corporation à la session pour laquelle il se présente.}
\alinea{Doit remplir son formulaire de mise en candidature. Le formulaire doit être électronique.}

\sousarticle{Candidat au conseil d'administration}
\alinea{Doit être membre régulier de la corporation.}
\alinea{Doit être membre régulier de la corporation à la période pour laquelle il se présente.}
\alinea{Doit remplir son formulaire de mise en candidature. Le formulaire doit être électronique.}
\alinea{Un candidat souhaitant se présenter à titre d'administrateur annuel, peut aussi se présenter administrateur saisonnier, dans le cas où il ne soit pas élu. Les administrateurs annuels sont donc élu avant les administrateurs saisonniers.}
\sousalinea{Si le candidat est exclut de la course faute de la confiance des électeurs, le candidat est aussi exclut des autres postes où il avait appliqué.}

\partie{Fonctionnement de l’élection}
\article{Lancement de la période d’élection}
\alinea{Le comité exécutif désigne un directeur d’élection.}
\alinea{Le directeur d’élection détermine les dates de la présente campagne électorale.}

\article{Mise en candidature }
\alinea{Tous les candidats doivent remplir le formulaire de mise en candidature.}
\alinea{Le directeur d'élection recevra les candidatures des candidats.}
\alinea{La période de mise en candidature doit durer un minimum de cinq (5) jours ouvrables.}
\sousalinea{Si possible, la période de mise en candidature doit englober la soirée d'implication de l'AGEG. Il est recommandé de terminer la période de mise en candidature avec la fin de la soirée d'implication de l'AGEG.}

\article{Campagne électorale}
\alinea{La campagne électorale débute à la fin de la période de mise en candidature et se termine la veille du jour du scrutin.}
\alinea{Les candidatures seront publiées par le président d'élection dès le début de la période de campagne électorale.}
\alinea{Pour les postes d'exécutif passif où aucune candidature n'aura été présentée à la fin de la période de mise en candidature régulière, la période de mise en candidature sera prolongée pendant toute la campagne électorale pour ce poste uniquement.}
\alinea{Pour les postes d'administrateur où il n'y a pas suffisamment de candidature pour combler tous les postes à la fin de la période de mise en candidature régulière, la période de mise en candidature sera prolongée pendant toute la campagne électorale, pour ces postes uniquement.}
\alinea{Dans le cas d'une prolongation de la période de mise en candidature, tous les candidats seront traités équitablement, peu importe le moment du dépôt de la candidature.}

\article{Période de scrutin}
\alinea{Les jours suivants la campagne électorale sont destinés au scrutin.}
\sousalinea{Si possible, une période de présentation des candidats aura lieu le premier jour du scrutin.}
\alinea{Pendant la période de scrutin, la seule publicité permise sera celle prévue par le directeur d'élection. Elle devra favoriser la participation au scrutin.}
\alinea{La période de scrutin doit être avant la fin des cours de la présente session.}
\alinea{Pour un vote électronique, la période doit durer un minimum de 24 heures consécutives. Il est recommandé que la période de scrutin soit de 48 heures.}

\sousarticle{Méthode de scrutin}
\alinea{La méthode de scrutin par défaut est le scrutin universel.}


\sousarticle{Élection du comité exécutif passif}
\alinea{Chaque poste constitue une élection à part entière. Les électeurs ont la possibilité de voter une seule fois par poste. En plus des candidats, l'électeur peut s'abstenir ou voter pour la chaise (laisser le poste vacant).}

\sousarticle{Élection du conseil d'administration}
\alinea{Tous les candidats sont réputés appliquer pour le même poste.}
\alinea{L'élection contient un vote de confiance implicite.}
\alinea{Les électeurs placent les candidats en ordre de préférence de 1 à n, où n est le nombre de candidats.}
\alinea{L'électeur peut attribuer une cote de non-confiance, ou 0, à un ou plusieurs candidats dont l'électeur n'a pas confiance.}
\alinea{Si plus du tiers des électeurs attribue la cote de 0 à un candidat, le candidat sera retiré de la course, faute de confiance des électeurs.}
\alinea{Chaque position des candidats équivaut à un nombre de point. Le candidat à la première position reçoit un nombre de point équivalent au nombre de candidats. Les candidats subséquents reçoivent un point de moins par position subséquente. }


\article{Résultat de l’élection}
\alinea{Un candidat au comité exécutif passif sera élu s'il récolte la majorité simple des voix.}
\alinea{Les candidats élus sont les candidats ayant reçu le plus de point.}
\sousalinea{En cas d'égalité, le candidat ayant reçu le moins de cote 0 sera élu.}
\alinea{Le président d’élection doit publiciser les résultats du scrutin, après l'entérinement du rapport par le CA.}
\alinea{Le président d’élection doit déposer un rapport au dernier CA de la session durant laquelle se tient le scrutin.}

\article{Litiges}
\alinea{En cas de litige, le directeur d'élection déterminera les procédures à suivre.}

\adoption{3 novembre 2019}{\add{À venir}}