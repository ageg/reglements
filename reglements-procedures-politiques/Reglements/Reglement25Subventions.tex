\reglement{Relatif aux subventions attribuées par l’AGEG aux groupes techniques}

\preambule{L’objectif du présent règlement est de donner des lignes directrices au comité d’attribution des subventions pour uniformiser les subventions données aux groupes.}
\partie{Dispositions générales}
\article{Mission de l’attribution des subventions de l’AGEG}
\alinea{Permettre aux groupes d’obtenir des subventions pour leurs activités et ainsi promouvoir l’implication au sein de la communauté étudiante et le rayonnement du génie à l’Université de Sherbrooke au travers de ces activités.}

\article{Vision de l’attribution des subventions de l’AGEG}
\alinea{La vision de l’attribution des subventions de l’AGEG est d’assurer la continuité des activités du plus grand nombre de groupes techniques au sein de la faculté.}

\article{Valeurs de l’attribution des subventions de l’AGEG}
\valeurs
{Ouverture}{Faire preuve de démocratie dans l’attribution des subventions de l’AGEG}
{Engagement}{Fournir aux membres la possibilité de subventionner leurs activités parascolaires}
{Intégrité}{Assurer une attribution honnête et équitable des subventions de l’AGEG}
{Fraternité}{Favoriser un sentiment d’appartenance des membres face à la Faculté et à l’Université}

\article{Rôles et pouvoirs}
\sousarticle{AGEG}
\alinea{Elle forme le comité d’attribution des subventions lors du CA1 de la session d’hiver.}
\alinea{Elle vote sur les recommandations du comité d’attribution des subventions suite à la réception du rapport du VPAX de l’AGEG au sujet des subventions à l’hiver.}
\alinea{Elle vote sur les recommandations du comité d’attribution des subventions suite à la réception du rapport du VPAX de l’AGEG au sujet de la réévaluation des dossiers à l’automne.}

\article{VPAF de l’AGEG}
\alinea{Il siège le comité d’attribution des subventions.}
\alinea{Il prépare et remet les chèques de subventions.}
\alinea{Il peut gérer les sommes du fonds déposées dans le compte bancaire de l’AGEG et effectuer des placements dans le meilleur intérêt des subventions. Ces placements ne doivent comporter aucun risque et les sommes doivent être disponibles en tout temps.}

\article{VPAX de l’AGEG}
\alinea{Il préside le comité d’attribution des subventions}
\alinea{Il est responsable de publiciser les formulaires de demandes ainsi que les modalités d’attributions des subventions auprès des membres.}
\alinea{Il organise les séances d’interview en vue des attributions de subventions.}
\alinea{Il recueille les demandes avant la date butoir à la session d’hiver.}
\alinea{Il recueille les bilans des différents groupes subventionnés avant la date butoir à la session d’automne.}
\alinea{Il dépose les recommandations du comité au sujet des subventions au CA sous forme d’un rapport détaillé au plus tard lors du CA3 de la session d’hiver.}
\alinea{Il dépose les recommandations du comité au sujet de la réévaluation des dossiers au CA sous forme d’un rapport détaillé au plus tard au CA2 de la session d’automne.}

\article{Comité d’attribution des subventions}
\alinea{Il assiste aux séances d’entrevue des groupes qui ont fait la demande d’une subvention.}

\article{Groupes technique}
\alinea{Ils doivent remettre leur demande de subvention avant la date limite fixée par le VPAF de l’AGEG à la session d’hiver.}
\alinea{Ils doivent être présents à la séance d’entrevue selon les modalités fixées par le comité d’attribution des subventions.}
\alinea{Ils sont tenus responsables de remettre un bilan financier, une liste mise à jour des outils et produits dangereux qu’ils utilisent, le rapport d’évaluation de sécurité des outils et du local, la liste des rapports d’incidents, le nom du coordonnateur SSE et un bilan des activités avant la date butoir fixée par le VPAF de l’AGEG à la session d’automne.}

\partie{Comité d’attribution des subventions}
\article{Composition du comité}
\sousarticle{Membres du comité}
\alinea{Le comité est formé de 3 étudiants du 1er cycle nommés par le CA ainsi que du VPAF et du VPAX de l’AGEG active.}

\sousarticle{Rémunération}
\alinea{Les membres du comité ne recevront aucune rémunération pour leur fonction.}

\article{Nomination et démission}
\sousarticle{Nomination des membres du comité}
\alinea{La durée du mandat des membres du comité est d’une année, du début de la session d’hiver à la fin de la session d’automne.}
\alinea{La priorité sera donnée aux étudiants ayant siégé sur le comité à l’année précédente.}

\sousarticle{Démission}
\alinea{Toute démission de tout membre du comité devra parvenir par écrit au VPAX de l’AGEG et sera effective après son approbation.}
\alinea{Un membre du comité peut se voir expulsé du comité par le CA de l’AGEG ou l’assemblée générale de l’AGEG pour des raisons jugées valables.}
\alinea{Un poste laissé vacant sera comblé par un étudiant du 1er cycle nommé par le VPAX de l’AGEG.}

\article{Décision}
\sousarticle{Critères}
\alinea{Les critères d’attribution permettent au comité de subventions d’évaluer et d’analyser les demandes afin de soumettre les meilleures recommandations au conseil d’administration.}
\alinea{Les critères sont les suivants : Le groupe technique doit :}
\sousalinea{Se définir comme un groupe;}
\sousalinea{Avoir un besoin de soutien financier;}
\sousalinea{Assurer une saine gestion financière;}
\sousalinea{Être ouvert aux membres (nombre de membres appartement au groupe et diversité au sein du groupe (concentration, promotion));}
\sousalinea{Favoriser l’apprentissage;}
\sousalinea{Favoriser le sentiment d’appartenance à la Faculté de génie;}
\sousalinea{Participer à diverses activités facultaires organisées par l’AGEG;}
\sousalinea{Contribuer au rayonnement de la faculté à l’extérieur par des actions concrètes (compétitions, actions à l’international, aide humanitaire, dimensions humaines en génie, média, etc.);}
\sousalinea{Se conformer au règlement 46 en matière de santé et sécurité étudiante.}
\sousalinea{Se conformer au règlement 50 sur les finances des groupes étudiants et des promotions.}

\sousarticle{Répartition des fonds disponibles}
\alinea{Le solde disponible pour les subventions est établi en date du 1er janvier de chaque année.}
\alinea{La totalité du fonds des groupes étudiants doit être attribuée à chaque année selon les modalités fixées par le comité d’attribution des subventions.}
\alinea{Le solde du fonds des nouveaux groupes étudiants est conservé, et ce, année après année.}
\alinea{15~\% des entrées des fonds doivent être réservés pour des subventions aux nouveaux groupes jusqu’à concurrence d’un solde du fonds des nouveaux groupes à 5000~\$.}
\alinea{Le comité de subvention doit considérer qu’un maximum de 5000~\$ est disponible en tout temps pour chacun des nouveaux groupes étudiants, et ce, indifféremment du nombre de demandes reçues.}
\alinea{Le fonds des nouveaux groupes étudiants peut être déficitaire, jusqu’à un maximum de 10 000~\$, après quoi le CA devra statuer sur l’attribution de fonds supplémentaires.}

\sousarticle{Quorum}
\alinea{Pour toute assemblée officielle du comité, il est nécessaire d’avoir la présence de tous les membres. Aucune décision ne sera prise à moins que cette assemblée n’ait le quorum requis.}
\alinea{Les décisions du comité se prendront par vote à majorité. Chaque membre, incluant le VPAF et le VPAX de l’AGEG, a droit à un vote.}

\article{Attribution}
\sousarticle{Attribution des fonds pour les groupes étudiants}
\alinea{Le VPAX de l’AGEG, suite à la décision du CA, informe les demandeurs de la subvention accordée.}
\alinea{Une somme égale à 75~\% du montant accordé par résolution du conseil d’administration sera versée avant la fin de la session d’hiver.}
\alinea{Une somme égale à 25~\% du montant accordé par résolution du conseil d’administration sera versée avant la fin de la session d’automne suite à la réévaluation du dossier du demandeur si et seulement si le bilan a été remis avant la date butoir et si son budget à jour a été fourni à chaque session.}

\sousarticle{Attribution des fonds spécifique pour les nouveaux groupes étudiants}
\alinea{Lors de la création d’un nouveau groupe, le CA pourra, selon les demandes faites par le groupe,  avancer un montant provenant du fonds de transition qui permettra au groupe de démarrer, jusqu’à concurrence du maximum attribuable par le comité d’attribution pour un nouveau groupe.}
\alinea{Le groupe devra faire une demande au prochain comité d’attribution des subventions.}
\alinea{Le montant avancé par le CA sera déduit du montant accordé par le comité d’attribution et remboursera le prêt fait via le fonds de transition.}
\alinea{Dans le cas où la somme remise par le comité d’attribution est inférieure au prêt, le montant excédentaire pourra être conservé par le groupe jusqu’à l’année suivante, où le montant sera de nouveau déduit de la subvention accordée.}

\partie{Fonds de subventions}
\article{Solde de l’année antérieure}
\alinea{Les montants non distribués avant la session de réception des demandes seront dilués dans le solde disponible.}

\article{Ristourne de la finissante}
\alinea{La promotion finissante doit verser une somme équivalente à 5~\% des profits générés lors des "Jeudi détente" sur toute la période où elle a eu la responsabilité d’organiser les "Jeudi détente". Cette somme sera versée au Fonds de subventions de l’AGEG lors de la passation des pouvoirs à la session d’automne.}

\article{Autres sources}
\alinea{Le CA ou l’AG peuvent attribuer des montants au Fonds sur résolution.}
\alinea{D’autres sources externes peuvent provenir d’ententes avec de nouvelles instances.}

\adoption{7 avril 2019}{27 juin 2019}