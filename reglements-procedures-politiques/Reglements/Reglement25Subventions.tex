\reglement{Relatif aux subventions attribuées par l’AGEG aux groupes techniques}

\preambule{L’objectif du présent règlement est de donner des lignes directrices au comité d’attribution des subventions pour uniformiser les subventions données aux groupes. À l'automne 2019, une mise à jour eu lieu afin de fusionner le Fonds d'équipement étudiant et le Fonds de subventions.}
\partie{Dispositions générales}
\article{Mission de l’attribution des subventions de l’AGEG}
\alinea{Permettre aux groupes d’obtenir des subventions pour leurs activités et leurs équipements pour ainsi promouvoir l’implication au sein de la communauté étudiante et le rayonnement du génie à l’Université de Sherbrooke au travers de ces activités.}

\article{Vision de l’attribution des subventions de l’AGEG}
\alinea{La vision de l’attribution des subventions de l’AGEG est d’assurer la continuité des activités du plus grand nombre de groupes techniques au sein de la faculté.}

\article{Valeurs de l’attribution des subventions de l’AGEG}
\valeurs
{Ouverture}{Faire preuve de démocratie dans l’attribution des subventions de l’AGEG}
{Engagement}{Fournir aux membres la possibilité de subventionner leurs activités parascolaires}
{Intégrité}{Assurer une attribution honnête et équitable des subventions de l’AGEG}
{Fraternité}{Favoriser un sentiment d’appartenance des membres face à la Faculté et à l’Université}

\article{Rôles et pouvoirs}
\sousarticle{CA de l'AGEG}
\alinea{Il nomine le directeur subvention au CA1 de la session d'automne.}
\alinea{Il vote sur les recommandations du comité d’attribution des subventions suite à la réception du rapport du directeur subvention au sujet des subventions à l'automne et à l’hiver.}

\article{VPAF de l’AGEG}
\alinea{Il siège le comité d’attribution des subventions.}
\alinea{Il prépare et remet les chèques de subventions.}
\alinea{Il est responsable de s'assurer d'obtenir les entrées au fonds.}
\alinea{Il doit assurer l'intérim jusqu'à ce qu'un directeur soit nommé.}

\article{Directeur subvention}
\alinea{Il préside le comité d’attribution des subventions.}
\alinea{Il est responsable de publiciser les formulaires de demandes ainsi que les modalités d’attributions des subventions auprès des membres.}
\alinea{Il est responsable d'organiser les séances d'entrevue des groupes.}
\alinea{Il rédige un rapport détaillé des recommandations du comité d'attribution du Fonds de subventions.}
\alinea{Il rédige les procès-verbaux des réunions du comité.}
\alinea{Il est responsable de faire le lien entre le VPAF, le VPAX, le comité d'attribution du fonds de subvention et les demandeurs.}
\alinea{La durée du mandat du directeur débute à sa nomination et se termine à la fin de la session d'été.}

\article{Comité d’attribution des subventions}
\alinea{Il reçoit les demandes et les dossiers de candidatures de chaque groupe faisant une demande au Fonds de subventions.}
\alinea{Il assiste aux séances d’entrevue des groupes qui ont fait la demande d’une subvention.}
\alinea{Il détermine la somme à répartir pour la période d'attribution en cours à partir des sommes disponibles dans le Fonds de subventions.}
\sousalinea{Il est recommandé de prévoir environ la moitié du budget annuel à chaque période d'attribution.}
\alinea{Il évalue les demandes reçues selon les critères du présent règlement.}
\alinea{Il rédige des recommandations sur l'attribution des fonds selon les critères du présent règlement.}
\alinea{Il peut recommander d'imposer des conditions particulières aux groupes sur les montants financés.}


\article{Groupes techniques}
\alinea{Ils sont tenus responsables de remettre un bilan financier, la liste des rapports d’incidents, le nom du coordonnateur SSE et un bilan des activités}
\alinea{Ils doivent remettre leur demande de subvention avant la date limite fixée par le directeur subvention à la session d'automne et d'hiver.}
\alinea{Ils doivent être présents à la séance d’entrevue selon les modalités fixées par le comité d’attribution des subventions.}
\alinea{Un nouveau groupe technique peut faire une demande au Fonds de subvention.}

\article{Délégations de l’AGEG aux compétitions québécoises de la CRÉIQ}
\alinea{La délégation pour les Jeux de génie et la délégation de la Compétition québécoise d'ingénierie peuvent faire une demande au Fonds de subvention.}
\alinea{La séance d'attribution de l'hiver permet de financer la délégation de l'AGEG à la Compétition canadienne d'ingénierie. Il est recommandé de financer le frais d'inscription et une partie des frais de transport si l'avion est nécessaire.}

\article{Demande extraordinaire}
\alinea{Tout groupe technique, groupe de l'AGEG, le Studio de création ou l'AGEG peut faire une demande au comité d'attribution du fonds de subvention}
\alinea{Le comité d'attribution établit les critères pour les demandes extraordinaires}
\alinea{On compte parmi ce genre de demande, des  demandes d'équipements pouvant bénéficier à plus d'un groupe technique}

\partie{Comité d’attribution des subventions}
\article{Composition du comité}
\sousarticle{Membres du comité}
\alinea{Le comité est formé de 3 membres de l'AGEG , du VPAX de l'AGEG, du VPAF de l'AGEG et du directeur subvention de l'AGEG. Ce dernier préside le comité.}

\sousarticle{Rémunération}
\alinea{Les membres du comité ne recevront aucune rémunération pour leur fonction.}
\article{Nomination et démission}

\sousarticle{Nomination des membres du comité}
\alinea{Après sa nomination, le directeur subvention à la responsabilité de trouver les membres de l'AGEG pour former le comité à l'automne. Le mandat des membres du comité est pour la session en cours.}
\alinea{La priorité sera donnée aux étudiants ayant siégé précédemment sur le comité.}

\sousarticle{Démission}
\alinea{Toute démission de tout membre du comité devra parvenir par écrit au directeur subvention de l’AGEG et sera effective après son approbation.}
\alinea{Un membre du comité peut se voir expulsé du comité par le CA de l’AGEG ou l’assemblée générale de l’AGEG pour des raisons jugées valables.}
\alinea{Un poste laissé vacant sera comblé par un membre de l'AGEG nommé par le directeur subvention.}
\alinea{Dans le cas de la démission du directeur subvention, le VPAF de l'AGEG assurera l'intérim jusqu'à ce que le directeur soit remplacé.}

\article{Décision}
\sousarticle{Critères}
\alinea{Les critères d’attribution permettent au comité de subventions d’évaluer et d’analyser les demandes afin de soumettre les meilleures recommandations au conseil d’administration.}
\alinea{Les critères sont les suivants : Le groupe technique doit :}
\sousalinea{Se définir comme un groupe;}
\sousalinea{Avoir un besoin de soutien financier;}
\sousalinea{Assurer une saine gestion financière;}
\sousalinea{Être ouvert aux membres (nombre de membres appartement au groupe et diversité au sein du groupe (concentration, promotion));}
\sousalinea{Favoriser l’apprentissage;}
\sousalinea{Favoriser le sentiment d’appartenance à la Faculté de génie;}
\sousalinea{Participer à diverses activités facultaires organisées par l’AGEG (portes ouvertes);}
\sousalinea{Contribuer au rayonnement de la faculté à l’extérieur par des actions concrètes (compétitions, actions à l’international, aide humanitaire, dimensions humaines en génie, média, etc.);}
\sousalinea{Se conformer au règlement 46 en matière de santé et sécurité étudiante.}
\sousalinea{Se conformer au règlement 50 sur les finances des groupes étudiants et des promotions.}

\sousarticle{Quorum}
\alinea{Pour toute assemblée officielle du comité, il est nécessaire d’avoir la présence de tous les membres. Aucune décision ne sera prise à moins que cette assemblée n’ait le quorum requis.}
\alinea{Les décisions du comité se prendront par vote à majorité. Chaque membre, incluant le directeur subvention, le VPAF et le VPAX de l’AGEG, a droit à un vote.}

\partie{Fonds de subventions}
\article{Solde antérieur}
\alinea{Les montants non distribués avant la session de réception des demandes seront dilués dans le solde disponible.}

\article{Ristourne de la finissante}
\alinea{La promotion finissante doit verser une somme équivalente à 5~\% des profits générés lors des "Jeudi détente" sur toute la période où elle a eu la responsabilité d’organiser les "Jeudi détente". Cette somme sera versée au Fonds de subventions de l’AGEG lors de la passation des pouvoirs à la session d’automne.}


\article{Subventions pour les délégations de l'AGEG aux compétitions québécoises de la CRÉIQ}
\alinea{Au CA1 de l'automne, un montant de 5000\$ est transféré du Fonds d'administration vers le Fonds de subvention afin de contribuer au financement des délégations de l'AGEG aux compétitions québécoises de la CRÉIQ. Le comité peut attribuer une subvention différente de ce montant.}

\alinea{Au CA1 de l'hiver, un montant de 3000\$ est transféré du Fonds d'administration vers le Fonds de subvention afin de contribuer au financement de la délégation de l'AGEG à la Compétition canadienne d'ingénierie. Le comité peut attribuer une subvention différente de ce montant.}

\article{Autres sources}
\alinea{Le CA ou l’AG peuvent attribuer des montants au Fonds sur résolution.}
\alinea{D’autres sources externes peuvent provenir d’ententes avec de nouvelles instances.}

\article{Restitution des fonds}
\alinea{L'argent attribuée par le comité d'attribution qui n'est pas dépensée par un groupe en date du 31 décembre chaque année sera restituée au 1er janvier.}

\adoption{24 novembre 2019}{\add{(À venir)}}                                   