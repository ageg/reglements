\reglement{Relatif à la gestion des ressources humaines de l’AGEG}

\preambule{L’objectif du présent règlement est de :
\begin{enumerate}
\item Uniformiser la gestion des ressources humaines de l’AGEG
\item Assurer un suivi et un contrôle efficace des ressources humaines
\end{enumerate}}

\partie{Dispositions générales}
\article{Mission du présent règlement}
\alinea{Avoir des lignes directrices claires en matière de ressources humaines.}
\alinea{Uniformiser la gestion de nos ressources humaines.}
\alinea{Procurer une façon d’évaluer la performance de nos ressources humaines.}

\article{Vision de l’AGEG par rapport la gestion des ressources humaines}
\alinea{La vision de la gestion des ressources humaines de l’AGEG est de mettre sur papier les pratiques déjà en cours et aussi d’ajouter des lignes directrices quant à la validation des performances de nos employés, à la sélection équitable des candidats et candidates aux postes ouverts.}

\article{Valeurs et rôles de la gestion des ressources humaines}
\valeurs
{Ouverture}{Promouvoir le dialogue entre l’exécutif et les ressources humaines de la corporation}
{Engagement}{Fournir à l’exécutif une structure efficace pour aider la gestion des employés}
{Fraternité}{Augmenter le sentiment d’appartenance de nos employés à la corporation}
{Intégrité}{Promouvoir la sélection équitable de nos employés}

\article{Rôles et pouvoirs}
\sousarticle{Conseil d’administration de l’AGEG}
\alinea{Adopte les contrats de travail ou les modifications au contrat de travail des employés.}
\alinea{Approuve l’embauche des employés, à l’exception des postes spécifiquement délégués au conseil exécutif.}

\sousarticle{Comité exécutif de l’AGEG}
\alinea{Prépare les contrats de travail. Les contrats de travail doivent contenir une clause exigeant le respect de la présente règlementation.}
\alinea{Approuve l’embauche des préposés à l’accueil.}

\sousarticle{Président de l’AGEG}
\alinea{Recueille les candidatures et choisit en collaboration avec le VPEC les candidats et candidates retenus.}
\alinea{Passe les candidats en entrevue avec la collaboration avec le VPEC.}

\sousarticle{VPAL de l’AGEG}
\alinea{Approuve les heures du coordonateur adminstratif et de l'adjoint administratif.}
\alinea{Assure le respect des normes du travail.}
\alinea{Assure le prélèvement et le paiement des retenues et cotisations gouvernementales.}
\alinea{Archive les évaluations des employés.}

\sousarticle{VPEC de l’AGEG}
\alinea{Recueille les candidatures et choisit en collaboration avec le président les candidats et candidates retenus.}
\alinea{Envoie la convocation d’entrevues aux candidats sélectionnés.}
\alinea{Passe les candidats en entrevue avec la collaboration du président.}
\alinea{Réalise les évaluations des employés et produit les rapports associés à chaque ressource humaine.}
\alinea{S'assure de la gestion des préposés à l'acceuil ainsi que de leur heures de travail.}

\sousarticle{Préposés à l’accueil de l’AGEG}
\alinea{Préparent les informations nécessaires à l’affichage des postes et réalise celui-ci.}

\partie{Définition des postes}
\article{Préposé à l’accueil}
\alinea{La personne préposée à l’accueil de l’AGEG est responsable d’accueillir les membres au local de l’association. Elle réalise diverses tâches de services aux membres. Elle se doit d’avoir de l’entregent, d’être polyvalente, d’avoir de l’initiative et d’aimer travailler avec le public. Finalement, elle doit avoir une très bonne maîtrise du français écrit.}

\article{Préposé à l’accueil senior}
\alinea{Le préposé à l’accueil senior est un préposé ayant réalisé au moins 2 sessions de travail au poste de préposé à l’accueil et dont les évaluations semestrielles attestent d'un leadership exemplaire. Des tâches supplémentaires sont attribuées à ce poste, tel que définies à l'article 3.2}

\article{coordonnateur administratif}
\alinea{Le coordonnateur administratif s'occupe de la tenu des livres comptables de l'AGEG, de l'archivage des documents importants et des relations avec les comptables de la corporation.}

\article{Adjoint administratif}
\alinea{L'adjoint administratif offre un poste de soutien au coordonnateur dans les différentes tâches au sein de la corporation}

\partie{Détails sur les fonctions des postes}
\article{Préposé à l’accueil}
\alinea{Assurer une présence au local}
\alinea{Rédiger et réviser des textes
\alinea{Mettre en page et corriger des documents}
\alinea{Préparer le café}
\alinea{Servir les membres}
\alinea{Faire des affiches à l’aide de Photoshop ou en carton}
\alinea{Garder le local de la corporation propre}
\alinea{Toutes autres tâches connexes}

\article{Préposé à l’accueil senior}
En plus des tâches du préposé à l’accueil, le préposé à l’accueil senior doit effectuer les tâches suivantes :
\alinea{Donner la formation aux autres préposés à l’accueil en début de session}
\alinea{Assurer la bonne coordination des autres préposés}
\alinea{Produire un rapport d’avancement en fin session}
\alinea{Toute autre tâche connexe}

\article{coordonnateur administratif}
\alinea{S'occupe de la mise à jour des livres comptables de la corporation.}
\alinea{S’occuper des relevés d’emploi des employés et entre les transactions dans le logiciel de gestion financière.}
\alinea{S'occupe de la paie des employés}
\alinea{Remplir les relevés de cessation d’emploi des employés.}
\alinea{Produire les rapports pour la Fondation FORCE à la fin de chaque semestre.}
\alinea{Produire les rapports de conciliation bancaire mensuellement.}
\alinea{S'assure que les liquidités et le logiciel de gestion financière balance.}
\alinea{Faire les ajustements nécessaires dans le logiciel de gestion lors de la vérification financière de fin d’année.}
\alinea{Produire les relevés gouvernementaux pour les employés.}
\alinea{Produire les rapports de la CSST.}
\alinea{Produire et envoyer l’état des résultats au gouvernement provincial et fédéral.}
\alinea{Donner la formation initiale au VPAF.}
\alinea{Produire mensuellement et trimestrielle le bilan financier de l'AGEG et de ses groupes étudiants.}
\alinea{Toutes autres tâches connexes.}

\article{Adjoint administratif}
\alinea{S'occupe de la mise à jour des livres comptables de la corporation.}
\alinea{Produire les rapports de conciliation bancaire mensuellement.}
\alinea{S'assure que les liquidités et le logiciel de gestion financière balance.}
\alinea{Faire les ajustements nécessaires dans le logiciel de gestion lors de la vérification financière de fin d’année.}
\alinea{Toutes autres tâches connexes.}

\partie{Échelle salariale}
\article{Préposé à l’accueil}
\alinea{Salaire de 15,00~\$ de l’heure.}

\article{Préposé à l’accueil senior}
\alinea{Salaire de 16,00~\$ de l’heure.}

\article{Coordonateur administratif}
\alinea{Le salaire et le contrat du Coordonatrice administrative}est défini à huis clos. Veuillez vous référer au document associé dans le coffre de la corporation.}

\article{Adjoint administratif}
\alinea{Le salaire et le contrat de l'adjoint administratif est défini à huis clos. Veuillez vous référer au document associé dans le coffre de la corporation.}

\partie{Évaluation des employés}
\article{Définition}
\alinea{L’évaluation des employés a pour but de conserver un dossier des performances de nos ressources humaines.}
\alinea{Il est nécessaire l’évaluation permette de juger de la qualité du travail de l’employé.}
\alinea{Les évaluations doivent être faites en utilisant le gabarit.}
\alinea{Les évaluations sont faites durant des rencontres individuelles.}

\article{Fréquence}
\alinea{La première partie de l’évaluation doit être faite tout juste après l’embauche de l’employé.}
\alinea{a deuxième partie doit être réalisée au plus court entre le mi-mandat et un semestre.}
\alinea{La troisième partie doit être réalisée tout juste avant la fin du mandat de l’employé.}
\alinea{S’il y a fin prématurée du contrat, la troisième partie de l’évaluation doit être réalisée et signée par l’employé.}
\alinea{La troisième partie de l’évaluation doit être réalisée avant le renouvellement du contrat ou de la période d’entrevue pour le même poste.}

\partie{Processus de sélection et démission}
\article{Sélection}
\alinea{L’affichage des postes doit se faire au moins 2 semaines avant l’entrée en fonction de l’employé.}
\alinea{Les préposés à l’accueil doivent avoir été sélectionnés avant le début de la session pour assurer leur entrée en fonction à la première journée ouvrable de la session.}
\alinea{Les préposés à l’accueil doivent être éligibles au programme études-travail de la fondation FORCE.}

\article{Démission, congédiement et licenciement}
\alinea{Toute démission de tout employé devra parvenir par écrit au CA de l’AGEG.}
\alinea{Entre 2 CA, le président de l’AGEG peut approuver la démission. Le CA devra entériner celle-ci.}
\alinea{Les normes du travail du Québec doivent être respectées durant ce processus.}
\alinea{Un employé peut se voir renvoyé par le CA de l’AGEG pour des raisons jugées valables.}
\alinea{Un avis de cessation d’emploi doit être produit et donné à l’employé.}
\alinea{Un poste laissé vacant devra être comblé par un nouvel employé le plus tôt possible.}

\adoption{19 février 2017}{27 juillet 2017}