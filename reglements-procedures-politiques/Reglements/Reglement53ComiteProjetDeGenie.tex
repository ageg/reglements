\reglement{Relatif au Comité Projets de Génie}

\preambule{L’objectif du présent règlement est de donner des lignes directrices au Comité Projets de Génie dans la gestion, l’attribution et le suivi des fonds et des activités en lien avec le comité.}

\partie{Dispositions générales}
\article{Vision du Comité Projets de Génie}
\alinea{La vision du comité est d’encourager la persévérance scolaire et de susciter l’intérêt pour les sciences et l’ingénierie chez les étudiants du primaire et du secondaire en offrant l’occasion aux groupes techniques de présenter leurs projets dans les écoles de la région ou en invitant les écoles à une conférence à l’université.}
\alinea{Promouvoir et impliquer les groupes techniques de l’AGEG dans la communauté.}
\alinea{Offrir une nouvelle source de financement aux groupes techniques qui se verront rémunérés par les commanditaires du comité en échange de leurs conférences.}

\article{Valeurs du Comité Projets de Génie}
\valeurs
{Ouverture}{Offrir aux groupes techniques une ouverture sur le monde de l’éducation}
{Engagement}{Encourager les groupes techniques à redonner à la communauté étudiante en dehors de l’Université}
{Intégrité}{Assurer une distribution honnête et juste des fonds et des opportunités d’offrir une conférence}
{Fraternité}{Favoriser un sentiment d’appartenance des groupes techniques au réseau scolaire de la région}

\article{Rôles et pouvoirs}
\sousarticle{CA de l’AGEG}
\alinea{Il entérine le plan d’action du comité au CA3 de l’hiver;}
\alinea{Il entérine le résultat des entrevues annuelles au CA3 de l’hiver;}
\alinea{Il entérine les prévisions budgétaires ainsi que le calcul des dons que recevront les groupes techniques selon la complexité de leur évènement tel que défini à l’Art. 3.4 du présent règlement, au CA3 de l’hiver.}

\sousarticle{VPAX de l’AGEG}
\alinea{Il rapporte les actions du comité au CA;}
\alinea{Il supervise le directeur Projets de Génie;}
\alinea{Il supervise le Comité Projets de Génie en l’absence de directeur.}

\sousarticle{Conseil exécutif de l’AGEG}
\alinea{Il entérine la composition du comité dans le respect de la charte de l’AGEG (article 8.5).}

\sousarticle{Directeur Projets de Génie}
\alinea{Il planifie les différentes réunions du Comité Projets de Génie;}
\alinea{Il veille au bon fonctionnement du comité;}
\alinea{Il s’occupe du recrutement des membres du comité.}

\sousarticle{Comité Projet de Génie}
\alinea{Il s’occupe de chercher des fonds, soit par des commanditaires ou autres pour financer les conférences;}
\alinea{Il définit les critères de sélection des groupes techniques qui pourront offrir une conférence;}
\alinea{Il sélectionne les groupes techniques ayant un contenu jugé suffisant pour meubler le temps de la conférence;}
\alinea{Il établit la rémunération par conférence selon le niveau de participation des commanditaires;}
\alinea{Il organise, avec les écoles de la région ainsi que les groupes techniques, les conférences.}


\partie{Comité Projets de Génie}
\article{Composition du comité}
\sousarticle{Membres du comité}
\alinea{La formation et la composition du comité est à la discrétion du directeur Projets de Génie, mais doit être entérinée par le CE.}

\sousarticle{Rémunération}
\alinea{Les membres du comité ne recevront aucune rémunération pour leur fonction.}

\article{Nomination et démission}
\sousarticle{Nomination du directeur}
\alinea{Le directeur du comité est nommé par le VPAX et doit être entériné par le CE de l’AGEG.}

\sousarticle{Nomination d’un représentant en commission}
\alinea{Advenant l’impossibilité pour le chef de délégation d’assister à une commission du congrès, il peut nominer un membre de sa délégation pour agir en tant que représentant de l’AGEG au sein de cette commission.}

\sousarticle{Nomination des membres du comité}
\alinea{Il est du rôle du directeur de s’occuper du recrutement, pour combler les besoins du comité;}
\alinea{Les membres n’ont pas nécessairement besoin de faire partie de l’AGEG.}
\alinea{Les critères de sélection seront à la discrétion du directeur.}

\sousarticle{Démission}
\alinea{a)	Toute démission de tout membre du comité devra parvenir par écrit au CE de l’AGEG et sera en vigueur après son approbation;}
\alinea{Un membre du comité peut se voir expulsé du comité par le CA de l’AGEG ou l’assemblée générale de l’AGEG pour des raisons jugées valables;}
\alinea{Un poste laissé vacant sera comblé par un étudiant du 1er cycle nommé par le CE;}
\alinea{Dans le cas de la démission du directeur du comité, le CE de l’AGEG devra nommer un membre du comité en tant que directeur intérimaire jusqu’à ce qu’un nouveau directeur soit choisi par le VPAX.}

\article{Décision}
\sousarticle{Critères}
\alinea{Le comité décide des activités qu’il veut organiser et des achats qu’il veut effectuer selon la mission, la vision et les valeurs décrites dans le présent règlement, ainsi que la mission, la vision et les valeurs véhiculées par l’AGEG et son plan directeur.}

\sousarticle{Quorum}
\alinea{a)	Le quorum est fixé à la majorité des membres du comité;}
\alinea{b)	Les décisions du comité se prendront par vote à majorité simple.}

\article{Gestion financière}
\sousarticle{Budget du comité}
\alinea{Provenant principalement de commanditaires, les fonds du comité seront distribués aux groupes techniques selon les ententes prises entre le comité et les commanditaires;}
\alinea{Le comité peut demander un soutien financier de l’AGEG pour l’achat ponctuel de matériel servant à leurs activités;}
\alinea{Le comité se doit de respecter les normes budgétaires de l’AGEG;}
\alinea{Le comité se doit de ne pas être déficitaire sans l’accord du CA de l’AGEG.}

\partie{Organisation des conférences}
\article{Financement}
\alinea{Le comité est responsable de trouver suffisamment de commanditaires pour pouvoir rémunérer adéquatement les groupes techniques pour leurs conférences.}

\article{Choix des écoles}
\alinea{Le comité est responsable de contacter les écoles de la région à tout moment de l’année selon les disponibilités de celles-ci et des groupes techniques intéressés;}
\alinea{Advenant un nombre d’écoles intéressées supérieure aux moyens financiers du comité ou à la disponibilité des groupes techniques, le comité aura le libre choix quant à la sélection des écoles qui seront visitées.}

\article{Rémunération des groupes techniques}
\alinea{À la session d’hiver, le comité doit établir le calcul servant à déterminer les dons que recevront les groupes techniques participant en fonction de la générosité des commanditaires, de la complexité de l’évènement, des priorités du comité, des commentaires des écoles;}
\alinea{Toutes modifications au calcul devront être adoptées en CA.}

\adoption{28 mai 2017}{27 juillet 2017}