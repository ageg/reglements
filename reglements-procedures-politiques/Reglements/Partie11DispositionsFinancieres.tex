\partie{Dispositions financières} 

\article{Année financière } 

\alinea{L'exercice financier de la corporation débute le 1er janvier et se termine le 31 décembre de la même année.} 

\article{Vérification} 

\alinea{Les livres et les états financiers de la corporation sont vérifiés chaque année, après l'expiration de l'exercice financier, par la personne ou firme vérificatrice interne nommée à cette fin lors de l'assemblée générale annuelle des membres.} 

\article{Livre de comptabilité} 

\alinea{Le conseil d'administration fait tenir par la personne qui exécute la coordination administrative ou par l'adjoint ou l'adjointe à l'administration de la corporation ou sous son contrôle, un ou des livres de comptabilité dans lequel ou lesquels sont inscrits tous les fonds reçus et déboursés par la corporation et toutes ses dettes ou obligations, de même que toute autre transaction financière de la corporation. Ce ou ces livres sont tenus au siège social de la corporation et sont ouverts à l'examen des personnes administratrices. En aucun temps ces livres ne pourront quitter le siège social de la corporation à moins que le conseil d'administration y consente.} 

\article{Effets bancaires} 

\alinea{Tous les chèques, billets et autres effets bancaires de la corporation sont signés par deux (2) des quatre (4) personnes élues aux postes suivants: 

\begin{itemize} 

 \item Présidence 

 \item Vice-Présidence aux Affaires Financières 

 \item Vice-Présidence aux Affaires Légales 

 \item Personne qui exécute la coordination administrative de la corporation 

\end{itemize} 

} 

\sousarticle{Compte à statut particulier} 

\alinea{Sous résolution du conseil d'administration, un compte bancaire peut être déclaré comme compte à statut particulier. Il est à noter que le compte principal de la corporation ne peut être déclaré comme compte à statut particulier.} 

\alinea{Les comptes à statut particulier peuvent avoir autant de signataires supplémentaires aux signataires du compte principal que le comité exécutif juge nécessaire. Ces signataires sont nommés par le conseil exécutif actif en début de session, et ce, pour toute la session du mandat de l'exécutif.} 

\article{Emprunts}
Le conseil d'administration doit approuver tout emprunt d'argent par la corporation par un vote à majorité simple.

\article{Prévisions budgétaires et états financiers}
\alinea{Les prévisions budgétaires sont préparées et adoptées au conseil d'administration avant le début de l'année financière à laquelle elles se rapportent.}
\alinea{Les dépenses et les revenus des groupes étudiants ne sont pas budgétés.}
\alinea{L’année financière doit avoir un budget de fonctionnement équilibré.} 
\alinea{Les prévisions budgétaires et les états financiers à jour de l'année en cours sont présentés à chaque séance régulière du conseil d'administration et lors d'au moins une assemblée générale par session.}
\alinea{Les états financiers de fin d'année sont adoptés en CA, puis lors de l'assemblée générale annuelle.}


\article{Fonds}

\alinea{La corporation entretient les fonds suivants pour remplir sa mission : 

\begin{itemize} 
 \item Fonds d'administration 
 \item Fonds d'immobilisations
 \item Fonds meubles
 \item Fonds de subvention des groupes étudiants
 \item Fonds de la direction
 \item Fonds de donation
 \item Fonds des finissantes
 \item Fonds d'équipement étudiant
\end{itemize} 
} 

\alinea{Les descriptions de ces fonds et les dispositions concernant l'utilisation et l'allocation des fonds est décrite dans le cahier de procédures.}

\article{Revenus et dépenses des groupes étudiants}
\alinea{Les dépenses et les revenus des groupes étudiants sont affectés à des projets dont le solde est reporté d'une année à l'autre. Le solde des projets de chacun des groupes étudiants ne peut être négatif à moins d'une approbation par le conseil d'administration. }

\article{Frais de déplacement}
\alinea{Les frais de déplacement des employées et employés et des membres du CE ou du CA agissant pour le compte de la corporation sont remboursés selon le taux fixé par le CA. À défaut d'un taux fixé par le CA, le taux raisonnable par kilomètre suggéré par le gouvernement du Canada sera utilisé.}

\alinea{Les groupes étudiants peuvent faire une demande afin de rembourser les frais de déplacement encourus pour le compte du groupe par un de leur membre. Il est suggéré d'utiliser le taux décidé par le CA, mais le groupe peut autoriser un taux différent.}

\article{Approbation des dépenses}

\alinea{Les dépenses appliquées sur un des fonds de la corporation sont autorisées par le conseil d'administration a moins d'une disposition dans le cahier de procédures spécifiant des modalités contraires.}

\alinea{Les dépenses des groupes étudiants sont approuvées par la personne à la présidence et à la trésorerie du groupe, puis par la vice-présidence aux affaires financières ou, en cas de vacance du poste, par la vice-présidence aux affaires extracurriculaires ou par la présidence de l'AGEG.}
\sousalinea{Les personnes autorisant les dépenses des groupes sont responsables de s'assurer de ne pas autoriser des dépenses dépassant le solde du projet du groupe demandeur.}

\alinea{Les autres dépenses sont approuvées par la vice-présidence aux affaires financières et par la présidence de l'AGEG. En cas de vacance de l'un des postes, un substitut est nommé par le conseil exécutif parmi ses membres.}

\sousalinea{Les personnes autorisant les dépenses des groupes sont responsables de s'assurer de ne pas autoriser des dépenses dépassant plus de 5\% la prévision au poste budgétaire à laquelle la dépense sera assignée.}

\article{Immobilisation des biens}
\alinea{Les achats de meubles ou d'équipements dont la valeur pour un item unique est de plus de 2000\$ avant taxes sont amortis selon la période de vie utile de ces meubles et équipements.}

\alinea{Les achats d'équipement informatique dont la valeur pour un item unique est de plus de 750\$ avant taxes sont amortis selon la période de vie utile de ces meubles et équipements.}

\alinea{À moins de dispositions contraires par le conseil d'administration un amortissement linéaire est utilisé.}


\article{Contrats} 

\alinea{Les contrats et autres documents requérant la signature de la corporation sont au préalable approuvés par le conseil d'administration et, sur telle approbation, sont signés par la présidence ou l'une des vice-présidences.} 

 