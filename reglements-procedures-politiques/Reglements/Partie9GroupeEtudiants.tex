\partie{Groupes étudiants}

\article{Dispositions générales}
\alinea{L'AGEG est composée de plusieurs groupes étudiants dont les finances sont administrées par la corporation. Le rôle de l'AGEG est de permettre à ses membres une certaine indépendance pour le déroulement d'activités extracurriculaires, tout en s'assurant de la pérennité de celles-ci et de la sécurité de ses membres.}

\article{Types de groupes}
\alinea{Il existe trois (3) types de groupe reconnus par l'AGEG:
\begin{itemize}
\item Les promotions;
\item Les groupes de l'AGEG;
\item Les groupes techniques.
\end{itemize}
}

\article{Personnes dirigeantes des groupes étudiants}
\alinea{Chaque groupe reconnu doit nommer une personne à la présidence et une personne à la trésorerie et informer l'AGEG de tout changement à ces postes.}
\alinea{Le rôle de la présidence est de faire le lien entre l'AGEG et le groupe ainsi que de confirmer les transactions financières du groupe.}
\alinea{Le rôle de la trésorerie est de remplir les demandes de remboursement et les demandes de dépôt du groupe ainsi que de tenir à jour le budget du groupe.}

\article{Santé sécurité}
\alinea{Une liste des groupes réalisant des activités comportant des risques pour la santé et la sécurité des personnes participantes est tenue à jour par la vice-présidence aux activités extracurriculaires en collaboration avec l'Université.}
\alinea{Chaque groupe dans cette liste doit nommer une personne à la direction de la santé et sécurité du groupe qui doit suivre une formation particulière auprès de l'université. Cette personne est responsable de faire appliquer les règles de santé et de sécurité dans son groupe.}
\alinea{Chaque groupe dans cette liste doit fournir une liste de ses membres à la vice-présidence aux activités extracurriculaires et s'assurer que ces membres suivent la formation sur la santé et la sécurité demandées par l'AGEG et l'Université.}


\article{Rémunération}
\alinea{Les membres des groupes étudiants ne peuvent recevoir aucune forme de rémunération pour leurs fonctions.}

\section{Promotions}
\article{Définition d'une promotion}
\alinea{Une promotion est un groupe étudiant représentant les intérêts du contingent de personnes étudiantes ayant débutées leur baccalauréat à la même session d'automne.}
\alinea{Une promotion est libre d'accepter des membres ne répondant pas à ce critère pour accommoder les personnes au parcours académique atypique.}
\alinea{Le comité des cycles supérieurs représente toutes les personnes aux cycles supérieurs membres de l'AGEG. Ce comité est considéré comme une promotion au sens du présent règlement.}

\article{Création et dissolution d'une promotion}
\alinea{Une promotion est automatiquement créée au début de chaque session d'automne pour représenter les personnes étudiantes qui débutent leur baccalauréat à cette session. L'AGEG est en charge d'appuyer les nouvelles personnes étudiantes dans la formation d'un comité de promotion.}
\alinea{La promotion est active pendant treize (13) sessions à partir de sa création. Une session supplémentaire est laissée à la promotion pour conclure ses affaires auprès de l'AGEG, mais la promotion n'est pas réputée active pendant cette session de transition. }
\alinea{Le comité des cycles supérieurs est réputé être toujours en activité.}
\article{Nomination des personnes dirigeantes}
\alinea{Les personnes dirigeantes d'une promotion sont nommées selon les directives prévues à la charte de la promotion. En l'absence de charte, les personnes dirigeantes sont nommées par les membres de la promotion lors d'une assemblée générale selon les procédures indiquées dans le règlement associé dans le cahier de procédures.}


\section{Groupe de l'AGEG}
\article{Définition d'un groupe de l'AGEG}
\alinea{Un groupe de l'AGEG est un groupe étudiant dont l'existence est due à l'AGEG ou qui est déjà financé par le budget général de l'AGEG. (Ex.: Radio, Organisation CQI, Délégation Jeux de Génie, Intégration) Un groupe de l’AGEG est formé pour un objectif précis. Cet objectif est défini à la création du groupe et fait l'objet d'un règlement au cahier de procédures. Ce règlement fait aussi mention du fonctionnement du groupe et de ses obligations envers l'AGEG.}

\article{Création et dissolution d'un groupe de l'AGEG}
\alinea{Un groupe de l'AGEG est créé et dissous par résolution du conseil d'administration. Le  conseil d'administration est aussi responsable d'approuver le règlement du cahier de procédures relatif à ce groupe.}
\article{Nomination des personnes dirigeantes}
\alinea{Les personnes dirigeantes d'un groupes de l'AGEG sont nommées selon les procédures indiquées dans le règlement associé à ce groupe dans le cahier de procédures.}

\section{Groupes techniques}
\article{Définition d'un groupe technique}
\alinea{Un groupe technique est un groupe étudiant, formé par l’initiative des membres de l’AGEG, qui réalise un projet matériel ou logiciel dans le but de participer à une compétition ou groupe étudiant qui réalise un projet d'ingénierie à long terme.}

\article{Types de groupes techniques}
\sousarticle{Groupes techniques réguliers}
\alinea{Un groupe technique régulier est un groupe technique appartenant à l'AGEG. L'AGEG est responsable légalement et financièrement des groupes techniques réguliers.}
\sousarticle{Groupes techniques affiliés}
\alinea{Un groupe technique affilié est un groupe technique qui n'appartient pas à l'AGEG, notamment parce qu'il est incorporé lui-même ou qu'il appartient à une autre organisation.}

\article{Accréditation d'un groupe technique}
\alinea{Le conseil d'administration à le pouvoir d'accréditer comme groupe technique régulier ou affilié tous regroupements étudiants qui en font la demande.}
\alinea{Le conseil d'administration doit prendre en considération les critères suivants:}
\sousalinea{Le nombre de membres fondateurs;}
\sousalinea{Les buts du groupe;}
\sousalinea{La viabilité du groupe;}
\sousalinea{L'autonomie financière du groupe;}
\sousalinea{Le sérieux de la mise en candidature.}

\alinea{Le groupe demandeur doit fournir au conseil d'administration de l'AGEG les documents suivants:}
\sousalinea{Une liste des membres fondateurs;}
\sousalinea{Une lettre de motivation au conseil d'administration;}
\sousalinea{Une lettre d'appui d'un professeur de la faculté de génie;}
\sousalinea{Une description de leurs buts, des activités et des besoins du groupe;}
\sousalinea{Un budget préliminaire au conseil d'administration.}

\article{Entente entre les groupes techniques et l'AGEG}
\alinea{Chaque groupe doit signer une entente avec l'AGEG chaque session stipulant les attentes de l'AGEG envers le groupe.}
\alinea{Un groupe technique est considéré inactif s'il ne signe pas d'entente avec l'AGEG pour trois (3) sessions consécutives.}

\article{Dissolution d'un groupe technique}
\alinea{Il est du devoir du groupe étudiant dont le projet est terminé et dont la dissolution est imminente de le signaler à l'AGEG.}
\alinea{Les groupes techniques inactifs peuvent être dissous par une résolution du conseil d'administration de l'AGEG.}
\alinea{Un groupe technique peut être dissous sur demande de celui-ci ou si le conseil d'administration juge qu'il ne remplit pas les obligations stipulées dans l'entente entre le groupe et l'AGEG. Dans ce cas, l'AGEG doit avertir les personnes dirigeantes du groupe et leur permettre de corriger la situation ou de s'expliquer au conseil d'administration.}
\alinea{Les actifs d'un groupe technique régulier dissous peuvent être redistribués par le conseil d'administration.}

\article{Nomination des personnes dirigeantes}
\alinea{Les groupes techniques sont indépendants dans la nomination des personnes dirigeantes. L'AGEG peut nommer des personnes dirigeantes pour le compte d'un groupe technique régulier si celui-ci n'a pas signé d'entente avec l'AGEG pendant la session courante et si celui-ci ne semble pas avoir de membres actifs.}

\article{Exigences particulières}
\alinea{Nonobstant les clauses du présent règlement, l'AGEG se réserve le droit d'ajouter toutes exigences particulières qu'elle juge nécessaires pour remplir ses obligations et sa mission dans une entente avec un groupe technique.}