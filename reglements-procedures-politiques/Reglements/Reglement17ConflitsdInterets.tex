\reglement{Relatif aux conflits d’intérêts}

\preambule{Le présent règlement vise à indiquer la marche à suivre pour le conseil d’administration dans le cas où l’une personne membre serait en conflit d’intérêts ou en apparence de conflit d’intérêts, afin d’assurer et de maintenir entière la crédibilité du conseil face à ses membres et à assurer le désintérêt total des personnes administratrices.}

\partie{Dispositions générales}
\article{Mission du règlement relatif aux conflits d’intérêts}
\alinea{De donner un cadre pour traiter les situations de conflit d’intérêts ou d’apparence de conflit d’intérêts;}
\alinea{Maintenir l’entière crédibilité du conseil d’administration face à ses membres;}
\alinea{Assurer l’absence totale de conflit d’intérêts des personnes administratrices lors des délibérations.}

\article{Vision du règlement relatif aux conflits d’intérêts}
\alinea{Permettre au conseil d’administration de traiter les cas de conflits d’intérêts ou d’apparences de conflits d’intérêts sans qu’ils se sentent personnellement visés lors des délibérations.}

\article{Valeurs du règlement relatif aux conflits d’intérêts}
\valeurs
{Ouverture}{Permettre à toutes les personnes administratrices de pouvoir exprimer leur opinion librement}
{Engagement}{Permettre de prendre la meilleure décision possible sur des questions qui touchent les membres}
{Intégrité}{Traiter toutes les questions relevantes du conseil d’administration avec équité et transparence}
{Fraternité}{Pouvoir assurer la délibération dans un esprit de collaboration}

\article{Rôles et pouvoirs}
\sousarticle{Personnes administratrices de l’AGEG}
\alinea{Elle est responsable de faire savoir rapidement à la présidence d’assemblée qu’elle est en conflit d’intérêts dès que la question est traitée;}
\alinea{Une personne administratrice en conflit d’intérêts peut alors donner son opinion et répondre aux questions;}
\alinea{Une personne administratrice en conflit d’intérêts doit sortir de la réunion pour le temps où se dérouleront les débats et les délibérations.}

\sousarticle{Présidence d’assemblée du CA de l’AGEG}
\alinea{Elle est le juge si une question fait état d’un conflit d’intérêts ou d’apparence de conflit d’intérêts;}
\alinea{Elle est responsable de faire appliquer le règlement lors des conseils d’administration.}

\sousarticle{Conseil d’administration de l’AGEG}
\alinea{Il possède toute l’autorité pour retirer le droit de vote et forcer le départ de la salle à une personne administratrice qui ne se conforme pas au présent règlement de lui-même.}
\alinea{Le conseil peut décider de tout autre mesure allant jusqu’à l’expulsion du conseil d’administration pour une personne administratrice qui aurait omis de signaler une situation de conflit d’intérêts ou d’apparence de conflit d’intérêts. La personne administratrice  visée  par  le  conflit  d’intérêts mentionné par l’une des personnes membres du conseil d’administration  de  l’AGEG  peut  faire  appel  et démontrer l’absence de conflit d’intérêts.}

\partie{Modalités d’application}
\article{Principe}
\alinea{Nulle personne administratrice ne peut voter sur une question concernant l’affiliation à un organisme externe ou concernant l’attribution de subsides ou avantages quelconques à des organismes ou groupes autant ceux de l’interne que de l’externe, s’elle a elle-même intérêt ou apparence d’intérêt dans l’organisme ou groupe visé ou dans l’un des organismes ou groupes visés par la question à l’étude;}
\alinea{Nulle personne administratrice ne peut voter sur une question qui aurait pour effet de lui octroyer directement ou indirectement un avantage financier quelconque en dehors des fonctions qu’elle exerce à l’AGEG, s’il y a lieu.}

\article{Nature}
\sousarticle{Situations conflictuelles types}
Sans restreindre la généralité de ce qui précède, est considérée comme situation conflictuelle type:
\alinea{L’attribution de subventions ou autres avantages aux groupes étudiants toute personne membre dudit groupe;}
\alinea{L’affiliation à un organisme externe, la modification de la cotisation à cet organisme ou l’octroi d’avantages quelconques à cet organisme toute personne membre de l’exécutif ou employé de cet organisme;}
\alinea{L’attribution d’un contrat ou la ratification d’une entente avec une société ou compagnie dont il est employé, actionnaire, dirigeant ou mandataire;}
\alinea{Toute autre situation jugée conflictuelle par la présidence d’assemblée du CA de l’AGEG.}

\sousarticle{Cas d’exception}
Il existe cependant quelques exceptions où les personnes administratrice peuvent se prononcer même si elle sont en conflit d’intérêts : 
\alinea{La défense d’une idéologie ou philosophie associée à un groupe étudiant ou à un organisme externe même si la personne est membre de ce groupe ;}
\alinea{Toute question où l’intérêt ou l’affiliation de la personne envers un organisme externe ou interne provient des fonctions qu’elle exerce pour et au nom de l’AGEG.}

\article{Quorum}
\sousarticle{Non-modification du statut du quorum}
\alinea{Le fait qu’une personne administratrice se retire de la réunion pour des raisons de conflit d’intérêts ou d’apparence de conflit d’intérêts ne peut modifier le statut du quorum évalué en début de réunion.}

\sousarticle{Cas extrême}
\alinea{Advenant le cas où seulement cinq personnes administratrice ou moins seraient aptes à voter en vertu du présent règlement, le conseil d’administration doit former un comité composé de huit personnes dont les personnes administratrices aptes à voter ainsi que d’autres membres de l’AGEG qui ne sont pas en conflit d’intérêts ou en apparence de conflit d’intérêts.}
\alinea{Ce comité se voit déléguer les pouvoirs nécessaires afin d’être en mesure de trancher la question.}

\sousarticle{Engagement}
\alinea{Les formulaires de mise en candidature pour les candidature au poste de personne administratrice doivent comporter l’engagement suivant, qui doit être remplis et signé afin qu'il soit valide :}
\quotation{Ma candidature au conseil d’administration a comme seul objectif le bénéfice de l’AGEG et de ses membres et je m’engage à ne jamais voter lorsque je serai en conflit d’intérêts ou lorsqu’il y aura apparence de conflit d’intérêts, et ce, conformément aux règlements de l’AGEG.  En outre, je déclare être membre des organismes suivants (espace pour entrer le nom de quelques organismes) et je m’engage à signaler toute autre situation où je serais en conflit d’intérêts en vertu des règlements de l’AGEG.}

\adoption{13 mai 2018}{16 mai 2018}