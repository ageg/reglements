\setcounter{reglement}{13}
\reglement{Relatif à l’affichage}

\preambule{Le présent règlement gère l’affichage sur les babillards de l’AGEG. Ces babillards sont identifiés AGEG. Dans aucun cas, le présent règlement n’a préséance sur la politique d’affichage de la Faculté.}

\partie{Dispositions générales}

\article{Mission de la politique d’affichage}
\alinea{De permettre la communication des activités et des services offerts aux membres de l’AGEG;}
\alinea{De respecter la signature graphique de l'AGEG selon ses normes.}

\article{Vision de la politique d’affichage}
\alinea{La vision de la politique d’affichage est de permettre la garantie d’une priorité et d’une équité dans l’affichage des activités, en vertu du règlement 28.}

\article{Valeurs de la politique d’affichage}
\valeurs
{Ouverture}{Offrir une tribune aux personnes membres de l’AGEG afin qu'elles puissent promouvoir leurs activités.}
{Engagement}{Offrir ce service afin de promouvoir et de favoriser la réussite des différentes activités organisées par les groupes étudiants.}
{Intégrité}{Traiter tous les groupes de façon équitable.}
{Fraternité}{Promouvoir le sentiment d’appartenance en favorisant l’organisation d’activités sociales.}

\article{Rôles et pouvoirs}
\sousarticle{Vice-présidence aux employé-e-s et à la communication de l’AGEG}
\alinea{Elle est responsable de l’application du présent règlement.}

\sousarticle{Permanence de l’AGEG}
\alinea{Elle est responsable d’autoriser l’affichage en vertu du présent règlement.}
\alinea{Elle appose le sceau sur les affiches valides.}
\alinea{Elle retire les affiches non valides et dont la date d’expiration est échue.}
\alinea{Elle est responsable de l’entretien des babillards sur une base quotidienne.}

\sousarticle{Comité exécutif de l’AGEG}
\alinea{En cas de litige, le CE est le seul juge de la décision qui doit être rendue.}

\sousarticle{Groupe étudiant ou promotion}
\alinea{Le CE d’un groupe étudiant peut refuser l’affichage de toute autre activité se tenant durant la semaine qui lui a été attribuée en début de session et qui compromettrait la rentabilité de son activité sociale.}

\partie{Politique d’affichage}

\article{Validité d’une affiche}
\sousarticle{Sceau}
\alinea{Pour ne pas être retirée des babillards, l’affiche doit posséder le sceau de l’AGEG.}
\sousarticle{\label{RG14DateExpiration} Date d’expiration}
\alinea{La durée maximale pour l’affichage d’une activité étudiante est de 3 semaines, après quoi, l’affiche sera retirée même si l’activité n’a pas eu lieu.}
\alinea{Les affiches sont retirées des babillards lorsque la date de l’activité est dépassée.}
\alinea{Les petites annonces ont une durée d’affichage de 2 semaines (voir article 3.1).}
\alinea{Les activités sociales de génie ont une durée d’affichage limitée à 1 semaine avant ladite activité.}

\sousarticle{Dimension d’une affiche}
\alinea{Les affiches ne doivent pas être plus grandes que le format 11 pouces par 17 pouces.}

\sousarticle{Quantité des affiches}
\alinea{Le nombre d’affiches maximum pouvant être estampillé pour une activité est de six.}
\alinea{Le nombre maximum d’affiches par babillard est de un.}

\article{Affichage permis} \label{RG14AffichagePermis}
\alinea{Toute activité organisée par les groupes étudiants de génie, durant leur semaine d’activités sociales.}
\alinea{Toute activité organisée par l’AGEG.}
\alinea{Toutes informations d’intérêt général pour les personnes étudiantes.}
\alinea{Les "Jeudi détente"  organisés par la promotion finissante et/ou sortante.}
\alinea{Tout autre party dont la publicité a été explicitement autorisée par le groupe étudiant qui s’est vu attribuer la semaine en début de session.}

\article{Affichage proscrit} \label{RG14AffichageProscrit}
\alinea{La publicité n’ayant aucun lien avec les personnes étudiantes}
\alinea{La publicité d’un bar qui contrevient avec le party de la semaine.}
\alinea{L’affiche contrevenant aux règlements de la Faculté ou à une loi établie.}

\partie{Complément}
\article{Respect de l’appellation contrôlée : « Party de Génie »}
\alinea{Les seules affiches pouvant porter l’appellation « Party de Génie » sont celles des partys organisés par les groupes étudiants durant leur semaine de financement.}

\partie{Politique interne à la faculté de génie au sujet d’un référendum de la FEUS}
\article{Politique}
\alinea{Un maximum de quatre (4) représentants actifs par comité est autorisé.}
\alinea{Seuls les membres de la FEUS et toute personne acceptée par le CE/CA de l’AGEG sont autorisés à faire de la mobilisation.}
\alinea{Les personnes représentantes des comités partisans doivent venir s’identifier au local de l’AGEG .}
\alinea{Les personnes représentantes doivent être identifiés conformément aux modalités de la FEUS et du contrat référendaire}
\alinea{Toute mobilisation partisane doit être faite à l’intérieur de la cafétéria et du Salon.}
\alinea{Le maraudage dans les classes et sur les terrains de la faculté est strictement interdit.}
\alinea{Toute mobilisation partisane est strictement interdite lors des évènements sociaux de la faculté de génie.}
\alinea{Toute personne contrevenant à l’une ou l’autre de ces règles pourra se voir interdire l’accès à la faculté de génie.}

\partie{Babillards électroniques}\article{Politique d'affichage}
\sousarticle{Date d'expiration}
\alinea{L'expiration des affiches suit les mêmes règles que pour l'affiche sur les babillards, tel que défini à l'article 2.1.2}
\sousarticle{Format des affiches}
\alinea{La résolution, le format et le type de fichier doit respecter les directives indiquées dans les normes d'affichage électroniques de l'AGEG.}
\article{Admissibilité à l'affichage}
\alinea{Les affiches électroniques suivent les même critères d'admissibilité que les affiches papier, tel que défini aux articles 2.2. et 2.3}
\article{Jeudi détente}
\alinea{Les babillards électroniques sont mis à la disposition du groupe organisant les "Jeudi détente" pour la durée de ceux-ci.}
\alinea{Le comité exécutif de l'AGEG peut retirer tout média considéré inadmissible lors des "Jeudi détente".}


\adoption{13 mai 2018}{16 mai 2018}