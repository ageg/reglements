\reglement{Relatif à l'utilisation des locaux de l'AGEG}

\preambule{L'objectif du présent règlement est de définir les modalités d'utilisation et d'accès aux locaux de l'AGEG incluant, sans s'y restreindre, le local des groupes techniques (C1-2047), le C1-2116 et le local de la finissante (C1-2048).}

\partie{Dispositions générales}
\article{Mission du règlement}
\alinea{Maintenir les locaux de l'AGEG dans un état d'utilisation convenable et propre}
\alinea{Conserver les locaux dans les normes MAPAQ}

\article{Valeur du règlement sur les locaux de l'AGEG}
\valeurs
{Ouverture}{Assurer aux membres de l'AGEG d'avoir accès à des locaux accessibles et propres}
{Engagement}{Favoriser l'engagement des groupes de l'AGEG}
{Intégrité}{Promouvoir la responsabilisation des membres et des groupes utilisant les locaux de l'AGEG}
{Fraternité}{Encourager la cohabitation civilisée entre l'AGEG et tous ses groupes}

\article{Liste des locaux appartenant à l'AGEG}
\alinea{C1-1016}
\alinea{C1-2044}
\alinea{C1-2045}
\alinea{C1-2046}
\alinea{C1-2047}
\alinea{C1-2048}
\alinea{C1-2058}
\alinea{C1-2058-1}
\alinea{C1-2058-2}
\alinea{C1-2058-3}
\alinea{C1-2116}
\alinea{C1-2116-1}
\alinea{C1-2116-2}
\alinea{C1-2116-3}
\alinea{C1-5010}

\article{Entreposage de la nourriture}
\alinea{La nourriture ne peut être entreposée ailleurs que dans le C1-2045, le C1-2047, le C1-2048 et le C1-2044.}
\alinea{La nourriture entreposée dans les réfrigérateurs et les congélateurs doit servir exclusivement pour les activités des semaines de financement, les jeudis-détentes ou les activités organisées par l'AGEG.}
\alinea{Tout groupe de l'AGEG, ainsi que toute promotion, peut conserver de la nourriture dans les réfrigérateurs et congélateurs à condition que celle-ci soit identifiée au nom du groupe, sanitaire et non périmée.}
\alinea{La direction MAPAQ peut, à n'importe quel moment, et ce, sans préavis, inspecter la nourriture se trouvant dans les locaux et disposer de tout ce qui semble impropre à la consommation et/ou à la vente ainsi que tout ce qui n'est pas identifié.}
\alinea{Après sa semaine de financement, le groupe doit vider sa nourriture du réfrigérateur ou avoir un arrangement avec le groupe suivant.}
\alinea{Toute nourriture ou contenant de nourriture qui n'a pas besoin d'être réfrigéré doit se trouver sur les tablettes.}
\alinea{Toute nourriture ou contenant de nourriture déposé par terre, et ce, même momentanément, devra être disposé de façon convenable immédiatement.}

\article{Entreposage du matériel dans le C1-2047}
\alinea{Tout groupe technique n'ayant pas déjà un local alloué peut y entreposer du matériel.}
\alinea{Tout groupe ayant une semaine de financement à venir peut y entreposer du matériel.}
\alinea{Tout matériel doit y être identifié et doit être non salissant.}
\sousalinea{Le nom du groupe doit être indiqué clairement sur le matériel.}
\sousalinea{Les sacs de sables doivent être rangés dans un bac de plastique}
\sousalinea{Les sacs de sables doivent être sortis des bacs à l'extérieur du local}
\alinea{Aucun matériel ne doit y bloquer l'accès et la circulation.}
\alinea{Aucun matériel ne peut se trouver dans le corridor de circulation ou les zones identifiées.}
\alinea{Tous les chariots doivent être placé à l'endroit clairement identifié.}

\partie{Les responsabilités}
\article{Direction MAPAQ}
\alinea{La direction MAPAQ doit passer au moins une fois par semaine pour inspecter le contenu des réfrigérateur et congélateurs ainsi que la nourriture.}
\alinea{La direction MAPAQ peut, à n'importe quel moment, et ce, sans préavis, inspecter le contenu des réfrigérateur et congélateur, et jeter tout ce qui semble impropre à la consommation et/ou à la vente ainsi que tout ce qui n'est pas identifié.} 
\alinea{La direction MAPAQ doit jeter toute nourriture qui se trouve sur le plancher, ou à moins de dix centimètres de celui-ci, même si elle est emballée et qu'elle n'a pas besoin d'être réfrigérée, sans préavis.}

\article{Les groupes de l'AGEG, groupes techniques ou promotions}
\alinea{Tout groupe, incluant l'AGEG, a la responsabilité d'identifier toute nourriture ainsi que tout matériel présent dans les locaux de manière visible.}
\sousalinea{Le nom du groupe doit être indiqué clairement sur le matériel.}
\alinea{Tout groupe entreposant du matériel a la responsabilité d'envoyer au moins un membre pour aider à faire un ménage important une fois par session, à la demande de la VPAI.}
\alinea{Toute nourriture du groupe doit être évacuée des locaux ou vendue au(x) groupe(s) ayant la prochaine semaine de financement, une fois leur semaine de financement terminée.}

\article{La VPAI de l'AGEG}
\alinea{Chaque session, elle doit prévenir tous les groupes actifs de la tenue du grand ménage et rappeler l'obligation de chaque groupe locataire du local de déléguer au moins un membre lors de cette journée, à moins d'une exemption accordée explicitement par la VPAI.}
\alinea{Avertir les groupes ayant laissé un ou des items salissants dans le local pour qu'ils régularisent la situation dans des délais raisonnables (2 jours) avant d'en disposer de façon permanente.}

\partie{Les sanctions}
\article{La nourriture}
\alinea{Si la direction MAPAQ a eu à intervenir plus de deux fois dans une même session, ou à trois reprises à l'intérieur de trois sessions consécutives, envers un groupe, le groupe perdra son droit de profiter des semaines de financement pour les trois sessions suivantes, à moins d'avis contraire de la VPAI.}

\article{Le matériel}
\alinea{Tout le matériel qui n'est pas clairement identifié, rangé dans des contenants non identifiés ou non rangé sur des étagères identifiées est réputé appartenir à l'AGEG. Celle-ci qui pourra le donner à un autre groupe ou en disposer autrement, à moins d'avis contraire de la VPAI.}
\alinea{Tout groupe n'ayant pas nettoyé et rangé de manière appropriée son matériel salissant dans des délais raisonables (définit précédemment) après l'avertissement de la VPAI verra son matériel saisi et perdra sa prochaine semaine de financement.}



\article{Le ménage}
\alinea{Tout groupe n'ayant pas contribué lors de la journée du grand ménage, et qui avait du matériel dans les locaux, perdra l'accès à sa prochaine semaine de financement, à moins d'avis contraire de la VPAI.}



\adoption{3 février 2019}{27 juin 2019}