\partie{Élections
}
\article{Direction d'élection}

\alinea{Une personne est nommée à la direction d'élection par le comité exécutif. Cette personne ne peut pas se présenter aux postes en élections. Ses mandats sont:}

\alinea{Déterminer les dates de la présente période électorale;}
\sousalinea{Si, pour quelques raisons que ce soient, les dates de la période électorale ne peuvent pas respecter les barèmes fixés dans le présent règlement, le directeur d'élection devra faire approuver son calendrier électoral par le comité exécutif actif.}

\alinea{Préparer la plateforme de vote, dans le cas d'un vote électronique;}

\alinea{Recevoir les candidatures, préparer les bulletins de vote et veiller à la bonne marche de l'élection, le tout en conformité avec les dispositions du présent règlement;}

\alinea{Établir et publiciser les modalités du vote;}

\alinea{Favoriser la participation au scrutin à l’aide de publicité durant la période de scrutin;}

\alinea{Déposer un rapport d’élection à la dernière réunion du conseil d'administration de la session durant laquelle se tient le scrutin, lequel n'inclut aucun résultat des personnes candidates;}

\alinea{Déposer un document indiquant le résultat du vote à la dernière réunion du conseil d'administration pour entériner les résultats du vote;}

\alinea{Publiciser les résultats du scrutin après l'entérinement par le conseil d'administration.}


\sousarticle{Mise en candidature}
\alinea{Tous les candidats doivent remplir le formulaire de mise en candidature.}

\alinea{La période de mise en candidature doit durer un minimum de cinq (5) jours ouvrables}

\sousarticle{Campagne électorale}
\alinea{Un minimum de trois (3) jours ouvrables suivants la période de mise en candidature est destiné à la campagne électorale.}

\alinea{Les candidatures seront publiées par la direction d’élection dès le début de la période de campagne électorale.}

\alinea{Pour les postes où aucune candidature n’aura été présentée à la fin de la période de mise en candidature régulière, la période de mise en candidature sera prolongée tout au long de la campagne électorale.}

\alinea{Dans le cas d'une prolongation de la période de mise en candidature, toutes les candidatures seront traitées équitablement, peu importe le moment du dépôt de celles-ci.}


\article{Période de scrutin}
\alinea{Les jours suivants la campagne électorale sont destinés au scrutin.}
\sousalinea{Si possible, une période de présentation des personnes candidates aura lieu le premier jour du scrutin.}
\alinea{Pour un vote électronique, la période doit durer un minimum de 24 heures consécutives. Il est recommandé que la période de scrutin soit de 48 heures.}
\alinea{Pendant la période de scrutin, la seule publicité permise sera celle prévue par le directeur d'élection. Elle devra favoriser la participation au scrutin.}
\alinea{La période de scrutin doit être avant la fin des cours de la présente session.}

\sousarticle{Méthode de scrutin}
\alinea{La méthode de scrutin par défaut est le scrutin universel.}

\sousarticle{Élection du comité exécutif}
\alinea{Chaque poste constitue une élection à part entière. Les personnes électrices ont la possibilité de voter une seule fois par poste. En plus des personnes candidates, les personnes électrices peuvent s'abstenir ou voter pour la chaise (laisser le poste vacant).}

\sousarticle{Élection du conseil d'administration}
\alinea{Toutes les personnes candidates sont réputées appliquer pour le même poste.}
\alinea{L'élection contient un vote de confiance implicite.}
\alinea{Les personnes électrices placent les candidatures en ordre de préférence de 1 à n, où n est le nombre de candidatures.}
\alinea{Les personnes électrices peuvent attribuer une cote de non-confiance, ou 0, à une ou plusieurs personnes candidates en qui elles n'ont pas confiance.}
\alinea{Si plus du tiers des personnes électrices attribuent la cote de 0 à une candidature, la personne candidate sera retirée de la course, faute de confiance des personnes électrices.}
\alinea{Chaque position des personnes candidates équivaut à un nombre de points. La personne candidate à la première position reçoit un nombre de points équivalent au nombre de candidatures. Les personnes candidates subséquentes reçoivent un point de moins par position subséquente. }


\article{Résultat de l’élection}
\alinea{Une personnes candidate au comité exécutif passif sera élue si elle récolte la majorité simple des voix.}
\alinea{Les personnes candidates élues au conseil d'administration sont les candidatures ayant reçues le plus de points.}
\sousalinea{En cas d'égalité, la candidature ayant reçue le moins de cotes 0 sera élue.}
\sousalinea{En cas d'égalité du nombre de cotes 0 pour le dernier poste à être comblé, un tirage au sort est effectué pour déterminer le candidature élue.}
\alinea{La direction d’élection doit publiciser les résultats du scrutin, après l'entérinement du rapport par le conseil d'administration.}
\alinea{La direction d’élection doit déposer un rapport au dernier conseil d'administration de la session durant laquelle se tient le scrutin.}

\article{Litiges}
\alinea{En cas de litige, la direction d'élection déterminera les procédures à suivre.}
