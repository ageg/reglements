\chapter{Dispositions générales}


\article{Dispositions générales}

\alinea{La présente section est un résumé des dispositions générales de la corporation, de sa mission et de sa constitution. Ces informations sont issues, entre autres, des lettres patentes de la corporation.}

\begin{tabular}[c]{m{.2\textwidth}|m{.7\textwidth}}

\hline

Incorporation & Incorporée le 23 décembre 1968, en vertu de la partie III de la Loi sur les compagnies (L.R.Q., c. C-38).\\\hline

Accréditation & L'AGEG est l'association étudiante accréditée pour représenter toutes les personnes étudiantes du premier cycle de la faculté de génie de l'Université de Sherbrooke en vertu de la Loi sur l'accréditation et le financement des associations de personnes élèves ou étudiantes.\\\hline

Langue & L'AGEG est une corporation francophone. Ses instances et ses documents administratifs sont en français. \\\hline

Siège social & Le siège social de la corporation est établi à l'Université de Sherbrooke dans la cité de Sherbrooke au local déterminé par le conseil d'administration de la corporation.\\\hline

Année financière & L'année financière de l'AGEG débute au premier janvier et se termine au 31 décembre de la même année. \\\hline

Raison sociale & La corporation est accréditée sous le nom Association Générale Étudiante en Génie de l'Université de Sherbrooke.\\\hline

Mission de la corporation & La mission de l'Association Générale Étudiante en Génie de l'Université de Sherbrooke est de représenter et de défendre les intérêts de ses membres ainsi que de favoriser l'épanouissement et la qualité de leur formation afin de les préparer à leur vie en société en tant que futures personnes professionnelles de l'ingénierie.\\\hline

Vision de la corporation & L'Association Générale Étudiante en Génie de l'Université de Sherbrooke vise à faire de ses membres les meilleures personnes étudiantes en génie du Canada en encourageant leur épanouissement sur les plans professionnel, social et personnel.\\\hline

\end{tabular}
