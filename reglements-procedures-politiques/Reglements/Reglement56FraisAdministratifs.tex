\textcolor{red}{
\reglement{Relatif aux frais administratifs (ABROGÉ)}
\preambule{L’objectif du présent règlement est de détailler le calcul relatif aux charges financières de l’AGEG pour en établir le montant à facturer en tant que Frais Administratifs à nos comités générant le plus de revenus.}
\partie{Dispositions générales}
\article{Objectif du règlement}
\alinea{Considérant le coût non nul des très nombreux services offerts par l’AGEG à des fins administratives. Considérant que ses dits services administratifs ne bénéficient qu’aux comités de l’AGEG et non à ses membres. Considérant que, grâce à ses services, certains comités génèrent des revenus importants. Il est du devoir de ces comités de contribuer de manière proportionnelle au paiement des charges administratives utilisées par ces derniers.}
\article{Valeur du règlement}
\valeurs
{Ouverture}{S’ouvrir aux coûts administratifs sous-jacent aux activités et services à nos membres.}
{Engagement}{Encourager les comités à réaliser les coûts réels de leurs activités.}
{Intégrité}{Assurer une distribution honnête et juste des frais administratifs.}
{Fraternité}{Soutenir l’AGEG dans ses efforts de développement de ses services.}
\article{Calculs des Frais Administratifs}
\sousarticle{Équation généralisée}
\begin{equation}
    \text{F.A.du comité X} = \frac{\sum \text{C.A.de l' AGEG} - \sum \text{Considérants exceptionnels}}{\sum \text{Comités ciblés}} - \sum \text{Contribution du comité X}
\end{equation} 
\alinea{Où F.A. du comité X sont les Frais Administratifs imposées au Comité X.}
\alinea{Où C.A. de l’AGEG sont les Charges Administratives de l’AGEG.}
\alinea{Où les Considérants Exceptionnels représentent tous frais non récurrents qui sortent de l’ordinaire.}
\alinea{Où la Contribution du comité X est l’argent que le comité donne déjà à l’AGEG ou ses membres par l’entremise des autres ententes.}
\alinea{Où les Comités Ciblés sont tous les comités qui recevront un Frais Administratifs.}
\sousarticle{Définition des Charges Administratives de l’AGEG}
\alinea{Toutes dépenses administratives qui ne servent qu’au fonctionnement des différents comités de l’AGEG, sans être un service direct à ses membres.}
\alinea{Toutes dépenses faites dans le but d’assurer la légalité des opérations de l’AGEG et de ses comités}
\sousarticle{Liste non exhaustive des Charges Administratives de l’AGEG}
\alinea{Examen financier annuel}
\alinea{Gestion comptable (salaires)}
\alinea{Réparation et entretien des équipements louables}
\alinea{Assurances}
\alinea{Service de la dette}
\sousarticle{Définition des Considérants Exceptionnels}
\alinea{Toutes dépenses administratives non récurrentes.}
\alinea{Toutes dépenses administratives non prévues ou non budgétées en date du CA1 de l’été.}
\sousarticle{Exemples de Considérants Exceptionnels}
\alinea{Frais supplémentaires causés par un changement de firme comptable.}
\alinea{L’excédent des frais d’avocat dépassant largement sa case budgétaire.}
\sousarticle{Définition des Contributions}
\alinea{Tout don d’un Comité Ciblé vers un fonds de l’AGEG.}
\alinea{Tout don d’un Comité Ciblé vers un groupe technique de l’AGEG. }
\alinea{Tout paiement d’un Comité Ciblé pour le service de la dette de l’AGEG}
\alinea{Tout don issu d’un règlement de l’AGEG est considéré.}
\sousarticle{Définition des Comités Ciblés}
\alinea{Tout comité ou promotion qui utilise les services administratifs de l’AGEG et dont les Frais Administratifs potentiels représentent moins de 5\% de leur revenu de l’année antérieure.}
\alinea{Les groupes techniques sont exclus de la présente définition.}
\alinea{Dans le cadre du présent règlement, l’AGEG elle-même est considérée comme étant un comité et doit payer une part des Charges Administratives.}
\sousarticle{Liste des Comités Ciblés en date de 2017}
\alinea{L’AGEG}
\alinea{L’Oktoberfest}
\alinea{La promotion finissante}
\sousarticle{Définition de Frais Administratifs}
\alinea{Frais représentant la contribution manquante des Comités Ciblés au paiement des services administratifs de l’AGEG.}
\alinea{Ces frais ne peuvent pas dépasser 5\% des revenus du Comité Ciblé.}
\article{Rôles et pouvoirs}
\sousarticle{CA de l’AGEG}
\alinea{Il entérine les calculs des différentes variables de l’équation définie précédemment.}
\sousarticle{VPAF de l’AGEG}
\alinea{Présenter au CA1 de l’automne la valeur des variables de l’équation définie précédemment.}
\alinea{Envoyer aux Comités Ciblés la somme des frais à payer ou à transférer avant le CA2 de l’automne.}
\sousarticle{Comités Ciblés}
\alinea{Si le Comités Ciblés possède un compte en banque externe, il doit s’acquitter des Frais Administratifs avant le CA3 de l’automne.}
\alinea{Si le Comité Ciblés a son argent sous l’AGEG, le transfert se fera automatiquement au CA3 de l’automne.}
\adoption{17 septembre 2017}{29 novembre 2017}}