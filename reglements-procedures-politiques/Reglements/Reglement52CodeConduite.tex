\reglement{Relatif au code de conduite de l’AGEG dans les partys externes}
\preambule{L’objectif du présent règlement est de faire perdurer la présence des membres et desemployés de l’AGEG lors des événements universitaires externes en leur imposant un code de conduitemenant à des conséquences en cas de non respect du dit code de conduite pour éviter des comportements répréhensibles. L’AGEG se porte garant uniquement de ses membres et de ses employés participant à un événement universitaire externe dont le transport est organisé par l’AGEG. Cependant, le Comitédu code de conduite peut sévir contre tous membres ayant eu un comportement inadéquat lors d’unévénement universitaire externe.}

\partie{Dispositions Générales}
\article{Mission du code de conduite}
\alinea{S’assurer du bon comportement des membres et des employés de l’AGEG et faire perdurer laprésence des membres et des employés de l’AGEG lors des événements universitaires externes.}

\article{Vision du code de conduite}
\alinea{La vision du code de conduite est de promouvoir le rayonnement de l'AGEG par ses membres dans les partys externes.}
\alinea{Le présent règlement vise à encadrer le mise en oeuvre du code de conduite.}

\article{Valeurs du code de conduite}
\valeurs
{Ouverture}{Favoriser aux membres de fraterniser avec les autres étudiants en ingénierie}
{Engagement}{Permettre aux membres de fraterniser avec les autres étudiants en ingénierie.}
{Intégrité}{Assurer la bonne représentation de l’AGEG dans les autres événements externes.}
{Fraternité}{Favoriser un sentiment de fierté par rapport à la représentation de l’AGEG dans les événements universitaires externes.}

\article{Rôles et pouvoirs}
\sousarticle{CA de l'AGEG}
\alinea{Elle forme le comité du code de conduite.}

\sousarticle{Préposés à l'accueil de l'AGEG}
\alinea{Remettent le formulaire du code de conduite aux membres et employés participant à un événement universitaire externe}

\sousarticle{VPAS de l'AGEG}
\alinea{Gère la collecte des codes de conduite avant la tenue d'un événement universitaire externe.}
\alinea{Coordonne les activités du Comité du code de conduite.}
\alinea{Siège sur le Comité du code de conduite.}
\alinea{Tient une liste des personnes exlues des événements universitaires externes et s’assure que la liste est bien à jour en collaboration avec le VPEX.}

\sousarticle{VPEX de l'AGEG}
\alinea{Communique avec les autres associations de génie et met en contacte les autres associations et leVPAS pour l’organisation de transport ou d’événement}
\alinea{Assure un suivi avec les autres associations lorsqu’un membre de l’AGEG a eu un comportementinadéquat lors d’un événement universitaire externe.}
\alinea{Informer les associations externes lorsqu’un de leurs membres a un comportement inadéquat lorsd’un événement de l’AGEG.}
\alinea{Tient une liste des personnes exlues des événements universitaires externes et s’assure que la listeest bien à jour en collaboration avec le VPAS.}

\sousarticle{Comité du code de conduite}
\alinea{Les fonctions du comité du code de conduite sont présentées à la partie III.}

\partie{Code de conduite}
\article{Remise du code de conduite}
\alinea{Le code de conduite sera remis aux membres et aux employés avant la tenue d’un événementuniversitaire externe.}
\alinea{Un membre ou un employé n’ayant pas signé le code de conduite ne pourra pas participer àl’événement universitaire externe.}

\partie{Comité du code de conduite}
\article{Formation et composition}
\alinea{Le Comité du code de conduite sera formé avant le CA1 de chaque session.}
\alinea{Les futurs membres du Comité ne doivent pas avoir reçu de pénalité de la part du Comité du codede conduite depuis au minimum un an.}
\alinea{Le VPAS siégera sur le Comité du code de conduite.}
\alinea{Le CA de l’AGEG désignera un membre du CA de l’AGEG comme deuxième membre du Comitédu code de conduite}
\alinea{Le CA de l’AGEG désignera un membre de l’AGEG comme troisième membre du comité ducode de conduite. Le membre déterminé par le CA ne doit pas nécessairement être membre du CA del’AGEG}

\article{Rémunération}
\alinea{Les membres du comité ne recevront aucune rémunération pour leur fonction.}

\article{Durée du mandat}
\alinea{La durée du mandat du comité est d’une session.}

\article{Convocation}
\alinea{Le VPAS s’occupe de convoquer le Comité lorsqu’un incident survient lors d’un évènement universitaire ou d’un évènement universitaire externe.}

\sousarticle{Quorum}
\alinea{Pour toute assemblée officielle du comité, il est nécessaire d’avoir la présence de tous les membresdu comité. Aucune décision ne sera prise à moins que cette assemblée n’ait le quorum requis.}
\alinea{Les décisions du comité se prendront par vote à majorité. Chaque membre du comité a droit à un vote.}

\article{Décision}
\sousarticle{Critères}
\alinea{Les critères permettent de déterminer la gravité des sanctions imposées au membre ayant commis un délit.}
\alinea{Les critères sont les suivants :}
\sousalinea{Gravité du délit commis}
\sousalinea{Récurrence d'un délit par le membre}
\sousalinea{Attitude du participant face au délit commis et face aux organisateurs de l'évènement suite au délit}

\sousarticle{Délai}
\alinea{Le comité doit évaluer les dossiers à l’intérieur des 2 semaines suivant l’événement.}% et aviser le membre de l’AGEG de la conséquence des actes \replace{à l’intérieur de ces deux semaines-là.}{avant l'échéance de ce délai.}
\alinea{Le comité doit aviser le membre de l’AGEG de sa décision avant l'échéance de ce délai.}
\sousarticle{Appel}
\alinea{Un membre pourra porter une décision du comité du code de conduite en appel devant le CA de l'AGEG.}% s'il juge qu'il a été lésé de façon disproportionnée}
\alinea{La décision du CA de l'AGEG, incluant le refus d'entendre l'appel, est finale et sans appel.}
\partie{Article promotionnel lors des événements}
\alinea{Les directeurs COPS doivent fournir un article promotionnel à échanger par événement externe.}
\partie{Communication}
\alinea{Le VPEX doit avertir l’association externe hôte lorsqu’un membre est refusé;}
\alinea{Le VPEX doit avertir le membre de l’AGEG lorsque celui-ci est expulsé.}
\adoption{23 octobre 2016}{27 juillet 2017}