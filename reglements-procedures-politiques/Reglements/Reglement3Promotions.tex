\reglement{Relatif à la promotion finissante}

\preambule{L’objectif du présent règlement est de définir les liens entre la promotion finissante et l’AGEG.}

\partie{Dispositions générales}
\article{Mission du présent règlement}
\alinea{Décrire les privilèges et services que l’AGEG offre aux promotions finissantes, sortantes et subséquentes.}
\alinea{Donner une ligne directrice sur les services de la promotion sortante finissante lors de son départ de la Faculté.}

\article{Vision de l’AGEG face à la promotion finissante}
\alinea{Donner une chance à chaque promotion d’accumuler de l’argent pour ses activités de fin de bac.}
\alinea{S’assurer qu’une équipe organise les "Jeudi détente".}

\article{Valeurs dans les relations entre la promotion finissante et l’AGEG}
\valeurs
{Ouverture}{Agir dans un esprit de dialogue pour éviter tous malentendus}
{Engagement}{Offrir des services de qualité}
{Intégrité}{Être égal envers tous les membres}
{Fraternité}{Promouvoir le sentiment d’appartenance}

\article{Rôles et pouvoirs}
\sousarticle{AGEG}
\alinea{Détient et alloue les "Jeudi détente"}
\alinea{Donne et facture des services à la promotion finissante}

\article{Promotion finissante}
\alinea{Aux "Jeudi détente", doit facturer à l’AGEG les consommations offertes pour la représentation et celles pour ses activités au prix coûtant.}

\partie{Privilèges accordés à la promotion finissante}
\article{Réservation des locaux}
\alinea{L'AGEG devra, pourvu que l'administration de la Faculté lui en reconnaisse le pouvoir, réserver l'usage exclusif du salon des étudiants et de la radio pour les "Jeudi détente"  à la promotion finissante. Pour plus de détails sur les modalités des prêts de la radio, se référer au règlement 32.}

\article{Organisation des "Jeudi détente"}
\alinea{La promotion finissante se verra donner le droit exclusif d’organiser et de faire des profits lors des "Jeudi détente".}
\alinea{Sur résolution du conseil d’administration, pour une durée indéterminée et pour des raisons exceptionnelles, l’AGEG peut retirer l’organisation des "Jeudi détente" à la promotion finissante. L’AGEG devient donc responsable de l’organisation des "Jeudi détente" pour cette période.}

\sousarticle{Cas d’exceptions}
Des profits pourront être réalisés par un groupe différent de la promotion finissante seulement dans les situations suivantes : 
\alinea{Les profits engendrés par la vente des billets si c’est la semaine de party du groupe.}
\alinea{Tout autre profit pourvu qu’ils aient été entérinés par la promotion finissante.}

\article{Dates de financement}
\alinea{Conformément au règlement 28 sur la distribution des dates de party, la promotion finissante se réserve certains privilèges lors de la sélection des dates. Se référer au règlement 28 pour plus de détails.}

\partie{Devoir de la promotion sortante finissante}
\article{Relatif à la promotion sortantes}
\alinea{Étant donné que la promotion sortante, qui n’est plus à l’école pour  recevoir les comptes après la dernière session, utilise certains services de l’AGEG dont la facturation, qui se fait le mois suivant, le conseil d’administration de l’AGEG adopte la résolution suivante~:}

\begin{quote}
« Le président et le trésorier du conseil exécutif de l’AGEG doivent convoquer le président et le trésorier de la promotion sortante avant le 1er décembre pour faire l’état des comptes et pour assurer un suivi post-bac. À cette occasion, le président et le trésorier de la promotion sortante devront fournir une adresse postale où la correspondance de la promotion sortante sera acheminée une fois la session terminée. Dans l’éventualité où cette réunion n’aurait pas lieu, le conseil exécutif de l’AGEG se réserve le droit de mettre un terme aux services offerts à la promotion sortante.»    
\end{quote}

\article{Relatif à la Taxe administrative de la finissante}
\alinea{Étant donné que la promotion finissante utilise certains services de l'AGEG, celle-ci se doit de payer une partie des frais administratif reliés au salaire des comptables et honoraires professionnels des comptables externes de l'AGEG.}

\alinea{La promotion finissante doit verser une somme équivalente à 5~\% des profits générés lors des "Jeudi détente" sur toute la période où elle a eu la responsabilité d’organiser les "Jeudi détente". Cette somme sera versée au Fonds d'administration de l’AGEG lors de la passation des pouvoirs à la session d’automne.}

\alinea{De cette somme peuvent être retirées les Contributions de la finissante, soit de l’argent que le comité donne déjà à l’AGEG ou ses membres par l’entremise des autres ententes. L'approbation des Contributions se fait par le VPAF à l'automne lors de la passation des pouvoirs. Les Contributions de la finissante sont : }
\sousalinea{Tout don de la finissante vers un fonds de l’AGEG n'étant pas prévu par les Règlements de l'AGEG.}
\sousalinea{Tout don de la finissante vers un groupe de l'AGEG ou un groupe technique de l’AGEG. Ces dons incluent le financement du verre de la rentrée aux intégrations.}
\sousalinea{Tout don issu de la finissante qui est au service de l'AGEG. Ces dons incluent les réparations effectuées sur les locations des services de l'AGEG.}



\iffalse

\sousarticle{Équation généralisée}
\add{\begin{equation}
    \text{F.A. de la finissante} = (\sum \text{C.A.de l' AGEG} - \sum \text{C.E.} - \sum \text{C.F.}) \times \text{\% des services}
\end{equation} }
\alinea{Où F.A. de la finissante sont ses frais administratifs}
\alinea{Où C.A. de l’AGEG sont les Charges Administratives de l’AGEG.}
\alinea{Où C.E. sont les considérants exceptionnels représentent tous frais non récurrents qui sortent de l’ordinaire.}
\alinea{Où C.F. es la contribution de la finissante qui est l’argent que le comité donne déjà à l’AGEG ou ses membres par l’entremise des autres ententes.}

\sousarticle{Liste des Charges Administrative de l'AGEG applicable à la promotion finissante}
\alinea{Salaire et avantage sociaux.}
\alinea{Honoraires professionnels.}

\sousarticle{Définition des Considérants Exceptionnels de l'AGEG}
\alinea{Toutes dépenses administratives non récurrentes.}
\alinea{Toutes dépenses administratives non prévues ou non budgétées en date du CA1 de l’été.}

\alinea{Exemples de Considérants Exceptionnels}
\sousalinea{Frais supplémentaires causés par un changement de firme comptable.}
\sousalinea{L’excédent des frais d’avocat dépassant largement sa case budgétaire.}

\sousarticle{Définition des Contributions}
\alinea{Tout don de la finissante vers un fonds de l’AGEG.}
\alinea{Tout don de la finissante vers un groupe technique de l’AGEG. }
\alinea{Tout don issu d’un règlement \add{ou d'une entente} de l’AGEG est considéré.}
\alinea{Tout don issu de la finissante qui est au service de l'AGEG. \add{Ces dons inclut les réparations effectuées sur les locations des services de l'AGEG.}}

\sousarticle{Définition \% des services}
\alinea{Représente le pourcentage des services de l'AGEG utilisé par la finissante qui est de 25\%.}
\alinea{Exemple des services utilisés par la finissante:}
\sousalinea{Services du coordonnateur administratif et adjoint administratif}

\sousarticle{Définition de Frais Administratifs}
\alinea{Frais représentant la contribution de la promotion finissante au paiement des services administratifs de l’AGEG.}
\alinea{Ces frais ne peuvent pas dépasser 5\% des revenus de la promotion finissante.}

\fi


\adoption{7 avril 2019}{27 juin 2019}