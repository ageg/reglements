\reglement{Relatif à la gestion du Fonds d’équipement étudiant (FÉÉ)}

\preambule{L’objectif du présent règlement est de donner des lignes directrices concernant la gestion de la cotisation étudiante des membres du 1er cycle au Fonds d'équipement étudiant.}

\partie{Dispositions générales}
\article{Mission du FÉÉ}
\alinea{Aider à l’amélioration ou à l’ajout d’équipement ayant des fins didactiques jugées utiles à la réalisation d’un projet aux étudiants du 1er cycle}

\article{Valeurs du FÉÉ}
\valeurs
{Ouverture}{Faire preuve de démocratie dans l’administration du fonds}
{Engagement}{Fournir aux membres un service structuré et efficace pour encourager les activités pédagogiques}
{Intégrité}{Faire preuve de bon jugement dans le processus d’attribution des fonds}
{Fraternité}{Encourager la coopération entre les membres pour favoriser l’efficacité du fonds}

\partie{Relatif au financement du Studio de création}
\article{Attribution des fonds}
\alinea{ Un montant de 100 000~\$ par année est soustrait du total des fonds à répartir et est transféré au financement de la construction du studio de création.} 
\alinea{Un montant de 65 000~\$ par année est soustrait du total des fonds à répartir et est transféré au financement de l’opération du studio de création.}
\alinea{Le montant restant est transféré dans le Fonds de subvention.}

\article{Modalité d'application}
\alinea{Ce règlement arrivera à échéance lorsque la contribution des étudiants au studio de création sera payée complètement (2022). Il devra à ce moment être révisé ou abrogé.}

\adoption{24 novembre 2019}{\add{(À venir)}}