\reglement{Relatif à la gestion du Fonds d’équipement étudiant (FÉÉ)}

\preambule{L’objectif du présent règlement est de donner des lignes directrices au comité FÉÉ dans la gestion, l’attribution et le suivi des subventions relatives au FÉÉ.}

\partie{Dispositions générales}
\article{Mission du FÉÉ}
\alinea{Aider à l’amélioration ou à l’ajout d’équipement ayant des fins didactiques jugées utiles à la réalisation d’un projet soit à l’ensemble des étudiants du 1er cycle, soit à l’ensemble des étudiants d’un département, soit à un groupe technique de l’AGEG.}

\article{Vision du comité FÉÉ}
\alinea{La vision du comité est de pouvoir offrir aux étudiants du premier cycle en génie un environnement de travail didactique varié et mis à jour au besoin.}

\article{Valeurs du FÉÉ}
\valeurs
{Ouverture}{Faire preuve de démocratie dans l’administration du comité}
{Engagement}{Fournir aux membres un service structuré et efficace pour encourager les activités pédagogiques}
{Intégrité}{Faire preuve de bon jugement dans le processus d’attribution des fonds}
{Fraternité}{Encourager la coopération entre les membres pour favoriser l’efficacité du fonds}

\article{Rôles et pouvoirs}
\sousarticle{AGEG}
\alinea{Lors de la dissolution d’un groupe technique, l’AGEG gère le matériel subventionné par le fonds et le redistribue à un département ou à un autre groupe étudiant admissible.}
\alinea{Elle entérine le choix du directeur FÉÉ lors du CA1 de la session d'automne.}
\alinea{Elle vote sur les recommandations du comité FÉÉ suite à la réception du rapport du directeur FÉÉ de l’AGEG au CA3 de la session d'automne.}

\sousarticle{VPAF de l’AGEG}
\alinea{Il est responsable de faire le lien entre l’AGEG et le comité FÉÉ.}
\alinea{Il a les mêmes pouvoirs que chacun des autres membres du comité. Il prend part au processus décisionnel, mais ne représente aucune concentration en particulier.}
\alinea{Il est responsable de nommer le directeur FÉÉ.}
\alinea{Il doit assurer l’intérim jusqu’à ce qu’un directeur soit nommé.}
\alinea{Il s’occupe de remettre les montants alloués.}

\sousarticle{VPAX de l’AGEG}
\alinea{Il informe tous les groupes techniques admissibles au fonds FÉÉ de la procédure de demande au fonds FÉÉ.}
\alinea{Il s’assure que les inventaires et les listes d’outils et de matériaux dangereux soient à jour}
\alinea{Il s’occupe du suivi de l’équipement auprès des groupes.}

\sousarticle{Directeur FÉÉ}
\alinea{Il informe la Faculté et les départements de la procédure de demande au fonds FÉÉ.}
\alinea{Il rédige un rapport détaillé des recommandations du comité FÉÉ qu’il présente ensuite au CA3 de la session d’automne.}
\alinea{Il rédige les procès-verbaux des réunions du comité.}
\alinea{Il est responsable de faire le lien entre le VPAF, le VPAX, le comité FÉÉ et les demandeurs.}
\alinea{Il s’occupe du suivi des décisions et du statut des remboursements.}
\alinea{Il rédige un rapport d’avancement des achats faits et des remboursements effectués qu’il présente au CA3 de la session d’été.}

\sousarticle{Comité FÉÉ}
\alinea{Il rédige des recommandations pour la répartition des sommes disponibles dans le fonds FÉÉ.}
\alinea{Il reçoit les demandes et les dossiers de candidatures selon les procédures de chaque groupe faisant une demande au fond FÉÉ.}
\alinea{Il évalue les demandes reçues selon les critères du présent règlement}
\alinea{Il rédige les recommandations sur l’attribution des fonds selon les critères du présent règlement.}
\alinea{Il peut recommander d’imposer des conditions particulières sur l’utilisation de l’équipement subventionné.}

\partie{Comité FÉÉ}
\article{Composition du comité}
\sousarticle{Membres du comité}
\alinea{Le comité est formé d’un étudiant du 1er cycle par département, du VPAF de l’AGEG, du VPAX de l'AGEG et du directeur FÉÉ de l’AGEG. Ce dernier préside le comité.}

\article{Nomination  et démission}
\sousarticle{Nomination du directeur}
\alinea{La nomination du directeur se fait au CA1 de la session d’automne.}
\alinea{La durée du mandat du directeur débute à sa nomination et se termine à la fin de la session d’été.}

\sousarticle{Nomination des membres du comité}
\alinea{Après sa nomination, le directeur du fonds a la responsabilité de trouver des étudiants du 1er cycle pour former le comité à l’automne. La durée du mandat des autres membres du comité est d‘une session.}
\alinea{La priorité sera donnée aux étudiants ayant siégé au comité à l’année précédente.}

\sousarticle{Démission}
\alinea{Toute démission de tout membre du comité devra parvenir par écrit au VPAF de l’AGEG et sera effective après son approbation.}
\alinea{Un membre du comité peut se voir expulsé du comité par le CA de l’AGEG ou l’assemblée générale de l’AGEG pour des raisons jugées valables.}
\alinea{Un poste laissé vacant sera comblé par un étudiant du 1er cycle nommé par le directeur  et faisant partie du département du membre sortant.}
\alinea{Dans le cas de la démission du directeur FÉÉ, le VPAF de l’AGEG assurera l’intérim jusqu’à ce que le directeur soit remplacé.}
\article{Décision}

\sousarticle{Critères}
\alinea{Le comité décide des projets qu’il veut financer ainsi que des montants alloués à chacun d’eux selon la mission, la vision et les valeurs décrites dans le présent règlement.}
\alinea{Pour être éligible aux fonds, les groupes techniques doivent avoir rempli toutes les exigences du règlement 46 sur la santé et la sécurité étudiantes}
\alinea{Pour être éligible aux fonds, les groupes étudiants doivent avoir rempli toutes les exigences du règlement 50 sur les finances des groupes étudiants et des promotions.}

\sousarticle{Répartition des fonds disponibles}
\alinea{Le solde disponible du FÉÉ est établi en date du 20 octobre de chaque année.}
\alinea{La totalité du fonds doit être attribuée chaque année.}
\alinea{Les montants non réclamés avant la session d’automne seront dilués dans le solde disponible.}
\alinea{Jusqu'à 13~\% des fonds peuvent être attribués à des projets de la faculté et de l’AGEG, c'est-à-dire au tronc commun.}
\alinea{Jusqu'à 25~\% des fonds peuvent être attribués à des projets de groupes techniques.}
\alinea{Le reste du solde, soit un minimum de 62~\% de l’attribution des fonds, se répartit au prorata du nombre d’étudiants par département en se basant sur la fréquentation étudiante du 1er cycle en génie cumulative depuis les cinq années précédant l’année en cours.}
\alinea{Dans une situation où un groupe, un département ou la Faculté effectue une demande de plusieurs équipements totalisant un montant plus élevé que le montant qui est disponible, le comité doit établir une liste de priorités dans les équipements. Ceci n’empêche pas le comité de refuser de subventionner un équipement.}

\sousarticle{Quorum}
\alinea{Pour toute assemblée officielle du comité, il est nécessaire d’avoir la présence de tous les membres du comité. Aucune décision ne sera prise à moins que cette assemblée n’ait le quorum requis.}
\alinea{Les décisions du comité se prendront par vote à majorité.  Chaque membre du comité a droit à un vote.}

\sousarticle{Retour de décision}
\alinea{Advenant le cas où un projet est accepté par l’AGEG, la subvention attribuée est finale, sauf si le projet change de façon majeure. Dans cette situation, le projet ne recevra aucune subvention.}

\sousarticle{Part de financement}
\alinea{Le comité peut décider de financer partiellement ou entièrement un équipement.}

\sousarticle{Validité de projet}
\alinea{Les projets acceptés ne sont valides que pour une année et ne sont pas transférables aux années ultérieures.}

\article{Attribution}
\sousarticle{Attribution des fonds}
\alinea{Suite à l’approbation des recommandations du comité en conseil d’administration, le directeur du fonds doit en informer les demandeurs. Une fois l’achat effectué, l’acquéreur doit fournir une preuve d’achat (photo et facture) au VPAF. Le VPAF peut ensuite verser l’argent attribué pour cet achat.}

\sousarticle{Utilisation des fonds}
\alinea{Les fonds accordés ne pourront être utilisés à d’autres fins que celles décrites dans le rapport de décision. Le cas échéant, les sommes ne seront pas versées. Par contre, avec l’autorisation du VPAF, l’équipement accordé peut être remplacé par un autre de forte similitude, poursuivant les mêmes objectifs pédagogiques que l’équipement initial.}
\alinea{Lorsque tous les équipements subventionnés ont été payés, mais que le montant alloué n’a pas été utilisé en entier, le VPAF peut autoriser l’utilisation des fonds restants pour les équipements suivants dans la liste de priorité déterminée par le comité FÉÉ.}

\article{Cotisations}
\sousarticle{Cotisation étudiante}
\alinea{La cotisation au fonds est établie à 4.17~\$ par crédit jusqu’à concurrence de 50~\$ par étudiant au 1e cycle par session d’étude. Cette cotisation est obligatoire et non remboursable}
\alinea{Seule l’assemblée générale peut modifier cette cotisation.}

\sousarticle{Surplus inutilisés}
\alinea{Les surplus inutilisés du Fonds (subventions non réclamées avant le 19 octobre) seront réinvestis dans le fonds pour la prochaine année d’attribution.}

\partie{Critères d'admissibilité au Fonds d'Équipement Étudiant}
\article{Équipement}
\sousarticle{Définition}
\alinea{Le terme équipement fait référence à tout matériel, machine, outil ou installation nécessaire à un projet d'un des demandeurs}

\sousarticle{Durabilité}
\alinea{La durée de vie prévue de l'équipement doit être d'au moins trois (3) ans}
\alinea{L'équipement ne doit pas être consommable}

\sousalinea{Un équipement utilisant du matériel consommable sera éligible, mais le matériel consommable devra être refusé. Exemple: Une soudeuse pourrait être achetée, mais les électrodes ne pourront être remboursées à travers le FÉÉ.}
\alinea{L'équipement ne doit pas pouvoir faire l'objet de demandes répétitives annuellement}

\article{Bénéficiaires}
\alinea{L'équipement devra bénéficier à au moins une des populations suivantes:}
\sousalinea{L'ensemble des étudiants du premier cycle;}
\sousalinea{L'ensemble des étudiants d'un département;}
\sousalinea{Un groupe technique de l'AGEG.}
\adoption{3 avril 2016}{7 avril 2016}

\addenda{Relatif au financement du studio de création}

\preambule{L’objectif du présent addenda est de modifier les lignes directrices au comité FÉÉ dans l’attribution des subventions relatives au FÉÉ dans l’intérêt de financer le projet de studio de création.}
\setcounter{partie}{2} % On passe directement a la partie 3

\partie{Ajustement Répartition des fonds}
\article{Retrait au total des fonds}
\alinea{ Un montant de 100 000~\$ par année est soustrait du total des fonds à répartir et est transféré au financement de la construction du studio de création.} 
\alinea{Un montant de 65 000~\$ par année est soustrait du total des fonds à répartir et est transféré au financement de l’opération du studio de création.}

\article{Répartition des fonds disponibles}
\alinea{Cet article modifie entièrement le sous-article 2.3.2 sujet à la répartition des fonds disponibles.}
\alinea{Le solde disponible du FÉÉ est établi en date du 20 octobre de chaque année.}
\alinea{La totalité du fonds doit être attribuée chaque année.}
\alinea{Les montants non réclamés avant la session d’automne seront dilués dans le solde disponible.}
\alinea{Jusqu'à 17~\% des fonds peuvent être attribués à des projets de l’AGEG.}
\alinea{Le reste du solde, soit un minimum de 83~\% de l’attribution des fonds, est attribués à des projets de groupes étudiants. Les groupes scolaires ne sont pas admissibles aux subventions des groupes étudiants.}
\alinea{Dans une situation où un groupe ou l’AGEG effectue une demande de plusieurs équipements totalisant un montant plus élevé que le montant qui est disponible, le comité doit établir une liste de priorités dans les équipements. Ceci n’empêche pas le comité de refuser de subventionner un équipement.}

\partie{Modalités d’application}
\article{Le terme du présent addenda}
\alinea{Cette adaptation aux règlements entrera en vigueur suite à son entérinement en conseil d’administration de l’AGEG et arrivera à échéance lorsque la contribution des étudiants au studio de création sera payée complètement.}

\adoption{7 avril 2019}{27 juin 2019}