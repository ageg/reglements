\reglement{Relatif à la délégation des Jeux de génie de l’Université de Sherbrooke}

\preambule{L’objectif du présent règlement est de donner des lignes directrices au comité de sélection de l’exécutif de la délégation des Jeux de génie ainsi qu’à l’exécutif afin de s’assurer d’une représentation forte aux Jeux de génie.}

\partie{Dispositions générales}
\article{Mission de la délégation des Jeux de génie}
\alinea{Permettre aux étudiants de génie de Sherbrooke de rencontrer des étudiants des autres facultés de génie du Québec dans le cadre des Jeux de génie.}
\alinea{Rapprocher des membres de l’AGEG.}

\article{Vision de la délégation des Jeux de génie}
\alinea{La vision de la délégation des Jeux de génie est d’assurer la continuité et le bon fonctionnement des activités de la délégation des Jeux de génie de l’Université de Sherbrooke.}

\article{Valeurs de la délégation des Jeux de génie}
\valeurs
{Ouverture}{Faire preuve de démocratie dans la formation de la délégation des Jeux de Génie}
{Engagement}{Permettre aux membres de fraterniser avec les autres étudiants en génie du Québec}
{Intégrité}{Assurer une sélection impartiale des représentants de l’AGEG aux Jeux de génie}
{Fraternité}{Favoriser un sentiment d’appartenance des membres à la Faculté et à l’Université}

\article{Rôles et pouvoirs}
\sousarticle{CA de l'AGEG}
\alinea{Elle forme le comité de sélection du conseil exécutif lors du CA4 de la session d’automne.}
\alinea{Elle vote les recommandations du comité remises lors du CA1 de la session d’hiver.}

\sousarticle{Comité de sélection du conseil exécutif}
\alinea {Le comité décide des membres du conseil exécutif de la délégation des Jeux de génie de l'Université de Sherbrooke et remet ses recommandations pour le CE des Jeux de génie de l’Université de Sherbrooke lors du CA1 de la session d'hiver.}
\alinea {Il forme le \textit{comité plus} avec tous les membres de l’Ageg se présentant à la chefferie de la délégation des Jeux de génie de l'Université de Sherbrooke, n'ayant pas le droit de vote.}
\alinea {Le \textit{comité plus} procède aux entrevues individuelles des membres de l’Ageg se présentant pour les postes du conseil exécutif de la délégation des Jeux de génie de l'Université de Sherbrooke.}
\alinea {Le comité procède aux entrevues individuelles de tous les membres de l’Ageg se présentant à la chefferie de la délégation des Jeux de génie de l'Université de Sherbrooke.}
\alinea{Le comité peut, s'il le juge, nommer un des candidats à la chefferie, co-chef.}

\sousarticle{VPEX de l’AGEG de l’hiver}
\alinea{Il préside le comité de sélection du conseil exécutif.}
\alinea{Il est responsable d’annoncer la disponibilité des formulaires de demandes ainsi que les modalités de sélection du conseil exécutif de la délégation.}
\alinea{Il recueille les demandes avant la date butoir à la session d’hiver.}
\alinea{Il dépose les recommandations du comité au sujet des membres de l’exécutif au conseil d'administration sous forme d’un rapport détaillé lors du CA1 de la session d'hiver.}

\article{Chef de la délégation des Jeux de génie}
\alinea{Il est cosignataire du compte avec le trésorier de la délégation des Jeux de génie.}
\alinea{Il supervise le travail fait par les vice-présidents et s’assure que les différents mandats des vice-présidents sont accomplis.}
\alinea{Il s’occupe des communications à l’intérieur de la délégation.}
\alinea{Il fait le lien entre l’AGEG et la délégation.}
\alinea{Il fait le lien entre l’Université hôtesse et la délégation.}
\alinea{Il représente la délégation à la CRÉIQ.}
\alinea{Il convoque et dirige les réunions de l’exécutif de la délégation.}
\alinea{Il doit produire un rapport sur l’année de son mandat après la tenue des Jeux de génie y incluant la liste de tous les membres de la délégation au plus tard pour le CA1 de la session d’hiver.}
\alinea{Il est le chef de la délégation de Sherbrooke aux Jeux de génie.}
\alinea{Il est responsable de motiver les membres de la délégation avant et pendant les Jeux de génie.}
\alinea{Il est responsable de la tenue sécuritaire des évènements en lien avec les jeux de génie (mini-jeux, sélections, etc).}
\alinea{Il est responsable de former les équipes pour les différentes compétitions des Jeux de génie.}
\alinea{Il est responsable de trouver le matériel nécessaire aux différentes compétitions.}
\alinea{Il s’assure que tous les participants sont au courant de leurs tâches durant les Jeux.}
\alinea{Il doit respecter les spécifications soumises par l’Université hôtesse des Jeux de génie quant aux compétitions.}
\alinea{Il doit voir à faire participer tous les membres de la délégation.}

\sousarticle{Co-chef de la délégation des Jeux de génie}
\alinea{Il supervise le travail fait par les vice-présidents et s'assure que les différents mandats de vice-présidents sont accomplis.}
\alinea{Il est responsable de motiver les membres de la délégation avant et pendant les Jeux de génie.}
\alinea{Il est responsable de la tenue sécuritaire des évènements en lien avec les jeux de génie (mini-jeux, sélections, etc.).}
\alinea{Il est responsable de former les équipes pour les différentes compétitions des Jeux de génie.}
\alinea{Il est responsable de trouver le matériel nécessaire aux différentes compétitions.}
\alinea{Il s'assure que tous les participants sont au courant de ler tâche durant les Jeux.}
\alinea{Il doit respecter les spécifications soumises par l'Université hôtesse des Jeux de génie quant aux compétitions.}
\alinea{Il doit voir à faire participier tous les membres de la délégation.}
\alinea{Dans le cas où il y a un co-chef, il est responsable des tâches du VP logistique}

\sousarticle{Trésorier de la délégation des Jeux de génie }
\alinea{Il est cosignataire du compte avec le président de la délégation des Jeux de génie.}
\alinea{Il est responsable de créer le cahier de commandites et de former une équipe pour la recherche de commanditaires.}
\alinea{Il est responsable de stimuler la recherche de commandites auprès de toute la délégation.}
\alinea{Il est responsable d’organiser des activités de financement en collaboration avec le vice-président social.}
\alinea{Il est responsable du budget de la délégation et doit le remplir selon les normes établies par le règlement 50 sur les finances des groupes étudiants et des promotions.}

\sousarticle{VP commandites de la délégation des Jeux de génie}
\alinea{Il est responsable de donner la visibilité adéquate aux commanditaires comme décrite dans le cahier de commandites.}
\alinea{Il est responsable de trouver des commanditaires pour la délégation.}
\alinea{Il est responsable des remerciements aux commanditaires suite aux Jeux de génie.}
\alinea{Il est responsable de faire la promotion de la délégation auprès des étudiants, de la population et des médias.}
\alinea{Il est responsable des photos, des vidéos et des documents qui concernent la délégation.}

\sousarticle{VP social de la délégation des Jeux de génie}
\alinea{Il est responsable de faire connaître les Jeux de génie auprès des étudiants de la Faculté.}
\alinea{Il est responsable d’organiser le recrutement des membres de la délégation.}
\alinea{Il est responsable, assisté par le VP logistique, de l'organisation de la course aux jeux.}
\alinea{Il est responsable de préparer des activités permettant aux membres de la délégation de se rapprocher avant la tenue des Jeux de génie.}
\alinea{Il est responsable d’organiser des activités de financement en collaboration avec le VP commandites.}


\sousarticle{VP articles promotionnels de la délégation des Jeux de génie}
\alinea{Il est responsable du signe distinctif et de l’uniforme de la délégation.}
\alinea{Il est responsable de la mise en œuvre de la thématique de la délégation.}
\alinea{Il est responsable de se former une équipe pour l’aider dans l’accomplissement de ses tâches.}

\sousarticle{VP machine de la délégation des Jeux de génie}
\alinea{Il est responsable de la réalisation de la conception de la machine.}
\alinea{Il doit former une équipe pour réaliser le concours de conception de la machine.}
\alinea{Il doit respecter les spécifications soumises par l’université hôtesse des Jeux de génie quant au concours de conception de la machine.}
\alinea{Il doit s’assurer que l’environnement de conception/construction de la machine est sécuritaire.}

\sousarticle{VP logistique de la délégation des Jeux de Génie}
\alinea{Dans le cas qu'un co-chef soit élu, les tâches du VP logistique seront transférées au co-chef. Le poste de VP logistique restera vacant.}
\alinea{Il est responsable des l'organisation des minis-jeux.}
\alinea{Il est responsable de la compétition machine II.}
\alinea{Il assiste le VP social dans l'organisation de la course des jeux.}
\alinea{Il est reponsable de l'approvisionnement de la délégation.}
\alinea{Il s'occupe du transport de la délégation au jeux.}

\sousarticle{VP arts et culture de la délégation des Jeux de Génie}
\alinea{Il est responsable d'organiser les pratiques des compétitions sociales (génie enherbe, débat oratoire, improvisation).}
\alinea{Il est responsable de la mise à jour du site internet de la délégation.}
\alinea{Il est responsable des évènements culturels organisés par la délégation durant les jeux (décors, dégrises, spectacle d'ouverture).}
\alinea{Il s'occupe avec le VP aux articles promotionnels de la confection des costumes pour les différents évènements cultures organisés par la délégation.}
\alinea{Il est responsable de la réalisation et du tournage de la vidéo présentant la machine.}

\sousarticle{VP Entrepreneuriat de la délégation des Jeux de Génie}
\alinea{Il est responsable de monter et diriger l'équipe entrepreneuriale pour ladite compétition.}
\alinea{Il est responsable de toute communication avec les directeurs de la compétition.}
\alinea{Il est responsable de recueillir l'information nécessaire pour la compétition et de la transmettre à son équipe.}
\alinea{Il est responsable de coordonner les activités propres à la compétition entrepreneuriale.}
\alinea{Il est responsable de la gestion, de la compétition et de la remise à temps des travaux exigés durant toute la durée de la compétition.}
\alinea{Il doit respecter les spécifications soumises par l'université hôtesse des Jeux de génie quant à la compétition entrepreneuriale}


\partie{Comité de sélection du conseil exécutif}
\article{Composition du comité}
\sousarticle{Membres du comité}
\alinea{Il est composé de l’ancien chef de la délégation des Jeux de génie de l'Université de Sherbrooke, du VPEX de la session d’hiver ainsi qu’un membre de l’Ageg.}

\sousarticle{Rémunération}
\alinea{Les membres du comité ne recevront aucune rémunération pour leur fonction.}

\article{Nomination et démission}
\sousarticle{Nomination des membres du comité}
\alinea{La durée du mandat des membres du comité est d’une session, la session d’hiver.}
\alinea{La priorité sera donnée aux membres du CA ayant déjà participé aux Jeux de génie.}

\article{Démission}
\alinea{Toute démission de tout membre du comité devra parvenir par écrit au VPEX de l’AGEG et sera effective après son approbation.}
\alinea{Un membre du comité peut se voir expulsé du comité par le CA de l’AGEG ou l’assemblée générale de l’AGEG pour des raisons jugées valables.}
\alinea{Un poste laissé vacant sera comblé par un membre de l’AGEG nommé par le VPEX de l’AGEG.}

\article{Décision}
\sousarticle{Critères}
\alinea{Les critères de sélection du conseil exécutif de la délégation des Jeux de génie permettent au comité d’évaluer et d’analyser les demandes afin de soumettre les meilleures recommandations au conseil d’administration.}
\alinea{Les critères sont les suivants~:}
\sousalinea{Avoir un CE pour les Jeux de Génie le plus uni possible }
\sousalinea{Motivations des candidats}
\sousalinea{Idées des candidats quant aux Jeux de génie}
\sousalinea{Implications antérieures et actuelles des candidats au niveau universitaire}
\sousalinea{Participation antérieure des candidats aux Jeux de génie}
\sousalinea{Tout autre point jugé pertinent}

\partie{Conseil exécutif de la délégation des Jeux de génie}
\article{Composition du comité}
\sousarticle{Membres du comité}
\alinea{Le comité est formé de 9 membres de l’AGEG nommés par le CA~:}
\sousalinea{Chef}
\sousalinea{Co-chef (facultatif)}
\sousalinea{Trésorier}
\sousalinea{Vice-président aux commanditaires}
\sousalinea{Vice-président logistique (facultatif)}
\sousalinea{Vice-président social}
\sousalinea{Vice-président articles promotionnels}
\sousalinea{Vice-président machine}
\sousalinea{Vice-président arts et culture}
\sousalinea{Vice-président entrepreneurial}

\article{Rémunération}
\alinea{Les membres du comité ne recevront aucune rémunération pour leur fonction.}
\article{Nomination et démission}

\sousarticle{Nomination des membres du comité}
\alinea{La durée du mandat des membres du comité est d’un an, débutant au CA1 de la session d’hiver.}

\sousarticle{Démission}
\alinea{Toute démission de tout membre du comité devra parvenir par écrit au Chef de la délégation des Jeux de génie et sera effective après son approbation.}
\alinea{Un membre du comité peut se voir expulsé du comité par le CA de l’AGEG ou l’assemblée générale de l’AGEG pour des raisons jugées valables.}
\alinea{Un poste laissé vacant sera comblé par un membre de l’AGEG nommé par le conseil exécutif de la délégation des Jeux de génie.}

\article{Décision}
\sousarticle{Critères}
\alinea{Les critères de sélection des participants de la délégation des Jeux de génie permettent au conseil exécutif d’évaluer et d’analyser les demandes afin de choisir les meilleurs candidats.}
\alinea{Les critères sont les suivants~:}
\sousalinea{Leur implication antérieure et actuelle}
\sousalinea{Participation aux sélections}
\sousalinea{Talents (sports, débats, machine, académique, etc.)}
\sousalinea{Dernière chance d’aller aux Jeux}
\sousalinea{N’est jamais allé aux jeux}
\sousalinea{Tout autre point jugé pertinent par le comité exécutif de la délégation}

\sousarticle{Conflit d’intérêt}
\alinea{Chaque membre du comité devra fournir une liste de candidats potentiels aux sélections avec qui ils ont une relation pouvant favoriser ou défavoriser leur candidature et ce avant la première sélection}
\alinea{Lors de la sélection de la délégation l’exécutant s'abstiendra d’influencer (positivement ou négativement) la décision du CE sur les candidats mentionnés sur sa liste}

\sousarticle{Quorum}
\alinea{Pour toute assemblée officielle du comité, il est nécessaire d’avoir la majorité des membres. Aucune décision ne sera prise à moins que cette assemblée n’ait le quorum requis.}
\alinea{Les décisions du comité se prendront par vote à majorité. Chaque membre a droit à un vote. En cas d’égalité, le chef a un vote de plus.}

\adoption{3 février 2019}{27 juin 2019}