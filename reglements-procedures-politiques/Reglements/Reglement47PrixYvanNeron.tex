\reglement{Relatif au prix Yvan Néron}

\preambule{L’objectif du présent règlement est de donner des lignes directrices pour la remise du Prix Yvan Néron. Le prix Yvan Néron est un prix remis à un membre personnel non-enseignant de la Faculté de génie chaque automne, lors du gala du mérite étudiant depuis l’automne 2013. Aux fins de ce règlement, le terme «~enseignant~» correspondra aux professeurs, chargés de cours ou techniciens qui participent à la formation des étudiants du premier cycle.}

\partie{Dispositions générales}
\article{Mission de l’attribution du Prix Yvan Néron}
\alinea{La mission du Prix Yvan Néron est de rendre hommage annuellement à un membre personnel non-enseignant apprécié des étudiants durant leur baccalauréat. Ce prix permet de reconnaître l’effort et la disponibilité auprès des étudiants du personnel non-enseignant, se décrivant par une bonne relation personnel/étudiant s’étalant même jusqu’à l’extérieur de leurs missions administratives.}

\article{Vision de l’attribution du Prix Yvan Néron}
\alinea{La vision de l’attribution du Prix Yvan Néron est de valoriser le travail du personnel non-enseignant de la Faculté de génie.}

\article{Valeurs et rôles du Prix Yvan Néron}
\valeurs{Ouverture}{Permettre aux étudiants de valoriser le travail du personnel administratif au cours de leurs études}
{Engagement}{Enrichir les liens entre les étudiants et les membres de la Faculté}
{Intégrité}{Tous les étudiants sont appelés à reconnaître ouvertement l’investissement du personnel non-enseignant}
{Fraternité}{Développer le sentiment d’appartenance à la communauté et à la Faculté de génie}

\article{Rôles et pouvoirs}
\sousarticle{CE de l’AGEG}
\alinea{Il doit définir le récipiendaire et le faire accepter en CA au plus tard lors du CA2 de l’automne;}
\alinea{Il doit préparer les prix et les remettre au gala du mérite étudiant organisé par la Faculté de génie.}

\adoption{19 octobre 2014}{27 novembre 2014}