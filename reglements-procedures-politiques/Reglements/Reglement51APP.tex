\reglement{Relatif à l’organisation de l’APP par les étudiants à la Faculté de génie de l’Université de Sherbrooke.}

\preambule{L’objectif du présent règlement est de :
1.	Donner des lignes directrices au comité organisateur de l’APP afin de le guider dans l’organisation de l’événement;
2.	Effectuer le contrôle et le suivi de cette activité essentielle au rayonnement des étudiants de la Faculté de génie de Sherbrooke.}

\partie{Dispositions générales}
\article{Mission de l’APP}
\alinea{Offrir un lieu de rassemblement étudiant interfacultaire et interuniversitaire;}
\alinea{Développer des aptitudes d'organisation et de gestion chez les bénévoles et les organisateurs;}
\alinea{Contribuer au rayonnement l'Association générale des étudiants en génie de l'Université de Sherbrooke.}

\article{Vision du comité organisateur de l'APP}
\alinea{La vision du comité organisateur de l'APP est d'offrir un événement majeur aux étudiants.}

\article{Valeurs et rôles des documents administratifs}
\valeurs
{Fraternité}{Enrichir les liens entre les étudiants afin de développer la fraternité}
{Engagement}{Encourager l'implication universitaire et promouvoir les activités de l'AGEG}
{Ouverture}{Permettre à tous les membres de l'AGEG de s'impliquer dans l'organisation d'un événement majeur dans le milieu universitaire}
{Appartenance}{Développer le sentiment d’appartenance à la communauté et à la Faculté de génie}

\article{Objectifs de l’APP}
\alinea{L’objectif général de l'APP est l’organisation, la tenue et la supervision d’une activité permettant aux étudiants de génie d'acquérir une expérience organisationnelle et de gestion de projet par :}
\sousalinea{L'organisation d'un événement socioculturel à saveur estivale;}
\sousalinea{La tenue d'une activité sociale à Sherbrooke organisée par les étudiants pour les étudiants;}
\sousalinea{L’offre d’un événement comportant le moins de risque possibles pour les participants;}
\sousalinea{Le développement de liens fraternels entre les étudiants de la Faculté, nouveaux comme anciens;}
\sousalinea{Le développement d’un sentiment d’appartenance.}

\article{Composition}
\alinea{Le comité organisateur de l’APP est constitué d’étudiants de l’Université de Sherbrooke, avec une priorité pour les membres de l’AGEG.}

\article{Mandat}
\alinea{Le comité organisateur de l’APP a comme mandat d’organiser, de tenir et de superviser l'APP. Il est responsable de former les sous-comités par un appel général.}
\alinea{Il doit transmettre les informations sur l’avancement de la planification des activités au conseil d'administration.}
\alinea{Il doit recruter et former les bénévoles nécessaires au bon déroulement de l'APP.}
\alinea{Il doit s'assurer de donner les formations nécessaires : Éduc’alcool, premiers-soins (5 \%) et toute autre formation jugée nécessaire.}
\alinea{Il doit veiller à l’application du guide sur le développement durable et de l’écoresponsabilité de l’AGEG.}

\partie{Rôles et pouvoirs}
\article{AGEG}
\alinea{L’AGEG, représentée par le VP social de la session en cours, s’assure du suivi des objectifs et de la mission de l'APP. En cas de non-respect des valeurs exprimées à travers les objectifs et la mission de l'APP, l’AGEG a un droit de véto sur toutes activités découlant du comité organisateur.}

\article{Comité organisateur}
\alinea{En conformité avec la mission, les valeurs et les objectifs du présent règlement, le comité organisateur de l’APP a le pouvoir décisionnel sur tous les aspects organisationnels des sous-comités et des activités reliées à l’événement.}
\alinea{Le comité organisateur de l’APP choisit la façon d’élire les directeurs des sous-comités.}
\alinea{Le comité organisateur de l’APP se réserve le droit de destituer un membre ou un directeur d’un sous-comité advenant un comportement inadmissible avant ou durant la tenue des activités de l'événement.}
\alinea{Le comité organisateur de l’APP est voté au CA2 de la session d’automne selon les recommandations du comité de sélection du comité organisateur.}
\alinea{Le comité organisateur ne sera pas rémunéré.}

\article{Président}
\alinea{Le Président doit être membre de l’AGEG;}
\alinea{Superviser le travail du comité organisateur et des sous-comités;}
\alinea{S’occuper des communications à l’intérieur du comité organisateur;}
\alinea{Convoquer et diriger les réunions du comité organisateur de l’APP et des sous-comités;}
\alinea{S’assurer que les dossiers de chacun des VP avancent normalement;}
\alinea{Faire le lien entre le comité et le CA de l’AGEG, en étant présent au minimum au CA4 de l’automne, au CA 2 et 4 de l’hiver pour faire part des avancements du comité au CA.}
\alinea{Être signataire du compte bancaire de l'APP;}
\alinea{Produire un rapport final sur la planification et le déroulement des activités de l'activité, qui sera déposé au CA3 de la session d’été;}
\alinea{Il doit s'assurer que les tâches des postes laissés vacants soient effectuées à chaque session;}
\alinea{Il doit rester sobre tout au long de l’événement.}

\article{VP Logistique}
\alinea{Faire le lien entre le comité et le CA de l’AGEG, en étant présent au minimum au CA4 de l’automne, au CA 2 et 4 de l’hiver pour faire part des avancements du comité au CA.}
\alinea{Être signataire du compte bancaire de l’APP;}
\alinea{Assurer une communication constante avec les responsables de l’endroit où se tient l’APP;}
\alinea{Communiquer avec le service incendie de la ville de Sherbrooke pour leur faire part de nos plans et confirmer les capacités;}
\alinea{Planifier et louer le matériel nécessaire à la tenue de l'APP(toilettes, véhicule, walkie-talkie, hôtel, etc.);}
\alinea{Faire le plan (électrique, entrée d'eau, terrasse, etc.) du site de l'APP;}
\alinea{Déterminer l'horaire logistique de la soirée (ouverture, bouffe, ouvertures, fermetures);}
\alinea{S’assurer d’un transport adéquat des participants entre le lieu de l’événement et les différents campus de l’Université de Sherbrooke;}
\alinea{Coordonner avec le VP Bière les demandes en électricité, eau et espace de la brasserie;}
\alinea{Il doit rester sobre tout au long de l’événement.}

\article{VP Finances}
\alinea{S'assurer de la sécurité de l'argent au cours de la soirée;}
\alinea{Apprendre le fonctionnement des caisses électroniques, leur mise en place, la production de rapports, etc.;}
\alinea{Communiquer les besoins en matière d'électricité au VP logistique;}
\alinea{Communiquer les besoins en matière d’éclairage au VP Logistique et Audio-visuel;}
\alinea{S'assurer de la formation du personnel pour la tenue des caisses;}
\alinea{Créer le cahier de commandite;}
\alinea{Commander tous les billets requis pour l'événement (entrée, remplissages, etc.);}
\alinea{S'assurer d'avoir la monnaie nécessaire pour la soirée;}
\alinea{Présenter un budget préliminaire au CA4 de la session d'automne et au CA2 de la session d'hiver;}
\alinea{Présenter un budget au CA4 de la session d'hiver et au CA1 de la session d'été;}
\alinea{Présenter un bilan final au CA3 de la session d'été, ou dès la réception de toutes les factures;}
\alinea{Être signataire du compte bancaire de l'APP;}
\alinea{S'occuper de la gestion des factures et de leur entrée adéquate dans le logiciel comptable de l’AGEG;}
\alinea{Obtenir un avenant pour l'APP (assurer l'événement);}
\alinea{Il doit rester sobre tout au long de l’événement.}

\article{VP Sécurité}
\alinea{S'assurer d'avoir une équipe professionnelle de sécurité;}
\alinea{S’assurer de la présence d’une équipe de premiers soins et d’urgence adéquate (ambulance, équipe premiers soins, etc.);}
\alinea{S’assurer de la communication entre l’organisation et l’équipe de sécurité en place;}
\alinea{Informer les membres de tous les comités des procédures de résolution de problèmes;}
\alinea{S'occuper des communications entre la police de Sherbrooke et l'événement;}
\alinea{Il doit rester sobre tout au long de l’événement.}

\article{VP Audio-visuel}
\alinea{Communiquer les besoins en matière d'électricité au VP logistique;}
\alinea{Coordonner les demandes d’équipement visuel (éclairage, écrans, projecteurs) avec le VP Logistique et le VP Communication}
\alinea{Réserver le système de son et le matériel visuel de l'APP;}
\alinea{S’assurer de décorer l’événement en lien avec la thématique, et coordonner tout avec les équipements audiovisuels ;}
\alinea{Veiller à l'installation et la désinstallation des systèmes audio-visuels;}
\alinea{Trouver les performances de l'événement (groupes musicaux, DJ, magicien);}
\alinea{Déterminer l'horaire de la soirée (animation, groupes musicaux, DJ);}
\alinea{Il doit rester sobre tout au long de l’événement.}

\article{VP Communications}
\alinea{Créer le logo annuel de l'événement;}
\alinea{Créer le design d’impression pour les items promotionnels de l'APP;}
\alinea{Prendre les commandes des associations du campus de Sherbrooke et des associations externes;}
\alinea{Rédiger les articles et communiqués officiels;}
\alinea{S'assurer de la présence d'un service de raccompagnement sur les lieux de l'événement;}
\alinea{Promouvoir l'événement à l'interne et à l'externe du campus;}
\alinea{Commander les articles promotionnels;}
\alinea{Il doit rester sobre tout au long de l’événement.}

\article{VP Bière et Bouffe}
\alinea{S'assurer d'avoir l'alcool et la nourriture nécessaire au bon déroulement de l'activité;}
\alinea{Obtenir le permis d'alcool;}
\alinea{S'assurer  d'avoir les équipements de cuisson et de service nécessaires;}
\alinea{Commander et préparer la nourriture pour maximiser la quantité de nourriture servie;}
\alinea{Communiquer les besoins en matière d'électricité au VP logistique;}
\alinea{S'assurer d'avoir le personnel et les installations nécessaires pour le service d'alcool et de nourriture;}
\alinea{Prévoir un plan de service de nourriture et de nettoyage;}
\alinea{Il doit rester sobre tout au long de l’événement.}

\partie{Comité de sélection du comité organisateur de l'APP}
\article{Composition du comité}
\sousarticle{Membres du comité}
\alinea{Le comité est formé de 4 membres de l'AGEG ou ayant déjà été membre d'un comité organisateur du Party APP Sherbrooke nommés au CA1 de la session d'automne.}
\alinea{La présidence du comité précédent a une place réservée sur le comité de sélection. Elle peut toutefois se retirer si elle le désire.}

\article{Comité plus}
\alinea {Le comité de sélection forme le \textit{comité plus} avec tous les membres de l’AGEG se présentant à la présidence de l'APP, n'ayant pas le droit de vote.}
\alinea {Le \textit{comité plus} procède aux entrevues individuelles des membres de l’AGEG se présentant pour les postes du comité organisateur.}
\alinea {Le comité procède aux entrevues individuelles de tous les membres de l’AGEG se présentant à la présidence de l'APP de l'Université de Sherbrooke.}

\sousarticle{Rémunération}
\alinea{Les membres du comité ne recevront aucune rémunération pour leur fonction.}

\article{Nomination et démission}
\sousarticle{Nomination des membres du comité}
\alinea{La durée du mandat des membres du comité est d’une session, la session d’été;}
\alinea{La priorité sera donnée aux membres ayant déjà participé à l'organisation de l'APP, puis aux membres du CA.}

\sousarticle{Démission}
\alinea{Toute démission d’un membre du comité devra être acheminée par écrit au VPAS de l’AGEG et sera effective après son approbation;}
\alinea{Un membre du comité peut se voir expulsé du comité par le CA de l’AGEG ou l’assemblée générale de l’AGEG pour des raisons jugées valables;}
\alinea{Un poste laissé vacant sera comblé par un membre de l’AGEG nommé par le VPAS de l’AGEG.}

\article{Décision}
\sousarticle{Critères}
\alinea{Les critères de sélection du comité organisateur de l’APP permettent au comité d’évaluer et d’analyser les demandes afin de soumettre les meilleures recommandations au conseil d’administration;}
\alinea{Les critères sont les suivants:}
\sousalinea{Compétences des candidats;}
\sousalinea{Idées des candidats quant à l'APP;}
\sousalinea{Implications antérieures et actuelles des candidats au niveau universitaire;}
\sousalinea{Participation antérieure des candidats quant à l'organisation de l'APP;}
\sousalinea{Représente la relève de l'implication étudiante;}
\sousalinea{Tout autre point jugé pertinent.}

\partie{Dispositions financières}
\article{Surplus et déficit zéro}
\alinea{L'APP ne doit pas engendrer de déficit;}
\alinea{Advenant un surplus, celui-ci devra être remis à l’AGEG à la fin de l’activité;}
\alinea{Advenant un déficit annuel, le comité organisateur devra en justifier clairement les causes au conseil d’administration dès que ce déficit est connu. Il devra aussi produire des recommandations dans le but que la situation soit régularisée au cours de la prochaine édition.}

\adoption{3 février 2019}{27 juin 2019}