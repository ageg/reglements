\reglement{Relatif aux groupes étudiants}

\preambule{L’objectif du présent règlement est de clarifier des points « nébuleux » , des règles non écrites et de fixer des balises quant à la mission d'un groupe, soit d’être ouvert à tous et d’être un phare pour l’Université, la Faculté et l’AGEG.}

\partie{Dispositions générales}
\article{Mission du présent règlement}
\alinea{D’assurer une bonne relation entre l’AGEG les groupes étudiants et le décanat.}
\alinea{Déterminer les critères qui font qu’un groupe est devenu inactif.}
\alinea{De spécifier les devoirs et responsabilités auxquels doivent répondre les groupes étudiants et l’AGEG.}
\alinea{De promouvoir l’implication dans la faculté.}

\article{Vision de l’AGEG face aux groupes étudiants}
\alinea{L’AGEG a le devoir d’encourager l’implication de ses membres au travers des groupes étudiants.}
\alinea{L’AGEG a aussi le devoir de s’assurer d’utiliser le plus efficacement les ressources à sa disposition pour rendre service aux groupes étudiants.}

\article{Valeurs dans l’accréditation des nouveaux groupes étudiants}
\valeurs
{Ouverture}{Donner la chance à tous les étudiants de faire partie d’un groupe étudiant}
{Engagement}{Soutenir les groupes étudiants dans la réalisation de leurs projets}
{Intégrité}{Être égal envers tous les groupes étudiants}
{Fraternité}{Promouvoir le sentiment d’appartenance par le biais de l’implication dans les groupes étudiants}

\article{Rôles et devoirs}
\sousarticle{AGEG}
\alinea{L’AGEG doit rendre disponible à tous les groupes étudiants l’achat de bière au prix avantageux du contrat de bière selon le Règlement 18.}
\alinea{L’AGEG doit permettre aux groupes techniques de soumettre une demande au comité d’attribution du fonds de subvention des groupes étudiants, selon le Règlement 25.}
\alinea{L’AGEG doit permettre aux groupes étudiants de soumettre une demande au comité d’attribution du fonds d’équipements étudiants, selon le Règlement 26.}
\alinea{L’AGEG doit permettre aux groupes étudiants de se présenter en CA pour faire une demande au fonds de la direction selon le Règlement 27.}
\alinea{L’AGEG doit permettre aux groupes étudiants de se présenter à la réunion de distribution des semaines de financement, selon le Règlement 28.}
\alinea{L’AGEG doit présenter au décanat toute compétition hors de l’ordinaire et mettre les groupes étudiants en relation avec celui-ci, selon le contrat entre la direction et l’AGEG.}
\alinea{L’AGEG a le devoir de supporter les groupes étudiants demandant de l’aide dans leurs démarches.}

\sousarticle{Groupe Étudiant}
\alinea{Le groupe étudiant doit s’assurer que son bilan financier est en santé.}
\alinea{Le groupe étudiant doit aviser l’AGEG par écrit concernant toutes les ententes contractées avec des parties externes ayant un impact sur la gestion des biens du groupe.}
\alinea{Le groupe étudiant a le devoir de se conformer à la politique de gestion des espaces de la faculté.}
\alinea{Le responsable du groupe étudiant doit s’assurer de transmettre l’information provenant de l’AGEG à ses membres et aux membres de l’autre alternance du groupe.}
\alinea{Le responsable du groupe a le devoir d’être informé des règlements et d’en informer ses membres.}
\alinea{Le groupe étudiant doit recruter activement des nouveaux membres.}
\alinea{Le groupe étudiant doit participer à des évènements qui augmentent la visibilité de la Faculté de génie, de l’Université de Sherbrooke ou de l’AGEG.}
\alinea{Le groupe étudiant doit aussi, par ses projets, augmenter les compétences ou les connaissances de ses membres.}
\alinea{Avant de signer l’entente à l’hiver, le responsable du groupe étudiant doit remettre le rapport annuel demandé par le doyen, selon le modèle qui lui aura préalablement remis.}
\alinea{Le groupe étudiant ne doit en aucun cas, commettre des actions entachant la réputation de l’Université de Sherbrooke, de la Faculté de Génie ou de l’AGEG.}
\alinea{Le groupe étudiant doit répondre aux exigences du Règlement 25.}
\alinea{Le groupe étudiant doit répondre aux exigences du Règlement 46.}
\alinea{Le groupe étudiant doit répondre aux exigences du Règlement 50.}
\alinea{Le groupe étudiant doit répondre aux exigences du Règlement 55.}

\partie{Accréditation d’un nouveau groupe technique}
\article{Pouvoirs du Conseil d’Administration}
\alinea{Le Conseil d’administration a le pouvoir d’accepter ou de refuser l’accréditation du groupe demandeur.}
\alinea{Le conseil d’administration doit prendre en considération les critères suivants :}
\sousalinea{Le nombre de membres fondateurs}
\sousalinea{Les buts du groupe (en valeur avec ceux de l’AGEG)}
\sousalinea{La viabilité du groupe}
\sousalinea{L’autonomie financière du groupe}
\sousalinea{Le sérieux de la mise en candidature}

\article{Devoirs du groupe demandeur}
\alinea{Le groupe doit remettre une liste des membres fondateurs.}
\alinea{Le groupe doit remettre une lettre de motivation au conseil d’administration.}
\alinea{Le groupe doit obtenir une lettre d’appui d’un professeur de la faculté de génie.}
\alinea{Le groupe doit remettre une description de leurs buts, des activités et des besoins du groupe.}
\alinea{Le groupe doit fournir un budget préliminaire au conseil d’administration.}
\alinea{Le groupe est responsable de se trouver au préalable un local pour la tenue de ses activités.}

\partie{Inactivité d’un groupe technique}
\article{critères d’inactivité d’un groupe technique}
\alinea{Un groupe est considéré actif pendant la validité de son entente.}
\alinea{Un groupe est considéré en sursis s’il ne renouvelle pas l’entente à la fin de sa période de validité. Tout renouvellement de l’entente au cours de l’année restaure le statut actif du groupe.}
\alinea{Un groupe est considéré inactif s’il ne renouvelle pas l’entente au cours de son année de sursis.}
\alinea{Le CA se réserve le droit de déclarer inactif (de dissocier), après convocation du groupe en CA, tout groupe ne respectant pas plus d'un alinéa de l’article 1.4.2 du présent règlement.}

\article{Conséquences de l’inactivité d’un groupe technique}
\alinea{Un groupe inactif se verra dans l’obligation de refaire une demande d’accréditation s’il veut redevenir actif.}
\alinea{Tous les équipements ayant été subventionnés par le fonds d’équipements étudiants appartenant au groupe déclaré inactif seront disposés tel que stipulé par le Règlement 26.}
\alinea{Tous les avoirs n’appartenant pas à l’un des membres du groupe déclaré inactif deviendront la possession de l’AGEG, sauf exception des groupes affiliés dont une clause sur les avoirs en cas d'incativité devra être présent dans l'entente entre le groupe et l'AGEG. L’association se chargera par la suite de redistribuer les avoirs.}

\adoption{16 septembre 2018}{27 novembre 2018}