\reglement{Relatif aux finances}
\preambule{Le présent règlement a pour but de faciliter le suivi des finances de l’AGEG et de donner des lignes directrices à la gestion financière des affaires courantes et des fonds de l’AGEG}

\partie{Dispositions générales}
\article{Mission du règlement}
\alinea{Délimiter les pouvoirs de l'Executif quant à la gestion des finances de l'AGEG}
\alinea{Définir l'utilisation des fonds de l'AGEG}
\alinea{Définir les politiques budgétaires de l'AGEG}

\article{Valeurs et rôles du règlement sur les finances}
\valeurs
{Ouverture}{Permettre aux membres d'avoir accès aux informations financières de l'AGEG présentées de façons claires.}
{Engagement}{Favoriser l'engagement des étudiants via des fonds pour financer leurs projets}
{Intégrité}{Remplir les devoirs fiscaux de l'AGEG}
{Fraternité}{Favoriser l'attachement à la faculté via le financencement d'objets promotionnels et en offrant des services ouverts à tous}

\partie{Préparation des états financiers}
\article{Prévisions budgétaires}
\alinea{La présentation des prévisions budgétaires annuelles doivent être faite durant le dernier CA de l'année financière précédante.}
\alinea{Les budgets doivent être adoptés séance tenante.}

\article{États financiers trimestriels (bilan)}
\alinea{Les états financiers mensuels sont préparés à \a chaque CA par le VPAF en poste.}

\article{États financiers annuels}
\alinea{Les états financiers annuels sont preparés par une firme comptable reconnue.}
\alinea{Ils doivent être présentés au CA3 de la session d’hiver.}

\article{Marge de manœuvre}
\alinea{Tous les exécédents de plus de 5\% du budget accordé devront être approuvés préalablement par le CA.}

\partie{Fonds}
\article{Fonds d’immobilisations}
\alinea{Le fonds d’immobilisation est un fonds créé selon les pratiques comptables habituelles et qui a pour but de gérer efficacement les transactions relatives aux immobilisations de l’AGEG.}
\alinea{On entend par immobilisations corporelles les terrains, l’équipement, le matériel de transport, le mobilier, les installations, etc. Les immobilisations incorporelles sont le droit au bail, les droits de propriété intellectuelle, etc. Il faut ajouter qu’une immobilisation est acquise dans le cadre de son exploitation plutôt que pour la revente. Tout ce qui touche l’entretien d’équipement ou les réparations mineures NE concerne PAS le fonds d’immobilisation.}

\article{Fonds de subvention des groupes étudiants}
\alinea{Le fonds de subvention des groupes étudiants est un fonds créé, selon les pratiques comptables habituelles et qui a pour but de gérer efficacement les transactions relatives aux subventions accordées par l’AGEG à différents groupes étudiants.}
\alinea{Les procédures spécifiques d’attribution et de gestion des subventions sont régies par le règlement 25.}

\article{Fonds de la direction}
\alinea{Le fonds de la direction est un fonds créé selon les pratiques comptables habituelles et qui a pour but de gérer efficacement les transactions relatives au fonds}
\alinea{Les procédures spécifiques de gestion du fonds sont régies par le règlement 27.}

\article{Fonds de Donation}
\alinea{Le fonds de donation est un fonds créé selon les pratiques comptables habituelles et qui a pour but de gérer efficacement les transactions relatives au fonds}
\alinea{Les procédures spécifiques de gestion du fonds sont régies par le règlement 29.}

\article{Fonds Équipement Étudiants}
\alinea{Le fonds Équipement Étudiants est un fonds créé selon les pratiques comptables habituelles et qui a pour but de gérer efficacement les transactions relatives au fonds.}
\alinea{Les procédures spécifiques de gestion du fonds sont régies par le règlement 26.}

\article{Fonds d'Administration}
\alinea{Le fonds d'Administration est créé selon les pratiques comptables habituelles et a pour but de gérer efficacement les surplus et les déficits réels de la corporation, ainsi que de donner à l’AGEG la capacité de gérer les mauvaises créances sans avoir recours à son budget général.}
\alinea{C’est dans ce fonds que sont placés les déficits ou les surplus réels de l’association à la fin de l’année financière.}
\alinea{Le financement de ce fonds est laissé à la discrétion du conseil d’administration, qui doit s’assurer que son solde reste positif en tout temps. Les montants des chèques non encaissés un an après leur date d’émission sont déposés dans ce fonds.}
\alinea{Le conseil d’administration a la liberté d’autoriser un transfert à l’extérieur de ce fonds. Cependant, le solde du fonds doit demeurer supérieur à 15 000~\$.}

\article{Fonds des finissantes}
\alinea{Le fonds des finissantes est créé selon les pratiques comptables habituelles et qui a pour but de contenir l'argent restant des finissantes afin de financer leurs retrouvailles. Il a aussi pour but d'éviter que les comptes inactifs soit fermés et de limiter les frais bancaires des promotions.}
\alinea{Les finissantes doivent fermer leur compte banquaire lors de leur départ de l'université et l'argent sera déposé dans ce fonds.}
\alinea{Le montant laissé dans le fonds par les finissantes doit être conservé dans le coffre-fort papier de l'AGEG.}
\alinea{Lors de leurs retrouvailles les finissantes pourront demander l'argent à l'AGEG. Tout argent non réclamé après 10 ans sera déposé dans le fonds de la réserve générale.}

\article{fonds de meuble}
\alinea{Le fonds de meuble sert à payer les frais reliés aux :}
\sousalinea{objets mobiles servant à l'aménagement des locaux.}
\sousalinea{biens tangibles destinés à être utilisés d’une manière durable durant le cycle d’exploitation de l’entreprise. On distingue dans cette catégorie les constructions, le matériel industriel, les agencements et les installations techniques, le matériel de transport, les équipements de bureau.}
\alinea{À chaque session, 6 \% du montant total des cotisations de l’AGEG de tous les membres doit être alloué à ce fonds.}


\partie{Politique budgétaire de fonctionnement}
\article{Politique de déficit zéro}
\sousarticle{Équilibre financier}
\alinea{L’année financière, débutant à la session d’hiver, doit avoir un budget de fonctionnement équilibré.}

\adoption{4 novembre 2018}{27 novembre 2018}