 \partie{Préambule} 


\article{Objectif des règlements généraux}
\alinea{Les règlements généraux de l'Association Générale Étudiante en Génie de l'Université de Sherbrooke décrivent la structure de l'association et de ces instances ainsi que les pratiques financières. Ils se veulent un engagement de l'association envers ses membres.}

\alinea{Les règlements généraux ont préséance sur le cahier de procédures ainsi que sur toute autre documentation de l'AGEG.}


\article{Objectif du cahier de procédures}
\alinea{Le cahier de procédures de l'Association Générale Étudiante en Génie de l'Université de Sherbrooke décrit l'ensemble des procédures en vigueur au sein de l'association. Il se veut un engagement de l'association envers elle-même et les groupes qui la composent.}

\article{Objectif du cahier de positions}
\alinea{Le cahier de positions de l'Association Générale Étudiante en Génie de l'Université de Sherbrooke décrit l'ensemble des positions en vigueur au sein de l'association.}


\article{Documents publics}
\alinea{Les présents règlements ainsi que le cahier de procédures sont des documents publics accessibles à tous les membres de la corporation.}


\article{Interprétation} 
\alinea{Dans les présents règlements:} 
\sousalinea{\textbf{Loi} désigne la Loi sur les compagnies, L.R.Q., c. C-38.} 

\sousalinea{\textbf{Association}, \textbf{corporation} ou \textbf{AGEG} désignent l'Association Générale Étudiante en Génie de l'Université de Sherbrooke.} 

\sousalinea{\textbf{Session} désigne une période de quatre mois correspondant aux sessions officielles d'études de l'Université de Sherbrooke.} 

\sousalinea{\textbf{Personne administratrice} désigne un ou une membre du conseil d'administration de la corporation.} 

\sousalinea{\textbf{Personne exécutante} désigne un ou une membre du comité exécutif de la corporation.} 

\sousalinea{\textbf{Université} désigne l'Université de Sherbrooke.} 


\article{Entrée en vigueur et modifications des règlements généraux} 
\alinea{Les règlements généraux sont adoptés, abrogés et modifiés par un vote au 2/3 du conseil d'administration et entrent en vigueur dès leur adoption par celui-ci. Ils doivent être ratifiés par les membres par un vote à majorité absolue lors de la prochaine assemblée annuelle ou générale suivant l'adoption des règlements généraux par le conseil d'administration.} 

\alinea{À défaut d'être ratifiés par les membres, ils cessent alors d'être en vigueur, mais sans effet rétroactif.} 

\alinea{Lors de leur entrée en vigueur, les présents règlements généraux abrogeront tous les règlements généraux antérieurs.} 


\article{Entrée en vigueur et modification du cahier de procédures}
\alinea{Le cahier de procédures est adopté ou modifié par un vote à la majorité simple du conseil d'administration et entre en vigueur dès son adoption par celui-ci. Les modifications sont présentées lors de la prochaine assemblée annuelle ou générale suivant l'adoption des règlements généraux par le conseil d'administration, mais une ratification n'est pas nécessaire.} 

\alinea{Lors de leur entrée en vigueur, le cahier de procédures abrogera toutes les versions antérieures du cahier de procédures.} 

\article{Entrée en vigueur et modification du cahier de positions}
\alinea{Le cahier de positions est adopté, abrogé et modifié par un vote à la majorité simple du conseil d'administration et entre en vigueur dès son adoption par celui-ci. Il doit être ratifié par les membres par un vote à majorité absolue lors de la prochaine assemblée annuelle ou générale suivant l'adoption du cahier de positions par le conseil d'administration.} 

\alinea{Lors de son entrée en vigueur, le cahier de positions abrogera toutes les versions antérieures du cahier de positions.} 

\alinea{Le cahier de position doit être révisé et adopté de nouveau lors de chaque assemblée générale annuelle pour s'assurer qu'il représente les positions actuelles des membres.}