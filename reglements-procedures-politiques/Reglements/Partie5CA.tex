\chapter{Conseil d'administration} 

\article{Composition et droit de vote} 
\alinea{Le conseil d'administration est composé d'un maximum de vingt-et-une (21) personnes administratrices votantes élues ou nommées et réparties en conformité avec les dispositions des présents règlements. De plus, le comité exécutif ainsi que la coordination administrative siègent au conseil d'administration.} 

\alinea{Les affaires de la corporation sont administrées par un conseil d'administration composé comme suit: 

\begin{itemize} 

 \item De la présidence, de la vice-présidence aux affaires légales, de la vice-présidence aux affaires financières, de la vice-présidence aux activités extracurriculaire et la personne agente de liaison. Toutes ces personnes possèdent un (1) vote; 

 \item Quatre (4) personnes administratrices annuelles avec un (1) vote chacune; 

 \item Un total de douze (12) personnes administratrices saisonnières avec un (1) vote chacune. 

\begin{itemize} 

\item Des personnes administratrices saisonnières, une place par promotion est réservée d'office pour des personnes représentantes de promotion avec un (1) vote chacune. Ces personnes sont nommées par les comités exécutifs des promotions. Une place est aussi réservée pour un représentant du comité des cycles supérieurs avec un (1) vote.

\item Des personnes administratrices saisonnières, une est réservée pour la personne responsable des activités des jeudis détentes. Ce siège n'est pas transférable en cas de vacance du poste ou si ladite personne est déjà élue en tant que personne administratrice. 

\end{itemize} 

\end{itemize} 
} 

\alinea{La personne agente de liaison est désignée par les personnes élues au conseil exécutif pour la session suivant la session courante. Son rôle est de représenter le prochain comité exécutif et les membres réguliers passifs. Son mandat est d'une session.}

\alinea{Une personne ne peut cumuler plusieurs sièges au sein du conseil d'administration, par exemple, un ou une  membre du comité exécutif ne peux pas occuper un poste de représentant de promotion.}

\alinea{Les personnes exécutantes qui n'ont pas de siège votant ainsi que la personne occupant la coordination administrative de la corporation sont des personnes observatrices privilégiées avec droit de parole.} 

\alinea{Deux (2) sièges de personnes observatrices privilégiées avec droit de parole sont réservés pour former de futures personnes administratrices n'ayant jamais effectuées un mandat au conseil d'administration. Ces sièges sont attribués par le conseil d'administration au début de chacune de ses réunions. Dans la mesure où les deux sièges ne sont pas comblés, un ou une membre ayant déjà effectué un mandat pourra obtenir le siège.}

\alinea{Toutes les personnes administratrices siègent à titre personnel et devront entre autres défendre les intérêts de toutes les membres et assurer la circulation bidirectionnelle de l'information entre l'ensemble des membres et le conseil d'administration.} 


\article{Éligibilité} 
\alinea{Tout membre régulier est éligible aux différents postes administratifs à condition de satisfaire aux critères relatifs au siège postulé.} 


\article{Durée des fonctions} 
\alinea{Le mandat de toute personne administratrice annuelle commence le 1er janvier suivant son élection et se termine le 31 décembre de la même année.} 

\alinea{Le mandat de toute autre personne administratrice commence le premier jour de la session pour laquelle la personne a été élue ou nommée et se termine a minuit la veille du premier jour de la session suivante.}


\article{Élection} 
\alinea{Les personnes administratrices annuelles sont élues à la session d'automne par les membres réguliers au cours d'une assemblée générale.} 

\alinea{Les personnes administratrices saisonnières qui ne représentent pas une des promotions, le comité organisateur des jeudis détentes ou le comité des cycles supérieurs sont élues deux sessions à l'avance par les membres réguliers au cours d'une assemblée générale.} 


\article{Mise en candidature} 
\alinea{Les procédures de mise en candidature pour les élections au conseil d'administration sont décrites dans le partie 8 du présent règlement.} 


\article{Vacances} 
\alinea{Le conseil d'administration a la responsabilité de pourvoir tout siège vacant sur son conseil pour le reste d'un mandat.} 

\alinea{Le conseil d'administration nomme des membres réguliers pour combler les postes vacants.} 

\sousalinea{Dans le cas où le poste vacant est celui d'une personne représentante de promotion, du comité organisateur des jeudis détentes, du comité des cycles supérieurs ou d'une personne agente de liason, ce poste doit être pourvu par le groupe concerné.} 
\sousalinea{Dans le cas où le poste vacant est celui d'une personne exécutante, celui-ci ne peut être comblé que par une personne nommée au même poste au sein du conseil exécutif.}


\article{Retrait d'une personne administratrice} 
\alinea{Cesse de faire partie du conseil d'administration et d'occuper sa fonction, toute personne administratrice qui:} 

\sousalinea{Présente par écrit sa démission au conseil d'administration;} 

\sousalinea{Décède, devient insolvable ou interdite;} 

\sousalinea{Cesse d'occuper le poste par lequel elle siège au conseil d'administration;} 

\sousalinea{Perd son éligibilité;} 

\sousalinea{Est destituée par un vote des deux tiers (2/3) des membres réguliers présentes à une assemblée générale spéciale convoquée à cette fin.} 


\article{Rémunération} 
\alinea{Les personnes administratrices de la corporation ne peuvent recevoir aucune forme de rémunération pour leurs fonctions.} 


\article{Indemnisation} 
\alinea{Toute personne administratrice, ses personnes héritières et ayants droit seront tenues, au besoin et à toute époque, à même les fonds de la corporation, indemne et couverte:} 
 

\sousalinea{de tous frais, charges et dépenses quelconques que ces personnes administratrices supportent ou subissent au cours ou à l'occasion d'une action, poursuite ou procédure intentée contre elle, à l'égard ou en raison d'actes ou choses accomplis ou permis par elle dans l'exercice ou pour l'exécution de ses fonctions;} 

\sousalinea{de tous autres frais, charges et dépenses qu'elle supporte ou subit au cours ou à l'occasion des affaires de la corporation ou relativement à ces affaires, exceptés ceux qui résultent de sa propre négligence ou de son omission volontaire.} 