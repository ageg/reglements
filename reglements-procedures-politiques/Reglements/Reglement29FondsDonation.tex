\reglement{Relatif à la gestion du Fonds de Donation}

\preambule{Attendu que le conseil de faculté a demandé qu’une partie significative des profits générés lors de la 3e heure des "Jeudi détente", soient versés à un organisme de la région de Sherbrooke, l’AGEG a cru bon de se doter d’un mécanisme de gestion des sommes recueillies lors d’activités de levée de fonds. L’objectif du présent règlement est de donner un cadre de fonctionnement au fonds de donation ainsi que de décrire les mécanismes de distribution des dons par l’AGEG. Ce réglement fut considérament modifiés à l'automne 2015, suite à une constatation que plusieurs financement d'organisme n'aboutissait qu'a de simple lettre de remerciement, alors que d'autre, amenés par les étudiants, étaient une grande source de fierté pour tous.}

\partie{Dispositions générales}
\article{Mission du Fonds de Donation}
\alinea{Établir un endroit unique où transiteront tous les montants recueillis lors de levées de fonds qui seront remis à des organismes de la région.}

\article{Vision du Fonds de Donation}
\alinea{La vision du Fonds de Donation est d’assurer une participation financière de l’AGEG à des organismes œuvrant dans la région de Sherbrooke.}
\alinea{Le présent réglement vise à encadrer le choix des organismes pour favoriser les partenariats participatifs en accord avec nos valeurs}

\article{Valeurs du Fonds de Donation}
\valeurs
{Ouverture}{Faire preuve de démocratie dans l’attribution des fonds par l’AGEG}
{Engagement}{Donner un engagement social à l’AGEG envers la communauté}
{Intégrité}{Assurer une attribution honnête et impartiale des fonds recueillis lors de levées de fonds.}
{Fraternité}{Augmenter la solidarité entre les étudiants et les organisme de charité de la région en favorisant des partenariats participatifs.}

\article{Rôles et pouvoirs}
\sousarticle{CA de l'AGEG}
\alinea{Il vote sur les propositions suite à la présentation du directeur du fonds de donation au sujet des projets d'attribution du fonds.}
\alinea{Il nomine le directeur du fonds de donations lors du CA1 de l’hiver.}
\alinea{Forme le comité décrit à l’article 1.4.3 lors du CA1 de l’hiver.}

\sousarticle{Directeur du fonds de donation}
\alinea{Il préside le comité d'attribution du fonds de donations}
\alinea{Il rédige un rapport à la fin de son mandat sur l'utilisation de l'argent du fonds et des partenariats développés à travers celui-ci.}
\alinea{Il présente au CA de l'AGEG les conclusions du comité d'attribution du fonds de donation.}
\alinea{Il fait le suivi avec le VPAF de l'argent au sein du fonds.}
\alinea{Il convoque le comité si nécessaire.}

\sousarticle{Comité}
\alinea{Il invite des membres à proposé des idées de partenariats avec des oeuvres de charité}
\alinea{Il propose l’attribution des montants recueillis selon les critères d’attribution du fonds définis dans le présent règlement.}
\alinea{Le comité est formé impérativement au début de la session d’hiver, mais il peut être modifié lors des autres sessions lorsqu’une activité nécessite une attribution immédiate et que certains membres du comité ne sont pas disponibles.}
\alinea{S’assure que les donations et les résultats soient publicisés.}

\partie{Comité d’attribution des subventions}
\article{Composition du comité}

\sousarticle{Membres du comité}
\alinea{Le comité est formé d’un représentant de chacune des promotions nommées par le CA, du directeur du fonds de donation ainsi que du VPAF actif de l’AGEG.}

\sousarticle{Rémunération}
\alinea{Les membres du comité ne recevront aucune rémunération pour leur fonction.}

\article{Décision}
\sousarticle{Critères}
\alinea{Les critères d’attribution permettent au comité du fonds de donation d’évaluer et d’analyser les demandes afin de soumettre les meilleures recommandations au conseil d’administration.}
\alinea{L'organisme doit :}
\sousalinea{Se définir comme un organisme ayant des valeurs qui sont soutenues par l’AGEG et œuvrant dans la région de Sherbrooke et qui n’est pas en lien direct avec la Faculté de génie (Ne pas être un groupe technique de l'AGEG par exemple);}
\sousalinea{Avoir un besoin de soutien financier;}
%\sousalinea{Présenter une demande officielle, par le biais du formulaire uniformisé, au Comité du Fonds de Donation afin que la demande soit traitée et analysée.}
\alinea{Le comité doit favorisé les partenariats qui:}
\sousalinea{Favorisent la participation des membres de l'AGEG}
\sousalinea{Permettent une visibilité à l'association et a ses membres}

\sousarticle{Répartition des fonds disponibles}
\alinea{La totalité du Fonds doit être attribuée à chaque année.}
\alinea{Le comité se rencontre lorsque le directeur le juge nécessaire, suite à des demandes des membres.}
\alinea{À la fin de l’année financière, soit avant le CA4 de l’automne, le comité se rencontre pour attribuer l’argent restant dans le fonds et pour faire un bilan qui sera présenté en CA4 de l’automne.}

\sousarticle{Quorum}
\alinea{Pour toute assemblée officielle du comité, il est nécessaire d’avoir la présence \remove{de tous les} membres en plus du VPAF. Aucune décision ne sera prise à moins que cette assemblée n’ait le quorum requis.}
\alinea{Les décisions du comité se prendront par vote à majorité. Chaque membre, incluant le VPAF, a droit à un vote. Toutefois, les décisions peuvent être prises par voie électronique, à l'unanimité.}

\article{Attribution}
\sousarticle{Attribution des fonds}
\alinea{Le directeur du fonds de donation, suite à la décision du CA, informe les demandeurs du don accordé.}

\partie{Fonds de Donation}
\article{Sources du Fonds}
\sousarticle{Provenant des "Jeudi détente"}
\alinea{La promotion finissante doit verser une somme équivalente à 5~\% des profits générés lors des "Jeudi détente" sur toute la période où elle a eu la responsabilité d’organiser les "Jeudi détente". Cette somme sera versée au fonds de donation de l’AGEG lors de la passation des pouvoirs à la session d’automne.}

\sousarticle{Provenant des activités à but charitable}
\alinea{Toutes les sommes recueillies lors d’une activité de levée de fonds à but charitable devront être déposées dans ce fonds avant leur attribution.}

\adoption{18 Juin 2017}{27 juillet 2017}