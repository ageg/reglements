\reglement{Relatif aux objets promotionnels de l’AGEG}

\preambule{Il définit les lignes directrices visant à augmenter l’appartenance des étudiants de Génie envers l’AGEG et la visibilité de celle-ci lors d’activités regroupant les associations membres de la CRÉIQ et la FCEEG ou de toutes autres compétitions majeures.}

\partie{Dispositions générales}
\article{Mission du présent règlement}
\alinea{Déterminer les sources de financements.}
\alinea{Régenter les processus de dépenses relatifs à ce fond.}
\alinea{Constituer une ligne directrice pour l'achat d'articles promotionnels.}
\alinea{Déterminer les types d'objets promotionnels.}
\alinea{Assurer une saine gestion financière du fonds.}

\article{Valeurs prônées par les objets promotionnels et le COPS}
\valeurs
{Ouverture}{Offre à l'AGEG une ouverture et une visibilité accrue}
{Engagement}{Engage l'AGEG dans sa pérennité et sa représentativité}
{Intégrité}{S'assure que les objets promotionnels ne causent pas de torts à qui que ce soit}
{Fraternité}{Promouvoir la fierté d'être à Sherbrooke à l'interne comme à l'externe}

\article{Rôles et devoirs du CA de l’AGEG}
\alinea{Nominer un directeur en CA.}
\alinea{Recevoir et adopter les idées d’objets et les dépenses du comité.}
\alinea{Fournir un espace de rangement sécurisé au comité.}

\article{Comité COPS}
\sousarticle{Formation et composition}
\alinea{Le directeur COPS doit être nommé au CA1 de la session.}
\alinea{Il doit y avoir au moins trois (3) personnes dans le comité.}
\alinea{Le directeur COPS nomme son comité.}

\sousarticle{Durée d'un mandat}
\alinea{Le mandat du comité est d’une durée d’une session.}

\sousarticle{Rôles et pouvoirs}
\alinea{Le comité est responsable de l'inventaire des objets promotionnels. Il doit assurer une saine gestion de l'inventaire, promouvoir les articles et s'assurer d'offrir des périodes de ventes adéquates.}
\alinea{Le comité peut présenter une ou plusieurs idées d'objet promotionnel à chaque CA de la session active. Le CA décidera si l'article promotionnel est adéquat pour les besoins de l'AGEG.}
\alinea{Le comité est responsable des contacts avec les groupes étudiants qui voudraient avoir des objets promotionnels dans le cadre de leurs fonctions.}
\alinea{Seuls les membres du comité et du CE de l’AGEG ont accès à l’inventaire des objets et seuls les membres du comité doivent en assumer la gestion physique et financière.}
\alinea{Une personne du comité sortant sera désignée comme agent de liaison avec le nouveau comité.}

\sousarticle{Rapport d’avancement}
\alinea{Le rapport d’avancement devra être remis au dernier CA de la session.}
\alinea{Le rapport devra comprendre, au minimum, les informations suivantes :}
\sousalinea{L’inventaire de tous les articles promotionnels;}
\sousalinea{Une liste de prix auxquels les articles promotionnels se vendent;}
\sousalinea{La soumission des articles promotionnels achetés au courant de la session;}
\sousalinea{Une liste de contacts détaillée des personnes ressources des fournisseurs;}
\sousalinea{Tous problèmes survenus avec un fournisseur lors de la commande, de la réception ou du suivi des dossiers avec celui-ci.}

\partie{Dispositions financières}
\article{Articles promotionnels récurrents}
\alinea{Se décrit comme article promotionnel récurent un ou des objets uniformes qui seront distribués lors d’activités de l’AGEG, de la CRÉIQ ou de la FCEEG.}
\alinea{L’objectif premier des articles promotionnels est la visibilité du Génie de l’UdeS. (ex. : S d’or plus que le logo de l’AGEG)}
\alinea{Le décanat pourra acheter au prix coutant les articles promotionnels de l’AGEG.}
\alinea{Il sera possible à tous les membres de l’AGEG, délégations et groupes étudiants de se procurer ces articles avant un événement suivant la disponibilité du comité.}

\article{Articles promotionnels non récurrents}
\alinea{Se décrit comme article promotionnel non récurrent un objet au tirage limité qui ne sera pas reproduit dans un délai de 1 an.}
\alinea{La moitié de l’inventaire de cet objet sera réservé à l’alternance suivante.}
\alinea{La distribution ne pourra en aucun cas favoriser un ou des membres l’AGEG en particulier.}
\alinea{La date de lancement de l’objet devra être annoncée à l’avance par les méthodes promotionnelles habituelles de l’AGEG.}
\alinea{Une personne ne pourra acheter qu’un nombre limité d’objets lors du lancement. À la fin de la période de lancement, toute personne peut s’en procurer le nombre voulu.}
\alinea{La période de lancement suggérée est de 2 jours 
ouvrables et sera décidée par le COPS.}

\adoption{3 avril 2016}{7 avril 2016}