\reglement{Relatif au prix Jacques Bazinet}

\preambule{L’objectif du présent règlement est de donner des lignes directrices pour la remise du Prix Bazinet. Le prix Jacques Bazinet est un prix remis à des enseignants de la Faculté de génie chaque automne, lors du gala du mérite étudiant. Aux fins de ce règlement, le terme « enseignant » correspondra aux professeurs, chargés de cours ou techniciens qui participent à la formation des étudiants du premier cycle.}

\partie{Dispositions générales}
\article{Mission de l’attribution du Prix Bazinet}
\alinea{La mission du Prix Bazinet est de rendre hommage annuellement aux enseignants les plus appréciés des étudiants durant leur baccalauréat. Ce prix permet de reconnaître l’effort des enseignants et leur disponibilité auprès des étudiants, se décrivant par une bonne relation enseignant/étudiant s’étalant même jusqu’à l’extérieur de leurs missions titulaires.}

\article{Vision de l’attribution du Prix Bazinet}
\alinea{La vision de l’attribution du Prix Bazinet est de valoriser le travail des enseignants de la Faculté de génie.}

\article{Valeurs et rôles du Prix Bazinet}
\valeurs
{Ouverture}{Permettre aux étudiants au cours de leurs études de valoriser le travail de leurs enseignants}
{Engagement}{Enrichir les liens entre les étudiants et les membres de la Faculté}
{Intégrité}{Tous les étudiants sont appelés à reconnaître ouvertement l’investissement des enseignants}
{Fraternité}{Développer le sentiment d’appartenance à la communauté et à la Faculté de génie}

\article{Catégories du prix Bazinet}
\alinea{Les catégories où les enseignants peuvent être mis en nomination sont les suivantes : génie du bâtiment, génie biotechnologique, génie chimique, génie civil, génie électrique, génie informatique, génie mécanique et génie robotique.}

\article{Rôles et pouvoirs}
\sousarticle{VPAP de l’AGEG}
\alinea{Il doit distribuer les feuilles de vote à tous les représentants de concentration de la promotion finissante au début de la dernière session du baccalauréat (S8).}
\alinea{Il doit s’assurer que les représentants feront bien comprendre la procédure du vote et le but du prix aux étudiants.}
\alinea{Il doit s’assurer que les représentants feront circuler les feuilles de vote et les lui remettre avant la date déterminée.}
\alinea{Il doit comptabiliser les votes avant le gala, préparer les prix et les remettre au gala du mérite étudiant organisé par la Faculté de génie.}

\partie{Fonctionnement du vote}
\article{Formulaire de votation}
\alinea{Les étudiants devront voter en inscrivant le nom de l’enseignant méritant selon leur catégorie.}
\alinea{Les formulaires doivent être identifiés, selon la catégorie, avant leur remise aux représentants de l'AGEG pour avoir un contrôle à leur retour.}
\alinea{Les formulaires doivent être disponibles à l'AGEG ou via courriel, accompagnés d’une lettre décrivant la nature du prix Jacques Bazinet, les critères et la procédure à suivre pour voter.}

\article{Vote}
\alinea{Chaque étudiant finissant membre de l’AGEG a le droit de vote.}
\alinea{Si un étudiant ne considère pas qu’un de ses enseignants se soit démarqué selon les critères définis ci-haut, il peut s’abstenir de voter.}

\article{Comptabilisation}
\alinea{Les votes doivent être comptabilisés au moins un mois avant le gala du mérite étudiant, afin d’avoir assez de temps pour préparer les prix.}
\alinea{Les trois meilleurs enseignants par catégorie seront nommés. Celui qui sera convoqué lors du gala du mérite étudiant sera l’enseignant ayant recueilli le plus de votes.}

\adoption{19 octobre 2014}{27 novembre 2014}