\reglement{Relatif à la gestion du matériel de la radio étudiante et du système de son}

\preambule{Le conseil d’administration a demandé la révision complète des méthodes de gestion d’opération et d’emprunt du matériel de la radio étudiante. Le présent règlement a pour but de normaliser et de clarifier les pratiques d’utilisation et de prêt du matériel cité en rubrique. Il faut aussi mentionner que l’équipement de la radio sera consigné au local de la radio qui lui sera assigné par l’AGEG. Le matériel de la radio étudiante est composé d’un kit de son mobile et d’un kit de son fixe installé dans le salon et la cafétéria de la Faculté de Génie.}

\partie{Dispositions générales}
\article{Mission de la radio étudiante et de son équipement}
\alinea{Favoriser la détente et le divertissement des participants des "Jeudi détente", de pouvoir prêter, aux groupes qui en font la demande et qui satisfont les conditions d’emprunt, un matériel de qualité.}

\article{Vision de la radio étudiante et de son équipement}
\alinea{La vision de la radio étudiante est d’assurer une communication efficace entre les organisateurs d’événements et les participants.}

\article{Valeur de la radio étudiante}
\valeurs
{Ouverture}{Respecter les opinions et les goûts \newline Dialogue entre les organisateurs et les participants}
{Engagement}{Promouvoir l’engagement social des membres}
{Intégrité}{Promouvoir le sens des responsabilités des groupes ou personnes qui empruntent l’équipement}
{Fraternité}{Promouvoir le sentiment d’appartenance à la faculté en fraternisant lors des activités}

\article{Rôles et pouvoirs}
\sousarticle{CA de l'AGEG}
\alinea{Elle vote sur les propositions du conseil exécutif relatif à l’ajout, la modification ou tout autre geste ayant trait à la vocation ou l’utilisation de l’équipement de la radio.}

\sousarticle{Vice-président affaires internes}
\alinea{Il est l’exécutant responsable de la location ainsi que de l’équipement de la radio.}

\sousarticle{Promotion finissante}
\alinea{Elle est le principal opérateur de l’équipement.}
\alinea{Elle est le gestionnaire de l’équipement et doit être au courant de tout mouvement ou usage de l’équipement de la radio.}
\alinea{Elle doit être présent lors de la sortie du matériel ainsi qu’au retour.}
\alinea{Elle est responsable de l’inventaire de l’équipement de la radio.}
\alinea{Elle est responsable de la propreté du local de la radio.}
\alinea{Elle est responsable de l’inspection de l’équipement lors de la sortie du matériel lors d’un prêt ainsi qu’à son retour.}
\alinea{Une clé du local de la radio sera allouée à la promotion finissante après la signature d’un contrat de prêt de clé renouvelable à chaque session}

\sousarticle{Directeur AGEGBand}
\alinea{L’AGEGBand aura à suivre tout autre règlement de location d’équipements tel que décrit dans la Partie II, l’article 2.1., à l’exception de la signature du contrat de prêt qui est fait à la formation de l’AGEGBand. Ce contrat doit être signé par tous les membres du dit groupe.}

\partie{Utilisation de l’équipement et du local}
\article{Location du kit de son mobile ou du kit de son fixe de la radio.}
\alinea{La location de l’équipement de la radio doit être faite selon les règles suivantes :}
\sousalinea{La réservation doit être faite via le formulaire d’emprunt disponible à l’AGEG.}
\sousalinea{Le dépôt nécessaire doit être acquitté lors de la sortie du matériel.}
\sousalinea{Le dépôt est fixé à 500~\$.}
\alinea{Le matériel doit être remis dans le même état qu’il était lors de la sortie.}
\alinea{Seul le directeur radio, le président, le VPAF ainsi que le VPAI ont accès à ce local en tout temps.}
\alinea{L’équipement est disponible en tout temps pendant la semaine. La promotion finissante à priorité sur l’équipement lors des "Jeudi détente" (voir Art.2.2). Afin de louer l’équipement les fins de semaines, le locataire doit venir le chercher le vendredi avant la fermeture des bureaux et le ramener le jour ouvrable suivant avant 10 h.}
\alinea{La manutention du kit de son mobile doit être faite avec le chariot et les housses prévues à cet effet.}

\article{"Jeudi détente"}
\alinea{L’équipement de la radio doit être en tout temps disponible pour la réalisation des "Jeudi détente" à la faculté.}
\alinea{Le directeur radio doit être présent lors de la manutention de l’équipement.}
\alinea{Cet événement a priorité sur tout autre événement qui nécessiterait l’équipement de la radio.}

\partie{Sanction}
\article{Remise du local de la radio en état}
\alinea{Suite à l’utilisation du local de la radio à des fins autres que pour les "Jeudi détente", le groupe doit remettre le local en état. Le défaut de respecter cette consigne résultera en la rétention du dépôt du groupe.}
\alinea{À la suite de la sortie de l’équipement du local par un groupe, le groupe doit remettre le local dans le même état qu’il était initialement. De plus, il doit s’assurer que tous les équipements sont rebranchés et prêts à être utilisés par le directeur radio en cas de besoin. Le défaut de respecter cette consigne résultera en la rétention du dépôt du groupe.}

\article{Dommage, bris ou perte d’équipement}
\alinea{En cas de dommage, bris ou perte d’équipement de la part d’un groupe qui emprunte l’équipement de la radio, la totalité du dépôt sera retenue jusqu’à la réparation ou remplacement de l’équipement. Les modalités de remboursement seront les suivantes :}
\sousalinea{Si le coût de la réparation ou remplacement de l’équipement est inférieur au montant du dépôt, l’AGEG remboursera la différence au groupe.}
\sousalinea{Si le coût de la réparation ou du remplacement de l’équipement est supérieur au montant du dépôt, le groupe perdra complètement son dépôt et sera exigé de défrayer tout montant supérieur à 500~\$.}

\adoption{3 avril 2016}{7 avril 2016}