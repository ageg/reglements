\reglement{Relatif à la gestion de l'imprimante 3D de l'Ageg}

\preambule{Dans le cadre d'une demande d'équipement pour le fond Féé, l'AGEG a décidé de s'équiper d'une imprimante 3D à grande capacité dans le but de démocratiser l'impression 3D auprès de ses membres. Ce règlement explique les différentes modalitées d'utilisaion et les bonnes pratiques nécessaires à l'utilisation et la durabilité de l'imprimante.}

\partie{Dispositions générales}
\article{Fonctionnement général}
La structure proposée dans ce règlement est une structure qui favorise l'apprentissage de tous les utilisateurs, une charge minimale de travail des responsables de l'imprimante et une rigueur monétaire assurée par les permanentes. Les membres de l'AGEG sont responsables de fournir des pièces fonctionnelles et imprimables aux permanentes. Celles-ci sont responsables de démarrer les impressions, de tenir un registre des impressions et récolter l'argent. Les directeurs Imprimante 3D sont responsables de l'entretien et peuvent aider les utilisateurs à préparer leurs fichiers. Le VPAI gère les litiges et s'assure du bon fonctionnement de la structure de gestion.

\article{Mission de l'imprimante 3D}
Rendre accessible l'impression 3D au plus grand nombre de membres de l'AGEG dans un cadre responsable et contrôlé qui permettra d'assurer la durabilité de l'imprimante et une gestion saine des commandes, du matériel d'apport et des impressions. Le système de gestion de ce règlement veut favoriser l'apprentissage des techniques d'impression 3D par l'utilisateur plutôt que d'offrir un service de programmation et d'impression. Cet apprentissage a pour but de favoriser ces trois éléments:
\alinea{L'acquisition de connaissances qui seront très utiles dans la carrière du membre de l'AGEG.}
\alinea{Minimiser de la charge de travail du VPAI, des directeurs imprimante et de la permanente.}
\alinea{Former de nouveaux directeurs Imprimante 3D potentiels.}
 
\article{Vision de l'utilisation de l'imprimante 3D}
La vision de l'utilisation de l'imprimante 3D est une vision à long terme qui assurera la pérennité de l'investissement fait par l'AGEG dans cette imprimante à grande capacité de qualité.

\article{Valeurs de l'imprimante 3D}
\valeurs
{Ouverture}{Donner un accès équitable à l'impression 3D à tout les membres de l'AGEG.}
{Engagement}{Promouvoir l'apprentissage et la maitrise d'une technologie d'avenir.}
{Intégrité}{Favoriser l'utilisation responsable et juste de l'imprimante.}
{Fraternité}{Promouvoir le sentiment d'appartenance à la faculté en donnant accès à des outils de développement et de design avant-gardistes et qui deviendront bientôt la norme.}

\article{Rôles et pouvoirs}

\sousarticle{Vice-président formation étudiante}
\alinea{Il est l’exécutant responsable de la gestion de l'imprimante 3D.}
\alinea{Il es responsable de l'entretien et d'assurer la périnité de l'imprimante 3D.}
\alinea{Il est responsable de trouver et choisir le(les) directeur(s) imprmimante 3D.}

\sousarticle{Directeurs Imprimante 3D}
\alinea{Ils sont les principaux responsables de l’imprimante 3D.}
\alinea{Ils sont responsables d'assurer une transmission des connaissances avec les prochains directeurs/VPAI.}
\alinea{Ils sont les responsables de l'entretien saisonnier de l’imprimante 3D.}
\alinea{Ils sont responsables de vérifier que l'entretien quotidien est fait et que l'imprimante est en ordre.}
\alinea{Ils sont responsables de former les permanentes au début de la session.}
\alinea{Ils sont responsables du débogage de l'imprimante.}
\alinea{Ils sont responsables d'aider les membres de l'AGEG qui ont des questions APRÈS AVOIR LU le Guide d'Impression 3D de l'AGEG.}

\sousarticle{Permanentes de l'AGEG}
\alinea{Les permanantes sont responsables  de tenir un registre des impressions à faire et des impressions faites.}
\alinea{Elles sont responsables de recevoir et ajouter les demandes d'impressions des membres de l'AGEG au registre.}
\alinea{Elles sont responsables de démarrer les impressions, de récupérer les pièces et de récupérer le paiment du membre.}

\sousarticle{Membre de l'AGEG}
\alinea{Le membre de l'AGEG est responsable de lire le Guide d'Impression 3D de l'AGEG.}
\alinea{Il est responsable des pièces qu'il envoie en impression, incluant les défauts d'impression qui peuvent résulter d'erreurs de sa part.}
\alinea{Il est responsable de connaître les paramètres d'impression et de les ajuster avant d'envoyer les pièces en impression.}
\alinea{Il est responsable de payer et venir chercher la pièce imprimée.}
\alinea{Il est responsable de respecter la procédure décrite dans le Guide de l'Impression 3D de l'AGEG sans quoi la pièce imprimée lui sera confisquée et il sera obligé de payer la pièce.}

\partie{Précisions}
\article{Tarif}
Afin d'assurer une accessibilité maximale, un prix au gramme de le plus bas possible est établi par le VPAI. Ce prix permet de couvrir les coûts du matériel utilisé et les différents coûts d'entretien. L'argent qui n'est pas utilisé pour acheter le matériel d'apport sera transféré vers le fonds d'administration de l'AGEG.

\article{Pièces avec des défauts}
L'impression 3D n'est pas une méthode de fabrication parfaite. En effet, l'imprimante 3D MakerBot Replicator Z18 peut se montrer parfois capricieuse et même un peu imprévisible. Dans l'éventualité où une pièce logique et bien construite aurait des défauts d'impression significatifs, les directeurs peuvent décider de ne pas facturer la pièce défectueuse au membre de l'AGEG et de la recommencer.

\article{Procédure de commande d'une impression 3D}
La marche à suivre afin de commander l'impression d'une pièce avec l'imprimante 3D de l'AGEG est détaillée dans le Guide d'Impression 3D de l'AGEG disponible sur "endroit à déterminer". Les étapes importantes sont les suivantes:
\alinea{Lire le Guide de l'Impression 3D de l'AGEG.}
\alinea{Produire les fichiers conformes selon le Guide de l'Impression 3D de l'AGEG.}
\alinea{Faire parvenir les fichiers nécessaires à la permanente.}
\alinea{Venir chercher la pièce imprimée à l'AGEG et payer la somme due.}

\article{Disponibilité}
\alinea{L'imprimante 3D est un service qui se veut comparable au service de techniciens du département de génie mécanique. C'est un service disponible sur les heures ouvrables (8h-17h) et pour lequel il faut prévoir un certain délai de traitement et de fabrication.}
\alinea{L'impression de pièces prioritaires pendant ou en dehors des heures d'ouverture reste à la discretion des directeurs. Pour compenser le dérangement et le délai imposé aux autres pièces, le coût demandé au membre de l'AGEG sera équivalent à 3 fois celui de la pièce.}

\adoption{3 avril 2016}{7 avril 2016}