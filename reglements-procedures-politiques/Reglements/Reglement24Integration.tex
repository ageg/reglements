\reglement{Relatif à l’intégration de la nouvelle promotion d’étudiants à la Faculté de génie de l’Université de Sherbrooke}

\preambule{Les objectifs du présent règlement sont de :
\begin{itemize}
\item[]{Donner des lignes directrices au comité d’intégration afin de le guider dans l’organisation de l’événement;}
\item[]{Effectuer le contrôle et le suivi de cette activité essentielle à l’adaptation de chaque étudiant.}
\end{itemize}
}

\partie{Dispositions générales}
\article{Mission de l’Intégration}
\alinea{Informer les membres de l’AGEG au sujet des différentes facettes de la vie étudiante, de mettre en œuvre des activités qui permettront de développer un sentiment de confort et d’appartenance ainsi que d’encourager l’adhésion aux valeurs et aux principes de l’AGEG et de la Faculté de génie.}

\article{Vision du comité d’intégration}
\alinea{La vision du comité d’intégration est d’offrir aux nouveaux étudiants en génie la meilleure intégration du campus de l’Université de Sherbrooke ainsi que des autres universités québécoises.}

\article{Valeurs et rôles de l’Intégration}
\valeurs
{Fraternité}{Enrichir les liens entre les étudiants afin de développer la fraternité dans le groupe}
{Soutien \& Adaptation}{Soutenir les étudiants au cours de leurs études tout en les aidant à s’adapter à leur nouveau style de vie}
{Information}{Transmettre toutes les informations nécessaires relatives à la vie étudiante et au domaine académique}
{Appartenance}{Développer le sentiment d’appartenance à la communauté et à la Faculté de génie}

\article{Objectifs de l’Intégration}
\alinea{L’objectif général de l’intégration est l’organisation, la tenue et la supervision d’activités permettant à la nouvelle promotion de génie de s’intégrer au milieu de vie de l’Université de Sherbrooke, via :}
\sousalinea{La transmission d’informations essentielles à l’intégration des nouveaux étudiants;}
\sousalinea{La mise en œuvre d’activités variées répondant aux besoins de chaque étudiant;}
\sousalinea{L’élimination complète de toute forme d’intimidation, d’abus de pouvoir et de risques pour la santé;}
\sousalinea{Le développement de liens fraternels entre les étudiants de la Faculté, nouveaux comme anciens;}
\sousalinea{Le développement d’un sentiment d’appartenance;}
\sousalinea{L'élimination complète de toute chanson à caractère sexuel, dégradante ou ne participant pas à l'intégration des nouveaux.}

\article{Composition}
\alinea{Le Comité organisateur de l’intégration est constitué de membres réguliers de l’AGEG, avec une priorité pour ceux issus de la plus récente promotion à l’exception de l’agent de liaison et des mentors.}

\article{Mandat}
\alinea{Le comité organisateur de l’intégration a comme mandat d’organiser, de tenir et de superviser les activités d’intégration. Il est responsable de  transmettre les informations sur l’avancement de la planification des activités au décanat et au CA.}
\alinea{Recruter et former tous les bénévoles nécessaires à l’activité d’intégration pour assurer le soutien des nouveaux étudiants dès leur arrivée à la faculté. Les formations recommandées sont Éduc-alcool, premiers-soins (5~\%) et MAPAQ.}
\alinea{Avertir les bénévoles et les chefs d’équipe que les chansons et activités à caractère sexuel, dégradantes ou ne participant pas à l’intégration ne seront pas acceptées sur les prémisses de l’université}

\partie{Rôles et pouvoirs}
\article{AGEG}
\alinea{L’AGEG, représenté par l’agent de liaison et les mentors, s’assure du suivi des objectifs et de la mission de l’intégration. En cas de non-respect des valeurs exprimées à travers les objectifs et la mission de l’intégration ou en cas de conflit avec les autorités de l’université, l’AGEG a un droit de veto sur toutes activités découlant du Comité organisateur.}

\article{Comité organisateur}
\alinea{En conformité avec la mission, les valeurs et les objectifs du présent règlement, le Comité organisateur de l’intégration a le pouvoir décisionnel sur tous les aspects organisationnels et des activités reliées à l’intégration;}
\alinea{Le Comité organisateur de l’intégration choisit la façon d’élire les directeurs et l'ensemble des bénévoles;}
\alinea{Le Comité organisateur de l’intégration se réserve le droit de destituer ou d'exclure toutes personnes advenant un comportement inadmissible avant ou durant la tenue des activités d’intégration;} 
\alinea{Faire une rencontre mensuelle de suivi avec l'agent de liaison;}
\alinea{Assurer la prise de notes et les procès-verbaux lors des réunions;}
\alinea{Organiser les activités de financement;}
\alinea{Être responsable du budget d'intégration;}
\alinea{Être signataire du compte bancaire de l'intégration;}
\alinea{Faire le suivi avec le VPAI au sujet des articles promotionnels et des possibilités d'achat commun;}
\alinea{Veiller à l'application de la politique de développement durable et de l'écoresponsabilité de l'Université de Sherbrooke;}
\alinea{S'assurer du rayonnement des étudiants et de l'association lors des activités d'intégration;}
\alinea{Être responsable de la planification et du bon déroulement de la sécurité des événements.}

\article{Président du comité}
\alinea{Superviser le travail du Comité organisateur;}
\alinea{S’occuper des communications à l’intérieur du Comité organisateur;}
\alinea{Convoquer et diriger les réunions du comité organisateur de l’intégration et des sous-comités;}
\alinea{Produire un rapport d’avancement déposé au CA4 de la session d'hiver et au CA3 de l'été;}
\alinea{Être signataire du compte bancaire de l’intégration;}
\alinea{Produire un rapport final sur la planification et le déroulement des activités d’intégration, qui sera déposé au CA3 de la session d’automne;}
\alinea{Séparer les tâches entre les différents vice-présidents du comité;}
\alinea{Être présent à la distribution des semaines d’activités sociales aux sessions d’hiver et d’été;}
\alinea{Rassembler et fournir au comité d’intégration suivant toute la documentation non formelle accumulée.}

\article{Agent de liaison}
\alinea{S’assurer que la sécurité et l’intégrité de tous les membres sont respectées durant l’ensemble des activités sur le campus comme à l’extérieur de l’Université;}
\alinea{Faire le lien entre le comité et le CA, la sécurité de l’Université, la Faculté et les départements;}
\alinea{Être en charge, durant l’intégration, du respect des directives de sécurité adoptées par le CO;}
\alinea{Produire un rapport sur les incidents importants s’étant déroulés durant la semaine;}
\alinea{Il doit rester sobre, disponible et présentable tout au long de l’événement;}
\alinea{Doit remettre le présent règlement au comité organisateur;}
\alinea{Porte-parole des intégrations auprès des institutions externes;}
\alinea{Participe à une rencontre mensuelle avec le comité organisateur.}

\article{Mentors}
\alinea{Assister le Comité organisateur de l'intégration dans ses prises de décisions selon son expérience, en participant à chacune des réunions;}
\alinea{Diriger le Comité organisateur de l’intégration vers les personnes-ressources de l’Université et de la Faculté pour assurer
le bon déroulement des activités d’intégration;}
\alinea{Assister le comité organisateurs lors des sélections des participants aux intégrations;}
\alinea{Avoir un comportement exemplaire lors de l’évenement.}

\article{Chef animateur}
\alinea{Est selectionné par le comité organisateur parmis les mentors;}
\alinea{Organise une rencontre avec les animateurs tous les matins de l’évènement;}
\alinea{S’assure de la présence des animateurs aux activités;}
\alinea{Est responsable de la gestion des animateurs lors de l’évènement;}
\alinea{Faire figure d’exemple.}

\partie{Sélections des organisateurs}

\article{Élection du Comité organisateur de l’intégration}
\alinea{Un des membres du Comité organisateur de la dernière année est nommé président d’élection pour le nouveau Comité organisateur. Advenant qu’aucun ne soit disponible, c’est l’AGEG qui s’en occupe;}
\alinea{La période de mise en candidature et de vote, le nombre de postes ouverts et la procédure de nomination sont décidés par le président d’élection;}
\alinea{La nouvelle composition du Comité organisateur de l’intégration doit être entérinée par le CA de l’AGEG;}

\article{Sélection des mentors et de l'agent de liaison}
\alinea{Le comité de sélection des mentors et de l’agent de liaison est composé du VPAX, du VPAS, d’un ancien membre du
CO et de deux membres de l’association;}
\alinea{Le comité de sélection doit remettre ses recommendations au CA4 de l’automne;}
\alinea{L’agent de liaison doit idéalement être un membre du CA ou un ancien membre du CA;}
\alinea{Au moins un mentor doit être présent à la session d’hiver et d’été.}


\partie{Dispositions financières}

\article{Gestion financière}
\alinea{Les signataires du compte des intégrations sont le président et deux membres du comité organisateur;}
\alinea{L’organisation et les activités de l’intégration ne doivent pas engendrer de déficit;}
\alinea{Advenant surplus, celui-ci devra être remis à l’AGEG à la fin de l’activité;}
\alinea{Advenant déficit, le comité organisateur devra justifier clairement les causes au conseil d’administration, au plus tard au CA2 de l’automne. Il devra aussi produire des recommandations dans le but qu’une situation semblable ne se reproduise plus;}
\alinea{Le budget de l'intégration doit respecter les normes établies par le règlement 50 sur les finances des groupes étudiants et des promotions.}

\article{Commandite}
\alinea{La corporation s’engage à fournir une commandite au comité exécutif de l’intégration du comité organisateur de l’intégration après approbation du rapport d’avancement détaillant l’organisation, le choix et le déroulement des activités.}
\alinea{La commandite est conditionnelle à la présentation d'un budget prévisonnel respectant les normes établies par le règlement 50.}

\adoption{2016}{6 avril 2017}