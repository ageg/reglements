\reglement{Relatif à l’emprunt d’argent par l’AGEG}

\preambule{Attendu que l’Association Générale des Étudiants en Génie de l’Université de Sherbrooke (« AGEG ») a besoin, pour répondre à ses fins, de faire, de temps à autre, des emprunts auprès d'une institution bancaire, les administrateurs de l’AGEG adoptent le règlement suivant :}

\partie{Dispositions générales}
\article{Rôles et pouvoirs}
\sousarticle{CA}
\alinea{Le CA est autorisé par un vote au deux tier, par les présentes, à faire, de temps à autre, des emprunts d'argent auprès d'une institution bancaire, à valoir sur le crédit de l’AGEG, pour les montants qu'ils jugent convenables, et sous forme d'emprunt à découvert ou autrement, à condition que lesdits montants ne dépassent pas cinq cents dollars (500,00 \$).}
\alinea{Pour tout montant excédant cinq cents dollars (500,00 \$), le CA doit obtenir au préalable le consentement des 2/3 des membres présents lors d’une assemblée générale dûment convoquée à cette fin.}

\partie{Modalités d’application}
\article{Documents}
Sous réserve de l’article 1.1. ci-dessus :
\alinea{Tout billet à ordre (promissory note) ou tout autre effet négociable (y compris  les renouvellements entiers ou partiels) couvrant lesdits emprunts, incluant les intérêts convenus y relatifs, donnés à ladite institution bancaire et signés pour le compte de l’AGEG par les administrateurs dûment autorisés à signer pour le compte de cette dernière des instruments négociables engage l’AGEG;}
\alinea{Les administrateurs, s'ils le jugent à propos, peuvent donner de temps à autre des garanties, notamment sous forme d'hypothèque, sur la totalité ou une partie des propriétés ou des valeurs formant l'actif de l’AGEG, couvrant la totalité ou une partie des emprunts contractés par l’AGEG auprès de l'institution bancaire, ou couvrant  toute  autre obligation contractée auprès de l'institution bancaire et toute garantie ainsi donnée est valide et engage l’AGEG si elle est signée par les administrateurs autorisés à signer les instruments négociables pour le compte de l’AGEG.}
\alinea{Tous les contrats, actes, documents, concessions et assurances qui peuvent être requis par l’institution bancaire relativement à l’objet du présent règlement doivent être exécutés, fournis et effectués par les administrateurs dûment autorisés et le sceau de l'AGEG doit être apposé lorsque requis.}

\partie{Entrée en vigueur}
\article{Ratification}
\alinea{Le présent règlement entre en vigueur dès sa ratification par le vote d’au moins les deux tiers (2/3) des membres de l’AGEG présents à une assemblée générale dûment convoquée à cette fin. Ainsi, tous les pouvoirs et les droits conférés aux termes des présentes demeurent en vigueur tant et aussi longtemps qu’un autre règlement au même effet ne le remplace et qu’une copie de la résolution y relative ait été remise à l’institution bancaire, le cas échéant.}

\adoption{3 avril 2016}{7 avril 2016}
