\chapter{Assemblée générale des membres}


\article{Date et lieu des assemblées générales}

\alinea{Les assemblées générales ont lieu sur le campus principal de l'Université de Sherbrooke, dans un local déterminé par le comité exécutif de l'association.}

\alinea{Au minimum une assemblée générale doit avoir lieu à chaque session sauf en cas de force majeure. La date et l'heure des assemblée générales est fixée par le comité exécutif ou le conseil d'administration.}


\article{Assemblée générale régulière}

\alinea{L'assemblée générale régulière peut être convoquée par le conseil d'administration ou le comité exécutif. Un avis de convocation doit être envoyé par courriel au moins quatre jours ouvrables avant l'assemblée et doit contenir le cahier de propositions ainsi que tous les règlements auxquels des modifications sont proposées.}


\article{Assemblée annuelle}

\alinea{L'assemblée générale annuelle des membres de la corporation aura lieu durant la session suivant la fin de l'année financière de la corporation. À l'assemblée générale annuelle sont approuvés les rapports financiers annuels et le vérificateur financier est nommé. Un avis de convocation est adressé par courriel à tous les membres au moins sept (7) jours avant l'assemblée.}


\article{Assemblée générales spéciale}

\alinea{Le conseil d'administration, le conseil exécutif ou vingt (20) membres réguliers peuvent, selon les besoins, convoquer une assemblée générales spéciales (AGS) au lieu, à la date et à l'heure désirés. Les personnes convoquant l'AGS doivent donner un délai de deux (2) jours ouvrables aux membres pour cette réunion. La présidence de l'AGEG est alors tenue de convoquer cette assemblée. Le conseil d'administration ou le conseil exécutif procède par résolution, tandis que le groupe de vingt (20) membres réguliers ou plus doit produire une demande écrite, signée par ces vingt (20) membres réguliers ou plus. L'avis de convocation doit énoncer le ou les buts de cette assemblée et l'ordre du jour ne peut être modifié.}

\alinea{Un avis de convocation est adressé par courriel à tous les membres dès la réception de la convocation par la présidence. L'avis doit contenir les sujets à être traités lors de l'assemblée.}

\alinea{La présence d'un ou d'une membre régulier à une assemblée couvre le défaut d'avis quant à cette personne.}

\alinea{L'omission accidentelle de cet avis ou la non-connaissance de cet avis par toute personne n'a pas pour effet de rendre nulles les résolutions adoptées à cette assemblée.}


\article{Assemblée générale extraordinaire}

\alinea{Une assemblée générale extraordinaire ne peut être convoquée que par une pétition signée par au moins cinquante (50) membres réguliers ou par le conseil d'administration. La pétition peut convoquer une assemblée générale extraordinaire au lieu, à la date et à l'heure souhaités, tout en respectant les délais de convocation d'une assemblée générale spéciale.}

\alinea{Un avis de convocation doit être envoyé par la présidence de l'AGEG et doit contenir le cahier de proposition de l'assemblée. L'avis de convocation doit énoncer le ou les buts de cette assemblée et l'ordre du jour ne peut être modifié.}

\alinea{Seulement une assemblée générale extraordinaire peut adopter une grève.}


\article{Quorum}

\alinea{Le quorum d'une assemblée générale régulière, annuelle ou spéciale est constitué de 2~\% des membres réguliers, tandis que celui pour une assemblée générale extraordinaire est de 10~\% . Il est nécessaire que le quorum subsiste pendant toute la durée de l'assemblée.}


\article{Conflit d'intérêts}

\sousarticle{Principe}

\alinea{Toute personne en conflit d'intérêts est responsable de faire savoir rapidement à la présidence d'assemblée qu'elle est en conflit d'intérêts dès que la question est traitée.}

\alinea{Ne sont pas considérées en conflit d'intérêts les personnes qui reçoivent un avantage financier du simple fait d'être membre de la corporation.}

\sousarticle{Nature}

\alinea{Sans restreindre la généralité de ce qui précède, est considérée comme situation conflictuelle typique ;}

\alinea{L’attribution de subventions ou autres avantages aux groupes étudiants tout membre dudit groupe ;}

\alinea{L’affiliation à un organisme externe, la modification de la cotisation à cet organisme ou l’octroi d’avantages quelconques à cet organisme, tout membre de l’exécutif ou employée de cet organisme ;}

\alinea{L’attribution d’un contrat ou la ratification d’une entente avec une société ou compagnie dont le ou la membre est employé, actionnaire, dirigeante ou mandataire}


\article{Vote}

\alinea{À une assemblée des membres, les membres réguliers ont droit à un vote chacun. Les votes par procuration ne sont pas acceptés.}

\alinea{Le vote se fait à main levée, à moins que trois (3) membres réguliers ne réclament le scrutin secret. En cas de vote au scrutin secret, la présidence d'assemblée nomme deux (2) personnes scrutatrices qui distribuent et recueillent les bulletins de vote, compilent les résultats et les communiquent à la présidence d'assemblée.}

\alinea{Il est possible de voter par anticipation par la méthode inscrite dans l'avis de convocation. La présidence d'assemblée compile les résultats reçus qui seront ajoutés au vote de l'assemblée. Le résultat final est communiqué aux membres par celui-ci.}

\alinea{Une proposition est adoptée à majorité absolue des voies. En cas d'égalité des voies, la proposition est battue.}


\article{Code de procédures des assemblées et réunions}

\alinea{Le code Morin est utilisé pendant toutes les assemblées générales. La procédure aux assemblées délibérantes est décrite dans le livre Procédure des assemblées délibérantes / Victor Morin; mise à jour par Michel Delorme Éditions Beauchemin, 1994, à moins de dispositions contraires prévues dans les présents règlements.}


\article{Présidence et secrétariat d'assemblée}

\alinea{La présidence et le secrétariat d'assemblée sont élus au début de l'assemblée. La présidence doit veiller au bon déroulement de l'assemblée et au respect des règles d'assemblée. Le secrétariat d'assemblée est responsable de l'écriture du procès-verbal de l'assemblée. La présidence et le secrétariat ne doivent idéalement pas être des membres réguliers de l'AGEG.}


\article{Procès verbaux publics}

\alinea{Les procès-verbaux de la corporation sont publics, donc ouverts aux membres de la corporation, exceptés les procès-verbaux des huis clos.}
