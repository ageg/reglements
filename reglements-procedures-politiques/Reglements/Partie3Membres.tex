h
\chapter{Membres}

\article{Membres}
\alinea{Sont membres de la corporation les membres réguliers actifs et passifs ainsi que les personnes membres honoraires. Les membres doivent adhérer aux objectifs de la corporation.}

\sousarticle{ Membres réguliers actifs}
\alinea{Est un ou une membre régulier actif de l'AGEG toute personne étudiante inscrite à un programme de premier cycle, de deuxième cycle, de troisième cycle ou à un stage postdoctoral de la Faculté de génie de l'Université de Sherbrooke en session d'étude ou de rédaction, exceptées les personnes rattachées uniquement à un programme du centre de développement professionnel. Ses droits sont de participer à toutes les activités de la corporation, de recevoir les avis de convocation aux assemblées des membres, d'assister à ces assemblées et d'être éligibles à titre de personnes administratrices ou exécutantes de la corporation. Ces membres ont un droit de vote.}
\sousarticle{Membres réguliers passifs}
\alinea{Est un ou une membre réguliers passifs de l'AGEG toute personne étudiante inscrite à un programme de premier cycle de la Faculté de génie de l'Université de Sherbrooke qui n'est pas en session d'étude. Ses droits sont de participer à toutes les activités de la corporation, de recevoir les avis de convocation aux assemblées des membres, d'assister à ces assemblées et d'être éligibles à titre de personnes administratrices de la corporation. Ces personnes membres ont un droit de vote.}

\sousarticle{Membres honoraires}
\alinea{Il est loisible au conseil d'administration, par résolution, de nommer membre honoraire de la corporation toute personne qui aura rendu service à la corporation par son travail ou par ses donations, ou qui aura manifesté son appui à la mission poursuivie par la corporation.}

\alinea{Les membres honoraires peuvent participer aux activités de la corporation et assister aux assemblées des membres, mais n'ont pas le droit de voter lors des assemblées. Ces personnes ne sont pas éligibles comme personnes administratrices de la corporation, et ne sont pas tenues de verser des cotisations ou contributions à la corporation.}


\article{Cotisations}
\sousarticle{Cotisation étudiante}
\alinea{Tous les membres réguliers, actifs ou passifs, paient une cotisation étudiante. Le conseil d'administration peut, par résolution, fixer le montant des cotisations, de même que le moment, le lieu et la manière d'effectuer le paiement. Cette résolution doit être entérinée lors d'une assemblée générale spéciale convoquée uniquement à cette fin. La cotisation étudiante est obligatoire et n'est pas remboursable même dans les cas de radiation, de suspension ou de retrait.}
\alinea{Cette cotisation est fixée pour les membres actifs au premier cycle à temps plein à 45\$ par session pour l'année de référence 2016. Cette cotisation est indexé de 2.5\% au début de chaque année financière. Les membres aux cycles supérieurs, paient une cotisation correspondant au 8/13 de la cotisation des membres actifs au premier cycle. Pour les étudiants à temps partiel, le montant est calculé au prorata du nombre de crédits, la cotisation régulière correspondant à 12 crédits. Les membres passifs au premier cycle paient une cotisation de 0.01\$.}

\sousarticle{Cotisation au fonds équipement étudiant}
\alinea{Seuls les membres réguliers actifs du premier cycle paient la cotisation au fonds équipement étudiant. Cette cotisation inclue l'accès au Studio de création. Le conseil d'administration peut, par résolution, fixer le montant des cotisations, de même que le moment, le lieu et la manière d'effectuer le paiement. Cette résolution doit être entérinée lors d'une assemblée générale spéciale convoquée uniquement à cette fin. La cotisation au fonds équipement étudiant est obligatoire et n'est pas remboursable même dans les cas de radiation, de suspension ou de retrait.}
\alinea{Cette cotisation est fixé à pour les étudiants actifs du premier cycle à temps plein 50\$ par session pour l'année de référence 2016. Cette cotisation est indexé de 2.5\% au début de chaque année financière.. Pour les étudiants à temps partiel, le montant est calculé au prorata du nombre de crédits, la cotisation régulière correspondant à 12 crédits.}

\sousarticle{Cotisation pour l'accès au studio de création}
\alinea{Seuls les membres des cycles supérieurs paient pour l'accès au Studio de création. Le conseil d'administration peut, par résolution, fixer le montant des cotisations, de même que le moment, le lieu et la manière d'effectuer le paiement. Cette résolution doit être entérinée lors d'une assemblée générale spéciale convoquée uniquement à cette fin. La cotisation pour l'accès au Studio de création est obligatoire et n'est pas remboursable même dans les cas de radiation, de suspension ou de retrait.}
\alinea{Cette cotisation est fixé à 11,40\$ par session pour les étudiants des cycles supérieurs.}

\sousarticle{Cotisation CRÉIQ-FCEG}
\alinea{Seuls les membres réguliers actifs du premier cycle paient la cotisation CREIQ-FCEG. Le montant de la cotisation CREIQ-FCEG est fixé par les instances de ces deux associations étudiantes. Ce montant est ensuite entériné par le conseil d'administration et lors d'une assemblée générale spéciale convoquée uniquement à cette fin. Si le conseil d'administration ou l'assemblée des membres refusent d'entériner la modification, l'association est tout de même tenue de faire le paiement nécessaire à l'affiliation à la CRÉIQ et à la FCEG. La cotisation CRÉIQ-FCEG est obligatoire et n'est pas remboursable même dans les cas de radiation, de suspension ou de retrait.}
\alinea{Cette cotisation est fixé à 0.50\$ par session pour tous les membres actifs du premier cycle.}

\sousarticle{Cotisation accès universel Vert et Or}
\alinea{Toutes les membres réguliers, actifs ou passifs paient la cotisation accès universel vert et or. Le conseil d'administration peut, par résolution, fixer le montant des cotisations, de même que le moment, le lieu et la manière d'effectuer le paiement. Cette résolution doit être entérinée lors d'une assemblée générale spéciale convoquée uniquement à cette fin. Le Vert et Or de l'Université de Sherbrooke est responsable des remboursements de cette cotisation.}
\alinea{Cette cotisation est fixé à 10\$ pour tous les membres de l'AGEG à la session d'hiver seulement.}

\article{Référendum ou plébiscite}
\alinea{Le conseil d'administration peut, par résolution, organiser un référendum ou plébiscite sur tout point qu'il jugera bon de consulter les membres. Pour ce faire, le conseil d'administration nomme, par résolution, une présidence d'élection chargée du scrutin. La présidence d'élection doit s'assurer que toutes les membres réguliers, ou un sous-groupe de membres réguliers identifié par le conseil d'administration, aient l'opportunité de voter. Elle peut s'adjoindre de personnes scrutatrices. Le scrutin est secret.}


\article{Retrait ou démission}
\alinea{Tout membre peut, en tout temps, se retirer ou démissionner de la corporation en signifiant ce retrait par écrit à la présidence de la corporation.}


\article{Code de conduite}
\alinea{L'AGEG est dotée d'un code de conduite adopté par le conseil d'administration. Ce code de conduite est un document public accessible à tous les membres de la corporation.}

\alinea{Le conseil d'administration se réserve le droit de limiter les services accordés à un ou une membre ayant enfreint le code de conduite.}


\article{Suspension et radiation}
\alinea{Dans l'éventualité où le conseil d'administration considère qu'un ou une membre ait enfreint les dispositions des présents règlements ou dont la conduite ou les activités sont jugées nuisibles, le conseil d'administration devra convoquer le membre afin qu'elle réponde de ses actes lors d'une séance spéciale du conseil d'administration convoquée à cet effet. Par la suite, le conseil d'administration pourra, par résolution, procéder à une suspension ou radiation du membre.}

\alinea{Le membre pourra porter appel pour chacune des décisions à un comité disciplinaire formé pour cette demande. La décision du comité sera finale et sans appel.}


\article{Comité disciplinaire}
\alinea{Le comité disciplinaire est formé uniquement lorsqu'une personne fait appel à une décision de suspension ou radiation.}

\alinea{Le comité disciplinaire est formé de trois personnes;}

\sousalinea{Si la personne à la direction des affaires étudiantes de la faculté est disponible, elle siège sur le comité avec un ou une membre ne connaissant pas de près la personne portant appel ainsi qu'un ou une membre du conseil d'administration.}

\sousalinea{Dans le cas où la personne à la direction des affaires étudiantes de la faculté n'est pas disponible, une autre personne du corps professoral ou de la direction de la faculté peut la remplacer.}

\sousalinea{Advenant le cas où aucune personne du corps professoral ou de la direction de la faculté est disponible, un ou une autre non-membre de l'AGEG siégera sur le comité.}


\article{Politique de harcèlement et sur les violences à caractère sexuel}
\alinea{L'association s'engage à tenir des activités et des instances exempt de harcèlement ou de violences à caractère sexuel sous toutes ses formes.}

\alinea{L'AGEG est dotée d'une politique contre le harcèlement et les violences à caractère sexuel adoptée par le conseil d'administration. Cette politique est un document public accessible à toutes les membres de la corporation.}
