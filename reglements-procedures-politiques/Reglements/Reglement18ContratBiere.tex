\reglement{Relatif au contrat de bière}

\preambule{L’objectif du présent règlement est de démontrer les principaux enjeux relatifs au contrat de bière.}

\partie{Dispositions générales}
\article{Mission de l’AGEG par rapport au contrat de bière}
\alinea{Le but du contrat de bière est d’augmenter et de faciliter l’organisation des activités sociales dans notre Faculté.}

\article{Vision de l’AGEG par rapport au contrat de bière}
\alinea{La vision de l’AGEG est d’offrir un produit de qualité à meilleur prix lors des activités de génie.}

\article{Valeurs de l’AGEG à propos du contrat de bière}
\valeurs{Intégrité}{Assurer le respect des clauses du contrat de bière}
{Ouverture}{Assurer une communication efficace entre tous les intervenants}
{Fraternité}{Être fier d’être lié à une brasserie de qualité}
{Engagement}{Offrir un service de qualité aux membres}

\article{Rôles et pouvoirs}
\sousarticle{Président de l’AGEG}
\alinea{Il est responsable de la gestion du contrat avec le vice-président aux affaires financières et la promotion finissante.}
\alinea{Il doit s’assurer qu’il ait, en tout temps, une étroite communication entre le vice-président aux affaires financières et la promotion finissante, afin d’éviter le dédoublement de tâches.}
\alinea{Il est signataire de tout document autorisé en CA pour et au nom du CA qui a trait à l’entente entre l’AGEG et la brasserie.}

\sousarticle{Vice-président aux affaires légales}
\alinea{Il est responsable de la gestion du contrat avec le président et la promotion finissante.}
\alinea{Il a la responsabilité de négocier les ententes ainsi que de mener les dossiers de fond, soit ce qui a trait aux prix de présence, à la résolution de problèmes, etc. En fait, tout ce qui n’a pas trait à la gestion des commandes comme telles.}
\alinea{Le VPAL de l’été a la responsabilité de faire l’évaluation du contrat sur la dernière année.  Il doit, pour ce faire, évaluer les forces et les faiblesses de notre entente afin d’optimiser le partenariat entre les deux parties.}
\alinea{L’année où le contrat arrive à échéance, le VPAL de l’été est responsable de négocier une nouvelle entente. S’il le juge pertinent, il peut faire le tour du marché pour voir ce qui pourrait nous être offert par une autre brasserie.}

\sousarticle{Vice-président aux affaires financières de l’AGEG}
\alinea{Il supervise la promotion finissante dans la gestion quotidienne, soit ce qui a trait à la facturation, aux commandes et aux relations avec la brasserie, etc.}
\alinea{Il est responsable de la distribution des argents remis par la brasserie selon l’entente survenue entre les deux parties.  Il se doit de distribuer les sommes de façon équitable, en respectant tous les règlements internes de l’AGEG et toutes les modalités du contrat.}

\sousarticle{Promotion finissante}
\alinea{Elle est responsable de la gestion du contrat avec le vice-président aux affaires financières et le président.}
\alinea{Elle est responsable de faire la promotion de l’entente auprès des différents regroupements et organismes de la Faculté autorisés à bénéficier de l’entente.}
\alinea{Elle est responsable de voir à ce qu’il n’y ait aucune action posée sur le campus de l’Université par des étudiants de génie qui pourrait nuire à notre entente.}
\alinea{Elle est responsable d’organiser les activités qui ont rapport à l’entente, par exemple des visites de la brasserie, etc.}
\alinea{Elle est responsable de la gestion quotidienne, soit ce qui a trait à la facturation, aux commandes et aux relations avec la brasserie, etc.}
\alinea{Elle est responsable de tenir à jour l’inventaire de tous les articles promotionnels attribués par la brasserie.  Elle doit pour ce faire tenir une liste détaillée des articles reçus, des inventaires accumulés et de tous les articles ayant été distribués ainsi que leur récipiendaire.}
\alinea{Elle reçoit les demandes de bière de la promotion finissante, de l’AGEG et des autres groupes, et passe les commandes de bière de manière hebdomadaire. Elle reçoit les commandes avec l’aide de la promotion finissante pour la réception.}

\partie{Contrat}
\article{Renégociation}
\alinea{Le contrat devra toujours être négocié pour se terminer au mois d’août.}

\article{Particularités}
\alinea{Aucun responsable de la gestion du contrat ne pourra recevoir de rémunération soit monétaire ou matérielle pour son travail}
\alinea{Le contrat est à huis clos et ne devra jamais devenir public, même après l’échéance de celui-ci.}
\alinea{Seuls les membres du CA ont le droit de prendre connaissance de l’entente globale.}
\alinea{L’entente sera conservée en permanence dans le coffre de la corporation. En aucun temps et sous aucun prétexte elle ne devra être divulguée à qui que ce soit.}
\alinea{Par contre, sur décision unanime du CE ou sur approbation du CA, une partie ou l’ensemble du contrat pourra être divulgué à un membre de l’AGEG dont les services seraient requis pour aider à gérer ledit contrat.}

\article{Organisme reconnu}
\alinea{Seuls les groupes et regroupements qui sont reconnus sur le campus comme étant des organismes ont le droit de se prévaloir des différents services offerts grâce à l’entente.}

\article{Exclusion de l’entente}
\alinea{Il est entendu que si un organisme ou l'un de ses membres posait des actions pouvant nuire à l’entente entre l’AGEG et la brasserie, il se verrait interdire l’accès aux différents services offerts par cette entente.}
\alinea{Il est entendu que tout organisme qui tenterait de passer outre les responsables, soit en contactant eux-mêmes la brasserie ou tout autre acte considéré comme pouvant porter préjudice à l’entente subira les sanctions citées plus haut.}
\alinea{Un organisme qui est associé de près ou de loin à une autre brasserie se verra exclu de l’entente.}

\article{Durée de l’exclusion}
\alinea{L’organisme qui est associé avec une brasserie concurrente sera exclu tant et aussi longtemps qu’il lui restera des obligations à remplir face à cette brasserie.}
\alinea{Pour les autres délits, une décision en CA permettra de juger de la longueur de l’exclusion.}

\article{Articles promotionnels provenant de 
l’entente}
\alinea{La vente d’articles promotionnels sera réservée à la promotion finissante.}
\alinea{L’AGEG se réserve le droit de vendre ces articles si le besoin se fait sentir.}
\alinea{Le prix de vente des articles promotionnels ne devra en aucun cas dépasser le prix de détail suggéré par la brasserie.}

\article{Prix de présence}
\alinea{Les organismes qui organisent des activités sociales sont tenus de distribuer aux participants les articles promotionnels qu’ils ont reçus de l’AGEG pour une activité déterminée et ce, le plus équitablement possible lors de ce même événement.}
\alinea{En aucun cas, un organisateur ne pourra conserver pour son usage personnel ni même utiliser le privilège offert par l’AGEG à des fins personnelles.}
\alinea{Tout organisme dont un de ses membres se placerait dans cette position devrait payer à l’AGEG le prix de vente suggéré par la brasserie du produit en question sous peine de se voir retirer l’accès aux différents services de l’entente tel que stipulé à l’article 2.4 du présent règlement.}
\alinea{Les produits qui ne seraient pas utilisés lors de l’événement pour lequel ils ont été attribués devront être remis aux responsables de l’AGEG sous peine des mêmes sanctions que précédemment.}

\article{Engagement de l’AGEG envers les organismes de la Faculté}
\alinea{La seule contribution que l’AGEG se réserve est l’attribution des gratuités.}

\adoption{3 avril 2016}{7 avril 2016}