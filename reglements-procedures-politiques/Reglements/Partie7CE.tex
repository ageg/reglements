
\chapter{Comité exécutif} 

\article{Composition} 
\alinea{Le comité exécutif est composé de douze (12) membres. Celui-ci est composé de : 

\begin{itemize} 

 \item la Présidence (PREZ); 

 \item la Vice-Présidence aux Affaires Légales (VPAL); 

 \item la Vice-Présidence aux Affaires Universitaires (VPAU); 

 \item la Vice-Présidence aux Affaires Internes (VPAI); 

 \item la Vice-Présidence à l'Image et aux Communications (VPIC); 

 \item la Vice-Présidence aux Affaires Sociales (VPAS); 

 \item la Vice-Présidence aux Activités Extracurriculaires (VPAX); 

 \item la Vice-Présidence aux Affaires Financières (VPAF); 

 \item la Vice-Présidence aux Affaires Externes (VPEX); 

 \item la Vice-Présidence aux Affaires Pédagogiques (VPAP); 

 \item la Vice-Présidence au Développement Durable (VPDD); 

 \item la Vice-Présidence aux Cycles Supérieurs (VPCS).

\end{itemize} 
} 


\article{Présidence} 

\alinea{La présidence est la personne exécutante en chef de la corporation. Elle voit à l'exécution des décisions du conseil d'administration. Elle remplit tous les devoirs inhérents à sa charge de même qu'elle exerce tous les pouvoirs qui pourront lui être attribués par le conseil d'administration. Elle s'assure que toutes les personnes exécutantes de son conseil exécutif remplissent les devoirs inhérents à leur charge.} 

\alinea{La présidence signe tous les documents qui requièrent sa signature. Elle a le contrôle général et la surveillance des affaires de la corporation.} 

\alinea{La présidence est la seule personne a avoir le droit d'autoriser le partage d'informations privilégiées.}


\article{Vice-Présidence aux Affaires Légales} 
\alinea{La vice-présidence aux affaires légales s'occupe de l'ensemble des dossiers légaux de la corporation. Elle s'assure que l'ensemble des déclarations annuelles envers le gouvernement sont produites et que les contrats en cours, les règlements généraux et le cahier de procédures de la corporation soient respectés. Elle est signataire des chèques et autres effets de commerce en l'absence de la présidence ou de la vice-présidence aux affaires financières. En cas d'absence ou d'incapacité d'agir de la présidence, elle la remplace et exerce tous ses pouvoirs et toutes ses fonctions. Elle est responsable d'appuyer la présidence dans l'ensemble de ses tâches.} 

\alinea{La vice-présidence aux affaires légales occupe aussi le poste de secrétariat de la corporation. Elle doit signer les procès-verbaux du conseil d'administration et du comité exécutif. Elle doit aussi s'assurer que l'ensemble des documents prévus par la loi est rempli et disponible pour les personnes ayant le droit de les consulter.} 


\article{Vice-Présidence aux Affaires Universitaires} 
\alinea{La vice-présidence aux affaires universitaires s'occupe des relations avec les associations, les instances et les organismes de l'Université de Sherbrooke. Elle s'occupe de la représentation de l'AGEG auprès de la FEUS.} 


\article{Vice-Présidence aux Affaires Internes} 
\alinea{La vice-présidence aux affaires internes s'occupe principalement des groupes et des ressources humaines de l'AGEG, de la qualité des services offerts aux membres, de l'entretien des biens meubles et des locaux de la corporation ainsi que de la gestion des produits vendus par celle-ci.}


\article{Vice-Présidence à l'Image et aux Communications} 
\alinea{La vice-présidence à l'image et aux communications s'occupe de l'image publique de la corporation ainsi que de la communication entre l'association, ses instances et ses membres. De plus, elle s'occupe des procédures de communication interne de l'exécutif, des employés et du recrutement des membres pour les divers comités, instances et groupes de la corporation. Elle est porte-parole de la corporation. Elle parle donc au nom de l'AGEG sur tous les dossiers nécessitant une intervention médiatique et promeut et/ou publicise les positions, actions et évènements de l'AGEG. Elle est responsable des demandes de commandites soumises à l'AGEG.} 

 
\article{Vice-Présidence aux Affaires Sociales} 
\alinea{La vice-présidence aux affaires sociales doit organiser et coordonner les activités sociales pour les membres de la corporation. Il supervise l'ensemble des comités en lien avec les activités sociales et fait la gestion des dossiers se rapportant aux activités sociales de la corporation.} 

 
\article{Vice-Présidence aux Activités Extracurriculaires } 
\alinea{La vice-présidence aux activités extracurriculaires assure la relation entre les groupes étudiants, les promotions, les comités, la Faculté de génie et l'AGEG. Elle gère, entre autres, l'aspect santé et sécurité relié aux groupes étudiants, les assurances et la partie liée aux groupes du Studio de création. Elle est responsable d'appuyer les groupes dans leur devoir financier et de s'assurer de leur saine gestion financière. } 

 
\article{Vice-Présidence aux Affaires Financières} 
\alinea{La vice-présidence aux affaires financières veille à l'administration financière courante de la corporation. Elle est signataire des chèques et autres effets de commerce de la corporation avec la présidence et la vice-présidence aux affaires légales.} 


\article{Vice-Présidence aux Affaires Externes} 
\alinea{La vice-présidence aux affaires externes s'occupe des relations avec les organismes extérieurs à l'Université. Elle fait aussi la promotion de la profession. Elle s'occupe de la représentation de l'AGEG à la CRÉIQ et à la FCEG.} 

 
\article{Vice-Présidence aux Affaires Pédagogiques} 
\alinea{La vice-présidence aux affaires pédagogiques assure la relation entre les membres et la Faculté de génie pour les dossiers relatifs à la formation des membres. Elle s'occupe aussi des diverses reconnaissances liées à la corporation ainsi que de l'aspect étudiant du Studio de Création.} 

 
\article{Vice-Présidence au Développement Durable} 
\alinea{La vice-présidence au développement durable doit s'assurer que l'AGEG adopte des pratiques écologiquement responsables en sensibilisant la population étudiante et non étudiante à divers enjeux environnementaux. Elle doit participer aux évènements organisés sur le campus en matière de développement durable et collaborer avec les différents groupes étudiants ayant aussi pour but celui-ci.} 

 
\article{Vice-Présidence aux Cycles Supérieurs} 
\alinea{La vice-présidence aux cycles supérieurs s'occupe des relations avec les étudiants aux cycles supérieurs, les associations facultaires des cycles supérieurs et le REMDUS. Elle est aussi responsable des relations avec le comité des cycles supérieurs et appuie la vice-présidence aux affaires pédagogiques sur les dossiers touchant les étudiants aux cycles supérieurs.} 

 
\article{Éligibilité} 
\alinea{Les personnes candidates aux différents postes du comité exécutif doivent être des membres réguliers de la corporation et être des membres réguliers actifs lors de leur entrée en poste au comité exécutif.} 
\sousalinea{Dans le cas où un poste ne présente aucun candidat au début du mandat de celui-ci, un membre régulier passif pourra occuper le poste.} 

\alinea{Dans le cas de la vice-présidence aux affaires externes, les personnes candidates doivent être des membres réguliers actifs au premier cycle et ce tout au long du mandat pour lequel elles postulent.} 

\alinea{Dans le cas de la vice-présidence aux affaires universitaires et de la vice-présidence aux affaires pédagogiques, les personnes candidates doivent être des membres réguliers actifs faisant partie du premier cycle au moment de leur entrée en poste au comité exécutif.}

\alinea{Dans le cas de la vice-présidence aux cycles supérieurs, les personnes candidates doivent être des membres réguliers actifs faisant partie des cycles supérieurs au moment de leur entrée en poste au comité exécutif.}


\article{Élection} 
\alinea{L'élection des membres du conseil exécutif se tient deux sessions avant le début du mandat de la personne exécutante.} 

\alinea{La vice-présidence aux affaires externes est élue pour deux sessions. La première session se déroulant deux sessions après son élection et la deuxième se déroulant quatre sessions après son élection.}

\alinea{Les procédures d'élections sont décrites dans le partie 8 du présent règlement.} 


\article{Nomination} 
\alinea{Le mandat des personnes exécutantes débute le premier jour de la session pour laquelle elles sont élues et se termine à minuit la veille du premier jour de la session suivante.} 


\article{Rémunération} 
\alinea{Les membres du comité exécutif ne peuvent recevoir aucune forme de rémunération pour leurs fonctions.} 

\article{Pouvoirs et devoirs des personnes exécutantes} 
\alinea{Les personnes exécutantes ont tous les pouvoirs et devoirs ordinairement inhérents à leur charge, sous réserve des dispositions de la loi ou des règlements, et elles ont en plus les pouvoirs et devoirs que le conseil d'administration leur délègue ou impose. Les pouvoirs des personnes exécutantes peuvent être exercés par toute autre personne spécialement nommée par le conseil d'administration à cette fin, en cas d'incapacité d'agir de ces personnes exécutantes.} 

\alinea{De façon générale, le comité exécutif: 

\begin{itemize} 

 \item Veille de près aux affaires courantes de la corporation dont il est le porte-parole et le représentant; 

 \item Décide des questions trop urgentes pour qu'il puisse y avoir réunion du conseil d'administration. Il devra rendre compte de ses décisions au conseil d'administration; 

 \item Prépare et convoque les réunions du conseil d'administration et les assemblées générales; 

 \item Informe le conseil d'administration sur ses activités et décisions les plus importantes;

 \item Doit rendre compte au conseil d'administration de l'utilisation de l'argent de la corporation et doit obtenir l'autorisation du conseil d'administration pour effectuer les dépenses supérieures de deux cents (200) dollars au poste budgétaire alloué par le conseil d'administration; 

 \item Veille à l'exécution des décisions du conseil d'administration; 

 \item Nomme la présidence d'élection pour l'élection du comité exécutif et du conseil d'administration; 

 \item Peut s'adjoindre des personnes et former des groupes de travail pour la conduite des affaires de la corporation. 

\end{itemize} 
} 


\article{Démission et destitution} 
\alinea{Toute personne exécutante peut démissionner en tout temps, en remettant un écrit à cet effet à la présidence ou à la vice-présidence aux affaires légales.} 

\alinea{Les personnes exécutantes sont sujettes à destitution par résolution du conseil d'administration.} 

\alinea{Les personnes exécutantes sont sujettes à destitution par résolution du de l'assemblée générale des membres.}

\article{Vacances} 
\alinea{Dans le cas d'une vacance au comité exécutif, le conseil d'administration, par résolution, peut nommer une autre personne qualifiée sous recommandation du conseil exécutif pour remplir cette vacance ou entamer une procédure d'élection.} 


\article{Accès aux documents privilégiés} 
Les personnes exécutantes sont tenues de ne pas divulguer les informations privilégiées auxquelles elles ont accès. La présidence du comité exécutif est la seule personne pouvant autoriser le partage d'informations privilégiées.


\article{Réunions comité exécutif} 
\alinea{Le comité exécutif doit tenir le nombre de réunions nécessaires au bon déroulement des affaires de la corporation.} 


\article{Convocation et lieu} 
\alinea{Les réunions du comité exécutif sont convoquées par la présidence. Elles sont tenues aux lieu, date et heure désignés par celle-ci après consultation de toutes les personnes exécutantes.} 


\article{Quorum} 
\alinea{Cinq (5) personnes exécutantes doivent être présentes à chaque réunion pour constituer le quorum requis.} 


\article{Conflit d'intérêts}  
\sousarticle{Principe}
\alinea{Toute personne en conflit d'intérêts est responsable de faire savoir rapidement aux membres du conseil exécutif qu'elle est en conflit d'intérêts dès que la question est traitée.} 

\alinea{Une personne exécutante en conflit d'intérêts peut alors donner son opinion et répondre aux questions.}

\alinea{Une personne exécutante en conflit d'intérêts entraînant des avantages financiers doit sortir de la réunion pour le temps où se dérouleront les débats, les délibérations et le vote.}

\alinea{Ne sont pas considérées en conflit d'intérêts les personnes qui reçoivent un avantage financier du simple fait d'être membre de la corporation.}

\sousarticle{Nature}
\alinea{Sans restreindre la généralité de ce qui précède, est considérée comme situation conflictuelle type;}

\alinea{L’attribution de subventions ou autres avantages aux groupes étudiants tout membre dudit groupe ;}

\alinea{L’affiliation à un organisme externe, la modification de la cotisation à cet organisme ou l’octroi d’avantages quelconques à cet organisme, tout membre de l’exécutif ou personne employée de cet organisme ;}

\alinea{L’attribution d’un contrat ou la ratification d’une entente avec une société ou compagnie dont la personne est employée, actionnaire, dirigeante ou mandataire.}

\article{Vote} 
\alinea{À une réunion du comité exécutif, les personnes exécutantes présentes ont droit à un vote chacun. Les votes par procuration ne sont pas acceptés. En cas d'égalité des voix, la présidence a un droit de vote prépondérant.} 


\article{Code de procédures des assemblées et des réunions} 
\alinea{Le code Morin est utilisé pendant toutes les réunions du conseil exécutif. La procédure aux assemblées délibérantes est décrite dans le livre Procédure des assemblées délibérantes / Victor Morin; mise à jour par Michel Delorme Éditions Beauchemin, 1994, à moins de dispositions contraires prévues dans les présents règlements.}


\article{Procès-verbaux publics}
\alinea{Les procès-verbaux de la corporation sont publics, donc ouverts aux membres de la corporation, exceptés les procès-verbaux des huis clos.}

























