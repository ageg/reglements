\reglement{Relatif aux règlements généraux de l’Association Générale  Étudiante en Génie de l’Université de Sherbrooke}
\preambule{Les règlements généraux de l’Association Générale Étudiante en Génie de l’Université de Sherbrooke ont préséance sur les autres règlements internes de l’AGEG ainsi que sur toute autre documentation de l’AGEG.}
\partie{Préambule}


\article{Interprétation}
\alinea{Dans les présents règlements:}
\sousalinea{\textbf{Loi} désigne la Loi sur les compagnies, L.R.Q., c. C-38.}
\sousalinea{\textbf{Association}, \textbf{corporation} ou \textbf{AGEG} désignent l’Association générale étudiante en génie de l’Université de Sherbrooke.}
\sousalinea{\textbf{Session} désigne une période de quatre mois correspondant aux sessions officielles d’études de l’Université de Sherbrooke.}
\sousalinea{\textbf{Fonds équipement étudiant} ou \textbf{FÉÉ} désigne un fonds servant à l’achat d’équipements pédagogiques pour les membres.}
\sousalinea{\textbf{Personne administratrice} désigne une personne membre du conseil d’administration de la corporation.}
\sousalinea{Personne exécutante désigne une personne membre du comité exécutif.}
\sousalinea{Promotion désigne l’ensemble des membres qui, selon leur cheminement d’étude, termineront leur baccalauréat au même moment.}

\article{Entrée en vigueur}
\alinea{Les règlements généraux sont adoptés, abrogés et modifiés par le conseil d’administration et entrent en vigueur dès leur adoption par celui-ci. Ils doivent être ratifiés par les membres, à la majorité simple, lors de la prochaine assemblée annuelle ou générale suivant l’adoption des règlements généraux par le conseil d’administration.}
\alinea{À défaut d’être ratifiés, ils cessent alors d’être en vigueur, mais sans effet rétroactif.}

\article{Abrogation}
\alinea{Lors de leur entrée en vigueur, les présents règlements généraux abrogeront tous les règlements généraux antérieurs.}

\partie{Dispositions générales}

\alinea{La présente section est un résumé des lettres patentes, elle ne peut donc pas être modifiée sans la modification des lettres patentes. Les lettres patentes ne peuvent être modifiées que selon les formalités prévues par la loi.}

\begin{tabular}[c]{m{.2\textwidth}|m{.7\textwidth}}
\hline
Incorporation & Incorporée le 23 décembre 1968, en vertu de la partie III de la Loi sur les compagnies (L.R.Q., c. C-38).\\\hline
Siège social & Le siège social de la corporation est établi à l’Université de Sherbrooke dans la cité de Sherbrooke au local déterminé par le conseil d'administration de la corporation.\\\hline
Raison sociale & La corporation est accréditée sous le nom Association générale étudiante en génie de l’Université de Sherbrooke.\\\hline
Mission de la corporation & La mission de l'Association générale étudiante en génie de l'Université de Sherbrooke est de représenter et défendre les intérêts de ses membres et de favoriser l'épanouissement et la qualité de leur formation afin de les préparer à leur vie en société en tant que futurs ingénieurs.\\\hline
Vision de la corporation & L'Association générale  étudiante en génie de l'Université de Sherbrooke vise à faire de ses membres les meilleures personnes étudiantes en génie du Canada en encourageant leur épanouissement sur les plans professionnel, social et personnel.\\\hline
\end{tabular}

\partie{Membres}
\article{Membres}
\alinea{Sont membres de la corporation les membres réguliers actifs et passifs ainsi que les membres honoraires. Les membres doivent adhérer aux objectifs de la corporation.}

\sousarticle{Personnes membres régulières actives}
\alinea{Est une personne membre régulière active de l’AGEG toute personne étudiante inscrite à un programme de premier cycle de la Faculté de génie de l’Université de Sherbrooke en session d’étude. Ses droits sont de participer à toutes les activités de la corporation, recevoir les avis de convocation aux assemblées des membres, assister à ces assemblées et sont éligibles à titre de personnes administratrices ou exécutantes de la corporation. Ces membres ont un droit de vote.}

\sousarticle{Personnes membres régulières passives}
\alinea{Est une personne membre régulier passif de l’AGEG tout étudiant inscrit à un programme de premier cycle de la Faculté de génie de l’Université de Sherbrooke qui n’est pas en session d’étude. Ses droits sont de participer à toutes les activités de la corporation, recevoir les avis de convocation aux assemblées des membres, assister à ces assemblées et sont éligibles à titre de personnes administratrices de la corporation. Ces membres ont un droit de vote.}

\sousarticle{Membres honoraires}
\alinea{Il est loisible au conseil d’administration, par résolution, de nommer membre honoraire de la corporation toute personne qui aura rendu service à la corporation par son travail ou par ses donations, ou qui aura manifesté son appui à la mission poursuivie par la corporation.}
\alinea{Les membres honoraires peuvent participer aux activités de la corporation et assister aux assemblées des membres, mais n’ont pas le droit de voter lors des assemblées. Ces personnes ne sont pas éligibles comme personnes administratrices de la corporation, et ne sont pas tenus de verser des cotisations ou contributions à la corporation.}

\article{Cotisations}
\sousarticle{Cotisation étudiante}
\alinea{Toutes les personnes membres, qu'elles soient actives ou passives paient une cotisation étudiante. Le conseil d’administration peut, par résolution, fixer le montant des cotisations, de même que le moment, le lieu et la manière d’effectuer le paiement. Cette résolution doit être entérinée par l’assemblée générale. La cotisation étudiante est obligatoire et n’est pas remboursable même dans les cas de radiation, de suspension, ou de retrait.}
\sousarticle{Cotisation au fonds équipement étudiant }
\alinea{Seuls les personnes membres régulières actives paient la cotisation au fonds équipement étudiant. Le conseil d’administration peut, par résolution, fixer le montant des cotisations, de même que le moment, le lieu et la manière d’effectuer le paiement. Cette résolution doit être entérinée par l’assemblée générale. La cotisation au fonds équipement étudiant est obligatoire et n’est pas remboursable même dans les cas de radiation, de suspension, ou de retrait.}

\sousarticle{Cotisation CRÉIQ-FCEG}
\alinea{Seuls les personnes membres régulières actives paient la cotisation CREIQ-FCEG. Le montant de la cotisation CREIQ-FCEG est fixé par les instances de ces deux associations étudiantes. Toutefois, l’AGEG est responsable de recueillir les fonds et de faire le paiement annuel à la CRÉIQ et la FCEG. La cotisation au fonds CRÉIQ-FCEG est obligatoire et n’est pas remboursable même dans les cas de radiation, de suspension, ou de retrait.}

\article{Retrait ou démission}
\alinea{Toute personne membre peut, en tout temps, se retirer ou démissionner de la corporation en signifiant ce retrait par écrit à la présidence de la corporation.}

\article{Suspension et radiation}
\alinea{Dans l’éventualité où le conseil d’administration considère qu’une personne membre aie enfreint les dispositions des présents règlements ou dont la conduite ou les activités sont jugées nuisibles, le conseil d’administration devra convoquer la personne membre afin qu’elle réponde de ses actes lors d’une séance spéciale du conseil d’administration convoquée à cet effet. Par la suite, le conseil d’administration pourra, par résolution, procéder à une suspension ou radiation du dit membre.}
\alinea{La personne membre pourra porter appel pour chacune des décisions à un comité disciplinaire formé pour cette demande.La décision du comité sera finale et sans appel.}

\article{Comité disciplinaire}
\alinea{Le comité disciplinaire est formé uniquement lorsqu’une personne fait appel à une décision de suspension ou radiation.}
\alinea{Le comité disciplinaire est formé de trois personnes;}
\sousalinea{Si la personne à la direction des affaires étudiantes de la faculté est disponible, elle siège sur le comité avec unepersonne membre ne connaissant pas de près la personne portant appel ainsi qu’une personne membre du CA.}
\sousalinea{Dans le cas où la personne à la direction des affaires étudiantes de la faculté n’est pas disponible, une autre personnedu corps professoral ou de la direction de la faculté peut la remplacer.}
\sousalinea{Advenant le cas où aucune personne du corps professoral ou de la direction de la faculté est disponible, une autrepersonne non-membre de l’AGEG siègera sur le comité.}


\partie{Assemblée générale des membres}

\article{Assemblée générale régulière}
\alinea{Il doit y avoir, lors d’une session sans assemblée générale annuelle, au moins une assemblée générale régulière. Celle-ci peut être convoquée par le conseil d’administration ou le comité exécutif. Un avis de convocation doit être envoyé par courriel au moins quatre jours ouvrables avant l’assemblée et doit contenir le cahier de propositions ainsi que tous les règlements auxquels des modifications sont proposés.}
\alinea{Chaque promotion doit organiser au moins une assemblée générale par session afin d’élire son comité exécutif pour une session déterminée ou pour informer la promotion sur ce qu’il se passe avec celle-ce. Le quorum à respecter est celui identifié à l’article 4.5}


\article{Assemblée annuelle}
\alinea{L'assemblée générale annuelle des membres de la corporation aura lieu durant la session suivant la fin de l’année financière de la corporation à une date fixée par le conseil d'administration. Elle sera tenue en la cité de Sherbrooke à tout endroit déterminé par le conseil d'administration. À l'assemblée générale annuelle sont approuvés les rapports financiers annuels datant de moins de quatre (4) mois, les autres rapports exigés par la loi et sont ratifiés les règlements, résolutions et actes adoptés ou posés par le conseil d'administration et les personnes exécutantes depuis la dernière assemblée générale. Un avis de convocation est adressé par courriel à tous les membres au moins 7 jours avant l’assemblée.}

\article{Assemblée générales spéciale}
\alinea{Le conseil d’administration, le conseil exécutif ou 20 personnes membres régulières peuvent, selon les besoins, convoquer une assemblée générales spéciales au lieu, à la date et à l'heure désirée. Les personnes convoquant l'AGS doivent donner un délai de 2 jours ouvrables aux membres pour cette réunion. La présidence de l'AGEG est alors tenue de convoquer cette assemblée. Le conseil d’administration ou le conseil exécutif procède par résolution, tandis que le groupe de 20 personnes membres régulières ou plus doit produire une demande écrite, signée par ces 20 personnes membres régulières ou plus. L’avis de convocation doit énoncer le ou les buts de cette assemblée et l'ordre du jour ne peut être modifié.}
\alinea{Un avis de convocation est adressé par courriel à tous les membres dès la réception de la convocation par la présidence. L’avis doit contenir les sujets à être traités lors de l’assemblée.}
\alinea{La présence d'une personne membre régulière à une assemblée couvre le défaut d'avis quant à cette pesonne.}
\alinea{L’omission accidentelle de cet avis ou la non-connaissance de cet avis par toute personne n’a pas pour effet de rendre nulles les résolutions adoptées à cette assemblée.}

\article{Assemblée générale extraordinaire}
\alinea{Une asemblée générale extraordinaire ne peut être convoquée que par une pétition signée par au moins cinquante (50) personnes membres régulières. La pétition peut convoquer une AGE au lieu, à la date  et à l'heure souhaitée, tout en respectant les délais de convocation d'une AGS.}
\alinea{Un avis de convocation doit être envoyé par la présidence de l'AGEG à la réception et doit contenir le cahier de proposition de l'assemblée. L'avis de convocation doit énoncer le ou les buts de cette assemblée et l'ordre du jour ne peut être modifié.}
\alinea{Seulement une AGE peut posséder un point grève sur son cahier de proposition.}
\alinea{Une AGE peut être convoqué afin de prendre une décision importante nécessitant l'avis de plus de membres que lors d'une AGS.}

\article{Quorum}
\alinea{Le quorum d'une assemblée générale régulière, annuelle ou spéciale est constitué de 3~\% des personnes membres régulières actives, tandis que celui pour une assemblée générale extroardinaire est de 10~\% . Il est nécessaire que le quorum subsiste pendant toute la durée de l’assemblée.}
\alinea{Une assemblée générale pour une promotion doit avoir un double quorum. Il doit y avoir 3~\%  des personnes membres actives de la promotion ainsi que 3~\%  des personnes membres actives de chaque programme. Toutefois, S'il s'agit d'une reprise d'AG, seules les personnes membres d'un programme ayant une personne à voter une personne représentante pour ce programme nécessaire pour ouvrir.}
\alinea{Le quorum d'une charte de promotion a préséance sur cet article.}

\article{Vote}
\alinea{À une assemblée des membres, les membres réguliers ont droit à un vote chacun. Les votes par procuration ne sont pas acceptés.}
\alinea{Le vote se fait à main levée, à moins que trois (3) des membres réguliers présents ne réclament le scrutin secret. En cas de vote au scrutin secret, le président d’assemblée nomme deux scrutateurs qui distribuent et recueillent les bulletins de vote, compilent les résultats et le communiquent au président d’assemblée.}
\alinea{Il est possible de voter par anticipation par la méthode inscrite dans l’avis de convocation. Le président d’assemblée compile les résultats reçus qui seront ajoutés au vote de l’assemblée. Le résultat final est communiqué aux membres par celui-ci.}
\alinea{Une proposition est adoptée à majorité absolue des voies. En cas d’égalité des voix, la proposition est battue.}
\alinea{Seules les personnes membres d'un programme peuvent se voter une personne pour les représenter sur leur CE de promotion.}

\article{Présidence et secrétariat d’assemblée}
\alinea{La présidence et la secrétariat d’assemblée sont élus au début de l’assemblée. La présidence doit veiller au bon déroulement de l’assemblée et au respect des règles d’assemblée.}

\partie{Conseil d'administration (CA)}
\article{Composition et droit de vote}
\alinea{Le conseil d'administration est composé d'un maximum de vingt-et-une (21) personnes administratrices votantes élues ou nommées et réparties en conformité avec les dispositions des présents règlements. De plus, le comité exécutif ainsi que la coordination administrative siègent au conseil d’administration. }
\alinea{Les affaires de la corporation sont administrées par un conseil d’administration composé comme suit:
\begin{itemize}
 \item De la présidence, de la VPAL, de la VPAF, de la VPAU et la personne agente de liaison. Toutes ces personnes possèdent un (1) vote
 \item Quatre (4) personnes administratrices annuelles avec un (1) vote chacunes
 \item Un total de douze (12) personnes administratrices saisonnières avec un (1) vote chacunes
 \begin{itemize}
  \item De ces personnes administratrices, une place par promotion présente en session d’études est réservée d’office pour des personnes représentantes de promotion avec un (1) vote chacunes. Ces personnes sont nommées par les comités exécutifs des promotions. 
  \item De ces personnes administratrices, une est réservée pour la personne responsable des activités des jeudis détentes. Ce siège n’est pas transférable en cas de vacance du poste ou si ladite personne est déjà élue en tant que personne administratrice.
 \end{itemize}
\end{itemize}
}
\alinea{Les autres personnes membres du comité exécutif sont des observatrices privilégiées avec droit de parole.}

\alinea{Deux (2) sièges d’observateur privilégié avec droit de parole sont réservés pour former de futurs personnes administratrices n’ayant jamais effectué un mandat au conseil d’administration. Ces sièges sont attribuées par le conseil d’administration au début de chacune de ses réunions. Dans la mesure où les deux sièges ne sont pas comblés, un membre ayant déjà effectué un mandat pourra obtenir le siège.}

\alinea{La coordonnation administrative de la corporation siège à titre d'observatrice privilégiée avec droit de parole.}

\alinea{Toutes les personnes administratrices siègent à titre personnel et devront entre autres défendre les intérêts de toutes les personnes membres et assurer la circulation de l’information bidirectionnelle entre l’ensemble des personnes membres et le conseil d’administration. Une personnes administratrice ne peut cumuler plus d’un poste au conseil d’administration.}

\article{Éligibilité}
\alinea{Toute personne membre régulière est éligible aux différents postes de personne administratrice à condition de satisfaire aux critères relatifs au siège postulé.}

\article{Durée des fonctions}
\alinea{Le mandat de toute personne administratrice annuelle commence le 1er janvier suivant son élection et se termine le 31 décembre de la même année.}
\alinea{Le mandat de toute personne administratrice saisonnière commence au début de la deuxième session suivant son élection et se termine à la fin de cette même session.}

\article{Élection}
\alinea{Les personnes administratrices annuelles sont élues à la session d’automne par les personnes membres régulières au cours d’une assemblée générale régulière, annuelle ou spéciale.}
\alinea{Les personnes administratrices saisonnières sont élues deux sessions à l’avance par les personnes membres régulières au cours d’une assemblée générale régulière, annuelle ou spéciale.}

\article{Mise en candidature}
\alinea{Les procédures de mise en candidature pour les élections au conseil d'administration sont décrites au règlement 19 de l'AGEG.}

\article{Personne agente de liaison}
\alinea{La personne agente de liaison est nommée par le comité exécutif passif. La personne agente de liaison siège à titre personnel et a pour fonction de représenter et défendre les intérêts des personnes membres régulières passives. Son mandat commence au début de la session suivant sa nomination et est d’une durée d’une session.}

\article{Vacances}
\alinea{Le conseil d’administration a la responsabilité de pourvoir à tout siège vacant sur son conseil pour le reste d’une session.}
\alinea{Le conseil d’administration nomme des personnes membres régulières pour combler les postes vacants.}
\sousalinea{Dans le cas où le poste vacant est celui d’une personne représentante de promotion, ce poste doit être pourvu par une nouvelle personne de la promotion concernée.}

\article{Retrait d’une personne administratrice}
\alinea{Cesse de faire partie du conseil d’administration et d’occuper sa fonction, toute personne administratrice qui:}
\sousalinea{Présente par écrit sa démission au conseil d’administration;}
\sousalinea{Décède, devient insolvable ou interdit;}
\sousalinea{Perds son éligibilité;}
\sousalinea{Est destitué par un vote des deux tiers (2/3) des personnes membres régulières présents à une assemblée générale spéciale convoquée à cette fin.}

\article{Rémunération}
\alinea{Les personnes administratrices de la corporation ne peuvent recevoir aucune forme de rémunération.}

\article{Indemnisation}
\alinea{Toutes personnes administratrices, ses personne héritières et ayants droit seront tenues, au besoin et à toute époque, à même les fonds de la corporation, indemne et couvert:}
\alinea{de tous frais, charges et dépenses quelconques que ces personnes administratrices supportent ou subient au cours ou à l’occasion d’une action, poursuite ou procédure intentée contre elle, à l’égard ou en raison d’actes ou choses accomplis ou permis par elle dans l’exercice ou pour l’exécution de ses fonctions;}
\alinea{de tous autres frais, charges et dépenses qu’elle supporte ou subit au cours ou à l’occasion des affaires de la corporation ou relativement à ces affaires, excepté ceux qui résultent de sa propre négligence ou de son omission volontaire.}

\partie{Réunion du conseil d'administration}
\article{Date et lieu}
\alinea{Le conseil d’administration doit tenir toutes les réunions nécessaires à la bonne marche de la corporation. Les réunions sont tenues à l’endroit et à la date désignés par le comité exécutif. Un minimum de trois (3) réunions doit être tenu par session.}

\article{Convocation et lieu}
\alinea{Les réunions du conseil d’administration sont convoquées par la présidence de la corporation. Un avis de convocation est adressé par courriel à tout le conseil d'administration au moins 5 jours avant la réunion. }

\alinea{Le conseil exécutif ou 5 personnes administratrices peuvent, selon les besoins, convoquer une réunion sans avis préalable au lieu, à la date et à l'heure qu’ils fixent. Ils doivent donner un délai de 2 jours ouvrables aux personnes administratrices. La présidence est alors tenu de convoquer cette réunion. L’avis de convocation doit énoncer le ou les buts de la réunion.}

\alinea{La présence d'une personne administratrice à une réunion couvre le défaut d'avis quant à celle-ci.}

\alinea{L’omission accidentelle de cet avis ou la non-connaissance de cet avis par toute personne administratrice n’a pas pour effet de rendre nulles les résolutions adoptées à cette réunion.}

\alinea{Toute personne peut assister aux réunions du conseil d'administration à titre de personne observatrice, sans droit de parole, à moins que le conseil d'administration n'en décide autrement.}

\alinea{Les personnes administratrices peuvent participer par téléphone ou vidéoconférence aux réunions.}

\alinea{Une résolution signée de tous les personnes administratrices est valide et a le même effet que si elle avait été adoptée dans une réunion dûment convoquée.}

\article{Quorum et vote}
\alinea{Les deux tiers (2/3) des personnes membres votantes doivent être présents à chaque réunion pour constituer le quorum requis. Les sièges vacants ne sont pas comptabilisés dans le calcul du quorum.}


\alinea{Toutes les questions soumises seront décidées à la majorité absolue des voix à moins de dispositions contraires dans les présents règlements, dans la loi ou dans le Code Morin. Seuls les personnes administratrices votantes ont droit de faire des propositions. }


\article{Comité}
\alinea{Le conseil d’administration peut former des comités d’étude ou de travail dont il détermine la composition et le mandat. Le conseil d’administration n’est pas tenu de donner suite aux recommandations des comités, mais il doit permettre à tous les membres de la corporation de prendre connaissance du rapport qu’il a commandé. Un comité est dissous lorsque son mandat est effectué.}


\article{Procès-verbaux}
\alinea{Les procès-verbaux de la corporation sont publics, donc ouverts aux membres de la corporation.}


\article{Droit de mise en dépôt d’une proposition de la personne agente de liaison}
\alinea{Lorsque la personne agente de liaison le juge nécessaire, elle peut demander qu'une consultation ait lieu entre le comité exécutif et le comité exécutif passif relativement à des décisions à prendre par le conseil d'administration pouvant porter préjudice aux membres du groupe qu'elle représente, et ce, avant qu'une décision finale soit prise par le conseil d'administration. Au cas où le conseil d'administration refuserait d'avoir recours à cette consultation, la personne agente de liaison peut forcer la mise en dépôt de la proposition.}


\article{Présidence et secrétariat d’assemblée}
\alinea{La présidence et le secrétariat d’assemblée sont élus au début de l’assemblée. La présidence doit veiller au bon déroulement de l’assemblée et au respect des règles d’assemblée.}

\partie{Le comité exécutif passif}
\article{Composition}
\alinea{Le comité exécutif passif est composé de onze (11)  membres. Les postes au sein du comité exécutif passif sont:
\begin{itemize}
 \item présidence passive;
 \item vice-présidence passive aux affaires légales;
 \item vice-présidence passive aux affaires universitaires;
 \item vice-présidence passive aux affaires internes;
 \item vice-présidence passive aux employés et aux communications;
 \item vice-présidence passive aux affaires sociales;
 \item vice-présidence passive aux activités extracurriculaires;
 \item vice-présidence passive aux affaires financières;
 \item vice-présidence passive aux affaires externes;
 \item vice-présidence passive aux affaires pédagogiques;
 \item vice-présidence passive au développement durable.
\end{itemize}
}

\article{Éligibilité}
\alinea{Les personnes candidates aux différents postes du comité exécutif passif doivent être des personnes membres régulières de la corporation et être des personnes membres régulières actives lors de leur entrée en poste au comité exécutif.}
\alinea{Dans le cas de la vice-présidence aux affaires externes, les personnes candidates doivent être des personnes membres régulères tout au long du mandat pour lequel elles postulent. La personne à la vice-présidence aux affaires externes alterne entre un poste sur le comité exécutif passif lorsqu’en stage et un poste sur le comité exécutif actif lorsqu’en session de cours.}
\alinea{Les autres exigences pour la mise en candidature aux postes du comité exécutif passif sont exposées dans le règlement 19 de l'AGEG.}


\article{Élection}
\alinea{Les procédures à respecter pour l'élection du comité exécutif passif sont définies dans le règlement 19 de l'AGEG.}
\alinea{Dans le cas de la vice-présidence passive aux affaires externes, l’élection a lieu à la session où le mandat d’une personne à la vice-présidence aux affaires externes se termine.}


\article{Rémunération}
\alinea{Les membres du comité exécutif passif ne peuvent recevoir aucune forme de rémunération.}
\article{Durée du mandat }
\alinea{Le comité exécutif passif entre en fonction au début des cours de la session suivant son élection, au moment où le comité exécutif passif précédent est nommé comité exécutif. La durée du mandat est d’une session.}


\article{Mandat}
\alinea{Le comité exécutif passif, sous la direction de la présidence passive, a pour mandat de nommer la personne agente de liaison qui représentera et défendra les intérêts des personnes membres régulières passives aux réunions du conseil d’administration et de préparer son futur mandat de comité exécutif.}


\partie{LE COMITÉ EXÉCUTIF}
\article{Composition}
\alinea{Le comité exécutif est composé de onze (11) membres. Celui-ci est composé de :
\begin{itemize}
 \item la présidence (PREZ);
 \item la vice-présidence aux affaires légales (VPAL);
 \item la vice-présidence aux affaires universitaires (VPAU);
 \item la vice-présidence aux affaires internes (VPAI);
 \item la vice-présidence aux employés et aux communications (VPEC);
 \item la vice-présidence aux affaires sociales (VPAS);
 \item la vice-présidence aux activités extracurriculaires (VPAX);
 \item la vice-présidence aux affaires financières (VPAF);
 \item la vice-présidence aux affaires externes (VPEX);
 \item la vice-présidence aux affaires pédagogiques (VPAP);
 \item la vice-présidence au développement durable (VPDD).
\end{itemize}
}


\article{Nomination}
\alinea{À la fin du mandat du comité exécutif, le comité exécutif passif est nommé comité exécutif.}


\article{Rémunération}
\alinea{Les membres du comité exécutif ne peuvent recevoir aucune forme de rémunération.}


\article{Durée du mandat}
\alinea{Le comité exécutif entre en fonction dès sa nomination. Le comité est en poste jusqu’à la fin de la session où il a été nominé. Le mandat de la vice-présidence aux affaires externes est d’une durée de 16 mois : 2 sessions actives et 2 sessions passives.}


\article{Pouvoirs et devoirs des personnes exécutantes}
\alinea{Les personnes exécutantes ont tous les pouvoirs et devoirs ordinairement inhérents à leur charge, sous réserve des dispositions de la loi ou des règlements, et elles ont en plus les pouvoirs et devoirs que le conseil d’administration leur délègue ou impose. Les pouvoirs des personnes exécutantes peuvent être exercés par toute autre personne spécialement nommée par le conseil d’administration à cette fin, en cas d’incapacité d’agir de ces personnes exécutantes.}


\alinea{De façon générale, le comité exécutif:
\begin{itemize}
 \item Veille de près aux affaires courantes de la corporation dont il est le porte-parole et le représentant;
 \item Décide des questions trop urgentes pour qu'il puisse y avoir réunion du conseil d'administration. Il devra rendre compte de ses décisions au conseil d’administration;
 \item Prépare et convoque les réunions du conseil d'administration et les assemblées générales;
 \item Informe le conseil d'administration sur ses activités et décisions les plus importantes;
 \item Doit rendre compte au conseil d'administration de l'utilisation de l'argent de la corporation et doit obtenir l'autorisation du conseil d'administration pour effectuer les dépenses supérieures de deux cents (200) dollars au poste budgétaire alloué par le conseil d’administration;
 \item Veille à l'exécution des décisions du conseil d'administration;
 \item Nomme la présidence d’élection pour l’élection du comité exécutif passif et du conseil d’administration;
 \item Peut s'adjoindre des personnes et former des groupes de travail pour la conduite des affaires de la corporation.
\end{itemize}
}

\article{Réunions comité exécutif}
\alinea{Le comité exécutif doit tenir le nombre de réunions nécessaires au bon déroulement des affaires de la corporation.}

\article{Convocation et lieu}
\alinea{Les réunions de comité exécutif sont convoquées par la présidence. Elles sont tenues aux lieu, date et heure désignés par celle-ci après consultation de toutes les personnes exécutantes.}

\article{Quorum}
\alinea{Cinq (5) personnes exécutantes doivent être présentes à chaque réunion pour constituer le quorum requis.}

\article{Vote}
\alinea{À une réunion du comité exécutif, les personnes exécutantes présentes ont droit à un vote chacun. Les votes par procuration ne sont pas acceptés. En cas d’égalité des voix, la présidence a un droit de vote prépondérant.}


\article{Présidence}
\alinea{La présidence préside de droit toutes les réunions du conseil d’administration et celles des membres, à moins qu’une présidence d’assemblée ne soit nommé et n’exerce cette fonction. Elle signe tous les documents qui requièrent sa signature. Elle a le contrôle général et la surveillance des affaires de la corporation.}
\alinea{La présidence est la personne exécutante en chef de la corporation. Elle voit à l'exécution des décisions du conseil d'administration. Elle remplit tous les devoirs inhérents à sa charge de même qu'elle exerce tous les pouvoirs qui pourront lui être attribués par le conseil d'administration. Elle s’assure que toutes les personnes exécutantes de son conseil exécutif remplissent les devoirs inhérents à leur charge.}


\article{Vice-présidence aux affaires légales}
\alinea{La vice-présidence aux affaires légales s’occupe de l’ensemble des dossiers légaux de la corporation. Elle s’occupe de s’assurer que l’ensemble des déclarations annuelles envers le gouvernement sont produites et que les contrats en cours et que les règlements généraux de la corporation soient respectés. Elle est signataire des chèques et autres effets de commerce en l’absence de la présidence ou de la vice-présidence aux affaires financières. En cas d'absence ou d'incapacité d'agir de la présidence, elle la remplace et exerce tous ses pouvoirs et toutes ses fonctions. Elle est responsable d’appuyer la présidence dans l’ensemble de ses tâches.}
\alinea{La vice-présidence aux affaires légales occupe aussi le poste de secrétariat de la corporation. Elle doit signer les procès-verbaux du conseil d’administration et du comité exécutif. Elle doit aussi s’assurer que l’ensemble des documents prévus par la loi est rempli et disponible pour les personnes ayant le droit de les consulter.}

\article{Vice-présidence aux affaires universitaires}
\alinea{La vice-présidence aux affaires universitaires s'occupe des relations avec les associations, instances et organismes de l’Université de Sherbrooke.}

\article{Vice-présidence aux affaires internes}
\alinea{La vice-présidence aux affaires internes assure principalement la qualité des services offerts aux membres, l’entretien des biens meubles et des locaux de la corporation ainsi que la gestion des produits vendus par celle-ci. Elle s’occupe aussi de faire la gestion de l’accès aux locaux ainsi que des objets promotionnels de la corporation. Puisqu’elle doit travailler en étroite collaboration avec les personnes employées, il appuie la vice-présidence aux employés et aux communications dans la sélection et l’évaluation de ceux-ci.}

\article{Vice-présidence aux employés et aux communications}
\alinea{La vice-présidence aux employés et aux communications s’occupe de la gestion des ressources humaines de la corporation ainsi que de la communication entre l’association, ses instances et ses membres. Il s’occupe aussi de l’image publique de la corporation. De plus, elle s’occupe des procédures de communication interne de l’exécutif et des employés et du recrutement des membres pour les divers comités, instances et groupes de la corporation.}

\article{Vice-présidence aux affaires sociales}
\alinea{La vice-présidence aux affaires sociales doit organiser et coordonner les activités sociales pour les membres de la corporation. Il supervise l’ensemble des comités en liens avec les activités sociales et fait la gestion des dossiers se rapportant aux activités sociales de la corporation.}

\article{Vice-présidence aux activités extracurriculaires }
\alinea{La vice-présidence à la formation étudiante assure la relation entre les groupes étudiants, les promotions, les comités, la Faculté de génie et l’AGEG. Elle gère entre autres l’aspect santé et sécurité relié aux groupes étudiants, les assurances et la partie reliée aux groupes du Studio de Création. Elle est responsable d’appuyer les groupes dans leur devoir financier et de s’assurer de leur saine gestion financière. }

\article{Vice-présidence aux affaires financières}
\alinea{La vice-présidence aux affaires financières veille à l’administration financière courante de la corporation. Elle est signataire des chèques et autres effets de commerce de la corporation avec la présidence et la vice-présidence aux affaires légales.}

\article{Vice-présidence aux affaires externes}
\alinea{La vice-présidence aux affaires externes s’occupe des relations avec les organismes extérieurs à l’Université. Elle fait aussi la promotion de la profession.}

\article{Vice-présidence aux affaires pédagogiques}
\alinea{La vice-présidence aux affaires pédagogiques assure la relation entre les membres et la Faculté de génie pour les dossiers relatifs à la formation des membres. Elle s’occupe aussi des diverses reconnaissances liées à la corporation ainsi que de l’aspect étudiant du Studio de Création.}

\article{Vice-présidence au développement durable}
\alinea{La vice-présidence au développement durable doit s’assurer que l’AGEG adopte des pratiques écologiquement respon-sables en sensibilisant la population étudiante et non étudiante à divers enjeux environnementaux. Elle doit participer aux évènements organisés sur le campus en termes de développement durable et collaborer avec les différents groupes étudiants ayant pour but celui-ci.}


\article{Porte-parole de la corporation}
\alinea{La personne porte-parole de la corporation est élue par décision du conseil exécutif. Elle parle au nom de l’AGEG sur tous les dossiers nécessitant une intervention médiatique et promeut ou publicise les positions, actions et évènements de l’AGEG.}

\article{Démission et destitution}
\alinea{Toute personne exécutante peut démissionner en tout temps, en remettant un écrit à cet effet à la présidence ou à la vice-présidence aux affaires légales.}
\alinea{Les personnes exécutantes sont sujets à destitution par résolution du conseil d’administration.}

\article{Vacances}
\alinea{Dans le cas d’une vacance au comité exécutif, le conseil d'administration, par résolution, peut nommer une autre personne qualifiée pour remplir cette vacance ou encore s'en remettre à un scrutin universel, conformément aux dispositions des présents règlements.}


\partie{Dispositions financières}
\article{Année financière }
\alinea{L'exercice financier de la corporation débute le 1er janvier et se termine le 31 décembre de la même année.}


\article{Vérification}
\alinea{Les livres et les états financiers de la corporation sont vérifiés chaque année, après l’expiration de l’exercice financier, par la personne ou firme vérificatrice interne nommée à cette fin lors de l’assemblée générale annuelle des membres.}


\article{Livre de comptabilité}
\alinea{Le conseil d'administration fait tenir par la coordonnation administrative de la corporation ou sous son contrôle, un ou des livres de comptabilité dans lequel ou lesquels sont inscrits tous les fonds reçus et déboursés par la corporation et toutes ses dettes ou obligations, de même que toute autre transaction financière de la corporation. Ce ou ces livres sont tenus au siège social de la corporation et sont ouverts à l'examen des personnes administratrices. En aucun temps ces livres ne pourront quitter le siège social de la corporation à moins que le conseil d’administration y consente.}

\article{Effets bancaires}
\alinea{Tous les chèques, billets et autres effets bancaires de la corporation sont signés par deux (2) des quatre (4) personnes élues aux postes suivants:
\begin{itemize}
 \item présidence
 \item vice-présidence aux affaires financières
 \item vice-présidence aux affaires légales
 \item coordonnation administrative de la corporation
\end{itemize}
}
\sousarticle{Compte à statut particulier}
\alinea{Sous résolution du conseil d’administration, un compte bancaire peut être déclaré comme compte à statut particulier. Il est à noter que le compte principal de la corporation ne peut être déclaré comme compte à statut particulier.}
\alinea{Nonobstant l’article 9.4 du présent document, les comptes à statut particulier peuvent avoir autant de signataires supplémentaires aux signataires du compte principal que le comité exécutif juge nécessaire. Ces signataires sont nommés par le conseil exécutif actif en début de session, et ce, pour toute la session du mandat de l’exécutif.}

\article{Contrats}
\alinea{Les contrats et autres documents requérant la signature de la corporation sont au préalable approuvés par le conseil d'administration et, sur telle approbation, sont signés par la présidence ou l'une des vice-présidences.}

\partie{Affiliation et désaffiliation}
\article{Proposition}
\alinea{Toute personne membre peut présenter en assemblée générale une proposition visant à ce que la corporation devienne ou cesse d’être membre d’un regroupement d’associations. L’adhésion ou le retrait de la corporation d’un regroupement d’associations entraîne l’adhésion ou le retrait individuel des membres.}

\partie{Procédure référendaire}
\article{Référendum ou plébiscite}
\alinea{Le conseil d'administration peut, par résolution, organiser un référendum ou plébiscite sur tout point qu'il jugera bon. Pour ce faire, le conseil d'administration nomme, par résolution, une présidence d'élection chargé du scrutin. La présidence d'élection devra faire en sorte que toutes les personnes membres régulières aient l'opportunité de voter. Elle pourra s'adjoindre de personnes scrutatrices. Le scrutin sera secret.}

\partie{Procédures aux assemblées et réunions}
\article{Code Morin}
\alinea{Pendant toutes les assemblées générales, les réunions du conseil d'administration et les réunions du comité exécutif, la procédure aux assemblées délibérantes est décrite dans le livre Procédure des assemblées délibérantes / Victor Morin; mise à jour par Michel Delorme Éditions Beauchemin, 1994, à moins de dispositions contraires prévues dans les présents règlements.}
\adoption{29 juillet 2018}{27 novembre 2018}