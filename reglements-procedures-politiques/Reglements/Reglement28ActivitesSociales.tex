\reglement{Relatif aux activités sociales de l’AGEG}

\preambule{L’objectif du présent règlement est de donner des lignes directrices au VPAS afin d’éliminer les sources de conflit associées aux activités sociales.}

\partie{Dispositions générales}
\article{Définitions}
Aux fins de ce règlement, les termes suivants ont les significations suivantes:
\alinea{\textbf{Activité sociale}~: toute activité visant le maintien d'un bon esprit du groupe ou des membres de la Faculté (ex: un souper, une activité sportive, une activité de financement, un party, etc.)}
\alinea{Les \textbf{"Jeudi détente"} et le \textbf{7@10 implication} ne sont pas considérés comme des activités sociales au sens de ce règlement.}
\alinea{Les \textbf{semaines} remises vont du lundi au dimanche suivant.}

\article{Mission de la distribution des semaines d’activités sociales}
\alinea{Permettre aux groupes étudiants et aux promotions d’obtenir la meilleure semaine d’activités sociales possible pour leurs activités tout en prônant le respect entre ces différentes instances et la corporation.}

\article{Vision de la distribution des semaines d’activités sociales}
\alinea{La vision de la distribution des semaines d’activités sociales est qu’il ne survienne aucun conflit entre les étudiants et les promotions dus à leur attribution  au début de chaque session.}

\article{Valeurs de la distribution des semaines d’activités sociales}
\valeurs
{Ouverture}{Faire preuve de démocratie dans la distribution des semaines d’activités sociales}
{Engagement}{Fournir aux membres la possibilité de subventionner leurs activités parascolaires}
{Intégrité}{Assurer une distribution équitable et impartiale des semaines d’activités sociales}
{Fraternité}{Favoriser la coopération entre les groupes et les promotions dans l’attribution des semaines d’activités sociales}

\article{Rôles et pouvoirs}
\sousarticle{AGEG}
\alinea{L’AGEG peut s’allouer une ou plusieurs semaines d’activités sociales avant la remise des dates aux autres groupes. L’AGEG se réserve aussi le droit de promouvoir toute activité externe à la Faculté.}
\alinea{En cas de conflit lié à la distribution des profits entre deux groupes associés pour l’organisation d’un party, l’AGEG servira d’arbitre auprès des parties en conflit.}
\alinea{En cas de non-respect du présent règlement par un groupe, l’AGEG pourra retirer le droit à une semaine d’activités sociales pour la prochaine session où le groupe sera actif. Dans le cas d’une récidive, ou lorsque la sanction ne s’applique pas, l’AGEG se réserve le droit d’appliquer des sanctions plus sévères ou adéquates.}
\alinea{Dans l’éventualité où l’AGEG est partie prenante dans un conflit, le CA sera appelé à trancher, sa décision étant finale; dans tous les autres cas, la décision du CE de l’AGEG sera finale}

\sousarticle{VPAS de l’AGEG}
\alinea{Il est responsable de tenir une réunion pour la distribution des semaines d’activités sociales dans les quatres dernières semaines de la session précédente.}
\alinea{Il est responsable d’avoir en main les dates des activités ou partys organisés par l’AGEG qui pourraient entrer en concurrence avec un party d’un groupe. Un groupe peut toujours choisir, avec son accord, de prendre cette semaine, mais en aucun cas le groupe ne peut par la suite déclarer un conflit avec l’AGEG.}
\alinea{Il est responsable de contacter les bars de la région pour les inviter à envoyer un représentant à la réunion.}

\sousarticle{VPAX de l’AGEG}
\alinea{Il est responsable de publiciser adéquatement auprès des groupes la date, l’heure et le lieu de la tenue de la réunion.}
\alinea{Il est responsable de publier le calendrier des semaines d'activités sociales dès la première semaine de la session courante.}

\sousarticle{Représentants des groupes et des promotions}
\alinea{Ils sont tenus responsables de s’informer auprès de l’AGEG de la date, l’heure et le lieu de la tenue de la réunion.}

\partie{Fonctionnement de la distribution des semaines d’activités sociales}
\article{Déroulement de la réunion}
\sousarticle{Choix de l’AGEG}
\alinea{L’AGEG doit arriver à la réunion avec les semaines qu’elle se réserve pour la session en cours. Le nombre de semaines est choisi par le VPAS. L’AGEG n’a pas à se jumeler pour ses semaines.}

\sousarticle{Choix de la promotion finissante}
\alinea{La promotion finissante obtient le premier choix de semaine d'activités sociales.}
\alinea{La promotion finissante n’a pas à se jumeler pour ses semaines.}

\sousarticle{Choix des promotions non-finissantes}
\alinea{Les promotions non-finissantes suivent ensuite par ordre d’ancienneté, avec un choix chacune.}
\alinea{Les promotions non-finissantes doivent se jumeler si elles sont sollicitées.}
\alinea{À la session d’automne, la promotion entrante sera automatiquement jumelée à la promotion de deuxième année.}
\alinea{La promotion doit respecter le règlement 50 sur les finances des groupes étudiants et des promotions pour obtenir à la session suivante sont droit de choisir une semaine.}

\sousarticle{Choix des groupes étudiants}
\alinea{Les groupes étudiants obtiennent leur choix par tirage au sort.}
\alinea{Lorsque toutes les semaines sont attribués, le tirage au sort continue et le groupe tirés doivent se jumeler avec un autre groupe ou promotion (exculant la finissante et l’AGEG)}
\alinea{Les groupes peuvent se jumeler avec une équipe avant d’être tiré au sort, si les deux groupes sont d’accord.}

\sousarticle{Semaines disponibles}
\alinea{S’il reste encore des semaines disponibles, une fois que tous les groupes ont obtenu leur semaine d’activités sociales, elles seront réservées à l’AGEG.}

\article{Absences à la réunion}
\alinea{Si un groupe est absent lors de la distribution, il sera laissé à la discrétion de l’AGEG de lui octroyer une semaine. Si un groupe, à l’exception de l’AGEG et de la promotion finissante, est resté seul à la suite de la réunion de distribution des semaines, il sera automatiquement jumelé à ce dernier.}

\partie{Détails concernant l’organisation d’activités sociales}
\article{Profits lors des activités sociales}
\alinea{Lorsqu’une activité sociale est coorganisée par plus d’un groupe, la distribution des profits entre ces groupes est laissée à leur discrétion.}

\article{Publicité des activités sociales}
\alinea{La publicité relative à une activité sociale, sous toutes ses formes, ne peut débuter hors de la semaine d’activités sociales, sauf sous autorisation du groupe ayant les droits d’affichage pour ladite semaine.}

\article{Organisation d’activités sociales hors des semaines attribuées}
\alinea{Aucune activité sociale ne peut être organisée par un groupe sauf pendant sa semaine d’activités sociales. Il est reconnu que d’autres groupes peuvent organiser des activités sociales; ces activités doivent être approuvées par le groupe qui détient la semaine durant laquelle l’activité doit être organisée.}

\article{Organisation des "Jeudi détente"}
\alinea{Aucune activité sociale ne peut se tenir au même moment que le "Jeudi détente", sauf sous approbation de l’organisateur du "Jeudi détente".}

\sousarticle{Publicité des "Jeudi détente"}
\alinea{Le groupe organisant le "Jeudi détente" peut publiciser en tout temps ladite activité.}

\article{Restrictions des objets promotionnels}
\alinea{L'objet doit être approuvé par le conseil exécutif de l'AGEG et ce, avant le premier événement servenant entre l'achat ou la vente de celui-ci.}
\alinea{En cas de doute, se référer à la Vice-présidence aux Affaires Extracurriculaire (VPAX) de l'AGEG.}

\adoption{18 mars 2018}{13 mai 2018}