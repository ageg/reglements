\reglement{Relatif à la délégation du congrès de la FCEG de l’Université de Sherbrooke}

\preambule{L’objectif du présent règlement est de guider la sélection de la délégation de l’Université de Sherbrooke pour le congrès annuel de la FCEG (Fédération Canadienne Étudiante en Génie).}

\partie{Dispositions générales}
\article{Mission de la délégation du congrès de la FCEG}
\alinea{Permettre à des étudiants de l’Université de Sherbrooke de participer au congrès annuel de la FCEG.}
\alinea{Représenter l’Université de Sherbrooke et les étudiants membres de l’AGEG lors du congrès de la FCEG.}

\article{Vision de la délégation du congrès de la FCEG}
\alinea{La vision de la délégation du congrès FCEG est d’assurer une présence soutenue et une représentation adéquate des membres de l’AGEG devant les associations d’étudiants en ingénierie du Canada.}

\article{Valeurs de la délégation du congrès de la FCEG}
\valeurs
{Ouverture}{Faire preuve de démocratie dans la sélection des délégués de l’AGEG}
{Engagement}{Encourager l’implication étudiante à l’Université de Sherbrooke et dans les milieux étudiants au Canada}
{Intégrité}{Assurer la sélection d’une délégation de membres ayant un excellent sens des responsabilités}
{Fraternité}{Encourager la coopération avec les associations étudiantes en génie du Canada}

\article{Rôles et pouvoirs}
\sousarticle{VPEX de l’AGEG}
\alinea{Il nomme le directeur FCEG de l’AGEG pour un mandat de 1 an lors du CA2 de la session d’hiver.}
\alinea{Il assure l’intérim du directeur FCEG si celui-ci est manquant.}

\sousarticle{Conseil d'administration de l’AGEG}
\alinea{Il nomme les membres du comité de sélection de la délégation du congrès FCEG durant le CA1 de la session d’automne.}
\alinea{Il détermine le nombre maximum de personnes dans la délégation à chaque année, selon le budget, lors du CA2 de la session d'hiver.}
\alinea{Il entérine le choix du chef de délégation au CA3 de l’été.}
\alinea{Il entérine la composition de la délégation au CA2 de l’automne.}

\sousarticle{Le directeur FCEG de l’AGEG}
\alinea{Lance l’appel à la mise en candidature qui doit être ouverte durant la session d’été et le début de la session d’automne.}
\alinea{Il sélectionne le chef de délégation avant le CA3 de l’été selon les critères de l’article 2.3.1 du présent règlement.}
\alinea{Il reçoit les candidatures des membres de l’AGEG.}
\alinea{Il siège sur le comité de sélection de la délégation du congrès FCEG.}
\alinea{Il remplit les tâches de chef de délégation si celui-ci est manquant.}
\alinea{Il présente à l’AGEG une proposition de formation de la délégation au CA3 de l’automne.}

\sousarticle{Le chef de délégation}
\alinea{Il siège sur le comité de sélection de la délégation du congrès FCEG.}
\alinea{Il assiste à la Rencontre du Président, à l’automne.}
\alinea{Il agit comme délégué votant  durant l’assemblée de plénière du congrès.}
\alinea{Il présente à l’AGEG une proposition de formation de la délégation au CA3 de l’automne.}

\partie{Comité de sélection de la délégation du congrès FCEG}
\article{Composition du comité}
\sousarticle{Membres du comité}
\alinea{Le comité est formé de 2 membres de l'AGEG, nommés par le CA, par le directeur FCEG de l’AGEG et par le chef de délégation.}

\sousarticle{Rémunération}
\alinea{Les membres du comité ne recevront aucune rémunération pour leur fonction.}
\article{Nomination et démission}

\sousarticle{Nomination des membres du comité}
\alinea{La durée du mandat des membres du comité est d’une session, soit la session d’automne.}
\alinea{Seuls les membres votants seront acceptés sur le comité.}

\sousarticle{Démission}
\alinea{Toute démission de tout membre du comité devra parvenir par écrit au VPEX de l’AGEG et sera effective après son approbation.}
\alinea{Un membre du comité peut se voir expulsé du comité par le CA de l’AGEG ou l’assemblée générale de l’AGEG pour des raisons jugées valables.}
\alinea{Un poste laissé vacant sera comblé par un membre du CA nommé par le VPEX de l’AGEG.}

\article{Décision}
\sousarticle{Critères}
\alinea{Les critères de sélection de la délégation du congrès FCEG permettent au comité d’évaluer et d’analyser les demandes afin de sélectionner les meilleurs candidats à envoyer au congrès de la FCEG.}
\alinea{Les critères  de sélection pour le chef de délégation sont les suivants :}
\sousalinea{Faire partie du CE de l’AGEG au moment du congrès, de préférence au poste de VPEX;}
\sousalinea{Avoir une connaissance appropriée de l’anglais.}
\alinea{Les critères de sélection pour les membres de la délégation sont les suivants :}
\sousalinea{Motivation des candidats;}
\sousalinea{Implications antérieures et actuelles dans le milieu universitaire;}
\sousalinea{Objectifs futurs dans l’implication universitaire;}
\sousalinea{Leurs attentes et objectifs par rapport au congrès.}

\partie{Délégation du congrès de la FCEG}
\article{Composition de la délégation}
\alinea{Le nombre de délégués est déterminé par le CA lors du CA2 de la session d’hiver.}
\alinea{La délégation est composée au minimum de deux personnes, dont le chef de délégation.}

\article{Rémunération}
\alinea{Les membres de la délégation ne recevront aucune rémunération pour leur fonction.}

\article{Démission}
\alinea{Toute démission de tout membre de la délégation devra parvenir par écrit au VPEX et sera effective après son approbation.}
\alinea{Un membre du comité peut se voir expulsé de la délégation par le CA de l’AGEG ou l’assemblée générale de l’AGEG pour des raisons jugées valables.}
\alinea{Un poste laissé vacant sera comblé par un membre de l’AGEG nommé par le VPEX.}

\adoption{3 avril 2016}{7 avril 2016}