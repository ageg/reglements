\chapter{Réunion du conseil d'administration}


\article{Date et lieu}
\alinea{Le conseil d'administration doit tenir toutes les réunions nécessaires au bon fonctionnement de la corporation. Les réunions sont tenues à l'endroit et à la date désignés par le comité exécutif. Un minimum de trois (3) réunions doit être tenu par session.}


\article{Convocation et lieu}
\alinea{Les réunions du conseil d'administration sont convoquées par la présidence de la corporation. Un avis de convocation est adressé par courriel à tout le conseil d'administration au moins cinq (5) jours avant la réunion.}

\alinea{Le conseil exécutif ou cinq (5) personnes administratrices peuvent, selon les besoins, convoquer une réunion sans avis préalable au lieu, à la date et à l'heure qu'ils fixent. Ils doivent donner un délai de deux (2) jours ouvrables aux personnes administratrices. La présidence est alors tenue de convoquer cette réunion. L'avis de convocation doit énoncer le ou les buts de la réunion.}

\alinea{La présence d'une personne administratrice à une réunion couvre le défaut d'avis quant à celle-ci.}

\alinea{L'omission accidentelle de cet avis ou la non-connaissance de cet avis par toute personne administratrice n'a pas pour effet de rendre nulles les résolutions adoptées à cette réunion.}

\alinea{Toute personne peut assister aux réunions du conseil d'administration à titre de personne observatrice, sans droit de parole, à moins que le conseil d'administration n'en décide autrement.}

\alinea{Les personnes administratrices peuvent participer par téléphone ou vidéoconférence aux réunions.}

\alinea{Une résolution signée de toutes les personnes administratrices est valide et a le même effet que si elle avait été adoptée dans une réunion dûment convoquée.}

\alinea{Une résolution peut être approuvée par vote électronique à condition d'obtenir l'unanimité des votes en faveur de la proposition.}
\sousalinea{Il est de la responsabilité des personnes administratrices de répondre au vote électronique le plus rapidement possible.}


\article{Conflit d'intérêts}
\sousarticle{Principe}
\alinea{Toute personne en conflit d'intérêts est responsable de faire savoir rapidement à la présidence d'assemblée qu'elle est en conflit d'intérêts dès que la question est traitée.}

\alinea{Une personne administratrice en conflit d'intérêts entraînant des avantages financiers peut donner son opinion et répondre aux questions. Elle doit par la suite sortir de la réunion pour le temps où se dérouleront les débats, les délibérations et le vote.}

\alinea{Une personne administratrice en conflit d'intérêts n'entraînant pas d'avantages financiers peut participer aux débats, aux délibérations et au vote.}

\alinea{Le conseil d'administration possède toute l'autorité pour retirer le droit de vote et pour forcer le départ de la salle à une personne administratrice qui ne se conforme pas au présent règlement d'elle-même.}

\alinea{Le conseil d'administration peut décider de toute autre mesure allant jusqu'à l'expulsion du conseil d'administration d'une personne administratrice qui aurait omis de signaler une situation de conflit d'intérêts ou d'apparence de conflit d'intérêts. La personne administratrice visée par le conflit d'intérêts mentionné par l'un des membres du conseil d'administration de l'AGEG peut faire appel et démontrer l'absence de conflit d'intérêts.}



\sousarticle{Nature}
\alinea{Sans restreindre la généralité de ce qui précède, est considérée comme situation conflictuelle type ;}

\alinea{L’attribution de subventions ou autres avantages aux groupes étudiants par tout membre dudit groupe ;}

\alinea{L’affiliation à un organisme externe, la modification de la cotisation à cet organisme ou l’octroi d’avantages quelconques à cet organisme, tout membre de l’exécutif ou personne employée de cet organisme ;}

\alinea{L’attribution d’un contrat ou la ratification d’une entente avec une société ou compagnie dont la personne est employée, actionnaire, dirigeante ou mandataire.}

\alinea{Ne sont pas considérées en conflit d'intérêts les personnes qui reçoivent un avantage financier du simple fait d'être membre de la corporation.}

\sousarticle{Quorum}
\alinea{Advenant le cas où seulement cinq (5) personnes administratrice ou moins seraient aptes à voter en vertu du présent article, le conseil d’administration doit former un comité composé de huit (8) personnes incluant les personnes administratrices aptes à voter ainsi que d’autres membres de l’AGEG qui ne sont pas en conflit d’intérêts ou en apparence de conflit d’intérêts. Ce comité se voit déléguer les pouvoirs nécessaires afin d’être en mesure de trancher la question.}

\alinea{Excepté le cas présenté à l'alinéa précédent, le fait qu’une personne administratrice se retire de la réunion pour des raisons de conflit d’intérêts ou d’apparence de conflit d’intérêts ne modifie pas le statut du quorum évalué en début de réunion.}


\article{Quorum et vote}
\alinea{Les deux tiers (2/3) des membres votantes doivent être présents à chaque réunion pour constituer le quorum requis. Les sièges vacants ne sont pas comptabilisés dans le calcul du quorum.}

\alinea{Toutes les questions soumises seront décidées à la majorité absolue des voix à moins de dispositions contraires dans les présents règlements, dans la loi ou dans le Code Morin. Seules les personnes administratrices votantes ont droit de faire des propositions. }


\article{Code de procédures des assemblées et des réunions}
\alinea{Le code Morin est utilisé pendant toutes les réunions du conseil d'administration. La procédure aux assemblées délibérantes est décrite dans le livre Procédure des assemblées délibérantes / Victor Morin; mise à jour par Michel Delorme Éditions Beauchemin, 1994, à moins de dispositions contraires prévues dans les présents règlements.}


\article{Comité}
\alinea{Le conseil d'administration peut former des comités d'étude ou de travail dont il détermine la composition et le mandat. Le conseil d'administration n'est pas tenu de donner suite aux recommandations des comités, mais il doit permettre à tous les membres de la corporation de prendre connaissance du rapport qu'il a commandé, sauf exception où la question traitée est confidentielle. Un comité est dissous lorsque son mandat est effectué.}


\article{Droit de mise en dépôt d'une proposition de la personne agente de liaison}
\alinea{Lorsque la personne agente de liaison le juge nécessaire, elle peut demander qu'une consultation ait lieu entre le comité exécutif actuel et le comité exécutif élu pour la session suivante relativement à des décisions à prendre par le conseil d'administration pouvant porter préjudice aux membres du groupe qu'elle représente, et ce, avant qu'une décision finale soit prise par le conseil d'administration. Au cas où le conseil d'administration refuserait d'avoir recours à cette consultation, la personne agente de liaison peut forcer la mise en dépôt de la proposition.}


\article{Accès aux documents privilégiés}
\alinea{Les personnes siégeant au conseil d'administration sont tenues de ne pas divulguer les informations privilégiées auxquelles elles ont accès.}


\article{Présidence et secrétariat d'assemblée}
\alinea{La présidence et le secrétariat d'assemblée sont élus au début de l'assemblée. La présidence doit veiller au bon déroulement de l'assemblée et au respect des règles d'assemblée. Le secrétariat d'assemblée est responsable de la rédaction du procès-verbal de l'assemblée. La présidence et le secrétariat ne doivent être occupés par des personnes administratrices de l'AGEG qu'en dernier recours.}


\article{Procès-verbaux}
\alinea{Les procès-verbaux de la corporation sont publics, donc ouverts aux membres de la corporation, exceptés les procès-verbaux des huis clos.}
