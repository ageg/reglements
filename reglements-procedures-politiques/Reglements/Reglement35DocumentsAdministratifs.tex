\reglement{Relatif aux documents administratifs de l’AGEG}

\preambule{L’objectif du présent règlement est d'uniformiser la gestion de documents relatifs à l'AGEG, d'assurer un suivi et un contrôle efficace de ses documents à long terme et de favoriser la transaction entre les exécutifs.}

\partie{Dispositions générales}
\article{Mission des documents administratifs}
\alinea{Avoir une existence corporative, d’assurer l’uniformité de la gestion, de favoriser les prises de décision, d’avoir une mémoire institutionnelle et de respecter les ententes avec les partenaires de la corporation.}

\article{Définition des documents administratifs}
\alinea{Les document administratifs de l’AGEG sont tous les documents à teneur légale et les documents relatifs au bon fonctionnement de la corporation.}
\alinea{Les documents administratifs comprennent, mais ne sont pas limités à~:}
\begin{enumerate}[i]
\item{Règlements internes}
\item{Charte}
\item{Rapports}
\item{Guide du parfait}
\item{Procès-verbaux des conseils d’administration, des conseils exécutifs, des comités décisionnels et d’assemblées générales}
\item{Contrats}
\item{Ententes}
\item{Les status de la corporation}
\item{Les documents financiers}
\end{enumerate}

\article{Valeurs et rôles des documents administratifs}
\valeurs
{Coopération}{Favoriser l’entraide entre les exécutants et leurs prédécesseurs}
{Intégrité}{Promouvoir le sens des responsabilités de l’ensemble des acteurs de l’AGEG}
{Professionnalisme}{Assurer une rigueur administrative de l’ensemble des acteurs de l’AGEG}
{Information}{Faciliter l’accès aux documents et aux savoirs ce qu’ils contiennent}

\article{Rôles et pouvoirs}
\sousarticle{AGEG}
\alinea{La corporation s’assure du suivi des documents entre les exécutants et s’engage à posséder un endroit sécurisé pour conserver les documents administratifs.}
\sousarticle{VPAL de l’AGEG}
\alinea{Gère, uniformise et conserve les documents confidentiels et les contrats de la corporation.}
\alinea{Possède un accès au coffre et à tous les documents administratifs de la corporation.}
\alinea{Effectue l’ensemble des tâches relatives à l’archivage des documents.}

\sousarticle{VPAI de l’AGEG}
\alinea{S’assure de gérer, uniformiser et conserver les autres documents administratifs de l’AGEG.}
\alinea{Demande aux organisateurs d’évènements majeurs de remettre les dossiers pertinents à l’organisation d’évènements semblables dans le futur.}
\alinea{S’assure d’avoir un directeur informatique compétent en fonction s’il n’est pas en mesure de remplir les exigences lui-même.}

\sousarticle{Président de l’AGEG}
\alinea{Possède l’accès à tous les documents de l’AGEG.}
\alinea{Peut autoriser, à un membre de l’AGEG, l’accès à un ou plusieurs de ces documents sauf dans le cas de conditions particulières liées à ces documents.}

\sousarticle{Directeur informatique de l’AGEG}
\alinea{S’assure du bon fonctionnement des services informatiques de l’AGEG.}
\alinea{S’assure que les services informatiques de l’AGEG répondent au présent règlement}

\sousarticle{Membres de l’AGEG}
\alinea{Chaque membre de l’AGEG responsable d’une activité corporative doit produire les rapports et les comptes rendus relatifs à ses activités.}

\partie{Détails sur les emplacements des documents administratifs}
\article{Coffre}
\alinea{Le coffre est un équipement physique servant à protéger et sécuriser les documents se trouvant à l’intérieur de celui-ci.}
\alinea{Seuls le président, le VPAF, le VPAL et le coordonateur administratif ont accès au coffre.}
\alinea{Il contient les documents importants de la corporation.}

¡color{red}\article\sout{Ouragan}\color{black}
\alinea{Emplacement informatique sécurisé contenant TOUS les documents de l’AGEG}
\alinea{Le service comporte les sections suivantes~:}
\begin{enumerate}[I]
\item{Comptabilité}
\item{Ententes confidentielles;}
\item{Documents de références;}
\item{Documents de travail de l’exécutif et des comités de l’AGEG;}
\item{Documents d’organisations;}
\item{Groupes étudiants;}
\item{Promotions.}
\end{enumerate}
\alinea{L’attribution des accès aux différentes sections du service sont détaillées dans la partie III du présent règlement.}
\alinea{Le service devra être muni de tous les dispositifs jugés nécessaire par le directeur informatique de l’AGEG pour s’assurer de la préservation des documents.}

\article{Classeurs et logiciel comptable}
\alinea{Ceux-ci se trouvent dans le local des officiers, ou encore, ils sont contenus dans le logiciel comptable.}
\alinea{Le logiciel comptable est accessible à~:}
\begin{enumerate}[I]
\item{coordonateur administratif;}
\item{Le président;}
\item{Le VPAF de l’AGEG;}
\item{Le vérificateur fiscal de l’AGEG;}
\item{Tout autre directeur de l'AGEG ayant besoin d’accéder au logiciel comptable pour une activité de la corporation.}
\end{enumerate}
\partie{Détails sur la gestion des documents administratifs}

\article{Charte et règlements internes}
\alinea{Ces documents sont conservés dans la section « Documents de Référence » d’Ouragan.}
\alinea{L’ensemble des membres de l’AGEG a accès en lecture à ces documents.}
\alinea{Une copie de ces documents doit être disponible sur le site Internet de la corporation.}
\alinea{Seule la dernière version adoptée en assemblée générale sera disponible sur le site web de l’AGEG.}
\alinea{À chaque fin de mandat d’un exécutif, une copie de ces documents est envoyée au service des archives de l’Université de Sherbrooke.}

\article{Procès-verbaux des assemblées générales, conseils d’administration et conseils exécutifs}
\alinea{Après adoption, ces documents sont conservés dans la section « Documents de référence » d’Ouragan.}
\alinea{L’ensemble des membres de l’AGEG a accès en lecture seule à ces documents.}
\alinea{À chaque fin de mandat d’un exécutif, une copie de l’ensemble de ces documents adoptés au cours de la session est envoyée au service des archives de l’Université de Sherbrooke.}

\article{Contrats et ententes confidentielles de l’AGEG et procès-verbaux sous huis clos.}
\alinea{Après la signature ou l’adoption de ces documents, ils sont conservés dans le coffre.}
\alinea{Le président, le VPAF, le VPAL et le coordonateur administratif et le vérificateur comptable de l’AGEG ont accès à ces documents.}
\alinea{Au besoin, le président de l’AGEG peut autoriser, à un membre de l’AGEG, l’accès à un ou plusieurs de ces documents sauf dans le cas de conditions particulières liées à ces documents.}
\alinea{Une copie numérique des documents devra être conservée dans la section « Ententes confidentielles » d’Ouragan.}
\alinea{Après dix ans, sauf indications contraires, une copie de ces documents doit être envoyée au service des archives de l’Université de Sherbrooke.}

\article{Contrats et ententes non confidentielles de l’AGEG}
\alinea{Après l’adoption ou la signature de ces documents, ils sont conservés dans la section « Ententes d’Ouragan.}
\alinea{L’ensemble des membres de l’AGEG a accès en lecture seule à ces documents sur demande, directement au local de l’AGEG.}
\alinea{À chaque fin de mandat d’un exécutif, une copie de l’ensemble de ces documents adoptés ou signés au cours de la session est envoyée au service des archives de l’Université de Sherbrooke}

\article{Rencontre de transition et rapport de fin de mandat}
\alinea{Une rencontre de transition doit être effectuée pour chacun des postes entre l’exécutant actif et son homologue afin d’assurer une continuité dans les dossiers en cours. Si une rencontre physique ne peut être effectuée, un rapport de transition doit être produit par l’exécutant sortant.}
\alinea{Un rapport de fin de mandat doit être écrit par chacun des exécutants sortants et servir de mémoire au bénéfice des sessions à venir avant deux semaines après la fin du mandat.}
\alinea{Le rapport de fin de mandat doit contenir les projets réalisés, leurs conclusions et leurs recommandations.}
\alinea{Ces rapports doivent être conservés dans la section « Documents de référence » d’Ouragan.}
\alinea{L’ensemble des membres de l’AGEG a accès en lecture seule à ces documents.}
\alinea{Une copie du rapport de fin de mandat de chaque exécutant sortant doit être envoyée au service des archives de l’Université de Sherbrooke.}

\article{Autres rapports, guides et plans de l’AGEG}
\alinea{Ces documents doivent être conservés dans Ouragan. L’ensemble des membres de l’AGEG a accès en lecture seule à ces documents.}
\alinea{À chaque fin de mandat d’un exécutif, une copie de l’ensemble de ses documents réalisés au cours de la session est envoyée au service des archives de l’Université de Sherbrooke.}

\article{Documents financiers de l’AGEG}
\alinea{Ces documents doivent être conservés dans les classeurs et le logiciel comptable.  Après cinq ans, les originaux sont envoyés au service des archives de l’Université de Sherbrooke.  Le président, le VPAF, le VPAL et le coordonateur administratif et le vérificateur comptable de l’AGEG ont accès à ces documents. Au besoin, le président de l’AGEG peut autoriser, à un membre de l’AGEG, l’accès à un ou plusieurs de ces documents.}

\article{Documents relatifs à l’organisation d’évènements majeurs}
\alinea{Ces documents doivent être conservés dans la section « Organisations » d’Ouragan.}
\alinea{L’ensemble des membres de l’AGEG a accès en lecture seule à ces documents, sur demande, directement au local de l’AGEG.}

\article{Évaluations des employés}
\alinea{Les parties 1 et 2 de l’évaluation doivent être conservées dans Ouragan.}
\alinea{Seulement les membres du CA et le CE ont accès à ces documents.}
\alinea{La partie 3 de l’évaluation doit être conservée en format papier dans les classeurs de la corporation.}
\alinea{Les évaluations doivent être détruites 5 ans après le dernier mandat de l’employé.}

\adoption{3 avril 2016}{7 avril 2016}