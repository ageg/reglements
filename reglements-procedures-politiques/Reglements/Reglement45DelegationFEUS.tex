\reglement{Relatif à la délégation de l’AGEG aux instances de la FEUS}

\preambule{L’objectif du présent règlement est de donner des lignes directrices à la délégation de l’AGEG afin d’assurer la meilleure efficacité possible aux conseils des membres et aux commissions de la FEUS pour défendre les intérêts des membres de l’AGEG.}

\partie{Dispositions générales}
\article{Mission de la délégation aux congrès de la FEUS}
\alinea{Permettre d’affirmer les valeurs de l’AGEG auprès des membres de la FEUS;}
\alinea{Faciliter les communications avec les autres associations membres de la FEUS ainsi qu’avec les autres instances du campus;}
\alinea{Rechercher des solutions aux problèmes récurrents et faire avancer des dossiers jugés opportuns de l’AGEG à l’aide de la FEUS.}

\article{Mission du VPAU à la FEUS}
\alinea{La Mission du VPAU à la FEUS est d’assurer la continuité et le bon fonctionnement de la FEUS, telles que la gestion des affaires courantes et l’organisation d’activités sociales tout en défendant les valeurs et intérets des étudiants membres de l'AGEG.}

\article{Valeurs de la délégation}
\valeurs
{Ouverture}{Travailler de concert avec les instances relatives au campus}
{Engagement}{Représenter l’AGEG à toutes les instances de la FEUS pour y faire entendre ses valeurs}
{Intégrité}{S’assurer de l’honnêteté et de la justesse de la délégation}
{Fraternité}{Favoriser un sentiment d’appartenance entre les étudiants de génie à l’AGEG et les étudiants du campus à la FEUS}

\article{Rôles et pouvoirs}
\sousarticle{AGEG}
\alinea{Elle détermine en CE si elle désire s’investir dans l’organisation d’un congrès ou d’une activité sociale;}
\alinea{Elle essaie de présenter le maximum de membres de l'AGEG dans les diverses instances de la FEUS et de l'université.}

\sousarticle{VPAU actif de l’AGEG}
\alinea{Il est le représentant officiel de l’AGEG à la FEUS;}
\alinea{Il est le chef de la délégation aux congrès de la FEUS;}
\alinea{Il est responsable de choisir les membres de la délégation.}

\sousarticle{Délégation de génie aux congrès de la FEUS}
\alinea{Elle représente l’AGEG dans les instances des congrès;}
\alinea{Elle vote selon les indications du VPAU de l’AGEG.}
\alinea{Si elle ne reçoit pas d’indications du VPAU, chaque membre vote selon ses valeurs profondes en considérant qu’elle représente l’opinion d’une partie des membres de l’AGEG}
\alinea{Elle rapporte au CA de l’AGEG tout écartement au mandat du VPAU}

\partie{Délégation de génie aux congrès de la FEUS}
\article{Composition}
\sousarticle{Membres de la délégation}
\alinea{La délégation est formée d’un nombre d’étudiants recompté chaque année par la FEUS;}
\alinea{La formation de la délégation est à la discrétion du chef de délégation.}

\sousarticle{Rémunération}
\alinea{Les membres de la délégation ne recevront aucune rémunération pour leur fonction.}

\sousarticle{Mandat}
\alinea{Le mandat de la délégation est d’un congrès.}

\article{Nomination et démission}
\sousarticle{Nomination du chef de délégation}
\alinea{Le chef de délégation est le VPAU actif.}

\sousarticle{Nomination d’un représentant en commission}
\alinea{Advenant l’impossibilité pour le chef de délégation d’assister à une commission du congrès, il peut nominer un membre de sa délégation pour agir en tant que représentant de l’AGEG au sein de cette commission.}

\sousarticle{Nomination des membres de la délégation}
\alinea{Il est du rôle du chef de délégation de recruter le nombre d’étudiants requis par la FEUS;}
\alinea{La priorité sera donnée aux étudiants ayant été sur la délégation précédente, tout en conservant au minimum un siège pour un étudiant qui n’y a jamais siégé.}

\sousarticle{Démission et destitution}
\alinea{Toute démission de tout membre du comité devra être annoncée au chef de délégation et sera effective après son approbation;}
\alinea{Un membre de la délégation peut se voir expulsé de la délégation par le  chef de délégation ou le CA de l’AGEG pour des raisons jugées valables;}
\alinea{Un poste laissé vacant sera comblé par un étudiant du 1er cycle et membre de l’AGEG nommé par le  chef de délégation.}
\alinea{Dans le cas de la démission du directeur de la délégation, le CE de l’AGEG devra nommer un membre de la délégation en tant que directeur intérimaire jusqu’à ce que le responsable de la charge soit de nouveau disponible.}
\alinea{En cas de conflit entre le membre et le chef de délégation le CA de l’ageg agit comme arbitre.}
\alinea{Toutes décision du CA relatif à cette article est final et sans appel.}

\article{Positions}
\sousarticle{Critères}
\alinea{La délégation choisit ses positions selon les valeurs et intérêts de l’AGEG et de ses membres;}
\alinea{La position finale de la délégation est donnée par le VPAU de l’AGEG.}

\sousarticle{Conflit d’intérêts}
\alinea{En accord avec le règlement sur les conflits d’intérêts de l’AGEG et de la FEUS, si un membre de la délégation se trouve en situation de conflit d’intérêts, il devra le signaler aux membres de la délégation et s’abstenir lors des discussions de la délégation ou en instance.}
\alinea{Un membre en conflit d’intérêts devra voter suivant la position du chef de délégation.}

\adoption{3 avril 2016}{7 avril 2016}