\reglement{Relatif à l’AGEG Band}

\preambule{L’objectif du présent règlement est de guider l’AGEG dans la sélection de l’AGEG Band et d’encadrer le groupe dans ses activités.}

\partie{Dispositions générales}
\article{Mission de l'AGEG Band}
\alinea{Permettre à des étudiants de s’épanouir dans un contexte musical et de connaître une expérience de scène semi-professionnelle.}
\alinea{Créer une ambiance de détente lors de différents événements et activités organisés par l’AGEG.}
\alinea{De toujours, en tant que représentant externe de l’AGEG, se présenter d’une façon professionnelle sur la scène musicale locale.}

\article{Vision de l’AGEG Band}
\alinea{La vision du AGEG Band est d’assurer une grande variété de chansons et de musique lors d’événements de l’AGEG; d’avoir une représentation soutenue, constante et adéquate de l’AGEG devant tous ses auditoires potentiels; de s’épanouir et de développer un sens des affaires dans le cadre de l’organisation, de la gestion, de la promotion et de toutes tâches connexes au groupe et aux spectacles.}

\article{Valeurs de l’AGEG Band}
\valeurs
{Ouverture}{Faire preuve de démocratie dans la sélection des membres de l’AGEG Band}
{Engagement}{Encourager l’implication étudiante à l’Université de Sherbrooke}
{Intégrité}{Assurer la sélection d’une délégation de membres ayant un excellent sens des responsabilités}
{Fraternité}{Encourager la coopération entre les musiciens qui étudient en génie}

\article{Rôles et pouvoirs}
\sousarticle{VPAS de l’AGEG}
\alinea{Il est le président du Comité de sélection de l’AGEG Band.}
\alinea{Il assure le bon fonctionnement du Comité de sélection et a le pouvoir décisionnel en cas d'égalité des votes.}
\alinea{Il lance l’appel à la mise en candidature, en collaboration avec le directeur AGEG Band, au plus tard avant le CA3 de la session actuelle.}
\alinea{Il assure un bon contact entre l’AGEG et l’AGEG Band.}
\alinea{Il veille au bon fonctionnement du groupe et agit comme médiateur en cas de litige.}

\sousarticle{VPAF de l’AGEG}
\alinea{Il s'occupe de la gestion du budget de l'AGEG Band}
\alinea{Il effectue le paiement relatif à la location du local de pratique au début de chaque session.}

\sousarticle{CA de l’AGEG}
\alinea{Il nomme deux (2) membres de l’AGEG, proposés par le VPAS, pour faire partie du Comité de sélection.}

\sousarticle{Membre actif de l'AGEG Band}
\alinea{Il a été sélectionné par le Comité de sélection de l'AGEG Band pour la session actuelle.}

\sousarticle{Membre passif de l'AGEG Band}
\alinea{Il a été sélectionné par le Comité de sélection de l'AGEG Band pour la prochaine session.}

\sousarticle{Directeur AGEG Band}
\alinea{Il est responsable de la formation actuelle de l'AGEG Band, dont il doit être un membre actif.}
\alinea{Il organise les pratiques et s'occupe de toutes les communications internes et externes du groupe.}
\alinea{Il fixe un calendrier des événements avec le VPAS de l'AGEG et le reste de sa formation.}
\alinea{Il est responsable des équipements du groupe et doit tenir l'inventaire du matériel 
à jour.}
\alinea{Il doit produire un rapport à la fin de sa session en poste, comportant le résumé des activités du groupe et la liste des musiciens membres et invités ainsi que la liste des chansons apprises.}

\partie{Comité de sélection de l’AGEG Band}
\article{Composition du comité}
\sousarticle{Membres du comité}
\alinea{Le comité doit être formé de deux (2) membres du CA de l’AGEG, du VPAS, du directeur AGEG Band, d'un (1) autre membre actif de l'AGEG Band ainsi que d'un à deux (1-2) anciens membres de l’AGEG Band, suggérés au CA par le VPAS.}
\alinea{Un ancien membre de l'AGEG Band doit avoir été membre actif d'au moins une des trois formations antérieures à la formation actuelle pour faire partie du Comité de sélection.}
\alinea{Si la formation passive est incomplète, un membre de l'AGEG Band passif doit également s'ajouter au Comité de sélection.}
\alinea{Les membres du Comité de sélection ne peuvent pas être candidat.}

\sousarticle{Rémunération}
\alinea{Les membres du Comité de sélection ne recevront aucune rémunération pour leurs fonctions.}

\article{Nomination et démission}
\sousarticle{Nomination des membres du comité}
\alinea{La durée du mandat des membres du Comité de sélection est d’une session.}
\alinea{Le Directeur AGEG Band et le VPAS siègent d’office.}

\sousarticle{Démission}
\alinea{Toute démission de tout membre du comité devra parvenir par écrit au VPAS de l’AGEG et sera effective après son approbation.}
\alinea{Un membre du comité peut se voir expulsé du comité par le CA de l’AGEG ou par l’assemblée générale de l’AGEG pour des raisons jugées valables.}
\alinea{Un poste laissé vacant sera comblé par un membre de l’AGEG ou un ancien membre de l’AGEG Band suggéré au CA par le VPAS de l’AGEG.}

\article{Sélection}
\sousarticle{But}
\alinea{La sélection permet de choisir les membres du groupe qui sera actif deux (2) sessions plus tard.}
\alinea{Si un ou plusieurs postes demeurent vacants suite à la sélection, ceux-ci devront être offerts à nouveau à la sélection de la session suivante.}

\sousarticle{Critères}
\alinea{Les critères de sélection de l’AGEG Band permettent au comité d’évaluer et de comparer les candidats afin de sélectionner les meilleurs pour représenter l’AGEG au sein d'un même groupe.}
\alinea{Les critères sont les suivants :}
\sousalinea{Talent musical de niveau suffisant pour être fonctionnel dans un groupe}
\sousalinea{Expériences musicales et implications, antérieures et actuelles}
\sousalinea{Versatilité du répertoire et du style de musique}
\sousalinea{Attitude, charisme et présence de scène}
\sousalinea{Motivation, intérêt et compréhension de ce que représente l'AGEG Band.}

\sousarticle{Préparation}
\alinea{Les candidatures sont ouvertes un minimum de deux (2) semaines avant la date des auditions.}
\alinea{Une liste des chansons à apprendre doit être rendue disponible à l'ouverture des candidatures.}
\alinea{Même si seuls les membres du Comité de sélection peuvent participer à la décision, la journée de sélection doit être annoncée publiquement et être ouverte à tous les membres de l'AGEG.}

\partie{L'AGEG Band}
\article{Composition de l’AGEG Band}
\alinea{La composition de l'AGEG Band est déterminée par le CA lors de la réunion du CA qui suit les auditions effectuées par le Comité de sélection.}
\alinea{Une formation complète de l'AGEG Band doit être constituée de un (1) chanteur, deux (2) guitaristes, un (1) bassiste et un (1) batteur. Le poste de chanteur peut être comblé en même temps qu'un autre poste par une même personne, si cela respecte des critères de sélection.}
\alinea{Tous les membres de l'AGEG Band doivent être membres de l'AGEG.}
\alinea{Si un poste demeure toujours vacant à son entrée en poste, le groupe actif pourra alors désigner lui-même un remplaçant. Ce dernier ne doit pas nécessairement être membre de l'AGEG.}

\article{Respect du calendrier}
\alinea{Après approbation du calendrier des événements par le VPAS, l’AGEG Band doit respecter ce calendrier.}
\alinea{Ce calendrier doit contenir au moins 3 spectacles par session. L’AGEG Band peut modifier le calendrier des évènements durant la session dans le cas spécifique d’un ajout de concerts. L’AGEG Band devra informer le VPAS de cet ajout. Aucune annulation ou modification du calendrier des évènements ne peut être effective sans l’approbation de l’AGEG.}

\article{Spectacles payants}
\alinea{L’AGEG Band doit organiser au minimum un (1) spectacle payant au cours de la session afin d’assurer son développement. Des recommandations et une liste de contact sont fournies dans le guide de l’AGEG Band.}
\alinea{Les spectacles payants pourront être organisés avec ou sans la collaboration d’un ou des groupes étudiants. L’AGEG Band devra cependant respecter les semaines d’activités sociales tel que prescrit par les règlements de l’AGEG.}
\sousalinea{Si l'AGEG Band joue dans un événement organisé en collaboration avec un ou des groupes étudiants, un minimum de 25\% de la répartition des profits de cet événement devra être remis à l'AGEG Band et déposé dans le fonds AGEG Band.}

\article{Rémunération}
\alinea{Les membres de l’AGEG Band ne recevront aucune rémunération pour leur fonction.}
\alinea{Tout argent gagné par l’AGEG Band lors de ses activités sera déposé dans les actifs de l'AGEG Band.}
\alinea{Le budget de l'AGEG Band doit servir au développement à long terme du groupe. Ce budget est administré par le CA de l’AGEG, selon les recommandations du directeur AGEG Band et du VPAS.}

\article{Style de musique}
\alinea{Les membres de l'AGEG Band sont responsables de choisir les chansons qu'ils performent. Cependant, leurs choix doivent respecter les critères suivants :}
\sousalinea{Le style de musique doit convenir à un large public et plaire à la communauté étudiante.}
\sousalinea{Les chansons doivent être entraînantes, populaires et dynamiques.}
\sousalinea{Le répertoire doit être varié et ne pas se limiter à un seul genre.}
\sousalinea{Le répertoire doit inclure la chanson thème de la promotion finissante (si applicable).}
\sousalinea{Chaque spectacle doit comporter un minimum de (2) chansons francophones.}

\article{Local de pratique}
\alinea{L'AGEG doit allouer un minimum de 1000\$ par session, de son budget courant, pour la location d'un local de pratique pour l'AGEG Band.}
\alinea{La sélection du local de pratique est entérinée par le CA de l'AGEG. Une copie du contrat de location doit être conservée par l'AGEG pour toute sa période de validité.}
\alinea{Le local de pratique ne doit être partagé avec aucun autre groupe. Seuls les membres actifs et passifs de l'AGEG Band et le VPAS doivent avoir les clés en leur possession.}
\alinea{Le local de pratique doit être gardé sous alarme en tout temps en dehors des heures de pratique.}

\article{Équipement et assurance}
\alinea{La liste des équipements appartenant à l’AGEG Band, excluant les équipements personnels des membres, doit être mise à jour et fournie à l’AGEG à la fin de chaque mandat.}
\alinea{Cette liste servira de preuve de matériel auprès de l’AGEG pour que les équipements soient enregistrés comme appartenant à l’AGEG pour pouvoir bénéficier des assurances.}
\alinea{Tout nouvel équipement acheté par l’AGEG Band doit être ajouté à la liste des équipements dans le guide de l’AGEG Band.}

\article{Démission, expulsion et remplacement}
\alinea{La démission d’un membre de l’AGEG Band devra parvenir par écrit au VPAS et sera effective après son approbation.}
\alinea{L’expulsion d’un membre de l’AGEG Band devra être entérinée par le directeur AGEG Band et devra parvenir par écrit au VPAS et sera effective après son approbation. La raison de l’expulsion doit être valable et doit être approuvée par le VPAS de l’AGEG.}
\alinea{Le remplacement du membre sera effectué par la liste de remplacement dicté par le dernier Comité de sélection.}

\adoption{21 octobre 2018}{27 novembre 2018}