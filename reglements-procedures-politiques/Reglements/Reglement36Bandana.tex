\reglement{Relatif au Bandana de Génie}

\preambule{L’objectif du présent règlement est de donner des lignes directrices au comité Bandana dans la gestion, l’attribution et le suivi du fonds et des activités en lien avec le Bandana de génie.}

\partie{Dispositions générales}
\article{Vision du comité Bandana}
\alinea{La vision du comité est de pouvoir innover dans diverses activités promouvant le Bandana de génie, toujours dans le but d’assurer la continuité des valeurs en lien avec ce dernier.}

\article{Valeurs du comité Bandana}
\valeurs
{Ouverture}{Faire preuve d’innovation dans les activités promouvant la tradition}
{Engagement}{Fournir aux étudiants de génie un item les identifiant et promouvoir le sentiment d'appartenance, tout en s’assurant qu’il conserve son importance}
{Intégrité}{Faire preuve de bon jugement dans le processus d’attribution des fonds}
{Fraternité}{Organiser des activités pour les porteurs dans un esprit de franche camaraderie}

\article{Rôles et pouvoirs}
\sousarticle{AGEG}
\alinea{Elle entérine le budget et le plan d’action du comité en CA.}

\sousarticle{VPAI de l’AGEG}
\alinea{Il explique les tâches aux membres du comité.}
\alinea{Il assure l’intégration d’activités sociales liées au Bandana pour les membres de l’association.}
\alinea{Il aide à planifier les différentes rencontres du comité Bandana.}
\alinea{Il rédige les résumés des différentes rencontres du comité et en rapporte les actions au CA.}
\alinea{Il supervise le directeur Bandana et le comité Bandana en l’absence du directeur Bandana.}

\sousarticle{Conseil exécutif de l’AGEG}
\alinea{Il entérine la composition du comité, dans le respect de la charte de l'AGEG (article 8.5).}

\sousarticle{Directeur Bandana}
\alinea{Il planifie les différentes réunions du comité Bandana.}
\alinea{Il veille au bon fonctionnement du comité Bandana.}
\alinea{Il supervise le bon déroulement de la remise du S d’or.}

\sousarticle{Comité Bandana}
\alinea{Il veille à s’assurer que le Bandana conserve son statut d’article d’exception et représentant la tradition et l’appartenance au génie sherbrookois.}
\alinea{Il organise des activités pour promouvoir la tradition.}
\alinea{Il s’assure de l’implantation des valeurs en lien avec le Bandana à la promotion entrante lors de la semaine d’intégration.}
\alinea{Il s’assure de rendre disponible aux étudiants l’accès aux écussons.}
\alinea{Il peut recommander l’imposition de conditions particulières sur l’utilisation du Bandana sous l’approbation du CE de l’AGEG.}
\alinea{Il veille au respect des traditions.}
\alinea{Il rédige les recommandations sur l’utilisation des fonds selon les critères du présent règlement.}
\alinea{Le comité aura la tâche de définir les critères d’obtention du S d’or par la promotion finissante. Ces critères seront approuvés par le CE de l’AGEG.}
\alinea{Il peut proposer un récipiendaire d’un Bandana honorifique (voir art 2.3.3), doit être accepté en CA.}

\article{Législation et distribution}
\sousarticle{Porteurs et récipiendaire}
\alinea{Tout étudiant membre de l’AGEG ou du G3 a droit à l’achat d’un Bandana de génie.}
\alinea{Tout professeur ou membre de l’administration  de la faculté de génie de Sherbrooke a droit à l’achat d’un Bandana de génie.}
\alinea{Toute personne ayant fait au moins 3 sessions de travail au sein de l’AGEG a droit à un Bandana.}
\alinea{Toute personne ayant faites valoir les valeurs du Bandana de génie et s’étant démarqué dans sa valorisation des étudiants de génie à travers ses actions peut se faire attribuer un Bandana honorifique (voir art 2.3.3).}

\sousarticle{Remise du S d’or}
\alinea{Toute personne possédant un Bandana peut recevoir un S d’Or. S’il s’agit d’un étudiant de première année, il doit attendre à la remise durant sa deuxième session.}

\partie{Comité Bandana}
\article{Composition du comité}
\sousarticle{Membres du comité}
\alinea{La formation et la composition du comité est à la discrétion du directeur Bandana et du VPAI, mais doit être entérinée par le CE.}

\sousarticle{Rémunération}
\alinea{Les membres du comité ne recevront aucune rémunération pour leur fonction.}
\sousarticle{Mandat}
\alinea{Le mandat du comité Bandana est d’une session.}

\article{Nomination et démission}
\sousarticle{Nomination du directeur}
\alinea{Le directeur du comité est nommé par le VPAI et doit être entériné par le CE de l’AGEG.}

\sousarticle{Nomination des membres du comité}
\alinea{Il est du rôle du directeur Bandana de recruter entre un et cinq étudiants inscrits en session d’études lors de leur mandat actif pour combler le comité Bandana.}
\alinea{Jusqu’à un membre du comité peut être un étudiant du G3.}
\alinea{Les principaux critères de sélection seront la motivation à contribuer à la valorisation du Bandana et le respect des traditions des fondateurs du Bandana.}

\sousarticle{Démission}
\alinea{Un membre du comité peut se voir expulsé du comité par le CA de l’AGEG ou l’assemblée générale de l’AGEG pour des raisons jugées valables.}
\alinea{Dans le cas de la démission du directeur du comité, le CE de l’AGEG devra nommer un nouveau directeur Bandana.}

\article{Décision}
\sousarticle{Critères}
\alinea{Le comité décide des activités qu’il veut organiser et des achats qu’il veut effectuer selon la mission, la vision et les valeurs décrites dans le présent règlement, ainsi que la mission, la vision et les valeurs véhiculées par l’AGEG et son plan directeur.}

\sousarticle{Quorum}
\alinea{Le quorum est fixé à la majorité des membres du comité.}
\alinea{Les décisions du comité se prendront par vote à majorité simple.}

\sousarticle{Bandana honorifique}
\alinea{Sur recommandation du comité Bandana, au plus un Bandana honorifique peut être donné par session.}
\alinea{Les écussons à apposer sont à la discrétion du comité Bandana.}
\alinea{La proposition doit être amenée en CA et la décision relève des administrateurs.}

\sousarticle{S d’or}
\alinea{Le comité Bandana est responsable d’approuver les demandes de S d’or.}
\alinea{Les formulaires doivent avoir été mis disponible à tous auparavant et la publicisation de l’événement doit avoir été suffisante.}
\alinea{Un Bandana refait car perdu ou volé peut orner le S doré; au jugement du comité Bandana.}

\article{Gestion financière}
\sousarticle{Respect des autres règlements de l’AGEG}
\alinea{Le prix d’un Bandana incluant l’écusson de concentration, l’écusson de promotion, l'écusson Bandanom et le S d’or est fixé à 22,50~\$ à l'intégration.}
\alinea{La gestion des montants dans le fonds est dictée par le règlement relatif aux finances (règlement 22), partie II, avec les spécifications suivantes :}
\sousalinea{Le comité Bandana peut dépenser un maximum de 300~\$ avant le CA1 de la session pour les activités de début de sessions.}
\sousalinea{Ces dépenses doivent être approuvées par le CE.}

\sousarticle{Achat, perte et rachat d’un Bandana}
\alinea{Lors de la perte de son Bandana, un étudiant aura la chance de racheter son Bandana incluant l'écusson de concentration et l'écusson de promotion au coût de 25,00~\$, sur acceptation du comité Bandana. Afin d’éviter que le Bandana soit trop répandu, une personne ne peut recevoir un troisième Bandana.}
\alinea{Sur recommandation du Comité Bandana, les exceptions devront être statuées par le CE de l’AGEG.}

\sousarticle{Gestion de l’inventaire}
\alinea{Le comité Bandana doit s'assurer, avant le denier CA de sa session, d'avoir les budgets nécessaires pour maintenir les inventaires essentiels au bon fonctionnement du comité pour la période qui va du dernier CA au CA1 de la session suivante.}
\alinea{Le comité Bandana devra s’assurer de faire des commandes judicieuses (s’assurer avant achat que la commande s’écoulera facilement et qu’il existe une demande) et s’assurer que l’inventaire s’écoule par des ventes régulières d’écussons.}
\alinea{Le comité devra faire un inventaire à chaque fin de session ; celui-ci devra se trouvé dans la section Bandana d'Ouragan}
\alinea{Un suivi des ventes devra être fait pour permettre la mise à jour de la valeur de l'inventaire.}

\sousarticle{Budget du comité}
\alinea{Les achats du comité devront être approuvés par le Conseil d'administration de l'AGEG.}

\adoption{21 octobre 2018}{27 novembre 2018}