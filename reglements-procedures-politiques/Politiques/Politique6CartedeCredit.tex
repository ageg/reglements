\politique{Relative aux cartes de crédit d'entreprise}

\preambule{Le comité exécutif de l'AGEG est responsable de ravitailler les consommables des services offerts par l'AGEG. Posséder une carte de crédit facilite ce ravitaillement et permet de sauver du temps sur la procédure de remboursement répétitive. La présente politique précise les responsabilités et les obligations qui s'appliquent aux utilisateurs d'une carte de crédit d'entreprise. La présente politique établit les règles relatives à l'émission et à l'utilisation de cartes de crédit d'entreprise qui doivent servir exclusivement aux activités de l'AGEG.}



\partie{Champs d'application}

\alinea{Ce présent document concerne tous les membres du comité exécutif et du personnel de l'AGEG qui désirent utiliser la carte de crédit d'entreprise.}




\partie{Définitions}

\article{Demandeur:}
\alinea{Personne demandant l'utilisation d'une carte de crédit d'entreprise.}

\article{Personne déléguée:}
\alinea{Personne ayant en sa possession une carte de crédit d'entreprise.}

\article{Titulaire de cartes:}
\alinea{Personne ayant la responsabilité légale de la carte de crédit.}
\alinea{Personne responsable de conserver sécuritairement les cartes de crédit d'entreprise lorsqu'elles ne sont pas déléguées.}

\article{Délai de grâce:}
\alinea{Le délai de grâce désigne le report du terme d'une dette ou l'échelonnement des échéances que l'institution bancaire accorde au titulaire de carte.}

\article{Avances de fonds}
\alinea{Retrait d'argent comptant à un guichet automatique ou à une succursale;}



\partie{Rôles et responsabilités}

\article{Conseil d'Administration}
\alinea{Le conseil d'administration s'assure que les cartes de crédit d'entreprise soit utilisées raisonnablement et sans abus.}
\alinea{Le conseil d'administration peut révoquer le droit d'utilisation de cartes de crédit d'entreprise advenant le non respect de cette politique par un membre du comité exécutif, un membre du personnel de l'AGEG ou tout autre personne déléguée autorisée.}

\article{Coordonnatrice administratrice}
\alinea{La coordonnatrice administratrice est nommée "Titulaire de cartes" et signe la documentation légale relié à la possession de cartes de crédit d'entreprise.}
\alinea{La coordonnatrice administratrice conserve les cartes de crédit d'entreprise en sécurité lorsqu'elles ne sont pas déléguées.}
\alinea{La coordonnatrice administratrice rembourse, par le compte bancaire de l'AGEG, le crédit utilisé en respectant son délai de grâce.}
\alinea{La coordonnatrice administratrice s'assure que le formulaire de délégation de la carte de crédit d'entreprise de l'AGEG soit dument remplis avant de déléguer une carte à un demandeur.}

\article{Vice-Présidence aux Affaires Financières}
\alinea{La Vice-Présidence aux Affaires Financières, en collaboration avec la coordonnatrice administratrice, s'assure que la carte de crédit se fasse rembourser dans son délai de grâce.}
\alinea{La Vice-Présidence aux Affaires Financières prévoit les frais associés aux cartes de crédit d'entreprise et modifie sont budget annuel en conséquence.}
\alinea{La Vice-Présidence aux Affaires Financières peut révoquer le droit d'utilisation de cartes de crédit d'entreprise advenant le non respect de cette politique par un membre du comité exécutif, un membre du personnel de l'AGEG ou tout autre personne déléguée autorisée. La raison doit être clairement spécifiée au conseil d'administration.}

\article{Présidence}
\alinea{La Présidence peut révoquer le droit d'utilisation de cartes de crédit d'entreprise advenant le non respect de cette politique par un membre du comité exécutif, un membre du personnel de l'AGEG ou tout autre personne déléguée autorisée. La raison doit être clairement spécifiée au conseil d'administration.}



\partie{Conditions et exigences}

\article{Possession et délivrance}
\alinea{La coordonnatrice administratrice est responsable de conserver sécuritairement les cartes de crédit d'entreprise. Elle n'est pas permise de s'autodéléguée une carte de crédit d'entreprise sans l'approbation de la Présidence ou de la Vice-Présidence aux Affaires Financières. Une carte de crédit d'entreprise est délégué à un demandeur seulement si les conditions d'admissibilités sont respectées et s'il rempli conformément et signe le registre de la carte de crédit d'entreprise de l'AGEG.}

\article{Admissibilité d'achat}
\alinea{La carte de crédit d'entreprise doit servir comme moyen de paiement uniquement pour les dépenses suivantes :}
\sousalinea{consommables reliés à un service offert par l'AGEG;}
\sousalinea{souper du Comité Exécutif avant une réunion du conseil d'administration;}
\sousalinea{fournitures de bureau, dépenses de bureau et équipements informatiques;}
\sousalinea{frais de déplacement CE à l'exception des frais de l'essence.}
\alinea{L’utilisation d’une carte de crédit d’entreprise est limitée aux activités de l'AGEG; il est donc strictement interdit de s’en servir à des fins personnelles.}

\article{Admissibilité de délégation}
\alinea{La carte de crédit d'entreprise peut seulement être déléguée aux personnes suivantes :}
\sousalinea{Présidence;}
\sousalinea{Vice-Présidence aux Affaires Financières;}
\sousalinea{Vice-Présidence aux Affaires Internes;}
\sousalinea{un membre du personnel de l'AGEG ou un membre du comité exécutif approuvé par une des trois personnes mentionnées ci-haut.}

\article{Avances de fonds}
\alinea{Les avances monétaires sont interdites sauf lors de situations d’extrême urgence.}

\article{Responsabilité}
\alinea{Toute opération effectuée au moyen de la carte de crédit doit être justifiée à l'aide d'une pièce justificative.}
\alinea{Les opérations effectuées au moyen de la carte de crédit sont la responsabilité de la personne déléguée jusqu’à ce qu’elles soient autorisées.}
\alinea{La personne déléguée peut se faire demander de payer toute opération pour laquelle les pièces justificatives ne sont pas fournies.}

\article{Rapport d'utilisation}
\alinea{La Vice-Présidence aux Affaires Financières fournit, à chaque réunion du conseil d'administration, un état de compte et un rapport sur l'utilisation et les dépenses relatives aux cartes de crédit d'entreprise.}

\article{Incessibilité}
\alinea{Les cartes de crédit d’entreprise sont à l’usage exclusif des personnes autorisées pour qui elles ont été déléguées. Ainsi, elles ne peuvent être ni transférées ni utilisées par aucune autre personne. Toutefois, le titulaire de carte peut, lorsque nécessaire pour l’efficience des opérations, demander à un membre du personnel de faire des achats particuliers en son nom. Ce faisant, le membre du personnel est tenu de respecter toutes les obligations prévues dans la présente Politique.}

\article{Carte perdue ou volée}
\alinea{La perte ou le vol d’une carte de crédit d’entreprise doit être signalé immédiatement à la banque émettrice, à la Vice-Présidence aux Affaires financières et à la Coordonnatrice administratrice. La personne déléguée ni la personne titulaire de cartes ne peuvent être tenue responsable des frais occasionnés à l'AGEG suite à la déclaration de carte perdue ou volée, excepté si une de ces deux personnes est prouvée coupable de vole.}

\article{Abus}
\alinea{En cas d’abus d’une carte de crédit d’entreprise, le Conseil d'administration, la Présidence ou la Vice-Présidence aux Affaires Financières peuvent la révoquer. La personne déléguée est sujet à des mesures disciplinaires déterminé par le Conseil d'administration.}

\adoptionPO{3 Février 2019}