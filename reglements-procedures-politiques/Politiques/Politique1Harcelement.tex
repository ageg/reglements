\politique{Relative au harcèlement, à la discrimination et aux violences à caractère sexuel}

\preambule{En tant qu’association étudiante, l'AGEG veut promouvoir et offrir un milieu sain et sécuritaire dans lequel les personnes sont traitées avec respect et dignité. Par cette politique, elle exprime clairement sa volonté de contrer toute forme de harcèlement, de discrimination et de violence. Elle  s’engage à sensibiliser le milieu, à prendre toutes les mesures préventives nécessaires et, le cas échéant, les décisions et actions qui s’imposent pour redresser les situations.}

\partie{Champs d'application}
\alinea{Cette politique s’applique, de manière générale, à tous les membres de l'AGEG incluant sans s'y limiter le personnel, les personnes administrantes, les personnes exécutantes et les personnes de l’extérieur qui, pour des motifs liés au travail, aux études ou à l’utilisation des services,  fréquentent l'AGEG (fournisseurs, clients, participants à des activités sociales, partenaires, etc.). Elle s'applique autant sur le campus que lors de toute autre activité où des membres de l'AGEG y sont en tant que représentants de l'association. Elle s'ajoute à la politiques universitaire 2500-015 sur le harcèlement et la discrimination et la politique 2500-042 sur la prévention des violences à caractère sexuel.}

\partie{Philosophie}

\article{Responsabilité}
\alinea{L'AGEG est responsable de la sécurité de ses membres, de ses employés et de toutes personnes qui prennent part aux activités qu'elle organise ou endosse ou qui côtoient une personne agissant au nom de l'AGEG.}

\article{Vigilance}
\alinea{L'AGEG de par son rôle d'organisatrice d'activités sociales avec alcool, doit faire d'autant plus preuve de vigilance, car ces activités sont fortement à risque de mener à du harcèlement ou des violences à caractère sexuel.}

\article{Prévention}
\alinea{L'AGEG doit agir en amont pour prévenir le harcèlement et les violences à caractère sexuel. La protection des personnes doit être l'objectif principal des décisions concernant le harcèlement et les violences à caractère sexuel.}

\article{Indépendance}
\alinea{L'AGEG fait appel à l'Université pour la gestion des plaintes par souci d'indépendance. L'AGEG est composée d'un petit groupe de personnes entretenant souvent des relations amicales, ce qui empêche l'AGEG de garantir l'indépendance des personnes pouvant être amené à traiter la plainte à l'interne et éviter les conflits d'intérêts réels ou potentiels.}

\article{Compétence}
\alinea{L'AGEG fait appel à l'Université pour la gestion des plaintes par souci de compétence. Les changements fréquents des dirigeants de l'AGEG empêchent l'AGEG de garantir la compétence dans la gestion de dossier de harcèlement et les violences à caractère sexuel des personnes pouvant être amenées à gérer ces dossiers à l'interne.}

\article{Exemplarité}
\alinea{Il est attendu que les personnes représentant l'AGEG agissent en tant qu'exemple pour leurs pairs. Représenter l'AGEG est un privilège.}

\article{Amélioration continue}
\alinea{L'AGEG doit constamment améliorer la présente politique et prendre action afin de réduire le harcèlement et les violences à caractère sexuel.}

\partie{Prévention}

\article{Formation}
\alinea{Tous les membres de l'AGEG doivent suivre tous les ans une formation sur le harcèlement et les violences à caractère sexuel de l'Université.}

\alinea{Toutes les personnes prenant part aux intégrations, les jeux de génie du Québec et la compétition québécoise d'ingénierie doivent recevoir une formation sur le consentement, les violences à caractère sexuel et la consommation responsable d'alcool.}

\alinea{Les personnes organisant des activités sociales pour le compte de l'AGEG (Oktoberfest, 5@8, Intégrations, etc.) doivent recevoir une formation sur le prévention du harcèlement et des violences à caractère sexuel.}

\article{Communication}
\alinea{L'AGEG doit maintenir sur son site internet, sur le site internet de ses activités sociales (site de l'Oktoberfest, site des Intégrations) et de ses délégations une section sur le harcèlement et les violences à caractère sexuel. Cette section doit au minimum contenir un lien vers la présente politique et un lien vers une page contenant les informations pour obtenir du support.}

\alinea{Les profils sur les réseaux sociaux de l'AGEG, de ses activités sociales et de ses délégations doivent aussi contenir un lien vers la présente politique et un lien vers une page contenant les informations pour obtenir du support et comment déposer une plainte.}

\alinea{L'AGEG, incluant ses groupes, doit encourager la prévention du harcèlement et des violences à caractère sexuel et faire connaître les façons d'obtenir du support notamment via de l'affichage dans les activités sociales avec alcool et des rappels sur les réseaux sociaux.}

\article{Pair(e) aidant(e)}

\alinea{Les pair(e)s aidant(e)s sont des personnes qui ont reçu une formation spécifique pour accompagner les personnes qui ont subis du harcèlement ou une violence à caractère sexuel montée en collaboration avec le bureau du respect des personnes de l'Université.}

\alinea{Le rôle des pair(e)s aidant(e)s est d'agir comme personne-ressource auprès des personnes qui les contacts pour les écouter, les rassurer, les informer sur les procédures possibles et les accompagner sur demande dans les démarches de résolution du problème.}

\alinea{La liste des pair(e)s aidant(e)s doit être disponible sur la page contenant les façons d'obtenir du support.}

\alinea{Les pair(e)s aidant(e)s sont tenus de tenir les informations qu'ils ou elles reçoivent strictement confidentiel.}

\alinea{Les pair(e)s aidant(e)s doivent être irréprochable vis-à-vis le harcèlement et les violences à caractère sexuel. Tout pair(e) aidant(e) dont l'irréprochabilité est mis en cause sera retiré de la liste.}

\alinea{L'AGEG doit, dans la mesure du possible, s'assurer de recruter des profils diversifiés de personnes en tant que pair(e) aidant(e).}

\article{Ange gardien}

\alinea{Les anges gardiens sont des personnes qui ont reçu une formation spécifique pour agir comme témoin actif et pour intervenir sur lors des activités sociales montée en collaboration avec le bureau du respect des personnes de l'Université.}

\alinea{Le rôle des anges gardiens est de protéger les personnes qui ne seraient pas à l'aise ou qui ne se sentiraient pas en sécurité lors d'activités sociales en collaboration avec les organisateurs de l'évènement.}

\alinea{Un nombre suffisant d'anges gardiens doivent être présents à toutes les activités sociales organisées par l'AGEG.}

\alinea{Les anges gardiens doivent être identifiés par un signe distinctif lors des activités sociales.}

\alinea{Des personnes participant aux jeux de génie du Québec et la compétition québécoise d'ingénierie doivent agir en tant qu'ange gardien.}

\alinea{L'AGEG doit, dans la mesure du possible, s'assurer de recruter des profils diversifiés de personnes en tant qu'Ange gardien.}

\article{Collaboration}

\alinea{L'AGEG doit collaborer avec les personnes organisant des activités sociales, qu'elles soient internes (Oktoberfest, Intégrations, 5@8 de génie, etc.) ou externes (5@8 d'une autre faculté, compétitions, party dans une autre université) et avec la sécurité de l'Université pour s'assurer de limiter les risques pour les personnes participantes à ses activités.}


\partie{Plainte}
\alinea{Pour garantir l'indépendance du traitement des plaintes de harcèlement, de la discrimination et de violences à caractère sexuel, l'AGEG a décidé de confier le processus de gestion des plaintes au bureau du respect des personnes de l'Université de Sherbrooke. Celui-ci recueille les plaintes et les traite de façon indépendante de l'AGEG et en toute confidentialité.}

\partie{Procédures d'accompagnement}

\alinea{Au besoin, un(e) pair(e) aidant(e) dirige la personne qui a subi du harcèlement, de la discrimination ou une violence à caractère sexuel sur le processus auprès du bureau du respect des personnes et l'accompagne dans sa démarche si la personne le demande.}

\alinea{Si la personne ne désire pas porter plainte, l'AGEG ou un(e) pair(e) aidant(e) l'informe tout de même des ressources disponibles pour obtenir du soutien.}

\partie{Sanctions}

\alinea{Toute sanction suggérée par l'Université à l'issue d'une plainte au bureau du respect des personnes sera automatiquement mise en application dès que l'AGEG en sera informée.}

\alinea{Les sanctions sont appliquées et traitées conformément au processus prévu aux politiques universitaires.}

\alinea{L'AGEG, par son conseil d'administration, se réserve le droit d'imposer des sanctions à ses membres, dirigeants ou employés en dehors du processus normal si celui-ci ne peut pas suivre son cours, par exemple parce que le bureau de harcèlement des personnes n'a pas juridiction ou que l'exemplarité des dirigeants de l'AGEG est remise en cause.}


\adoptionPO{26 juillet 2020}
