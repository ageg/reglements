\politique{Relative à la prévention du harcèlement}

\preambule{En tant qu’association étudiante, l'AGEG veut promouvoir et offrir un milieu sain et sécuritaire dans lequel les personnes sont traitées avec respect et dignité. Par cette politique, elle exprime clairement sa volonté de contrer toute forme de harcèlement et de violence. Elle  s’engage à sensibiliser le milieu, à prendre toutes les mesures préventives nécessaires et, le cas échéant, les décisions et actions qui s’imposent pour redresser les situations.
En adoptant la présente politique, l'Association Générale Étudiante en Génie souhaite également promouvoir une culture organisationnelle de civilité, de respect mutuel et de responsabilisation des individus face à leurs comportements et à leurs impacts dans l’organisation. En ce sens, la civilité comporte à la fois un caractère de droit et d’obligation pour chaque personne. Les principes et règles compris dans la présente politique ainsi que dans la procédure qui en découle s’accordent avec la Charte québécoise des droits et libertés de la personne et les lois qui régissent les relations entre les personnes et les relations de travail, plus particulièrement les droits et obligations des employeurs.}

\partie{Champs d'application}
\alinea{Cette politique s’applique, de manière générale, à tous les membres de l'AGEG incluant sans s'y limiter le personnel, les personnes administrantes, les personnes exécutantes et les personnes de l’extérieur qui, pour des motifs liés au travail, aux études ou à l’utilisation des services,  fréquentent l'AGEG (fournisseurs, clients, partenaires, etc.). Elle s'applique autant sur le campus que toute autre activité où des membres et de l'AGEG y sont en tant que représentants de l'association.}

\partie{Définitions}
\article{Violence physique}
\alinea{Brutalité (menaces, coups ou contraintes physiques) à l’endroit d’objets ou à l’égard d’une ou de plusieurs personnes dans le but d’intimider et de contraindre.}
\article{Violence à caractère sexuel}
\alinea{Elle se définit comme toute forme de violence, physique ou psychologique, perpétrée par le biais de pratiques sexuelles ou ciblant la sexualité qui porte atteinte à la dignité ou à l’intégrité physique et psychologique de la personne.}
\alinea{Les violences à caractère sexuel incluent les expériences sexuelles non désirées sans contact.}

\alinea{La violence ou harcèlement sexuel n’est pas :}
\sousalinea{Un comportement social normal de simple camaraderie ou de badinage;}
\sousalinea{Un flirt ou une aventure entre deux personnes consentantes. Ces relations sont consenties, basées sur une attraction mutuelle et ne supposent aucune intimidation (notion de libre consentement).}

\article{Violence ou harcèlement discriminatoire}
\alinea{Conduite se manifestant, entre autres, par des paroles, des actes ou des gestes à caractère vexatoire ou méprisant, à l’égard d’une personne ou d’un groupe de personnes, en raison de l’un ou l’autre des motifs énumérés à l’article 10 de la Charte québécoise des droits et libertés de la personne, à savoir la race, la couleur, le sexe, la grossesse, l’orientation sexuelle, l’état civil, l’âge, sauf dans la mesure prévue par la loi, la religion, les convictions politiques, la langue, l’origine ethnique ou nationale, la condition sociale, le handicap ou l’utilisation d’un moyen pour pallier ce handicap.}

\article{Harcèlement sexuel}
\alinea{Le harcèlement sexuel se définit comme un comportement à connotation sexuelle se manifestant par des paroles, des gestes, des actes répétés, non désirés par la personne visée, soit pour obtenir sous pression des faveurs sexuelles, soit pour ridiculiser ses caractéristiques sexuelles et humilier la personne, en portant ainsi atteinte à sa dignité ou à son intégrité psychologique ou physique, en créant pour elle des conditions de travail ou d’études défavorables. }
\alinea{Une seule conduite de nature sexuelle, grave et non désirée qui produit des effets nocifs dans le temps, peut constituer du harcèlement sexuel.}
\article{Harcèlement psychologique}
\alinea{Conduite qui se manifeste soit par des comportements, des paroles, des actes ou des gestes répétés, qui est hostile ou non désiré, laquelle porte atteinte à la dignité ou à l’intégrité psychologique ou physique d’une personne et qui entraîne, pour elle, un milieu de travail ou d’études néfaste.}
\alinea{Une seule conduite grave peut aussi constituer du harcèlement psychologique si elle porte une telle atteinte et produit un effet nocif continu pour la personne.}

\article{Abus de pouvoir ou d’autorité}
\alinea{Comportement ou geste d’une personne pouvant se manifester par de l’intimidation, des menaces, du chantage, de la coercition ou une surveillance excessive et injustifiée visant à profiter indûment de son statut d’autorité ou d’une situation de pouvoir dans l’intention d’humilier ou diminuer une personne ou un groupe de personnes et susceptible de compromettre le travail et l'intégrité de cette ou ces personnes.}
\alinea{L'abus de pouvoir peut avoir lieu dans tous les milieux de travail tel qu'un comité organisateur, conseil exécutif, conseil d'administration, groupe technique ou toute autre groupe où une ou plusieurs personnes est en position d'autorité sur d'autres personnes.}
\alinea{L’abus de pouvoir ou d’autorité ne peut être assimilé à l’exercice légitime des droits de gérance, notamment la gestion habituelle de la discipline, du rendement au travail ou de l’absentéisme, lorsqu’exercés de façon raisonnable.}

\article{Incivilité}
\alinea{Acte ou comportement qui dénotent un rejet des règles élémentaires de la vie sociale, lesquelles règles visent le bien-être d’un groupe. Parmi ces règles, mentionnons le respect, la collaboration, la politesse, la courtoisie et le savoir-vivre.\bigskip}
\hspace*{1.5 cm}\underline{\textbf{Mise en garde}}\bigskip
\newline{Les différends ou situations de tension pouvant survenir entre deux personnes ou entre une personne et un groupe ne constituent pas nécessairement des manifestations de violence ou de harcèlement au sens de la présente politique.}

\partie{Rôles et responsabilités}

\article{Conseil d'administration}
\alinea{Le conseil d'administration assume la responsabilité générale de la politique.}

\article{Comité exécutif}
\alinea{Le comité exécutif informe les personnes, attachées de près ou de loin à l'AGEG, de cette politique, comme le personnel, les membres des groupes techniques et les comités organisateurs.}
\alinea{Le comité exécutif promouvoit les services de l'Université. Cela inclut les politiques (xx) et (yy) de l'Université}


\partie{La procédure applicable}

\article{Le dépôt de la plainte}
\alinea{Toute personne incluse dans le champs d'application qui estime faire l'objet ou être témoin de harcèlement ou de violence peut, si elle le désire, en discuter avec tout membre du comité exécutif ou membre du conseil d'administration avec qui cette personne se sent à l'aise d'en parler.}
\alinea{La personne qui désire faire une plainte doit remplir le formulaire de dépôt de plainte de l'AGEG et le faire parvenir à l'exécutif.}
\alinea{Si nécessaire, une intervention d'urgence est mise en place pour faire cesser immédiatement le harcèlement ou le comportement violent.}

\article{Traitement de la plainte}
\alinea{L'AGEG ne traite pas la plainte directement. L'AGEG accompagne la personne plaignante tout au long du traitement de la plainte.}
\alinea{La personne ayant reçu la plainte met en contact la personne plaignante avec le Service à la Vie Étudiante (SVE).}
\alinea{Toute plainte sera traitée selon les politiques (xx) et (yy).}

\article{Confidentialité}
\alinea{Toute plainte sera traitée confidentiellement.}
\alinea{Toute personne impliquée dans la gestion d’une plainte sera tenue de ne pas discuter des faits entourant la plainte avec d’autres personnes, sauf à des fins autorisées par la Loi, par cette politique ou à des fins de consultation auprès d’un conseiller externe, le cas échéant.}
\alinea{Les personnes qui, de bonne foi, se prévalent de la politique sont protégées contre les représailles qui pourraient être prises à leur endroit parce qu’elles ont utilisé la politique. Il en est de même des témoins.}

\article{Les plaintes frivoles ou vexatoires}
\alinea{Une plainte doit être sérieuse et faite de bonne foi.}
\alinea{Un employé ou un membre de l'AGEG qui déposerait une plainte frivole ou une accusation de mauvaise foi ou avec l’intention de nuire se verra imposer des mesures disciplinaires.}

\article{Contacts}
\alinea{La personne repère au niveau de l'Université (SVE) est:}
\newline {\color{white}.\color{black}\hspace{1cm}Luc Sauvé}
\newline {\color{white}.\color{black}\hspace{1cm}Directeur général du Service à la Vie Étudiante}
\newline {\color{white}.\color{black}\hspace{1cm}819 821-8000, poste 63660}
\newline {\color{white}.\color{black}\hspace{1cm}luc.sauvé@usherbrooke.ca}
\newline {\color{white}.\color{black}\hspace{1cm}B1-1043-2}
\par\leavevmode
\alinea{Les personnes ressources à la facutlé de génie:}
\newline {\color{white}.\color{black}\hspace{1cm}Stéphan Roux}
\newline {\color{white}.\color{black}\hspace{1cm}Secrétaire général de la faculté}
\newline {\color{white}.\color{black}\hspace{1cm}819 821-8000, poste 63113}
\newline {\color{white}.\color{black}\hspace{1cm}stephan.roux@usherbrooke.ca}
\newline {\color{white}.\color{black}\hspace{1cm}C1-3001}
\par\leavevmode



\adoptionPO{\add{(à venir)}}